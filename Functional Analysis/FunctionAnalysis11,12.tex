\documentclass[12pt, a4paper, oneside]{ctexart}
\usepackage{amsmath, amsthm, amssymb, bm, color, graphicx, geometry, mathrsfs,extarrows, braket, booktabs, array, wrapfig, enumitem}
\usepackage[colorlinks,linkcolor=red,anchorcolor=blue,citecolor=blue,urlcolor=blue,menucolor=black]{hyperref}
\setCJKmainfont{方正新书宋_GBK.ttf}[ BoldFont = 方正小标宋_GBK, ItalicFont = 方正楷体_GBK]
\setmainfont{Times New Roman}  % 设置英文字体
\setsansfont{Calibri}
\setmonofont{Consolas}

\linespread{1.4}
%\geometry{left=2.54cm,right=2.54cm,top=3.18cm,bottom=3.18cm}
\geometry{left=1.84cm,right=1.84cm,top=2.18cm,bottom=2.18cm}
\newcounter{problem}  % 问题序号计数器
\newenvironment{problem}[1][]{\stepcounter{problem}\par\noindent\textbf{题目\arabic{problem}. #1}}{\smallskip\par}
\newenvironment{solution}[1][]{\par\noindent\textbf{#1解答. }}{\smallskip\par}  % 可带一个参数表示题号\begin{solution}{题号}
\newenvironment{note}{\par\noindent\textbf{注记. }}{\smallskip\par}

%%%% 图片相对路径 %%%%
\graphicspath{{figure/}} % 当前目录下的figure文件夹, {../figure/}则是父目录的figure文件夹

%%%% 缩小item,enumerate,description两行间间距 %%%%
\setenumerate[1]{itemsep=0pt,partopsep=0pt,parsep=\parskip,topsep=5pt}
\setitemize[1]{itemsep=0pt,partopsep=0pt,parsep=\parskip,topsep=5pt}
\setdescription{itemsep=0pt,partopsep=0pt,parsep=\parskip,topsep=5pt}

\everymath{\displaystyle} % 默认全部行间公式
\DeclareMathOperator*\uplim{\overline{lim}} % 定义上极限 \uplim_{}
\DeclareMathOperator*\lowlim{\underline{lim}} % 定义下极限 \lowlim_{}
\let\leq=\leqslant % 将全部leq变为leqslant
\let\geq=\geqslant % geq同理
\DeclareRobustCommand{\rchi}{{\mathpalette\irchi\relax}}
\newcommand{\irchi}[2]{\raisebox{\depth}{$#1\chi$}} % 使用\rchi将\chi居中

%%%% 一些宏定义 %%%%
\def\bd{\boldsymbol}        % 加粗(向量) boldsymbol
\def\disp{\displaystyle}    % 使用行间公式 displaystyle(默认)
\def\weekto{\rightharpoonup}% 右半箭头
\def\tsty{\textstyle}       % 使用行内公式 textstyle
\def\sign{\text{sign}}      % sign function
\def\wtd{\widetilde}        % 宽波浪线 widetilde
\def\R{\mathbb{R}}          % Real number
\def\N{\mathbb{N}}          % Natural number
\def\Z{\mathbb{Z}}          % Integer number
\def\Q{\mathbb{Q}}          % Rational number
\def\C{\mathbb{C}}          % Complex number
\def\K{\mathbb{K}}          % Number Field
\def\P{\mathbb{P}}          % Polynomial
\def\d{\mathrm{d}}          % differential operator
\def\e{\mathrm{e}}          % Euler's number
\def\i{\mathrm{i}}          % imaginary number
\def\re{\mathrm{Re}}        % Real part
\def\im{\mathrm{Im}}        % Imaginary part
\def\res{\mathrm{Res}}      % Residue
\def\ker{\mathrm{Ker}}      % Kernel
\def\vspan{\mathrm{vspan}}  % Span  \span与latex内核代码冲突改为\vspan
\def\L{\mathcal{L}}         % Loss function
\def\wdh{\widehat}          % 宽帽子 widehat
\def\ol{\overline}          % 上横线 overline
\def\ul{\underline}         % 下横线 underline
\def\add{\vspace{1ex}}      % 增加行间距
\def\del{\vspace{-1.5ex}}   % 减少行间距

%%%% 定理类环境的定义 %%%%
\newtheorem{theorem}{定理}

%%%% 基本信息 %%%%
\newcommand{\RQ}{\today} % 日期
\newcommand{\km}{泛函分析} % 科目
\newcommand{\bj}{强基数学002} % 班级
\newcommand{\xm}{吴天阳} % 姓名
\newcommand{\xh}{2204210460} % 学号

\begin{document}

%\pagestyle{empty}
\pagestyle{plain}
\vspace*{-15ex}
\centerline{\begin{tabular}{*5{c}}
    \parbox[t]{0.25\linewidth}{\begin{center}\textbf{日期}\\ \large \textcolor{blue}{\RQ}\end{center}} 
    & \parbox[t]{0.2\linewidth}{\begin{center}\textbf{科目}\\ \large \textcolor{blue}{\km}\end{center}}
    & \parbox[t]{0.2\linewidth}{\begin{center}\textbf{班级}\\ \large \textcolor{blue}{\bj}\end{center}}
    & \parbox[t]{0.1\linewidth}{\begin{center}\textbf{姓名}\\ \large \textcolor{blue}{\xm}\end{center}}
    & \parbox[t]{0.15\linewidth}{\begin{center}\textbf{学号}\\ \large \textcolor{blue}{\xh}\end{center}} \\ \hline
\end{tabular}}
\begin{center}
    \zihao{-3}\textbf{第十次作业}
\end{center}
\vspace{-0.2cm}
% 正文部分
\begin{problem}[(2.4.5)]设$X$为$B^*$空间,$X_0\subset X$为闭子空间,证明:
    \begin{equation*}
        \rho(x,X_0) = \sup\{|f(x)| : f\in X^*, ||f|| = 1,f(X_0) = 0\}\quad(\forall x\in X),
    \end{equation*}
    其中$\rho(x,X_0) = \inf_{y\in X}||x-y||$.
\end{problem}
\begin{proof}
    $\forall x\in X$,由于$X_0$为$X$的闭子空间,则由Hahn-Banach定理可得$\exists f\in X^*$,使得$f(x) = \rho(x,X_0)$且$f|_{X_0} = 0,\ ||f|| = 1$.

    设$f\in X^*,\ ||f|| = 1,\ f|_{X_0} = 0$,只需证$|f(x)|\leq \rho(x,X_0)$. 由于$\forall y\in X_0$有
    \begin{equation*}
        1=||f||\geq \left|\frac{f(x+y)}{||x+y||}\right| = \frac{|f(x)|}{||x+y||}\Rightarrow |f(x)|\leq ||x+y||\Rightarrow |f(x)|\leq \inf_{y\in X_0}||x+y|| = \rho(x,X_0)
    \end{equation*}
\end{proof}
\begin{problem}[(2.4.6)]
    设$X$是$B^*$空间,给定$X$中$n$个线性无关的元素$x_1,\cdots, x_n$与数域$\K$中$n$个数$C_1,\cdots, C_n$,及$M>0$. 求证:\add 为了$\exists f\in X^*$使得$f(x_k) =C_k(k=1,2,\cdots,n)$且$||f||\leq M$,当且仅当,$\forall \alpha_1,\cdots, \alpha_n\in \K$有$\left|\sum_{k=1}^n\alpha_kC_k\right|\leq M\left|\left|\sum_{k=1}^n\alpha_k x_k\right|\right|$.
\end{problem}
\begin{proof}
    充分性,由于$f(x_k) = C_k(k=1,2,\cdots, n)$,则
    \begin{equation*}
        \left|\sum_{k=1}^n\alpha_kC_k\right| = \left|\sum_{k=1}^n\alpha_kf(x_k)\right| = \left|f\left(\sum_{k=1}^n\alpha_kx_k\right)\right|\leq ||f||\cdot\left|\left|\sum_{k=1}^n\alpha_kx_k\right|\right|\leq M\left|\left|\sum_{k=1}^n\alpha_kx_k\right|\right|
    \end{equation*}

    必要性,令$X_0 = \vspan\{x_1,\cdots, x_n\}$,令$f_0\left(\sum_{k=1}^n\alpha_kx_k\right) = \sum_{k=1}^n\alpha_k C_k$,则$f_0$是线性泛函,$\forall y\in X_0$,有
    \begin{equation*}
        ||f_0||\leq \left|\frac{f_0(y)}{||y||}\right| = \frac{|\sum_{k=1}^n\alpha_kC_k|}{||\sum_{k=1}^n\alpha_k x_k||}\leq M
    \end{equation*}
    于是$f_0$有界,则$f_0\in X_0^*$,由Hahn-Banach延拓定理可知,$\exists f\in X^*$使得$||f||=||f_0||\leq M$,且$f|_{X_0} = f_0$,则$f(x_k) = C_k$.
\end{proof}
\begin{problem}[(2.4.7)]
    设$X$为$B^*$空间,$\{x_1,\cdots, x_n\}\subset X$是$n$个线性无关的元素,证明:$\exists f_1,\cdots, f_n\in X^*$,使得$\langle f_i,x_j\rangle = \delta_{ij}\quad(i,j=1,2,\cdots, n)$.
\end{problem}
\begin{proof}
    设$X_0^k = \vspan\{x_1,\cdots, x_{k-1},x_{k+1},\cdots, x_n\}$,由于$x_1,\cdots, x_n$线性无关,则$x_k\notin X_0^k$,由Hahn-Banach定理可得$\exists f_k'\in X^*$使得$f_k'(x_k)\neq 0,\ f_k'|_{X_0^k}=0$,令$f_k = \frac{f_k'(x_k)}{d_k}$,则$f_k\in X^*$且$f_k(x_k) = 1,\ f_k|X_0^k = 0$即$f_k(x_j) = \delta_{kj}$.
\end{proof}
\begin{problem}[(2.4.13)]
    设$M$是(应该加上$\K=\R$)$B^*$空间$X$中的闭凸集,求证:$\forall x\in X\backslash M$,$\exists f_1\in X^*$,使得$||f_1|| = 1$且$\sup_{y\in M}f_1(y)\leq f_1(x)-d(x)$,其中$d(x) = \inf_{z\in M}||x-z||$.
\end{problem}
\begin{proof}
    由于$d(x) = \inf_{z\in M}||x-z||$且$M$为闭集,则$B(x,d(x))\cap M = \varnothing$,又由于$M$为凸集,由Hahn-Banach第一几何形式可得$\exists f_0\in X^*$使得$\forall y\in M,\ |z| < 1$有
    \begin{equation*}
        f_0(y)\leq f_0(x+d(x)z) = f_0(x)+d(x)f_0(z)
    \end{equation*}
    令$f_1 = \frac{f_0}{||f_0||}$,则有$f_1(y)\leq f_1(x)+d(x)f_1(z)$且$||f_1|| = 1$,由于$||f_1|| = \sup_{|z|<1}|f_1(z)| = 1$,则$\forall \varepsilon > 0$,$\exists |z_0| < 1$使得
    \begin{equation*}
        1-\varepsilon \leq |f_1(z_0)| \leq 1\Rightarrow -1\leq f_1(z_0)\leq -(1-\varepsilon)\text{或}-1\leq f_1(-z_0)\leq -(1-\varepsilon)
    \end{equation*}
    则有$f_0(y)\leq f_0(x) - (1-\varepsilon)d(x)$,由$y$和$\varepsilon$的任意性可知$\sup_{y\in M}f_0(y)\leq f_0(x)-d(x)$.
\end{proof}
\begin{problem}[(2.4.14)]
    设$M$为实$B^*$空间$X$内的闭凸集,求证:
    \begin{equation*}
        \inf_{z\in M}||x-z|| = \sup_{\substack{f\in X^*\\ ||f||=1}}\{f(x)-\sup_{z\in M}f(z)\}\quad (\forall x\in X).
    \end{equation*}
\end{problem}
\begin{proof}
    $\forall f\in X^*,\ ||f||=1$则
    \begin{equation*}
        f(x)-\sup_{z\in M}f(z) = \inf_{z\in M}\{f(x)-f(z)\} = \inf_{z\in M}f(x-z)\leq \inf_{z\in M}||x-z|| = d(x)   
    \end{equation*}
    则$\sup_{\substack{f\in X^*\\ ||f||=1}}\left\{f(x)-\sup_{z\in M}f(z)\right\}\leq d(x)$. 由上题可知$\exists f_0\in X^*,\ ||f_0|| = 1$使得$f_0(x)-\sup_{z\in M}f_0(z)\geq d(x)$,则$\sup_{\substack{f\in X^*\\ ||f||=1}}\{f(x)-\sup_{z\in M}f(z)\} = d(x) = \inf_{z\in M}||x-z||$.
\end{proof}
\begin{problem}[(2.5.1)]
    求证:$(l^p)^* = l^q\ \left(1\leq p < \infty, \frac{1}{p}+\frac{1}{q} = 1\right)$.
\end{problem}
\begin{proof}
    一方面:令$\eta = \{y_n\} \in l^q$,则$\forall \xi = \{x_n\}\in l^p$,定义$\langle\eta,\xi\rangle = \sum_{n\geq 1}y_nx_n$,由\textbf{2.3.8, 2.3.9}可知$\langle\eta,\xi\rangle\in l^p$,则$\eta\in (l^p)^*\Rightarrow l^q\subset (l^p)^*$.

    另一方面:$\forall f\in (l^p)^*$,$\forall \xi = \{x_n\}\subset l^p$,令$e_n = (\underbrace{0,\cdots, 0,1}_{n\text{个}},0,\cdots)$,则$\xi = \sum_{n\geq 1}x_ne_n$,于是$f(\xi) = f\left(\sum_{n\geq 1}x_n e_n\right) = \sum_{n\geq 1}x_nf(e_n)$收敛,\add 由于$\{x_n\}\in l^p$,则由\textbf{2.3.8, 2.3.9}可知$\eta = \{f(e_n)\}\in l^q$,且$||\eta||_{q} = ||f||$,于是$f(\xi) = \langle\eta, \xi\rangle$.

    综上,$l^q = (l^p)^*$.
\end{proof}
\begin{problem}[(2.5.2)]
    设$C$为收敛数列全体,赋以范数$||\cdot||:\{\xi_k\}\in C\mapsto \sup_{k\geq 1}|\xi_k|$,求证:$C^* = l^1$.
\end{problem}
\begin{proof}
    令$T:l^1\to C^*,\ y=\{\eta_n\}\mapsto f_y(x) = \sum_{n\geq 1}\xi_n\eta_n=:\langle y,x\rangle,\ (\forall x = \{\xi_n\}\in C)$,则$T$是线性的,先证$f_y$有界,由于$\forall x\in \{\xi_n\}\in C$有
    \begin{equation*}
        |f_y(x)| = \sum_{n\geq 1}|\xi_n|\,|\eta_n\leq \left(\sup_{n\geq 1}|\xi_n|\right)\sum_{n\geq 1}|\eta_n| = ||x||_{\infty}\cdot ||y||_{1}
    \end{equation*}
    则$||f_y||\leq ||y||_1$,故$f_y\in C^*$,$T$有意义.

    下证$T$为满射,令$e_n = (\underbrace{0,\cdots, 0, 1}_{n\text{个}},0,\cdots)$,则$e_n\in C$,$\forall x\in \{\xi_n\}\in C$有$x=\sum_{n\geq 1}\xi_ne_n$,于是$\forall f\in C^*$有
    \begin{equation*}
        f(x) = f\left(\sum_{n\geq 1}\xi_n e_n\right) = \sum_{n\geq 1}\xi_nf(e_n)
    \end{equation*}
    下证$\{f(e_n)\}\in l^1$,令$x_n=\{\xi_k^{(n)}\} = \begin{cases}
        \e^{-\i\theta_k},&\quad k\leq n, \theta_k = \arg f(e_k),\\
        0,&\quad k > n.
    \end{cases}$则$\forall n=1,2,\cdots$有
    \begin{equation*}
        |f(x_n)| = \sum_{k=1}^n\e^{-\i\theta_k}f(e_k) = \sum_{k= 1}^n|f(e_k)| \leq ||f||\cdot ||x_n|| = ||f||
    \end{equation*}
    于是$\{f(e_k)\}\in l^1$,令$y = \{f(e_n)\}$,则$||f_y||\geq ||y||_1$,故$||f_y|| = ||y||_1$,所以$T$是等距映射.

    再证明$T$为单射,$\forall y_1,y_2\in l^1$满足$\langle y_1,x\rangle = \langle y_2,x\rangle$,由于$T$的线性性,$\langle y_1-y_2,x\rangle = \langle 0,x\rangle$,则$y_1-y_2$为$C^*$中的零元$\theta$,由于$T$是等距映射,则$||y_1-y_2||_1 = ||\theta|| = 0\Rightarrow y_1=y_2$.

    综上,$T$是等距同构映射,则$C^* = l^1$.
\end{proof}
\begin{problem}[(2.5.3)]
    设$C_0$是以$0$为极限的数列全体,赋以范数$||\cdot||:\{\xi_k\}\in C_0\mapsto \sup_{k\geq 1}|\xi_k|$,求证:$C_0^* = l^1$.
\end{problem}
\begin{proof}
    由于$e_n = (\underbrace{0,\cdots, 0, 1}_{n\text{个}},0,\cdots)$,所以$e_n\in C_0$,其他与上题证明完全相同.
\end{proof}
\begin{problem}[(2.5.4)]
    求证:有限维$B^*$空间必是自反的.
\end{problem}
\begin{proof}
    令$X$为有限维$B^*$空间,$\{x_1,\cdots, x_n\}$是$X$中一组线性无关的基,由\textbf{2.4.7}可知,\\$\exists \{f_1,\cdots, f_n\}\subset X^*$且$f_i(x_j) = \delta_{ij}\ (i,j=1,2,\cdots,n)$,则$\forall f\in X^*$,\ $\forall x = \sum_{i=1}^n f_i(x)x_i\in X$有$f(x) = f\left(\sum_{i=1}^nf_i(x)x_i\right) = \sum_{i=1}^nf(x_i)f_i(x)$,于是$f = \sum_{i=1}^nf(x_i)f_i$,则$\{f_1,\cdots, f_n\}$是$X^*$中的一组基,则$\dim X^*= \dim X$,同理可证,$\dim X^{**}=\dim X^*$,于是$\dim X^{**} = \dim X$,由于$J(X)\subset X^{**}$且$J$为等距同构映射,则$\dim J(X) = \dim X = \dim X^{**}$,故$J(X) = X^{**}$,$X$是自反空间.
\end{proof}
\begin{problem}[(2.5.6)]
    设$X$为$B^*$空间,$T$为$X$到$X^{**}$的自然嵌入映射,证明:$R(T)$是闭的充要条件为$X$完备.
\end{problem}
\begin{proof}
    充分性,令$\{x_n\}$为$X$中的Cauchy列,由$T$的定义可知
    \begin{equation*}
        ||Tx_n-Tx_m|| = ||T(x_n-x_m)|| = \sup_{\substack{||f|| = 1\\ f\in X^*}}|f(x_n-x_m)|\leq ||x_n-x_m||\to 0,\quad (n,m\to\infty)
    \end{equation*}
    则$\{Tx_n\}$为$X^{**}$中的Cauchy列,由于$X^{**}$为$B$空间,则$Tx_n$收敛,由于$R(T)$是闭的,则$\exists x\in X$使得$Tx_n\to Tx$,由于$T^{-1}\in L(X^{**},X)$,则
    \begin{equation*}
        ||x_n-x|| = ||T^{-1}T(x_n-x)||\leq c||T(x_n-x)|| = c||Tx_n-Tx||\to 0
    \end{equation*}
    于是$x_n\to x$,故$X$完备.

    必要性,$\forall \{x_n\}\in X$,$Tx_n$收敛于$y\in X^{**}$,由$T^{-1}\in L(X^{**},X)$,则
    \begin{equation*}
        ||x_n-x_m||\leq c||Tx_n-y+y-Tx_m|| \leq c||Tx_n-y||+c||Tx_m-y||\to 0,\quad (n,m\to\infty)
    \end{equation*}
    则$\{x_n\}$为$X$中的Cauchy列,$\exists x\in X$,使得$x_n\to X$,由于$T$连续,则$Tx=y\Rightarrow y\in R(T)$,故$R(T)$是闭的.
\end{proof}
\begin{problem}[(2.5.7)]
    在$l^1$中定义算子
    \begin{equation*}
        T:(x_1,\cdots, x_n,\cdots)\mapsto(0,x_1,x_2,\cdots, x_n,\cdot),
    \end{equation*}
    求证:$T\in L(l^1)$并求$T^*$.
\end{problem}
\begin{proof}
    由于$\forall x\in l^1$,$||Tx|| = ||x||$,则$||T|| = 1$.

    $\forall \alpha,\beta\in \K$,$x,y\in l^1$,$T(\alpha x+\beta y) = (0,\alpha x_1+\beta y_1,\alpha x_2+\beta y_2,\cdots) = \alpha Tx + \beta Ty$,则$T\in L(l^1)$.

    由于$(l^1)^* = l^{\infty}$,$\forall f\in (l^1)^*$,令$f = (f_1,f_2,\cdots, f_n,\cdots)$,则
    \begin{equation*}
        \langle f,Tx\rangle = (f_1,\cdots, f_n,\cdots)\cdot (0,x_1,x_2,\cdots) = \sum_{n\geq 1}f_{n+1}x_n = (f_2,f_3,\cdots)\cdot (x_1,x_2,\cdots) = \langle T^*f, x\rangle
    \end{equation*}
    则$T^*f = (f_2,f_3,\cdots, f_n,\cdots),\quad \forall f = (f_1,f_2,\cdots, f_n,\cdots)\in (l^1)^*$.
\end{proof}
\begin{problem}[(2.5.8)]
    在$l^2$中定义算子
    \begin{equation*}
        T:(x_1,x_2,\cdots, x_n,\cdots)\mapsto \left(x_1,\frac{x_2}{2},\cdots, \frac{x_n}{n},\cdots\right),
    \end{equation*}
    求证:$T\in L(l^2)$并求$T^*$.
\end{problem}
\begin{proof}
    $\forall \xi=\{x_n\}\in l^2$,$||T\xi|| = \left(\sum_{n\geq 1}\left|\frac{x_n}{n}\right|^2\right)^{\frac{1}{2}}\leq \left(\sum_{n\geq 1}|x_n|^2\right)^{\frac{1}{2}} = ||\xi||$,则$||T||\leq 1$.

    $\forall \alpha,\beta\in \K$,$\xi=\{x_n\},\eta=\{y_n\}\in l^2$,则
    \begin{align}
        T(\alpha\xi+\beta\eta) =&\ \left(\alpha x_1+\beta y_1,\frac{\alpha x_2+\beta y_2}{2},\cdots, \frac{\alpha x_n+\beta y_n}{n},\cdots\right)\\
        =&\ \alpha\left(x_1,\frac{x_2}{2},\cdots,\frac{x_n}{n},\cdots\right)+\beta\left(y_1,\frac{y_2}{2},\cdots,\frac{y_n}{n},\cdots\right)\\
        =&\ \alpha T\xi+\beta T\eta
    \end{align}
    综上$T\in L(l^2)$.

    $\forall f\in (l^2)^*$,则$f\in l^2$,令$f = (f_1,\cdots, f_n,\cdots)$,则$\forall \xi = \{x_n\}\in l^2$.
    \begin{equation*}
        \langle f,T\xi\rangle = \sum_{n\geq 1}f_n\frac{x_n}{n} = \sum_{n\geq 1}\frac{f_n}{n}x_n = \left(f_1,\frac{f_2}{2},\cdots,\frac{f_n}{n},\cdots\right)\cdot (x_1,\cdots, x_n,\cdots) = \langle T^*f, \xi\rangle
    \end{equation*}
    则$T^* f = \left(f_1,\frac{f_2}{2},\cdots, \frac{f_n}{n},\cdots\right)$.
\end{proof}
\begin{problem}[(2.5.12)]
    设$X,Y$为$B$空间,$T$为$X$到$Y$的线性算子,$\forall g\in Y^*$,$g(Tx)\in X^*$,证明$T$连续.
\end{problem}
\begin{proof}
    由于$X,Y$为$B$空间,由闭图像定理,只需证$T$为闭算子. 设$\{x_n\}\in X$收敛于$x$且$Tx_n\to y\in Y$,下证$Tx=y$.

    令$T^* : Y^*\to X^*$,$\forall g\in Y^*$定义$T^*g(x) = g(Tx)$,则
    \begin{equation*}
        \left.\begin{aligned}
            &\ T^* g(x_n) = g(Tx_n)\xrightarrow{g\text{连续}}g(y)\\
            =&\ (T^* g)(x_n)  \xrightarrow{T^*g\text{连续}}(T^*g)(x) = g(Tx)
        \end{aligned}\right\} g(y) = g(Tx)\Rightarrow Tx = y
    \end{equation*}
\end{proof}
\begin{problem}[(2.5.15)]
    设$H$为Hilbert空间,$\{e_n\}$为$H$的标准正交基,证明$x_n\weekto x_0$等价于$||x_n||$有界且$(x_n,e_k)\to (x_0,e_k),\ (n\to\infty, k=1,2,\cdots)$.
\end{problem}
\begin{proof}
    首先证明充分性,由于$x_n\weekto x_0$,则$\forall f\in H^*$,有$f(x_n)\to f(x_0),\ (n\to\infty)$,令由于$e_k$对应有界泛函$f_{e_k}(x) = (x,e_k)$,则$(x_n,e_k)\to (x_0,e_k),\ (n\to\infty, k=1,2,\cdots)$. 由于$||x|| = ||J_x||$,而$x_n$弱收敛等价于$J_{x_n}$*弱收敛,由于$H^{**}$为$B$空间,则由孔明定理可知,$||J_{x_n}||$有界,故$||x_n||$有界.
    
    由于$x_n\weekto x_0$等价于$||x_n||$有界且$G$为$X$的稠密子集$\forall f\in G$有$f(x_n)\to f(x_0)$. 充分性得证. 于是只需证明,$(x_n,e_k)\to (x_0,e_k),\ (n\to\infty, k=1,2,\cdots)$可推出在$\{e_n\}$中有$f(x_n)\to f(x_0),\ (f\in H^*)$,不妨令$x_0 = 0$,若$x_0\neq 0$,取$x_n\leftarrow x_n-x_0$即可. 由Riesz表示定理可知,$\forall f\in H^*$,$\exists y\in H$使得$f(x) = (x,y)$,由于$\{e_n\}$为$H$中的一组基,则令$y = \sum_{k\geq 1}\alpha_ke_k$,于是
    \begin{align*}
        f(x_n) =&\ \left(x_n,\sum_{k\geq 1}\alpha_k e_k\right) = \sum_{k\geq 1}\bar{\alpha}_k(x_n,e_k)\\
        =&\ \lim_{N\to\infty}\sum_{k=1}^N\bar{\alpha}_k(x_n,e_k) \to\lim_{N\to\infty}\sum_{k=1}^N\bar{\alpha}_k(0,e_k)= \lim_{N\to\infty} 0 = f(x_0)
    \end{align*}
\end{proof}
\begin{problem}[(2.5.16)]
    设$S_n:L^p(\R)\to L^p(\R)$满足$
    (S_nu)(x) =\begin{cases}
        u(x),&\quad |x|\leq n,\\
        0,&\quad |x|>n.
    \end{cases}$
    其中$u\in L^p(\R)$是任意的,证明:$\{S_n\}$强收敛于恒通算子$I$,但不一致收敛于$I$.
\end{problem}
\begin{proof}
    由于$S_nu = u\cdot \rchi_{[-n,n]}$,于是
    \begin{equation*}
        ||S_nu-Iu||_p^p = ||u\rchi_{(-\infty,u)\cup(u,\infty)}||_p^p = \int_{-\infty}^{-n}|u|^p\,\d x+\int_{n}^\infty |u|^p\,\d x\to 0,\quad (n\to\infty)
    \end{equation*}
    所以$S_n\overset{s}{\to} I$. 由于$\int_n^\infty \e^{-x}\,\d x=e^{-n}$,令$u_n(x) = \rchi_{[n,\infty)}\left(\e^{n-x}\right)^{\frac{1}{p}}$,则$||u_n||_p^p = \e^{n}\int_n^\infty \e^{-x}\,\d x=1$,于是
    \begin{equation*}
        ||S_n-I||\geq ||S_nu_n-u_n||_p = ||u_n||_p = 1
    \end{equation*}
    故$S_n\nrightarrow I$.
\end{proof}
\begin{problem}[(2.5.17)]
     设$H$为Hilbert空间,在$H$中$x_n\weekto x_0$且$y_n\to y_0$,证明$(x_n,y_n)\to (x_0,y_0)$.
\end{problem}
\begin{proof}
    由于$x_n\weekto x_0$,则$||x_n||$一致有界,$\exists M>0$使得$||x_n||\leq M$,有Riesz表示定理知$f_y(x) = (x,y)\in H^*$,则
    \begin{align*}
        |(x_n,y_n)-(x_0,y_0)| =&\ |(x_n,y_n-y_0)+(x_n,y_0)-(x_0,y_0)|\\
        \leq&\ M\cdot ||y_n-y_0||+||f_{y_0}(x_n)-f_{y_0}(x_0)||\to 0
    \end{align*}
    所以$(x_n,y_n)\to (x_0,y_0)$.
\end{proof}
\begin{problem}[(2.5.18)]
    设$\{e_n\}$为Hilbert空间空间$H$中的标准正交基,证明:在$H$中$e_n\weekto \theta$,但$e_n\nrightarrow \theta$.
\end{problem}
\begin{proof}
    由Bessel不等式可知:$\forall x\in H$,有$\sum_{n\geq 1}|(x,e_n)|^2\leq ||x||^2$,于是$|(x,e_n)|\to 0$,由Riesz表示定理可知,$\forall f\in H^*$,$\exists y\in H$,使得$f(x) = (x, y)$,于是$|f(e_n)| = |(e_n, y)|\to 0,\ (n\to\infty)$,所以$e_n\weekto \theta$.

    由于$\{e_n\}$为标准正交基,则$||e_n|| = 1$,故$e_n\nrightarrow \theta$.
\end{proof}
\begin{problem}[(2.5.19)]
    设$H$为Hilbert空间,证明:在$H$中$x_n\to x$充要条件为$||x_n||\to ||x||$且$x_n\weekto x$.
\end{problem}
\begin{proof}
    只需证$H$为一致凸的$B$空间,由于$H$为Hilbert空间,所以只需证$H$的一致凸性. $\forall \varepsilon>0$,$\exists \delta >0$,使得$\forall x,y\in X,\ ||x||\leq 1,\ ||y||\leq 1,\ ||x-y|| > \varepsilon$,由于$H$中有平行四边形公式成立,则
    \begin{align*}
        &\ ||x+y||^2 = 2(||x||^2+||y||^2)-||x-y||^2< 4-\varepsilon^2\\
        \Rightarrow&\ \left|\left|\frac{x+y}{2}\right|\right|^2\leq 1-\frac{\varepsilon^2}{4} < 1-\delta
    \end{align*}
    故$H$满足一致凸性.
\end{proof}
\begin{problem}[(2.5.21)]
    证明:$B^*$空间$X$中的闭凸集是弱闭的.
\end{problem}
\begin{proof}
    设$M$为$X$中的闭凸子集,$\{x_n\}\subset M$,$x_n\weekto x$,则由Mazur定理可知,$x\in \ol{\text{co}\{x_n\}}$,于是只需证$\ol{\text{co}\{x_n\}}\subset M$,由于$M$是闭的,只需证$\text{co}\{x_n\}\subset M$,由于$M$为凸的,则$\alpha\in (0,1),\ \forall y_1,y_2\in\{x_n\}$有$\alpha y_1+(1-\alpha)y_2 \in M$,断言$\forall y_1,\cdots, y_n\in \{x_n\},\ \sum_{k=1}^n\alpha_k = 1$有$\sum_{k=1}^n\alpha_k y_k\in M$. 利用归纳假设当命题在$n$时成立,下推导在$n+1$时情况,任意的$\alpha_1,\cdots,\alpha_{n+1}\in \K$满足$\sum_{k=1}^{n+1}\alpha_k = 1$,由归纳假设可知$\frac{\sum_{k=1}^n\alpha_ky_k}{\sum_{k=1}^n\alpha_k}\in M$,于是由$M$的凸性可知
    \begin{equation*}
        \sum_{k=1}^{n+1}\alpha_ky_k = \sum_{k=1}^n\alpha_k\frac{\sum_{k=1}^n\alpha_ky_k}{\sum_{k=1}^n\alpha_k}+\alpha_{n+1}y_{n+1}\in M
    \end{equation*}
    故$\ol{\text{co}\{x_n\}}\subset M$.
\end{proof}
\begin{problem}[(2.5.22)]
    设$X$是自反空间,$M$为$X$中的有界闭凸子集,$\forall f\in X^*$,证明:$f$在$M$上达到最大值和最小值.
\end{problem}
\begin{proof}
    $\forall f\in X^*$,由于$M$有界,则$\forall x\in M$,$||x||\leq A$,则$\sup_{x\in M}||f(x)||\leq ||f||\cdot||x||\leq M||f||$,于是$f$在$M$上上极限和下极限存在,令$d = \sup_{x\in M}f(x)$,则$\exists \{x_n\}\in M$使得$d-1/n\leq f(x_n)\leq d$,由于$X$为自反空间且$\{x_n\}\subset M$有界,则$\{x_n\}$有弱收敛子列$\{x_{n_k}\}$使得$x_{n_k}\weekto x\in X$,由上题可知,由于$M$为闭凸集,则$x_n\weekto x\in M$,于是$d-1/n_k\leq f(x_{n_k})\leq d$,令$k\to\infty$可得$f(x) = d$,于是$x$是$f$在$M$中的最大值,同理可证,$f$在$M$中可取到最小值.
\end{proof}
\begin{problem}[(2.5.23)]
    设$X$为自反空间,$M$为$X$中的非空闭凸集,证明$\exists x_0\in M$,使得$||x_0|| = \inf\{||x||:x\in M\}$.
\end{problem}
\begin{proof}
    由于$M$非空,则$\exists x\in M$,取$M' = \ol{B(\theta, x)}\cap M$,于是$\inf\{||x||:x\in M'\} = \inf\{||x||:x\in M\}$,下证在有界闭凸集$M'$中能取到$||\cdot||$最小值. 令$d = \inf\{||x||:x\in M'\}$,则$\exists\{x_n\}\subset M'$,使得$||x_n||\to d$,由于$X$自反,则$\{x_n\}$存在弱收敛子列$\{x_{n_k}\}$使得$x_{n_k}\weekto x\in X$,由\textbf{2.5.21}可得$x_{n_k}\weekto x\in M'$. 又由Hahn-Banach定理可得$\exists f\in X^*$使得$f(x)=||x||,\ ||f|| = 1$,于是
    \begin{equation*}
        d\leq ||x|| = f(x) = \lim_{k\to\infty}f(x_{n_k})\leq \lim_{k\to\infty}||x_{n_k}|| = d
    \end{equation*}
    于是$||x|| = \inf\{||x||:x\in M'\} = \inf\{||x||:x\in M\}$.
\end{proof}
\begin{problem}
    证明:设$X$是一致凸空间,则$X$是严格凸空间.
\end{problem}
\begin{proof}
    $\forall x,y \in X$,$x\neq y$,$||x|| = ||y|| = 1$,$\forall \alpha\in (0,1)$,下证$\alpha x+(1-\alpha)\beta|| < 1$. 由于$||\cdot||$连续,只需证存在$(0,1)$的稠密子集$M$使得$\alpha\in M$,有$||\alpha x+(1-\alpha)\beta|| < 1$成立.\add

    由于$X$是一致凸空间,又由于$x\neq y$,则取$\varepsilon = \frac{||x-y||}{2}$,$\exists \delta > 0$,使得$\left|\left|\frac{x+y}{2}\right|\right|\leq 1-\delta < 1$,于是$\alpha = 1/2$严格凸性质成立. 由于$\left|\left|\frac{x+y}{2}\right|\right|\leq 1$,于是再次使用一致凸性质,可得
    \begin{equation*}
        \left|\left|\frac{1}{4}x+\frac{3}{4}y\right|\right| < 1,\ \left|\left|\frac{3}{4}x+\frac{1}{4}y\right|\right| < 1,
    \end{equation*}
    于是$\alpha = 1/4, 3/4$时严格凸性质成立. 同理可得,$\alpha = 1/8,3/8,5/8,7/8,\cdots$也即
    \begin{equation*}
        M = \left\{\frac{1}{2^n},\frac{3}{2^n},\cdots,\frac{2^n-1}{2^n}:n=1,2,\cdots\right\}
    \end{equation*}
    上有$\alpha\in M$均满足严格凸性,由于$M$是$(0,1)$的稠密子集,则$\forall \alpha\in (0,1)$,$\exists\{\alpha_n\}\in M$,使得$\alpha_n\to \alpha$且$||\alpha_kx+(1-\alpha_k)y||< 1$,则$||\alpha x+(1-\alpha)y|| < 1$,故$X$是严格凸空间.
\end{proof}
\begin{problem}
    设$X$为$B^*$空间,则$X$是一致凸空间$\iff\ ||x_n||\to 1,||y_n||\to 1,||x_n+y_n||\to 2$,则$||x_n-y_n||\to 0$.
\end{problem}
\begin{proof}
    必要性:反设,当$||x_n||\leq 1,||y_n||\leq 1$,$\exists \varepsilon > 0$,$\forall n\geq 1$当$\left|\frac{x+y}{2}\right|\geq 1-\frac{1}{n}$时都有$||x-y||\geq \varepsilon$.

    由于$||x_n||\to 1$不一定满足$||x_n||\leq 1$,考虑$\frac{x_n}{||x_n||}$,于是$\left|\left|\frac{x_n}{||x_n||}\right|\right| = \left|\left|\frac{y_n}{||y_n||}\right|\right| = 1$,则
    \begin{align*}
        &\ \Biggl|\left|\left|\frac{x_n}{||x_n||}+\frac{y_n}{||y_n||}\right|\right|-||x_n+y_n||\Biggl|\leq \Biggl|\left|\left|\frac{x_n}{||x_n||}-x_n+\frac{y_n}{||y_n||}-y_n\right|\right|\Biggl|\\
        \leq&\ \Biggl|\left|\left|\frac{x_n}{||x_n||}-x_n\right|\right|+\left|\left|\frac{y_n}{||y_n||}-y_n\right|\right|\Biggl|\leq \big|1-||x_n||\big|+\big|1-||y_n||\big|\to 0,\quad(n\to\infty)
    \end{align*}
    \add 于是$\left|\left|\frac{x_{n_k}}{||x_{n_k}||}+\frac{y_{n_k}}{||y_{n_k}||}\right|\right|\to 2$则存在子列满足$\left|\left|\frac{x_{n_k}}{||x_{n_k}||}+\frac{y_{n_k}}{||y_{n_k}||}\right|\right|\geq 2-\frac{2}{n_k}$,由条件可得\\$\left|\left|\frac{x_n}{||x_n||}-\frac{y_n}{||y_n||}\right|\right|\to 0$但与\add $\left|\left|\frac{x_{n_k}}{||x_{n_k}||}-\frac{y_{n_k}}{||y_{n_k}||}\right|\right|\geq \varepsilon$矛盾,于是原命题成立.

    充分性:设$||x_n||\to 1,||y_n||\to 1,||x_n+y_n||\to 2$,由必要性证明可知$\frac{x_n}{||x_n||}\to x_n$,$\frac{y_n}{||y_n||}\to y_n$,不妨令$||x_n|| = 1, ||y_n|| = 1$,于是$\forall \varepsilon > 0,\exists\delta >0$任意$||x||\leq 1,||y||\leq 1$满足$\left|\left|\frac{x+y}{2}\right|\right|<1-\delta$有$|x-y| < \varepsilon$. 由于$||x_n+y_n||\to 2$,则$\exists N >0$使得$\forall n\geq N$有$||x_n+y_n|| > 2-2\delta$,于是$||x_n-y_n|| < \varepsilon$,由$\varepsilon$任意性可知$||x_n-y_n||\to 0$.
\end{proof}
\begin{problem}
    (1). $p\geq 2$时,$\left|\left|\frac{f+g}{2}\right|\right|_p^p+\left|\left|\frac{f-g}{2}\right|\right|_p^p\leq \frac{1}{2}\left(||f||_p^p+||g||_p^p\right),\quad \forall f,g\in L^p$.

    (2). $1<p\leq 2$时,$\left|\left|\frac{f+g}{2}\right|\right|_p^q+\left|\left|\frac{f-g}{2}\right|\right|_p^q\leq \left(\frac{1}{2}||f||_p^p+\frac{1}{2}||f||_p^p\right)^{\frac{1}{p-1}}$,其中$\frac{1}{p}+\frac{1}{q}=1$.
\end{problem}
\begin{proof}
    (1). 先证明$a^p+b^p\leq (a^2+b^2)^{\frac{p}{2}}\ (\forall a,b > 0)$,不妨令$b\geq a$,令$t = b/a$,只需证$x(t) = (1+t^2)^{\frac{p}{2}}-t^p-1\geq 0\ (\forall t\geq 1)$. 由于$x'(t) = pt\left((\sqrt{1+t^2})^{p-2}-t^{p-2}\right)\geq 0$,所以$x(t)\geq x(1) = 2^{\frac{p}{2}}-2\geq 0$.

    令$a = \left|\frac{f+g}{2}\right|,b=\left|\frac{f-g}{2}\right|$,由于$x^{\frac{p}{2}}\ (p\geq 2)$为凸函数得
    \begin{equation*}
        \left|\frac{f+g}{2}\right|^p+\left|\frac{f-g}{2}\right|^p\leq \left(\frac{|f|^2+|g|^2}{2}\right)^{\frac{p}{2}}\leq \frac{|f|^p+|g|^p}{2}
    \end{equation*}
    再对其积分可得$\left|\left|\frac{f+g}{2}\right|\right|_p^p+\left|\left|\frac{f-g}{2}\right|\right|_p^p\leq \frac{1}{2}\left(||f||_p^p+||g||_p^p\right)$.
\end{proof}
\end{document}