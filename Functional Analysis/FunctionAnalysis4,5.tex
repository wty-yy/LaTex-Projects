\documentclass[12pt, a4paper, oneside]{ctexart}
\usepackage{amsmath, amsthm, amssymb, bm, color, graphicx, geometry, mathrsfs,extarrows, braket, booktabs, array}
\usepackage[colorlinks,linkcolor=red,anchorcolor=blue,citecolor=blue,urlcolor=blue,menucolor=black]{hyperref}
\setCJKmainfont{方正新书宋_GBK.ttf}[BoldFont=方正宋黑简体, ItalicFont=方正楷体_GBK]
\setmainfont{Times New Roman}  % 设置英文字体
\setsansfont{Calibri}
\setmonofont{Consolas}

\linespread{1.4}
%\geometry{left=2.54cm,right=2.54cm,top=3.18cm,bottom=3.18cm}
\geometry{left=1.84cm,right=1.84cm,top=2.18cm,bottom=2.18cm}
\newcounter{problem}  % 问题序号计数器
\newenvironment{problem}{\stepcounter{problem}\par\noindent\textbf{题目\arabic{problem}. }}{\smallskip\par}
\newenvironment{solution}{\par\noindent\textbf{解答. }}{\smallskip\par}
\newenvironment{note}{\par\noindent\textbf{注记. }}{\smallskip\par}

%%%% 图片相对路径 %%%%
\graphicspath{{figure/}} % 当前目录下的figure文件夹, {../figure/}则是父目录的figure文件夹

\everymath{\displaystyle} % 默认全部行间公式
\DeclareMathOperator*\uplim{\overline{lim}} % 定义上极限 \uplim_{}
\DeclareMathOperator*\lowlim{\underline{lim}} % 定义下极限 \lowlim_{}
\let\leq=\leqslant % 将全部leq变为leqslant
\let\geq=\geqslant % geq同理

%%%% 一些宏定义 %%%%
\def\bd{\boldsymbol}        % 加粗(向量) boldsymbol
\def\disp{\displaystyle}    % 使用行间公式 displaystyle(默认)
\def\tsty{\textstyle}       % 使用行内公式 textstyle
\def\sign{\text{sign}}      % sign function
\def\wtd{\widetilde}        % 宽波浪线 widetilde
\def\R{\mathbb{R}}          % Real number
\def\N{\mathbb{N}}          % Natural number
\def\Z{\mathbb{Z}}          % Integer number
\def\Q{\mathbb{Q}}          % Rational number
\def\C{\mathbb{C}}          % Complex number
\def\d{\mathrm{d}}          % differential operator
\def\e{\mathrm{e}}          % Euler's number
\def\i{\mathrm{i}}          % imaginary number
\def\re{\mathrm{Re}}        % Real part
\def\im{\mathrm{Im}}        % Imaginary part
\def\res{\mathrm{Res}}      % Residue
\def\L{\mathcal{L}}         % Loss function
\def\wdh{\widehat}          % 宽帽子 widehat
\def\ol{\overline}          % 上横线 overline
\def\ul{\underline}         % 下横线 underline
\def\add{\vspace{1ex}}      % 增加行间距
\def\del{\vspace{-3.5ex}}   % 减少行间距

%%%% 定理类环境的定义 %%%%
\newtheorem{theorem}{定理}

%%%% 基本信息 %%%%
\newcommand{\RQ}{\today} % 日期
\newcommand{\km}{泛函分析} % 科目
\newcommand{\bj}{强基数学002} % 班级
\newcommand{\xm}{吴天阳} % 姓名
\newcommand{\xh}{2204210460} % 学号

\begin{document}

%\pagestyle{empty}
\pagestyle{plain}
\vspace*{-15ex}
\centerline{\begin{tabular}{*5{c}}
    \parbox[t]{0.25\linewidth}{\begin{center}\textbf{日期}\\ \large \textcolor{blue}{\RQ}\end{center}} 
    & \parbox[t]{0.2\linewidth}{\begin{center}\textbf{科目}\\ \large \textcolor{blue}{\km}\end{center}}
    & \parbox[t]{0.2\linewidth}{\begin{center}\textbf{班级}\\ \large \textcolor{blue}{\bj}\end{center}}
    & \parbox[t]{0.1\linewidth}{\begin{center}\textbf{姓名}\\ \large \textcolor{blue}{\xm}\end{center}}
    & \parbox[t]{0.15\linewidth}{\begin{center}\textbf{学号}\\ \large \textcolor{blue}{\xh}\end{center}} \\ \hline
\end{tabular}}
\begin{center}
    \zihao{-3}\textbf{第三次作业}
\end{center}
\vspace{-0.2cm}
% 正文部分
\begin{problem}
    在度量空间$l^2$中,证明:$A = \{\xi = \{x_n\}\in l^2:n|x_n|\leq 1\}$是$l^2$中的紧集.
\end{problem}
\begin{proof}
    只需证明$A$是自列紧集,设$\{\xi_n\}\subset A$是Cauchy列,则$\rho(\xi_n,\xi_m) \to 0,\ (n,m\to\infty)$,于是$\sum_{i=1}^\infty|x_i^{(n)}-x_i^{(m)}|^2 \to 0$,所以$\forall i \geq 1$,$\{x_i^{(n)}\}$为$\R$中的Cauchy列,于是$\exists x_i$使得$\lim_{n\to\infty}x_i^{(n)} = x_i$.
    
    令$\xi = \{x_1,x_2,\cdots\}$,则$\forall \varepsilon > 0$,$\exists n_0\in \N$,有$\sum_{i=1}^\infty |x_i^{(n_0)}-x_i|^2 < \varepsilon^2$,则$|x_i^{(n_0)}-x_i| < \varepsilon$. 又由于$\xi_{n_0}\in A$,则$\exists N\geq [\frac{1}{\varepsilon}]+1$使得$\forall k\geq N$有$|x_k^{(n_0)}|\leq \frac{1}{N} < \varepsilon$,于是
    \begin{equation*}
        |x_i|\leq |x_i-x_i^{(n_0)}|+|x_i^{(n_0)}| < 2\varepsilon
    \end{equation*}
    则$\lim_{n\to \infty}\xi_n = \xi\in A$,所以$A$是自列紧集,故$A$是紧集.
\end{proof}
\begin{problem}
    用闭区间套定理证明压缩映射原理.
\end{problem}
\begin{proof}
    设度量空间为$(X,\rho)$,$T$为$X$上的压缩映射. 下面证明集列$A_n = \{x\in X:\rho(x, Tx) < \frac{1}{n}\}$是单调递减直径趋于$0$的非空闭集列.

    单调递减:$\forall x\in A_{n+1}$,则$\rho(x, Tx) < \frac{1}{n+1} < \frac{1}{n}$,故$x \in A_n$.

    非空:设$x_0\in A_n$且$\rho(x_0, Tx_0) = C$,记$x_1 = Tx_0, \cdots, x_{n+1} = Tx_{n}$,则
    \begin{equation*}
        \rho(x_n, Tx_n) \leq \alpha\rho(Tx_{n-1},T^2x_{n-1}) \leq \cdots\leq \alpha^n\rho(x_0, Tx_0) = \alpha^nC\to 0,\ (n\to\infty)
    \end{equation*}
    则$A_n\neq \varnothing$.

    闭集:$\forall m\in \N$,只需证$A_m$的对极限封闭,设$\{x_n\}\subset A_m$收敛于$x\in X$,$\forall \varepsilon > 0$,$\exists N > 0$使得$\forall n\geq N$有
    \begin{equation*}
        \rho(x, Tx)\leq \rho(x, x_n) + \rho(x_n, Tx_n) + \rho(Tx_n, Tx) \leq (\alpha+1)\varepsilon + \frac{1}{m}
    \end{equation*}
    由$\varepsilon$的任意性可知$x\in A_m$,所以$A_m$是闭集.

    直径趋于$0$:$\forall x, y\in A_n$,则
    \begin{equation*}
        \rho(x, y) \leq \rho(x, Tx)+\rho(Tx, Ty) + \rho(Ty, y)\leq \frac{2}{(1-\alpha)n}\to 0,\quad(n\to\infty)
    \end{equation*}
    所以$\lim_{n\to\infty}\text{dim }A_n = 0$.

    综上,$\{A_n\}$是直径趋于零的非空闭子集套,所以存在唯一的$x_0\in \bigcap_{i=1}^\infty A_i$,则$\rho(x_0, Tx_0) = 0$,压缩映射原理得证.
\end{proof}
% 下面给一些功能的写法
\iffalse
% 图片模板
\centerline{
    \includegraphics[width=0.8\textwidth]{figure.png}
}
% 表格模板
\renewcommand\arraystretch{0.8} % 设置表格高度为原来的0.8倍
\begin{table}[!htbp] % table标准
    \centering % 表格居中
    \begin{tabular}{p{1cm}<{\centering}p{1cm}<{\centering}p{3cm}<{\centering}p{5cm}<{\centering}} % 设置表格宽度
    %\begin{tabular}{cccc}
        \toprule
        $x_i$ & $f[x_1]$ & $f[x_i,x_{i+1}]$ & $f[x_i,x_{i+1},x_{i+2}]$ \\
        \midrule
        $x_0$ & $f(x_0)$ &                  &                          \\
        $x_0$ & $f(x_0)$ & $f'(x_0)$        &                          \\
        $x_0$ & $f(x_1)$ & $\frac{f(x_1)-f(x_0)}{x_1-x_0}$ & $\frac{f(x_1)-f(x_0)}{(x_1-x_0)^2}-\frac{f'(x_0)}{x_1-x_0}$\\
        \bottomrule
    \end{tabular}
\end{table}

\def\Log{\text{Log}} % 一个简单的宏定义
$\Log$ % 调用方法
\fi

\end{document}