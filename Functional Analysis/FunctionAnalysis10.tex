\documentclass[12pt, a4paper, oneside]{ctexart}
\usepackage{amsmath, amsthm, amssymb, bm, color, graphicx, geometry, mathrsfs,extarrows, braket, booktabs, array, wrapfig, enumitem}
\usepackage[colorlinks,linkcolor=red,anchorcolor=blue,citecolor=blue,urlcolor=blue,menucolor=black]{hyperref}
\setCJKmainfont{方正新书宋_GBK.ttf}[ BoldFont = 方正小标宋_GBK, ItalicFont = 方正楷体_GBK]
\setmainfont{Times New Roman}  % 设置英文字体
\setsansfont{Calibri}
\setmonofont{Consolas}

\linespread{1.4}
%\geometry{left=2.54cm,right=2.54cm,top=3.18cm,bottom=3.18cm}
\geometry{left=1.84cm,right=1.84cm,top=2.18cm,bottom=2.18cm}
\newcounter{problem}  % 问题序号计数器
\newenvironment{problem}[1][]{\stepcounter{problem}\par\noindent\textbf{题目\arabic{problem}. #1}}{\smallskip\par}
\newenvironment{solution}[1][]{\par\noindent\textbf{#1解答. }}{\smallskip\par}  % 可带一个参数表示题号\begin{solution}{题号}
\newenvironment{note}{\par\noindent\textbf{注记. }}{\smallskip\par}

%%%% 图片相对路径 %%%%
\graphicspath{{figure/}} % 当前目录下的figure文件夹, {../figure/}则是父目录的figure文件夹

%%%% 缩小item,enumerate,description两行间间距 %%%%
\setenumerate[1]{itemsep=0pt,partopsep=0pt,parsep=\parskip,topsep=5pt}
\setitemize[1]{itemsep=0pt,partopsep=0pt,parsep=\parskip,topsep=5pt}
\setdescription{itemsep=0pt,partopsep=0pt,parsep=\parskip,topsep=5pt}

\everymath{\displaystyle} % 默认全部行间公式
\DeclareMathOperator*\uplim{\overline{lim}} % 定义上极限 \uplim_{}
\DeclareMathOperator*\lowlim{\underline{lim}} % 定义下极限 \lowlim_{}
\let\leq=\leqslant % 将全部leq变为leqslant
\let\geq=\geqslant % geq同理

%%%% 一些宏定义 %%%%
\def\bd{\boldsymbol}        % 加粗(向量) boldsymbol
\def\disp{\displaystyle}    % 使用行间公式 displaystyle(默认)
\def\tsty{\textstyle}       % 使用行内公式 textstyle
\def\sign{\text{sign}}      % sign function
\def\wtd{\widetilde}        % 宽波浪线 widetilde
\def\R{\mathbb{R}}          % Real number
\def\N{\mathbb{N}}          % Natural number
\def\Z{\mathbb{Z}}          % Integer number
\def\Q{\mathbb{Q}}          % Rational number
\def\C{\mathbb{C}}          % Complex number
\def\K{\mathbb{K}}          % Number Field
\def\P{\mathbb{P}}          % Polynomial
\def\d{\mathrm{d}}          % differential operator
\def\e{\mathrm{e}}          % Euler's number
\def\i{\mathrm{i}}          % imaginary number
\def\re{\mathrm{Re}}        % Real part
\def\im{\mathrm{Im}}        % Imaginary part
\def\res{\mathrm{Res}}      % Residue
\def\ker{\mathrm{Ker}}      % Kernel
\def\L{\mathcal{L}}         % Loss function
\def\wdh{\widehat}          % 宽帽子 widehat
\def\ol{\overline}          % 上横线 overline
\def\ul{\underline}         % 下横线 underline
\def\add{\vspace{1ex}}      % 增加行间距
\def\del{\vspace{-1.5ex}}   % 减少行间距

%%%% 定理类环境的定义 %%%%
\newtheorem{theorem}{定理}

%%%% 基本信息 %%%%
\newcommand{\RQ}{\today} % 日期
\newcommand{\km}{泛函分析} % 科目
\newcommand{\bj}{强基数学002} % 班级
\newcommand{\xm}{吴天阳} % 姓名
\newcommand{\xh}{2204210460} % 学号

\begin{document}

%\pagestyle{empty}
\pagestyle{plain}
\vspace*{-15ex}
\centerline{\begin{tabular}{*5{c}}
    \parbox[t]{0.25\linewidth}{\begin{center}\textbf{日期}\\ \large \textcolor{blue}{\RQ}\end{center}} 
    & \parbox[t]{0.2\linewidth}{\begin{center}\textbf{科目}\\ \large \textcolor{blue}{\km}\end{center}}
    & \parbox[t]{0.2\linewidth}{\begin{center}\textbf{班级}\\ \large \textcolor{blue}{\bj}\end{center}}
    & \parbox[t]{0.1\linewidth}{\begin{center}\textbf{姓名}\\ \large \textcolor{blue}{\xm}\end{center}}
    & \parbox[t]{0.15\linewidth}{\begin{center}\textbf{学号}\\ \large \textcolor{blue}{\xh}\end{center}} \\ \hline
\end{tabular}}
\begin{center}
    \zihao{-3}\textbf{第九次作业}
\end{center}
\vspace{-0.2cm}
% 正文部分
\begin{problem}
    设$X$是复Hilbert空间,$T\in L(X)$,证明:$T^* = T\iff (Tx,x)\in \R,\ (\forall \in X)$.
\end{problem}
\begin{proof}
    “$\Rightarrow$”:$(Tx,x) = (x,T^* x) = (x,Tx)=\ol{(Tx,x)}$,则$(Tx,x)\in\R$.

    “$\Leftarrow$”:由于$(Tx,x)\in\R$于是$(Tx,x) = (x,Tx) = (x,T^*x)$,于是$(x,(T-T^*)x) = 0,\ (\forall x\in X)$,于是$T-T^* = 0\Rightarrow T = T^*$.
\end{proof}
\begin{problem}
    设$X$为Hilbert空间,$T_1,T_2\in L(X)$,$T_1^*=T_1,T_2^* = T_2$,证明:$T_1T_2=T_2T_1\iff T_1T_2=(T_1T_2)^*$.
\end{problem}
\begin{proof}
    由于$T_1T_2 = (T_1T_2)^{**} = (T_2^*T_1^*)^* = (T_2T_1)^*$

    “$\Rightarrow$”:由于$T_1T_2 = T_2T_1$,于是$T_1T_2 = (T_1T_2)^*$.

    “$\Leftarrow$”:由于$T_1T_2 = (T_1T_2)^* = (T_2T_1)^*$,两边同取共轭可得$T_1T_2 = T_2T_1$.
\end{proof}
\begin{problem}
    设$X$为Hilbert空间,$T\in L(X)$,证明$\ker(T^*) = R(T)^{\perp}$.
\end{problem}
\begin{proof}
    一方面,$\forall x\in \ker(T^*)$,则$0=(T^*x, y) = (x, Ty),\ (\forall y\in X)$,则$x\in R(T)^{\perp}\Rightarrow \ker(T) \subset R(T)^\perp$.
    另一方面,$\forall x\in R(T)^\perp$,则$0=(x,Ty) = (T^*x, y),\ (\forall y\in X)$,则$T^*x = 0\Rightarrow x\in \ker(T^*)$.

    综上:$\ker (T^*) = R(T)^\perp$.
\end{proof}
\begin{problem}
    证明$(^\perp M)^\perp = \bar{M}$.
\end{problem}
\begin{proof}
    $\forall x\in M$,$\forall f\in {^\perp M}$有$\langle f,x\rangle = 0$,则$x\in (^\perp M)^\perp$,令$\{x_n\}\subset M$且$x_n\to x\in X$,由于$f$的连续性,则$\langle f,x\rangle =\lim_{n\to\infty}f(x_n) = 0\Rightarrow x\in (^\perp M)^\perp$,于是$\bar{M}\subset (^\perp M)^\perp$.

    假设$\bar{M}$是$(^\perp M)^\perp$的真子集,则$\exists x_0\in (^\perp M)^\perp - \bar{M}$,且$d:=\rho(x_0,\bar{M}) > 0$,由Hahn-Banach定理推论可得$\exists f\in X^*$使得$f(x_0) = d,\ f|_{\bar{M}} = 0$,于是$f\in {^\perp M}$且$f(x_0) = d > 0$,则$x_0\notin(^\perp M)^\perp$与$x_0\in(^\perp M)^\perp$矛盾. 故$\bar{M} = (^\perp M)^\perp$.
\end{proof}
\begin{problem}
    设$X,Y$为$B^*$空间,$T\in L(X,Y)$则$\ker(T^*) = {^\perp R(T)}$,$\ker(T) = \R(T^*)^\perp$.
\end{problem}
\begin{proof}\ \vspace*{-1cm}
    \begin{align*}
        \left.\begin{aligned}
            \forall f\in\ker(T^*),\ \text{则}\langle f, Tx\rangle = \langle T^*f,x\rangle = 0,\ \text{则}f\in{^\perp R(T)}\quad\\
            \forall f\in {^\perp R(T)},\ \text{则}\langle T^*f,x\rangle = \langle f,Tx\rangle = 0,\ \text{则}f\in\ker (T^*)\quad
        \end{aligned}\right\}&\ \ker(T^*) = {^\perp R(T)}\\
        \left.\begin{aligned}
            \forall x\in\ker(T),\ \text{则}\langle T^*f,x\rangle = \langle f, Tx\rangle =f(0) = 0,\ \text{则}x\in{R(T^*)^\perp}\quad\\
            \forall x\in {R(T^*)^\perp},\ \text{则}\langle f,Tx\rangle = \langle T^*f,x\rangle = 0,\ \text{则}x\in\ker (T)\quad
        \end{aligned}\right\}&\ \ker(T) = {R(T^*)^\perp}
    \end{align*}
\end{proof}
\begin{problem}
    设$X = \{\xi = (x_1,\cdots, x_n)\in l^2:\sum_{n\geq 1}|nx_n|^2 < \infty\}$,$T:x\to l^2$,$Tx=x$,证明$\ol{R(T)} = l^2$.
\end{problem}
\begin{proof}
    先证明$X$为$l^2$的子空间,$\forall \alpha,\beta\in \K,\ \forall \xi,\eta\in X$,令$\xi = \{x_n\},\ \eta = \{y_n\}$,$\forall N > 0$,有
    \begin{equation*}
        \sum_{1\leq n\leq N}|n(\alpha x_n+\beta y_n)|^2 = \alpha^2\sum_{1\leq n\leq N}n^2x_n^2+2\alpha\beta\sum_{1\leq n\leq N}n^2x_ny_n + \beta^2\sum_{1\leq n\leq N}n^2y_n^2
    \end{equation*}
    由于$\xi,\eta\in X$,于是$\sum_{1\leq n\leq N}n^2x_n^2,\sum_{1\leq n\leq N}n^2y_n^2$关于$N$收敛,又由于
    \begin{equation*}
        \sum_{1\leq n\leq M}n^2x_ny_n\leq \left(\sum_{1\leq n\leq M}|nx_n|^2\right)^{\frac{1}{2}}\left(\sum_{1\leq n\leq M}|ny_n|^2\right)^{\frac{1}{2}}
    \end{equation*}
    于是
    \begin{equation*}
        \lim_{N\to\infty}\sum_{1\leq n\leq N}|n(\alpha x_n+\beta y_n)|^2 = \sum_{n\geq 1}|n(\alpha x_n+\beta y_n)|^2 < \infty
    \end{equation*}
    所以$\alpha \xi+\beta\eta\in X$,$X$是$l^2$的闭子空间.

    由Hahn-Banach定理推论可得,要证$\ol{R(T)} = l^2$即$R(T)$在$l^2$中稠密,只需证:$\forall f\in (l^2)^*,f|_{R(T)} = 0\Rightarrow f=\theta$. 反设,存在$f\neq \theta$使得$f|_{R(T)}=0$,由于$l^2$是Hilbert空间,由Riesz表示定理可知,存在$\eta_f\in l^2,\ \eta_f = \{y_n\}$使得$\forall \xi\in X,\ \xi = \{x_n\}$有
    \begin{equation*}
        f(\xi) = (\xi,\eta_f) = \sum_{n\geq 1}x_n\bar{y}_n = 0
    \end{equation*}
    由于$f\neq \theta$,于是$\eta_f\neq \theta$,即$\exists y_n\neq 0$,令$\xi = (\underbrace{0,\cdots,0,1}_{n\text{个}},0,\cdots)\in X$,则$f(\xi) = \bar{y}_n\neq 0$与$f(\xi) = 0$矛盾,则$f|_{R(T)}=\theta\Rightarrow f= \theta$.
\end{proof}

\end{document}