\documentclass[12pt, a4paper, oneside]{ctexart}
\usepackage{amsmath, amsthm, amssymb, bm, color, graphicx, geometry, mathrsfs,extarrows, braket, booktabs, array, wrapfig, enumitem}
\usepackage[colorlinks,linkcolor=red,anchorcolor=blue,citecolor=blue,urlcolor=blue,menucolor=black]{hyperref}
\setCJKmainfont{方正新书宋_GBK.ttf}[ BoldFont = 方正小标宋_GBK, ItalicFont = 方正楷体_GBK]
\setmainfont{Times New Roman}  % 设置英文字体
\setsansfont{Calibri}
\setmonofont{Consolas}

\linespread{1.4}
%\geometry{left=2.54cm,right=2.54cm,top=3.18cm,bottom=3.18cm}
\geometry{left=1.84cm,right=1.84cm,top=2.18cm,bottom=2.18cm}
\newcounter{problem}  % 问题序号计数器
\newenvironment{problem}[1][]{\stepcounter{problem}\par\noindent\textbf{题目\arabic{problem}. #1}}{\smallskip\par}
\newenvironment{solution}[1][]{\par\noindent\textbf{#1解答. }}{\smallskip\par}  % 可带一个参数表示题号\begin{solution}{题号}
\newenvironment{note}{\par\noindent\textbf{注记. }}{\smallskip\par}

%%%% 图片相对路径 %%%%
\graphicspath{{figure/}} % 当前目录下的figure文件夹, {../figure/}则是父目录的figure文件夹

%%%% 缩小item,enumerate,description两行间间距 %%%%
\setenumerate[1]{itemsep=0pt,partopsep=0pt,parsep=\parskip,topsep=5pt}
\setitemize[1]{itemsep=0pt,partopsep=0pt,parsep=\parskip,topsep=5pt}
\setdescription{itemsep=0pt,partopsep=0pt,parsep=\parskip,topsep=5pt}

\everymath{\displaystyle} % 默认全部行间公式
\DeclareMathOperator*\uplim{\overline{lim}} % 定义上极限 \uplim_{}
\DeclareMathOperator*\lowlim{\underline{lim}} % 定义下极限 \lowlim_{}
\let\leq=\leqslant % 将全部leq变为leqslant
\let\geq=\geqslant % geq同理
\DeclareRobustCommand{\rchi}{{\mathpalette\irchi\relax}}
\newcommand{\irchi}[2]{\raisebox{\depth}{$#1\chi$}} % 使用\rchi将\chi居中

%%%% 一些宏定义 %%%%
\def\bd{\boldsymbol}        % 加粗(向量) boldsymbol
\def\disp{\displaystyle}    % 使用行间公式 displaystyle(默认)
\def\weekto{\rightharpoonup}% 右半箭头
\def\tsty{\textstyle}       % 使用行内公式 textstyle
\def\sign{\text{sign }}      % sign function
\def\supp{\text{supp }}      % supp
\def\wtd{\widetilde}        % 宽波浪线 widetilde
\def\R{\mathbb{R}}          % Real number
\def\N{\mathbb{N}}          % Natural number
\def\Z{\mathbb{Z}}          % Integer number
\def\Q{\mathbb{Q}}          % Rational number
\def\C{\mathbb{C}}          % Complex number
\def\K{\mathbb{K}}          % Number Field
\def\P{\mathbb{P}}          % Polynomial
\def\d{\mathrm{d}}          % differential operator
\def\e{\mathrm{e}}          % Euler's number
\def\i{\mathrm{i}}          % imaginary number
\def\re{\mathrm{Re}}        % Real part
\def\im{\mathrm{Im}}        % Imaginary part
\def\res{\mathrm{Res}}      % Residue
\def\ker{\mathrm{Ker}}      % Kernel
\def\vspan{\mathrm{vspan}}  % Span  \span与latex内核代码冲突改为\vspan
\def\L{\mathcal{L}}         % Loss function
\def\wdh{\widehat}          % 宽帽子 widehat
\def\ol{\overline}          % 上横线 overline
\def\ul{\underline}         % 下横线 underline
\def\add{\vspace{1ex}}      % 增加行间距
\def\del{\vspace{-1.5ex}}   % 减少行间距

%%%% 定理类环境的定义 %%%%
\newtheorem{theorem}{定理}

%%%% 基本信息 %%%%
\newcommand{\RQ}{\today} % 日期
\newcommand{\km}{泛函分析} % 科目
\newcommand{\bj}{强基数学002} % 班级
\newcommand{\xm}{吴天阳} % 姓名
\newcommand{\xh}{2204210460} % 学号

\begin{document}

%\pagestyle{empty}
\pagestyle{plain}
\vspace*{-15ex}
\centerline{\begin{tabular}{*5{c}}
    \parbox[t]{0.25\linewidth}{\begin{center}\textbf{日期}\\ \large \textcolor{blue}{\RQ}\end{center}} 
    & \parbox[t]{0.2\linewidth}{\begin{center}\textbf{科目}\\ \large \textcolor{blue}{\km}\end{center}}
    & \parbox[t]{0.2\linewidth}{\begin{center}\textbf{班级}\\ \large \textcolor{blue}{\bj}\end{center}}
    & \parbox[t]{0.1\linewidth}{\begin{center}\textbf{姓名}\\ \large \textcolor{blue}{\xm}\end{center}}
    & \parbox[t]{0.15\linewidth}{\begin{center}\textbf{学号}\\ \large \textcolor{blue}{\xh}\end{center}} \\ \hline
\end{tabular}}
\begin{center}
    \zihao{-3}\textbf{第十二次作业}
\end{center}
\vspace{-0.2cm}
% 正文部分
\begin{problem}
    设$X$为$B$空间,$T\in \mathfrak{C}(X)$,则$\ker(I-T)=\{\theta\}\Rightarrow R(I-T)=X$.
\end{problem}
\begin{proof}
    反设$R(I-T)\subsetneqq X$. 则$\exists x\in X$没有$I-T$下的原像,断言$R(I-T)^2\subsetneqq R(I-T)$,反设$R(I-T)^2=R(I-T)$,由于$(I-T)x\in R(I-T)$,则$\exists y\in X$使得$(I-T)^2y = (I-T)x\Rightarrow (I-T)\big((I-T)y-x\big)=\theta$,由于$\ker(I-T)=\{\theta\}$,则$(I-T)y=x$与$x$没有$I-T$下的原像矛盾,则$R(I-T)^2\subsetneqq R(I-T)$.

    依此类推,由Riesz引理,$\exists y_n\in R(I-T)^n$且$||y_n||=1$使得$\rho(y_n,R(I-T)^{n+1}) > 1/2$,于是$\forall p \geq 1,\ n\geq 1$有
    \begin{equation*}
        ||Ty_{n+p}-Ty_n|| = ||Ty_{n+p}-y_{n+p}+y_{n+p}-y_n+y_n-Ty_n||\geq \rho(y_n,R(I-T)^{n+1}) > 1/2
    \end{equation*}
    上述第一个不等号是因为:$Ty_{n+p}-y_{n+p} = (T-I)y_{n+p}\in R(I-T)^{n+p+1}\subsetneqq R(I-T)^{n+1},\ y_{n+p}\in R(I-T)^{n+p}\subsetneqq R(I-T)^{n+1},\ y_n-Ty_n = (I-T)y_n\in R(I-T)^{n+1}$.
    
    故$\{Ty_n\}$没有收敛子列,与$T$是紧算子矛盾. 所以$I-T$是满射.
\end{proof}
\begin{problem}
    $D(\Omega)$是序列完备的,即若$\{\varphi_j\}$满足:(1) 存在紧集$K\subset \Omega$使得$\supp\varphi_j\subset K$.\\
    (2) $\forall \varepsilon > 0, \forall \alpha$,$\exists N > 0$使得$\max_{K}|\partial^\alpha \varphi_n(x)-\partial^\alpha\varphi_m(x)| < \varepsilon,\ (n,m > N)$,则$\exists \varphi\in D(\Omega)$使得$\varphi_j\to\varphi$.
\end{problem}
\begin{proof}
    由于$\{\varphi_j\}$满足$\max_{K}|\varphi_i-\varphi_j|\to 0,\ (i,j\to\infty)$,由于$(C,||\cdot||_{\infty})$是完备的,则存在函数$\varphi$使得$\varphi_j\stackrel{||\cdot||_{\infty}}{\longrightarrow}\varphi$.

    又由于$\forall \alpha=(\alpha_1,\cdots,\alpha_n)$,令$m=|\alpha|$,有$\max_{K}|\partial^\alpha\varphi_i-\partial^\alpha\varphi_j|\to0,\ (i,j\to\infty)$,由于$(C^m,||\cdot||)$是完备的,其中范数的定义为$||\varphi||:=\sup_{|\alpha|\leq m}\max_{K}|\partial^\alpha\varphi|$,则存在函数$\psi$使得$\partial^\alpha\varphi_j\stackrel{||\cdot||_\infty}{\longrightarrow}\psi$.

    由于收敛极限的导数就是导函数收敛的极限,所以$\partial^\alpha\varphi = \psi$,由于$\alpha$的任意性可知,$\varphi$的任意阶导数都存在,即$\varphi\in C_0^\infty$. 故$\forall \alpha$有$\max_{K}|\partial^\alpha\varphi_j-\partial^\alpha\varphi|\to 0$,则在$D(\Omega)$中有$\varphi_j\to\varphi$.
\end{proof}
\begin{problem}
    在$\R^1$中$f_j(x) = \frac{1}{\pi}\frac{\sin jx}{x}\ (j=1,2,\cdots)$,则$f_j\to\delta$.
\end{problem}
\begin{proof}
    由于$\forall \varphi\in D(\R^1)$有
    \begin{eqnarray*}
        |\langle f_j,\varphi\rangle-\langle\delta, \varphi\rangle| = \left|\frac{1}{\pi}\int_{\R^1}\frac{\sin jx}{x}\varphi(x)\,\d x-\varphi(0)\right| = \left|\frac{1}{\pi}\int_{\R^1}\frac{\varphi(x)-\varphi(0)}{x}\sin jx\,\d x\right|
    \end{eqnarray*}
    第二个等号是因为可通过留数定理构造挖去原点的围道计算$\int_{\R^1}\frac{\sin jx}{\pi x}\,\d x = \int_{\R^1}\frac{\sin x}{\pi x}\,\d x = 1$,又由于$\varphi$在$0$处导数存在,于是通过Riemann-Lebesgue引理可知,当$j\to\infty$时,$|\langle f_j,\varphi\rangle-\langle \delta,\varphi\rangle|\to 0$,故$f_j\to\delta$.
\end{proof}
\begin{problem}
    设$T\in D'(\Omega)$,则$\frac{\partial^2 T}{\partial x_k\partial x_j} = \frac{\partial^2 T}{\partial x_j\partial x_k}$.
\end{problem}
\begin{proof}
    \begin{align*}
        \langle\frac{\partial^2 T}{\partial x_k\partial x_j}, \varphi\rangle =&\ -\langle\frac{\partial T}{\partial x_k}, \frac{\partial \varphi}{\partial x_j}\rangle = \langle T, \frac{\partial^2 \varphi}{\partial x_j\partial x_k}\rangle = \langle T, \frac{\partial^2 \varphi}{\partial x_k\partial x_j}\rangle\\
        =&\ -\langle\frac{\partial T}{\partial x_j}, \frac{\partial \varphi}{\partial x_k}\rangle = \langle\frac{\partial^2 T}{\partial x_j\partial x_k},\varphi\rangle
    \end{align*}
\end{proof}
\end{document}