\documentclass[12pt, a4paper, oneside]{ctexart}
\usepackage{amsmath, amsthm, amssymb, bm, color, graphicx, geometry, mathrsfs,extarrows, braket, booktabs, array, wrapfig, enumitem}
\usepackage[colorlinks,linkcolor=red,anchorcolor=blue,citecolor=blue,urlcolor=blue,menucolor=black]{hyperref}
\setCJKmainfont{方正新书宋_GBK.ttf}[ BoldFont = 方正小标宋_GBK, ItalicFont = 方正楷体_GBK]
\setmainfont{Times New Roman}  % 设置英文字体
\setsansfont{Calibri}
\setmonofont{Consolas}

\linespread{1.4}
%\geometry{left=2.54cm,right=2.54cm,top=3.18cm,bottom=3.18cm}
\geometry{left=1.84cm,right=1.84cm,top=2.18cm,bottom=2.18cm}
\newcounter{problem}  % 问题序号计数器
\newenvironment{problem}[1][]{\stepcounter{problem}\par\noindent\textbf{题目\arabic{problem}. #1}}{\smallskip\par}
\newenvironment{solution}[1][]{\par\noindent\textbf{#1解答. }}{\smallskip\par}  % 可带一个参数表示题号\begin{solution}{题号}
\newenvironment{note}{\par\noindent\textbf{注记. }}{\smallskip\par}

%%%% 图片相对路径 %%%%
\graphicspath{{figure/}} % 当前目录下的figure文件夹, {../figure/}则是父目录的figure文件夹

%%%% 缩小item,enumerate,description两行间间距 %%%%
\setenumerate[1]{itemsep=0pt,partopsep=0pt,parsep=\parskip,topsep=5pt}
\setitemize[1]{itemsep=0pt,partopsep=0pt,parsep=\parskip,topsep=5pt}
\setdescription{itemsep=0pt,partopsep=0pt,parsep=\parskip,topsep=5pt}

\everymath{\displaystyle} % 默认全部行间公式
\DeclareMathOperator*\uplim{\overline{lim}} % 定义上极限 \uplim_{}
\DeclareMathOperator*\lowlim{\underline{lim}} % 定义下极限 \lowlim_{}
\let\leq=\leqslant % 将全部leq变为leqslant
\let\geq=\geqslant % geq同理
\DeclareRobustCommand{\rchi}{{\mathpalette\irchi\relax}}
\newcommand{\irchi}[2]{\raisebox{\depth}{$#1\chi$}} % 使用\rchi将\chi居中

%%%% 一些宏定义 %%%%
\def\bd{\boldsymbol}        % 加粗(向量) boldsymbol
\def\disp{\displaystyle}    % 使用行间公式 displaystyle(默认)
\def\weekto{\rightharpoonup}% 右半箭头
\def\tsty{\textstyle}       % 使用行内公式 textstyle
\def\sign{\text{sign}}      % sign function
\def\wtd{\widetilde}        % 宽波浪线 widetilde
\def\R{\mathbb{R}}          % Real number
\def\N{\mathbb{N}}          % Natural number
\def\Z{\mathbb{Z}}          % Integer number
\def\Q{\mathbb{Q}}          % Rational number
\def\C{\mathbb{C}}          % Complex number
\def\K{\mathbb{K}}          % Number Field
\def\P{\mathbb{P}}          % Polynomial
\def\d{\mathrm{d}}          % differential operator
\def\e{\mathrm{e}}          % Euler's number
\def\i{\mathrm{i}}          % imaginary number
\def\re{\mathrm{Re}}        % Real part
\def\im{\mathrm{Im}}        % Imaginary part
\def\res{\mathrm{Res}}      % Residue
\def\ker{\mathrm{Ker}}      % Kernel
\def\vspan{\mathrm{vspan}}  % Span  \span与latex内核代码冲突改为\vspan
\def\L{\mathcal{L}}         % Loss function
\def\wdh{\widehat}          % 宽帽子 widehat
\def\ol{\overline}          % 上横线 overline
\def\ul{\underline}         % 下横线 underline
\def\add{\vspace{1ex}}      % 增加行间距
\def\del{\vspace{-1.5ex}}   % 减少行间距

%%%% 定理类环境的定义 %%%%
\newtheorem{theorem}{定理}

%%%% 基本信息 %%%%
\newcommand{\RQ}{\today} % 日期
\newcommand{\km}{泛函分析} % 科目
\newcommand{\bj}{强基数学002} % 班级
\newcommand{\xm}{吴天阳} % 姓名
\newcommand{\xh}{2204210460} % 学号

\begin{document}

%\pagestyle{empty}
\pagestyle{plain}
\vspace*{-15ex}
\centerline{\begin{tabular}{*5{c}}
    \parbox[t]{0.25\linewidth}{\begin{center}\textbf{日期}\\ \large \textcolor{blue}{\RQ}\end{center}} 
    & \parbox[t]{0.2\linewidth}{\begin{center}\textbf{科目}\\ \large \textcolor{blue}{\km}\end{center}}
    & \parbox[t]{0.2\linewidth}{\begin{center}\textbf{班级}\\ \large \textcolor{blue}{\bj}\end{center}}
    & \parbox[t]{0.1\linewidth}{\begin{center}\textbf{姓名}\\ \large \textcolor{blue}{\xm}\end{center}}
    & \parbox[t]{0.15\linewidth}{\begin{center}\textbf{学号}\\ \large \textcolor{blue}{\xh}\end{center}} \\ \hline
\end{tabular}}
\begin{center}
    \zihao{-3}\textbf{第十一次作业}
\end{center}
\vspace{-0.2cm}
% 正文部分
\begin{problem}[(2.6.2)]
    设$A$是闭线性算子,$\lambda_1,\lambda_2,\cdots,\lambda_n\in \sigma_p(A)$两两互异,又设$x_i$是对应于$\lambda_i$的特征元$(i=1,2,\cdots,n)$. 证明:$\{x_1,\cdots,x_n\}$是线性无关的.
\end{problem}
\begin{proof}
    反设$\{x_1,\cdots,x_n\}$线性相关,不妨令$x_n=\sum_{k=1}^{n-1}\alpha_kx_k$且$\{x_1,\cdots,x_{n-1}\}$线性无关,则
    \begin{equation*}
        (\lambda_n I-A)x_n = 0 = \sum_{k=1}^{n-1}\alpha_k(\lambda_nI-A)x_k = \sum_{k=1}^{n-1}\alpha_k(\lambda_nx_k-\lambda_kx_k) = \sum_{k=1}^{n-1}\alpha_k(\lambda_n-\lambda_k)x_k
    \end{equation*}
    由于$\{x_1,\cdots,x_{n-1}\}$线性无关,则$\alpha_k(\lambda_n-\lambda_k) = 0,\ (k=1,2,\cdots,n-1)$,又由于$\lambda_1,\cdots,\lambda_n$两两互异,则$\alpha_k = 0,\ (k=1,2,\cdots,n-1)$,于是$x_n = 0$与$x_n$为特征向量矛盾,故$\{x_1,\cdots,x_n\}$线性无关.
\end{proof}
\begin{problem}[(2.6.3)]
    在双边$l^2$空间上,考虑右推移算子
    \begin{align*}
        A:x=&\ (\cdots,\xi_{-n},\xi_{-n+1},\cdots,\xi_{-1},\xi_0,\xi_1,\cdots,\xi_{n-1},\xi_n,\cdots)\in l^2\\
        &\ \quad \mapsto Ax = (\cdots, \eta_{-n},\eta_{-n+1},\cdots,\eta_{-1},\eta_0,\eta_1,\cdots,\eta_{n-1},\eta_n,\cdots),
    \end{align*}
    其中$\eta_m = \xi_{m-1}(m\in \Z)$. 求证:$\sigma_c(A)=\sigma(A)=\text{单位圆周}.$
\end{problem}
\begin{proof}
    设$x = \{\xi_n\}\in l^2$,满足$(\lambda I-A)x = 0\Rightarrow \lambda x = Ax\Rightarrow \lambda \xi_k = \xi_{k-1},\ (k\in \Z)$,则
    \begin{equation*}
        x = \left(\cdots, \lambda^n\xi_0,\cdots,\lambda\xi_0,\xi_0,\frac{\xi_0}{\lambda},\cdots,\frac{\xi_0}{\lambda^n},\cdots\right)
    \end{equation*}
    
    由于$x\in l^2$,则$\sum_{n\in\Z}|\xi_n|^2 = |\xi_0|^2+\sum_{n\geq 1}\left|\frac{\xi_0}{\lambda^n}\right|^2+\sum_{n\leq 1}|\lambda^{-n}\xi_0|^2 < \infty$,则\\$|\xi_0|^2\left(1+\sum_{n\geq 1}\frac{1}{|\lambda|^{2n}}+|\lambda|^{2n}\right) < \infty$,若第二项为$0$,则$\frac{1}{\lambda}\to 0,\ \lambda\to 0$矛盾,于是$\xi_0 = 0$,则$x = 0$.

    综上,$(\lambda I-A)x = 0$只有零解,故$\sigma_p(A) = \varnothing$.

    下证$\sigma_r(A) = \varnothing$,只需证$\ol{R(\lambda I-A)} = l^2$,只需证$R(\lambda I-A)^{\perp} = \{0\}$,设$y=\{\eta_n\}\in R(\lambda I-A)^\perp$,则$((\lambda I-A)x,y) = \sum_{k\in \Z}(\lambda\xi_k-\xi_{k-1})\eta_k = 0$,取$x = e_n = (\underbrace{0,\cdots,0, 1}_{n\text{个}}, 0,\cdots)$则
    \begin{equation*}
        ((\lambda I-A)e_n,y) = \lambda \eta_n-\eta_{n+1} = 0
    \end{equation*}
    类似上述证明可知$y=0$,故$\sigma_r(A) = \varnothing$.

    所以$\sigma(A) = \sigma_p(A) + \sigma_c(A) + \sigma_r(A) = \sigma_c(A)$.
\end{proof}
\begin{problem}[(2.6.4)]
    在$l^2$空间上,考虑左推移算子$A:(\xi_1,\xi_2,\cdots)\mapsto(\xi_2,\xi_3,\cdots)$.

    证明:$\sigma_p(A) = \{\lambda\in \C:|\lambda|<1\},\sigma_c(A) = \{\lambda \in \C:|\lambda|=1\}$,且
    \begin{equation*}
        \sigma(A) = \sigma_p(A)\cup\sigma_c(A).
    \end{equation*}
\end{problem}\del\del\del\del
\begin{proof}
    由于$||Ax||\leq ||x||$,则$||A||\leq 1$,则$|\lambda| > 1$时,$\lambda \in \rho(A)$. 下面讨论$|\lambda|\leq 1$的情况.

    当$|\lambda| < 1$时,$\sum_{n\geq 1}|\lambda|^{2n} < \infty$,于是$(1,\lambda,\lambda^2,\cdots)\in l^2$,则
    \begin{equation*}
        A_n(1,\lambda,\lambda^2) = (\lambda,\lambda^2,\lambda^3,\cdots) = \lambda(1,\lambda,\lambda^2,\cdots)
    \end{equation*}
    则$\lambda$为特征值,$(1,\lambda,\lambda^2,\cdots)\in l^2$是对应的特征向量,故$\lambda\in \sigma_p(A)$.

    当$|\lambda|=1$时,$\forall x = \{\xi_n\} \in l^2$,
    \begin{equation*}
        (I-A)x = 0\Rightarrow (\lambda\xi_1,\lambda\xi_2,\cdots) = (\xi_2,\xi_3,\cdots)
    \end{equation*}
    于是$\xi_k = \lambda^{k-1}\xi_1$,由于$x\in l^2$,则$\sum_{n\geq 1}|\xi_1|^2 < \infty\Rightarrow \xi_1 = 0$,故$x=0$.

    令$G = \{\{\xi_n\}\in l^2:\{\xi_n\}\text{中非零项有限}\}$,则$\forall y = \{\eta_k\}\in G$,不妨令$\eta_k = 0(k > n)$,于是
    \begin{equation*}
        (\lambda I-A)x = y\Rightarrow (\lambda\xi_1-\xi_2,\lambda\xi_2-\xi_3,\cdots) = (\eta_1,\eta_2,\cdots, \eta_n,0,\cdots)
    \end{equation*}
    于是
    \begin{equation*}
        \begin{cases}
            \lambda\xi_1-\xi_2 = \eta_1\\
            \lambda\xi_2-\xi_3 = \eta_2\\
            \quad\vdots\\
            \lambda\xi_n-\xi_{n+1} = \eta_n\\
            \lambda\xi_{n+1}-\xi_{n+2} = 0\\
            \quad\vdots
        \end{cases}\Rightarrow\begin{cases}
            \xi_1 = \sum_{k=1}^n\eta_k/\lambda^k\\
            \quad\vdots\\
            \xi_{n-1} = \eta_{n-1}/\lambda+\eta_n/\lambda\\
            \xi_n=\eta_n/\lambda\\
            \xi_{k+1}=0,\quad(k\geq n)
        \end{cases}
    \end{equation*}
    由$y$的任意性可知,对于$(\lambda I-A)$存在逆元,则$G\subset R(\lambda I-A)$,又由于$\bar{G} = l^2$,故$\ol{R(\lambda I-A)} = l^2$,所以$\lambda\in \sigma_c(A)$.

    综上,$\sigma_r(A) = \varnothing$,故$\sigma(A) = \sigma_p(A)\cup\sigma_c(A)$.
\end{proof}
\end{document}