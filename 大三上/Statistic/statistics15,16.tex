\documentclass[12pt, a4paper, oneside]{ctexart}
\usepackage{amsmath, amsthm, amssymb, bm, color, graphicx, geometry, mathrsfs,extarrows, braket, booktabs, array}
\usepackage[colorlinks,linkcolor=red,anchorcolor=blue,citecolor=blue,urlcolor=blue,menucolor=black]{hyperref}
%%%% 设置中文字体 %%%%
\setCJKmainfont{方正新书宋_GBK.ttf}[BoldFont=方正小标宋_GBK, ItalicFont=方正楷体_GBK]
%%%% 设置英文字体 %%%%
\setmainfont{Times New Roman}
\setsansfont{Calibri}
\setmonofont{Consolas}

\linespread{1.4}
%\geometry{left=2.54cm,right=2.54cm,top=3.18cm,bottom=3.18cm}
\geometry{left=1.84cm,right=1.84cm,top=2.18cm,bottom=2.18cm}
\newcounter{problem}  % 问题序号计数器
\newenvironment{problem}[1][]{\stepcounter{problem}\par\noindent\textbf{题目\arabic{problem}. #1}}{\smallskip\par}
\newenvironment{solution}[1][]{\par\noindent\textbf{#1解答. }}{\smallskip\par}  % 可带一个参数表示题号\begin{solution}{题号}
\newenvironment{note}{\par\noindent\textbf{注记. }}{\smallskip\par}

%%%% 图片相对路径 %%%%
\graphicspath{{figure/}} % 当前目录下的figure文件夹, {../figure/}则是父目录的figure文件夹
\setlength{\abovecaptionskip}{-0.2cm}  % 缩紧图片标题与图片之间的距离
\setlength{\belowcaptionskip}{0pt} 

\everymath{\displaystyle} % 默认全部行间公式
\DeclareMathOperator*\uplim{\overline{lim}} % 定义上极限 \uplim_{}
\DeclareMathOperator*\lowlim{\underline{lim}} % 定义下极限 \lowlim_{}
\DeclareMathOperator*{\argmax}{arg\,max}  % \argmin
\DeclareMathOperator*{\argmin}{arg\,min}  % \argmax
\let\leq=\leqslant % 将全部leq变为leqslant
\let\geq=\geqslant % geq同理
\DeclareRobustCommand{\rchi}{{\mathpalette\irchi\relax}}
\newcommand{\irchi}[2]{\raisebox{\depth}{$#1\chi$}} % 使用\rchi将\chi居中

%%%% 一些宏定义 %%%%
\def\bd{\boldsymbol}        % 加粗(向量) boldsymbol
\def\disp{\displaystyle}    % 使用行间公式 displaystyle(默认)
\def\tsty{\textstyle}       % 使用行内公式 textstyle
\def\sign{\text{sign}}      % sign function
\def\wtd{\widetilde}        % 宽波浪线 widetilde
\def\R{\mathbb{R}}          % Real number
\def\N{\mathbb{N}}          % Natural number
\def\Z{\mathbb{Z}}          % Integer number
\def\Q{\mathbb{Q}}          % Rational number
\def\C{\mathbb{C}}          % Complex number
\def\N{\mathbb{N}}          % Natural number
\def\Z{\mathbb{Z}}          % Integer number
\def\E{\mathbb{E}}          % Exception
\def\var{\text{Var}}        % Variance
\def\cov{\text{Cov}}        % Coefficient of Variation
\def\bias{\text{bias}}      % bias
\def\d{\mathrm{d}}          % differential operator
\def\e{\mathrm{e}}          % Euler's number
\def\i{\mathrm{i}}          % imaginary number
\def\re{\mathrm{Re}}        % Real part
\def\im{\mathrm{Im}}        % Imaginary part
\def\res{\mathrm{Res}}      % Residue
\def\L{\mathcal{L}}         % Loss function
\def\wdh{\widehat}          % 宽帽子 widehat
\def\ol{\overline}          % 上横线 overline
\def\ul{\underline}         % 下横线 underline
\def\add{\vspace{1ex}}      % 增加行间距
\def\del{\vspace{-1.5ex}}   % 减少行间距

%%%% 定理类环境的定义 %%%%
\newtheorem{theorem}{定理}

%%%% 基本信息 %%%%
\newcommand{\RQ}{\today} % 日期
\newcommand{\km}{数理统计} % 科目
\newcommand{\bj}{强基数学002} % 班级
\newcommand{\xm}{吴天阳} % 姓名
\newcommand{\xh}{2204210460} % 学号
\newcommand{\id}{50} % 序号

\begin{document}

%\pagestyle{empty}
\pagestyle{plain}
\vspace*{-15ex}
\centerline{\begin{tabular}{*6{c}}
    \parbox[t]{0.25\linewidth}{\begin{center}\textbf{日期}\\ \large \textcolor{blue}{\RQ}\end{center}} 
    & \parbox[t]{0.2\linewidth}{\begin{center}\textbf{科目}\\ \large \textcolor{blue}{\km}\end{center}}
    & \parbox[t]{0.2\linewidth}{\begin{center}\textbf{班级}\\ \large \textcolor{blue}{\bj}\end{center}}
    & \parbox[t]{0.1\linewidth}{\begin{center}\textbf{姓名}\\ \large \textcolor{blue}{\xm}\end{center}}
    & \parbox[t]{0.15\linewidth}{\begin{center}\textbf{学号}\\ \large \textcolor{blue}{\xh}\end{center}}
    & \parbox[t]{0.1\linewidth}{\begin{center}\textbf{序号}\\ \large \textcolor{blue}{\id}\end{center}}
     \\ \hline
\end{tabular}}
\begin{center}
    \zihao{3}\textbf{第八次作业}
\end{center}\vspace{-0.2cm}
% 正文部分
\begin{problem}[(41)]
    投掷一个骰子$300$次,记录得到以下结果:
% 表格模板
\renewcommand\arraystretch{0.8} % 设置表格高度为原来的0.8倍
\begin{table}[!htbp] % table标准
    \centering % 表格居中
    %\begin{tabular}{p{1cm}<{\centering}p{1cm}<{\centering}p{3cm}<{\centering}p{5cm}<{\centering}} % 设置表格宽度
    \begin{tabular}{ccccccc}
        结果:& 1& 2& 3& 4& 5& 6\\
        频率:& 43& 49& 56& 45& 66& 41
    \end{tabular}
\end{table}\vspace{-0.3cm}
\ \\请问是否有$0.05$的检验水平认为该骰子是均匀的?
\end{problem}
\begin{solution}
    由题可知,$\alpha = 0.05, k=6,n_1=43,n_2=49,n_3=56,n_4=45,n_5=66,n_6=41,n=300,p_j^0=1/6$.

    \textbf{卡方拟合优度检验}:广义似然比为$\lambda = \prod_{j=1}^k\left(\frac{np_j^0}{n_j}\right)^{n_j} \approx 0.0133$,通过查表可知$\rchi^2_{1-\alpha}(k-1) = \rchi^2_{0.95}(5)\approx 11.07$,由于$-2\log \lambda \approx 8.634 < 11.07$,所以有0.05的检验水平认为该骰子是均匀的.
    
    \textbf{Pearson统计量}:求解Pearson统计量$q_5^0 = \sum_{j=1}^k\frac{(n_j-np_j^0)^2}{np_j^0} = 8.96$,由于$8.96 < 11.07$,所以也有$0.05$的检验水平认为该骰子是均匀的.
\end{solution}
\begin{problem}[(42)]
    豚鼠杂交的$64$个后代中,有$34$个红色的,$10$个黑色的,$20$个白色的. 根据遗传学模型,这些数字的比例应该满足$9/3/4$. 这些数据是否有$0.05$的检验水平满足该模型?
\end{problem}
\begin{solution}
    由题可知,$\alpha = 0.05, k = 3, n_1 = 34, n_2 = 10, n_3 = 20, n = 64, p_1^0 = 9/16, p_2^0 = 3/16, p_3^0 = 4/16$.

    \textbf{卡方拟合优度检验}:广义似然比为$\lambda = \prod_{j=1}^k\left(\frac{np_j^0}{n_j}\right)^{n_j} \approx 0.498$,通过查表可知$\rchi^2_{1-\alpha}(k-1) = \rchi^2_{0.95}(2)\approx 5.994$,由于$-2\log \lambda \approx 1.3925 < 5.994$,所以有0.05的检验水平认为该骰子是均匀的.
    
    \textbf{Pearson统计量}:求解Pearson统计量$q_2^0 = \sum_{j=1}^k\frac{(n_j-np_j^0)^2}{np_j^0} = 1.444$,由于$1.444 < 5.994$,所以也有$0.05$的检验水平认为该骰子是均匀的.
\end{solution}
\begin{problem}[(47)]
    对于下面性别关于是否色盲的$2\times 2$的列联表,检验患色盲与性别是否独立.
\renewcommand\arraystretch{0.8} % 设置表格高度为原来的0.8倍
\begin{table}[!htbp] % table标准
    \centering % 表格居中
    \begin{tabular}{p{2cm}<{\centering}p{2cm}<{\centering}p{2cm}} % 设置表格宽度
        \toprule
         &男性&女性\\
        \midrule
        正常&442&514\\
        色盲&38&6\\
        \bottomrule
    \end{tabular}
\end{table}\vspace{-0.3cm}
\end{problem}
\begin{solution}
    由题可知,$n=1000, n_{11} = 442, n_{12} = 514, n_{21} = 38, n_{22} = 6, \hat{p}_{1.} = 0.956, \hat{p}_{2.} = 0.044, \hat{p}_{.1}=0.48, \hat{p}_{.2}=0.52$.\add

    \textbf{卡方拟合优度检验}:广义似然比为$\lambda = \frac{\prod_{i=1}^2\hat{p}_{i.}^{n_{i.}}\prod_{j=1}^2\hat{p}_{.j}^{n_{.j}}}{\prod_{i=1}^2\prod_{j=1}^2\hat{p}_{ij}^{n_{ij}}}\approx 3.426\times 10^{-7}$,
    通过查表可知$\rchi^2_{0.95}(1)\approx 3.838$则$-2\log \lambda \approx 29.773 > 3.838$,故可以以$0.05$的检验水平否定色盲与性别独立.

    \textbf{Pearson统计量}:Pearson统计量为$q = \sum_{i=1}^2\sum_{j=1}^2\frac{(n_{ij}-n\hat{p}_{i.}\hat{p}_{.j})^2}{n\hat{p}_{i.}\hat{p}_{.j}}\approx 27.1387 > 3.838$,故也可以以$0.05$的检验水平否定色盲与性别独立.
\end{solution}
\end{document}