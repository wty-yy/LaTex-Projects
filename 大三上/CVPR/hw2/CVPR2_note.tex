% Options for packages loaded elsewhere
\PassOptionsToPackage{unicode}{hyperref}
\PassOptionsToPackage{hyphens}{url}
%
\documentclass[
]{article}
\usepackage{amsmath,amssymb}
\usepackage{lmodern}
\usepackage{iftex}
\ifPDFTeX
  \usepackage[T1]{fontenc}
  \usepackage[utf8]{inputenc}
  \usepackage{textcomp} % provide euro and other symbols
\else % if luatex or xetex
  \usepackage{unicode-math}
  \defaultfontfeatures{Scale=MatchLowercase}
  \defaultfontfeatures[\rmfamily]{Ligatures=TeX,Scale=1}
\fi
% Use upquote if available, for straight quotes in verbatim environments
\IfFileExists{upquote.sty}{\usepackage{upquote}}{}
\IfFileExists{microtype.sty}{% use microtype if available
  \usepackage[]{microtype}
  \UseMicrotypeSet[protrusion]{basicmath} % disable protrusion for tt fonts
}{}
\makeatletter
\@ifundefined{KOMAClassName}{% if non-KOMA class
  \IfFileExists{parskip.sty}{%
    \usepackage{parskip}
  }{% else
    \setlength{\parindent}{0pt}
    \setlength{\parskip}{6pt plus 2pt minus 1pt}}
}{% if KOMA class
  \KOMAoptions{parskip=half}}
\makeatother
\usepackage{xcolor}
\usepackage{graphicx}
\makeatletter
\def\maxwidth{\ifdim\Gin@nat@width>\linewidth\linewidth\else\Gin@nat@width\fi}
\def\maxheight{\ifdim\Gin@nat@height>\textheight\textheight\else\Gin@nat@height\fi}
\makeatother
% Scale images if necessary, so that they will not overflow the page
% margins by default, and it is still possible to overwrite the defaults
% using explicit options in \includegraphics[width, height, ...]{}
\setkeys{Gin}{width=\maxwidth,height=\maxheight,keepaspectratio}
% Set default figure placement to htbp
\makeatletter
\def\fps@figure{htbp}
\makeatother
\setlength{\emergencystretch}{3em} % prevent overfull lines
\providecommand{\tightlist}{%
  \setlength{\itemsep}{0pt}\setlength{\parskip}{0pt}}
\setcounter{secnumdepth}{-\maxdimen} % remove section numbering
\ifLuaTeX
  \usepackage{selnolig}  % disable illegal ligatures
\fi
\IfFileExists{bookmark.sty}{\usepackage{bookmark}}{\usepackage{hyperref}}
\IfFileExists{xurl.sty}{\usepackage{xurl}}{} % add URL line breaks if available
\urlstyle{same} % disable monospaced font for URLs
\hypersetup{
  hidelinks,
  pdfcreator={LaTeX via pandoc}}

\author{}
\date{}

\begin{document}

\textbf{⼀、图像变换(}图像⾃选,编程语⾔⾃选)

✓ 图像的参数化⼏何变换原理(参考书:计算机视觉-算法与应⽤,3.6章节);

五种变换

✓ 图像的向前变换(forward warping)与图像的逆向变换(inverse warping);

✓ 图像的下抽样原理与图像的内插⽅法原理(近邻插值与双线性插值);

✓
完成图像的⼏何变换实验,包括:平移变换;旋转变换;欧式变换;相似变换;仿射变换。

✓
完成图像的⾼斯⾦字塔表示与拉普拉斯⾦字塔表示,讨论前置低通滤波与抽样频率的关系。

\textbf{⼆、特征检测(}图像⾃选,编程语⾔⾃选\textbf{)}

✓
基于⾼斯⼀阶微分的图像梯度(幅值图与⽅向图),分析⾼斯⽅差对图像梯度的影响;

✓
掌握Canny边缘检测原理,完成图像的边缘检测实验,展示每个环节的处理结果(梯度图、NMS、边缘链接);

✓
掌握Harris⻆点检测原理,完成图像的⻆点检测实验,分析窗⼝参数对⻆点检测的影响,讨论⻆点检测的不变

性、等变性与定位精度等。

\hypertarget{ux5b8cux6210ux4e0aux6b21ux672aux5b8cux6210ux7684ux586bux5145ux64cdux4f5c}{%
\paragraph{完成上次未完成的填充操作}\label{ux5b8cux6210ux4e0aux6b21ux672aux5b8cux6210ux7684ux586bux5145ux64cdux4f5c}}

\begin{figure}
\centering
\includegraphics{C:/Users/admin/Documents/GitHub/CVPR_homeworks/code/hw2/CVPR2_note.figure/填充操作1.png}
\caption{}
\end{figure}

\begin{figure}
\centering
\includegraphics{C:/Users/admin/Documents/GitHub/CVPR_homeworks/code/hw2/CVPR2_note.figure/填充操作2.png}
\caption{}
\end{figure}

\hypertarget{ux51e0ux4f55ux53d8ux6362ux5b9eux9a8c5ux79cdux53d8ux6362uxff09}{%
\subsubsection{几何变换实验(5种变换)}\label{ux51e0ux4f55ux53d8ux6362ux5b9eux9a8c5ux79cdux53d8ux6362uxff09}}

图像坐标系默认是按照左上角为原点,纵轴向下为 \(x\) 轴正方向,横轴向右为
\(y\)
轴正方向,为便于输出查看,变化后的图像保持与原图像相同的大小,但这样就会发生图像大部分空白,所以需要使用平移矩阵将变化后的图像中心保持在输出框的中心,具体来说,假设图像大小为
\(N\times M\),几何变换为 \(T\),记图像中心点为
\(\boldsymbol{x}_{mid} = (N/2,M/2)\) ,则考虑平移向量和对应的平移矩阵为

\[\boldsymbol{t} = \boldsymbol{x}_{mid} - T\boldsymbol{x}_{mid},\quad T_{translation}=\left[\begin{matrix}1&0&t_1\\ 0&1&t_2\\0&0&1\end{matrix}\right]\]

为了保证输出效果,处平移操作和旋转操作外,其他操作都进行平移.

\begin{figure}
\centering
\includegraphics{C:/Users/admin/Documents/GitHub/CVPR_homeworks/code/hw2/CVPR2_note.figure/几何变换实验.png}
\caption{}
\end{figure}

\hypertarget{gaussux91d1ux5b57ux5854ux548claplaceux91d1ux5b57ux5854}{%
\subsubsection{Gauss金字塔和Laplace金字塔}\label{gaussux91d1ux5b57ux5854ux548claplaceux91d1ux5b57ux5854}}

Gauss金字塔:使用Gauss核与图像做卷积,设定Gauss核移动步长为stride=2,于是每次可将整个图像缩小
\(1/4\) 倍.

\textbf{实验步骤:}
在使用matplotlib进行绘图时,由于将图像同时输出到同一个画板上时,其会自动将子图进行放大,以保持相同的图像大小,所以为了输出图像金字塔的效果,需自行设定每个子图的坐标,四种距离参数包括:左侧,底部,横向,竖向.

\begin{figure}
\centering
\includegraphics{C:/Users/admin/Documents/GitHub/CVPR_homeworks/code/hw2/CVPR2_note.figure/下采样缩小结果.png}
\caption{}
\end{figure}

在进行上采样放大图像过程中,由于要对原图进行0填充补其奇数行与奇数列,然后再用大小为5的Gauss核做卷积,这样会是的图像的整体亮度偏低,最后通过对全体像素乘
\(4\) 来提高亮度(经过尝试,感觉这个大小和原图亮度最接近).
在这里进行Gauss做卷积的过程中,我尝试了 \(4\)
中不同的边界填充方法,其中边界镜像的效果最好,所以在下述上采样操作中都是用该边界填充方法.

\begin{figure}
\centering
\includegraphics{C:/Users/admin/Documents/GitHub/CVPR_homeworks/code/hw2/CVPR2_note.figure/上采用使用不同边界填充.png}
\caption{}
\end{figure}

\begin{figure}
\centering
\includegraphics{C:/Users/admin/Documents/GitHub/CVPR_homeworks/code/hw2/CVPR2_note.figure/上采样放大结果.png}
\caption{}
\end{figure}

\begin{figure}
\centering
\includegraphics{C:/Users/admin/Documents/GitHub/CVPR_homeworks/code/hw2/CVPR2_note.figure/Laplace金字塔.png}
\caption{}
\end{figure}

讨论低通滤波和抽样频率的关系,由Shannon-Nyquist定理可知,采样频率应至少大于最大频率的2倍以上才不会发生图像像素混淆.
图像使用的是``黄鹤楼'',因为该建筑上具有非常多的细节纹理,从上次作业Fourier变换结果来看,该图具有较高的频率,像素重叠现象更为明显.
下图分布以直接进行下采样和做Gauss模糊以后再进行采样进行比较,体现Gauss模糊确实能有效降低图像的频率.

\begin{figure}
\centering
\includegraphics{C:/Users/admin/Documents/GitHub/CVPR_homeworks/code/hw2/CVPR2_note.figure/直接进行采样.png}
\caption{}
\end{figure}

\begin{figure}
\centering
\includegraphics{C:/Users/admin/Documents/GitHub/CVPR_homeworks/code/hw2/CVPR2_note.figure/Gauss模糊以后进行下采样.png}
\caption{}
\end{figure}

\hypertarget{gaussux4e00ux9636ux5faeux5206ux7684ux56feux50cfux68afux5ea6}{%
\subsubsection{Gauss一阶微分的图像梯度}\label{gaussux4e00ux9636ux5faeux5206ux7684ux56feux50cfux68afux5ea6}}

\textbf{目标:幅度图与方向图,分析高斯方差对梯度的影响.}

\textbf{实验原理:}

二维Gauss函数表示如下

\[G_{\sigma}(x, y)=\frac{1}{2\pi \sigma^2}e^{-\frac{x^2+y^2}{2\sigma^2}}\]

对其计算 \(x, y\) 方向上偏导得

\[\frac{\partial G_{\sigma}(x, y)}{\partial x}=-\frac{x}{2\pi \sigma^4}e^{-\frac{x^2+y^2}{2\sigma^2}},\quad \frac{\partial G_{\sigma}(x, y)}{\partial y}=-\frac{y}{2\pi \sigma^4}e^{-\frac{x^2+y^2}{2\sigma^2}}\]

再将两种方向偏导对应的高斯偏导核作用在图像上,分别得到图像在两个方向上的偏导,记为
\(f_x, f_y\),于是根据下式可以分别计算出每个像素处梯度的幅度图(范数)和方向图

\[||\nabla f||(m,n) = \sqrt{f_x(m,n)^2+f_y(m,n)^2},\quad \theta(m,n) = \arctan\left(\frac{f_y(m,n)}{f_x(m,n)}\right).\]

\textbf{实验步骤:}

\(\sigma = 3\),大小为\(13\times 13\) 的高斯核分别在 \(x,y\)
轴方向上的偏导如下图所示(这里以图像纵轴向下为 \(x\) 正方向,横轴向右为
\(y\) 轴正方向).

\textbf{注}:此处Gauss一阶微分核无需再进行归一化处理.

\begin{figure}
\centering
\includegraphics{C:/Users/admin/Documents/GitHub/CVPR_homeworks/code/hw2/CVPR2_note.figure/两种Gauss一阶微分核.png}
\caption{}
\end{figure}

实际操作中,要显示幅度谱需要使用\textbf{线性正规化方法},而显示方向图要使用\textbf{截断正规化方法},下图实验了将
\(\sigma=0.5, 1, 5\) 分别作用在图像上,得到梯度的幅度图和方向图

\begin{figure}
\centering
\includegraphics{C:/Users/admin/Documents/GitHub/CVPR_homeworks/code/hw2/CVPR2_note.figure/梯度的幅度图和方向图.png}
\caption{}
\end{figure}

从中可以看出,高斯核方差越大,幅度图中边缘提取更加多,但是边缘厚度太大,导致后续进一步处理困难,所以选取适合的方差十分关键,从图中可看出
\(\sigma=2\) 时提取效果较好.

\hypertarget{cannyux8fb9ux7f18ux68c0ux6d4b}{%
\subsubsection{Canny边缘检测}\label{cannyux8fb9ux7f18ux68c0ux6d4b}}

\textbf{目标:梯度图、NMS、边缘连接.}

\textbf{实验原理:}Canny边缘检测算法共分为以下三步:

\begin{enumerate}
\def\labelenumi{\arabic{enumi}.}
\item
  \textbf{梯度提取},取方差为 \(\sigma\)
  的Gauss一阶微分核做卷积,得到幅度图和方向图.
\item
  \textbf{非极大值像素梯度抑制}(Non-Maximum Suppression,
  NMS),对每个像素考虑其梯度方向上的两个临近点,利用之前写好的\textbf{双线性插值}获得两点处的梯度值,然后判断中间点的梯度范数大于两端的梯度范数,若大于则保留,反之舍弃(归零).

  具体来讲,设当前像素点为 \(\boldsymbol{q}: (x, y)\), 幅度函数记为
  \(||\nabla f||(x, y)\),\(\boldsymbol{q}\) 点处的梯度方向记为
  \(\theta\),则考虑沿梯度方向上的两点

  \[\boldsymbol{r}:(x+\cos \theta, y+\sin\theta),\quad
  \boldsymbol{p}=(x-\cos\theta, y-\sin\theta)\]

  若
  \(||\nabla f||(\boldsymbol{q}) > \max\{||\nabla f||(\boldsymbol{r}),||\nabla f||(\boldsymbol{p})\}\),则保留该点,否则令
  \(f(\boldsymbol{q}) = 0\).
\item
  \textbf{高低阈值处理,边缘连接},考虑对NMS结果进一步处理,首先设定高低阈值,这里我使用的是按照幅度值的分位数确定阈值,可以根据不同图像进行自适应阈值大小.
  我们称所有幅度值大于高阈值的点为\textbf{高阈值点},所有幅度值大于低阈值的点为\textbf{低阈值点}.
  然后,将高阈值点从大到小进行枚举,每个点使用深度优先搜索查找边缘,假设当前搜索点为
  \(\boldsymbol{x}\),考虑周围八个方向
  \(\boldsymbol{y}_i\),分为以下四种情况:

  \begin{itemize}
  \item
    若 \(\boldsymbol{y}_i\)
    为搜索树上的父节点或上两层节点(这里判断两层是因为从其他结点到达该节点,在八个方向的领域内,最多是通过存在两个祖先节点),则跳过该节点.
  \item
    若存在 \(\boldsymbol{y}_i\) 为高阈值点,则从高阈值中幅度值最大
    \(\boldsymbol{y}_i\) 开始进行搜索,连接
    \((\boldsymbol{x}, \boldsymbol{y}_i)\),并对 \(\boldsymbol{y}_i\)
    进行迭代搜索.
  \item
    若不存在 \(\boldsymbol{y}_i\) 为高阈值点,且存在
    \(\boldsymbol{y}_i\) 为低阈值点,则取低阈值中幅度值最大的
    \(\boldsymbol{y}_i\),连接
    \((\boldsymbol{x}, \boldsymbol{y}_i)\),返回迭代.
  \item
    若无低阈值点,则返回迭代.
  \end{itemize}
\end{enumerate}

为保持边缘的连续性,连接 \((\boldsymbol{x}, \boldsymbol{y}_i)\)
的具体操作为,若 \(\boldsymbol{y}_i\) 是 \(\boldsymbol{x}\)
的正上下左右四个位置时,则直接进行连接,若 \(\boldsymbol{y}_i\) 是
\(\boldsymbol{x}\)
的左上、右上、左下、右下四个位置时,以左上为例,继续判断
\(\boldsymbol{x}\)
的正上和正左两个位置的幅度值,取较大者设定为边界点,并将其加入结点
\(\boldsymbol{y}_i\) 的父节点中,其他方向同理.

最后,将凸出像素点删去,若一个边界点周围的边界点个数 \(\leqslant 1\)
则删去该边界点,可以使最终边缘图像更加平滑.

\textbf{实验步骤:}

NMS过程中,发现梯度的幅度图均值仅有
\(0.00858\),为了使得进行线性插值时便于加权求和,对幅度谱先均值提升到1.
下图所展示的NMS过程(fox2),NMS过程共计删除 \(38945\) 个像素点,占比
\(14.856339\%\),设定低阈值为分位数 \(50\%\) 的像素值,高阈值为分位数
\(92\%\) 的像素值,边缘连接过程共计连接 \(14756\) 个像素点,占比
\(5.628967\%\).

\begin{figure}
\centering
\includegraphics{C:/Users/admin/Documents/GitHub/CVPR_homeworks/code/hw2/CVPR2_note.figure/Canny边缘检测.png}
\caption{}
\end{figure}

\begin{figure}
\centering
\includegraphics{C:/Users/admin/Documents/GitHub/CVPR_homeworks/code/hw2/CVPR2_note.figure/Canny边缘检测1.png}
\caption{}
\end{figure}

\begin{figure}
\centering
\includegraphics{C:/Users/admin/Documents/GitHub/CVPR_homeworks/code/hw2/CVPR2_note.figure/Canny边缘检测2.png}
\caption{}
\end{figure}

\hypertarget{harrisux89d2ux70b9ux68c0ux6d4b}{%
\subsubsection{Harris角点检测}\label{harrisux89d2ux70b9ux68c0ux6d4b}}

\begin{figure}
\centering
\includegraphics{C:/Users/admin/Documents/GitHub/CVPR_homeworks/code/hw2/CVPR2_note.figure/角点检测1.png}
\caption{}
\end{figure}

\begin{figure}
\centering
\includegraphics{C:/Users/admin/Documents/GitHub/CVPR_homeworks/code/hw2/CVPR2_note.figure/角点检测2.png}
\caption{}
\end{figure}

\begin{figure}
\centering
\includegraphics{C:/Users/admin/Documents/GitHub/CVPR_homeworks/code/hw2/CVPR2_note.figure/角点检测31.png}
\caption{}
\end{figure}

\begin{figure}
\centering
\includegraphics{C:/Users/admin/Documents/GitHub/CVPR_homeworks/code/hw2/CVPR2_note.figure/角点检测32.png}
\caption{}
\end{figure}

\end{document}
