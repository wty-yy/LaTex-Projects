% Options for packages loaded elsewhere
\PassOptionsToPackage{unicode}{hyperref}
\PassOptionsToPackage{hyphens}{url}
%
\documentclass[
]{article}
\usepackage{amsmath,amssymb}
\usepackage{lmodern}
\usepackage{iftex}
\ifPDFTeX
  \usepackage[T1]{fontenc}
  \usepackage[utf8]{inputenc}
  \usepackage{textcomp} % provide euro and other symbols
\else % if luatex or xetex
  \usepackage{unicode-math}
  \defaultfontfeatures{Scale=MatchLowercase}
  \defaultfontfeatures[\rmfamily]{Ligatures=TeX,Scale=1}
\fi
% Use upquote if available, for straight quotes in verbatim environments
\IfFileExists{upquote.sty}{\usepackage{upquote}}{}
\IfFileExists{microtype.sty}{% use microtype if available
  \usepackage[]{microtype}
  \UseMicrotypeSet[protrusion]{basicmath} % disable protrusion for tt fonts
}{}
\makeatletter
\@ifundefined{KOMAClassName}{% if non-KOMA class
  \IfFileExists{parskip.sty}{%
    \usepackage{parskip}
  }{% else
    \setlength{\parindent}{0pt}
    \setlength{\parskip}{6pt plus 2pt minus 1pt}}
}{% if KOMA class
  \KOMAoptions{parskip=half}}
\makeatother
\usepackage{xcolor}
\usepackage{graphicx}
\makeatletter
\def\maxwidth{\ifdim\Gin@nat@width>\linewidth\linewidth\else\Gin@nat@width\fi}
\def\maxheight{\ifdim\Gin@nat@height>\textheight\textheight\else\Gin@nat@height\fi}
\makeatother
% Scale images if necessary, so that they will not overflow the page
% margins by default, and it is still possible to overwrite the defaults
% using explicit options in \includegraphics[width, height, ...]{}
\setkeys{Gin}{width=\maxwidth,height=\maxheight,keepaspectratio}
% Set default figure placement to htbp
\makeatletter
\def\fps@figure{htbp}
\makeatother
\setlength{\emergencystretch}{3em} % prevent overfull lines
\providecommand{\tightlist}{%
  \setlength{\itemsep}{0pt}\setlength{\parskip}{0pt}}
\setcounter{secnumdepth}{-\maxdimen} % remove section numbering
\ifLuaTeX
  \usepackage{selnolig}  % disable illegal ligatures
\fi
\IfFileExists{bookmark.sty}{\usepackage{bookmark}}{\usepackage{hyperref}}
\IfFileExists{xurl.sty}{\usepackage{xurl}}{} % add URL line breaks if available
\urlstyle{same} % disable monospaced font for URLs
\hypersetup{
  hidelinks,
  pdfcreator={LaTeX via pandoc}}

\author{}
\date{}

\begin{document}

\hypertarget{ux9ad8ux65afux6ee4ux6ce2ux5668}{%
\subsubsection{高斯滤波器}\label{ux9ad8ux65afux6ee4ux6ce2ux5668}}

\[G(x, y) = \frac{1}{2\pi\sigma^2}e^{-\frac{x^2+y^2}{2\sigma^2}}\]

对高斯函数在 \((i,j)\ (-1\leqslant i,j\leqslant 1)\) 点处进行采样.
再进行归一化处理

\hypertarget{ux9ad8ux65afux6838ux65b9ux5deeux548cux6ee4ux6ce2ux5668ux5927ux5c0fux5173ux7cfb}{%
\paragraph{高斯核方差和滤波器大小关系}\label{ux9ad8ux65afux6838ux65b9ux5deeux548cux6ee4ux6ce2ux5668ux5927ux5c0fux5173ux7cfb}}

利用高斯函数的性质,在距离中心 \(\sigma\) 中取到的体积占据总体积的
\(68\%\),在 \(2\sigma\) 中占比为 \(95\%\),在 \(3\sigma\) 中占比为
\(99.7\%\),如下图所示:

\begin{figure}
\centering
\includegraphics{C:/Users/admin/Documents/GitHub/CVPR_homeworks/code/hw1/note.figure/sigma原则.png}
\caption{}
\end{figure}

这里取 \(2\sigma\) 为高斯函数的边界,设滤波器大小为
\(2k+1\times 2k+1\),则 \(\sigma = \frac{k}{2}\). 以下为 \(k=1\),大小为
\(3\times 3\) 的高斯滤波器

\begin{figure}
\centering
\includegraphics{C:/Users/admin/Documents/GitHub/CVPR_homeworks/code/hw1/note.figure/3x3高斯核.png}
\caption{}
\end{figure}

\hypertarget{ux8bbeux8ba1ux8fb9ux754cux5904ux7406ux65b9ux6cd5}{%
\paragraph{设计边界处理方法}\label{ux8bbeux8ba1ux8fb9ux754cux5904ux7406ux65b9ux6cd5}}

\begin{figure}
\centering
\includegraphics{C:/Users/admin/Documents/GitHub/CVPR_homeworks/code/hw1/note.figure/零填充.png}
\caption{}
\end{figure}

\hypertarget{ux9ad8ux65afux6838ux4e0eux9ad8ux65afux6838ux7684ux5377ux79ef}{%
\paragraph{高斯核与高斯核的卷积}\label{ux9ad8ux65afux6838ux4e0eux9ad8ux65afux6838ux7684ux5377ux79ef}}

\begin{figure}
\centering
\includegraphics{C:/Users/admin/Documents/GitHub/CVPR_homeworks/code/hw1/note.figure/高斯模糊效果.png}
\caption{}
\end{figure}

\begin{figure}
\centering
\includegraphics{C:/Users/admin/Documents/GitHub/CVPR_homeworks/code/hw1/note.figure/高斯核卷积.png}
\caption{}
\end{figure}

取最后两张图片的中心 \(5\times 5\)
大小做差取绝对值,可以看出两者差值非常小,可近似相等,说明两个标准差为
\(\sigma_1,\sigma_2\) 的卷积核,通过卷积可得到
\(\sigma = \sqrt{\sigma_1^2+\sigma_2^2}\) 的卷积核:

\begin{figure}
\centering
\includegraphics{C:/Users/admin/Documents/GitHub/CVPR_homeworks/code/hw1/note.figure/两个Gauss核卷积差值.png}
\caption{}
\end{figure}

\hypertarget{ux9ad8ux65afux6838ux53efux5206ux79bbux5b9eux9a8c}{%
\paragraph{高斯核可分离实验}\label{ux9ad8ux65afux6838ux53efux5206ux79bbux5b9eux9a8c}}

标准差为 \(\sigma\) 的高斯核可分解为两个标准差为 \(\sigma\)
一维高斯核的卷积.

\begin{figure}
\centering
\includegraphics{C:/Users/admin/Documents/GitHub/CVPR_homeworks/code/hw1/note.figure/高斯核分离.png}
\caption{}
\end{figure}

\hypertarget{ux9ad8ux65afux6838ux4e4bux5deedog}{%
\paragraph{高斯核之差DOG}\label{ux9ad8ux65afux6838ux4e4bux5deedog}}

\begin{figure}
\centering
\includegraphics{C:/Users/admin/Documents/GitHub/CVPR_homeworks/code/hw1/note.figure/DOG1.png}
\caption{}
\end{figure}

\begin{figure}
\centering
\includegraphics{C:/Users/admin/Documents/GitHub/CVPR_homeworks/code/hw1/note.figure/DOG2.png}
\caption{}
\end{figure}

\hypertarget{ux56feux50cfux9510ux5316ux6ee4ux6ce2ux5668}{%
\paragraph{图像锐化滤波器}\label{ux56feux50cfux9510ux5316ux6ee4ux6ce2ux5668}}

\begin{align*}
f_{sharp} =&\ f+\alpha(f-f_{blur})\\
=&\ (1+\alpha)I*f-\alpha G_{\sigma}*f\\
=&\ ((1+\alpha)I-\alpha G_\sigma)*f
\end{align*}

所以 \((1+\alpha)I-\alpha G_\sigma\) 为锐化的核,\(I\)
为全通滤波器(输入与输出图像相同),例如 \(3\times 3\) 的如下

\[I = \left[\begin{matrix}0&0&0\\0&1&0\\0&0&0\end{matrix}\right]\]

在锐化滤波器操作中,由于做差后的图像正负值差距较大,如果直接使用线性正规化方法处理溢出部分,即:

\[f(m,n)\leftarrow \frac{f(m,n)-\min(f)}{max(f)-min(f)}\]

由于存在部分像素值较大,而低像素值的像素偏多,所以会导致整体色彩偏低.
为改进图像的亮度,使用与 \(1\) 做最大值截断的方法进行正规化处理:

\[f(m,n)\leftarrow \begin{cases}
1,& \quad f(m,n)-\min(f)>1,\\
f(m,n-\min(f)),& \quad\texttt{otherwise}.
\end{cases}\]

\begin{figure}
\centering
\includegraphics{C:/Users/admin/Documents/GitHub/CVPR_homeworks/code/hw1/note.figure/线性正规化与截断正规化.png}
\caption{}
\end{figure}

执行相减后图像的亮度可能非常低,需要手动提高亮度,我将调节后的亮度均值维持在
\(0.5\) 附近,这个调节亮度的操作就类似于锐化滤波器中的
\(\alpha\),\(\alpha\) 越大则最终图像的亮度越高.

\begin{figure}
\centering
\includegraphics{C:/Users/admin/Documents/GitHub/CVPR_homeworks/code/hw1/note.figure/锐化sigma=3.png}
\caption{}
\end{figure}

\begin{figure}
\centering
\includegraphics{C:/Users/admin/Documents/GitHub/CVPR_homeworks/code/hw1/note.figure/锐化sigma=6.png}
\caption{}
\end{figure}

\begin{figure}
\centering
\includegraphics{C:/Users/admin/Documents/GitHub/CVPR_homeworks/code/hw1/note.figure/锐化sigma=10.png}
\caption{}
\end{figure}

可以看出 \(\alpha=1,\sigma=6\) 时锐化效果较好.

\hypertarget{ux53ccux8fb9ux6ee4ux6ce2}{%
\paragraph{双边滤波}\label{ux53ccux8fb9ux6ee4ux6ce2}}

普通的高斯模糊只用到了图像\textbf{像素之间的距离关系}(空域),对每个像素使用相同的模糊处理,所以在图像的边缘部分处理效果不好.
而双边滤波器通过加入\textbf{像素值之间的关系}(值域),从而能较好的对边界部分进行处理.

记图像为 \(I\),\(\boldsymbol{p},\boldsymbol{q}\)
为像素点对应的向量,\(I_{\boldsymbol{p}}\) 表示图像中 \(\boldsymbol{p}\)
点对应的像素值,\(S\) 为滤波器的向量空间,\(G_\sigma\) 表示标准差为
\(\sigma\) 的高斯函数,\(||\cdot||\) 表示2-范数,\(W_p\)
表示对滤波器进行归一化处理(保持前后图像亮度一致),则双边滤波器为

\begin{aligned}
BF[I]_{\boldsymbol{p}} =&\ \frac{1}{W_\boldsymbol{p}}\sum_{q\in S}G_{\sigma_1}(||\boldsymbol{p}-\boldsymbol{q}||)G_{\sigma_2}(|I_{\boldsymbol{p}}-I_{\boldsymbol{q}}|)I_{\boldsymbol{q}}\\
W_\boldsymbol{p}=&\ \sum_{q\in S}G_{\sigma_1}(||\boldsymbol{p}-\boldsymbol{q}||)G_{\sigma_2}(|I_{\boldsymbol{p}}-I_{\boldsymbol{q}}|)\quad(\text{归一化常数})
\end{aligned}

由上述公式可知,\(G_{\sigma_1}(||\boldsymbol{p}-\boldsymbol{q}||)G_{\sigma_2}(|I_{\boldsymbol{p}}-I_{\boldsymbol{q}}|)\)
为点 \(\boldsymbol{p}\) 处的双边滤波器核,而
\(G_{\sigma_1}(||\boldsymbol{p}-\boldsymbol{q}||)\)
就是高斯核,\(G_{\sigma_2}(|I_{\boldsymbol{p}}-I_{\boldsymbol{q}}|)\)
是值域之差作用高斯函数后的核,两个核做内积即得到在点 \(\boldsymbol{p}\)
处的双边滤波器核,然后进行归一化处理.

\begin{figure}
\centering
\includegraphics{C:/Users/admin/Documents/GitHub/CVPR_homeworks/code/hw1/note.figure/局部滤波效果1.png}
\caption{}
\end{figure}

\begin{figure}
\centering
\includegraphics{C:/Users/admin/Documents/GitHub/CVPR_homeworks/code/hw1/note.figure/局部滤波效果2.png}
\caption{}
\end{figure}

\begin{figure}
\centering
\includegraphics{C:/Users/admin/Documents/GitHub/CVPR_homeworks/code/hw1/note.figure/双通道不同效果-黄鹤楼.png}
\caption{}
\end{figure}

\begin{figure}
\centering
\includegraphics{C:/Users/admin/Documents/GitHub/CVPR_homeworks/code/hw1/note.figure/双边滤波-噪声.png}
\caption{}
\end{figure}

\begin{figure}
\centering
\includegraphics{C:/Users/admin/Documents/GitHub/CVPR_homeworks/code/hw1/note.figure/双通道-人脸1.png}
\caption{}
\end{figure}

\begin{figure}
\centering
\includegraphics{C:/Users/admin/Documents/GitHub/CVPR_homeworks/code/hw1/note.figure/双通道-人物2.png}
\caption{}
\end{figure}

\begin{figure}
\centering
\includegraphics{C:/Users/admin/Documents/GitHub/CVPR_homeworks/code/hw1/note.figure/双通道-建筑1.png}
\caption{}
\end{figure}

\hypertarget{ux5085ux91ccux53f6ux53d8ux6362}{%
\subsubsection{傅里叶变换}\label{ux5085ux91ccux53f6ux53d8ux6362}}

一般的二维傅里叶变换公式为

\[\hat{f}(u,v) = \iint_{\R^2}f(x, y)e^{-2\pi i(\frac{ux}{M}+\frac{vy}{N})}\,dxdy\]

可以形象的理解为将图像 \(f(x,y)\)
向\textbf{不同平面不同方向}的复平面波做内积,也即是求在
\(e^{-2\pi i(ux+vy)}\) 上的投影.

\hypertarget{ux53efux89c6ux5316}{%
\paragraph{可视化}\label{ux53efux89c6ux5316}}

幅度谱和相位谱

注:幅度谱输出需要用线性变换将像素压缩到 \([0,1]\) 中.

\begin{figure}
\centering
\includegraphics{C:/Users/admin/Documents/GitHub/CVPR_homeworks/code/hw1/note.figure/幅度谱与相位谱.png}
\caption{}
\end{figure}

我们将二维傅里叶变化后的空间称为 \(K\) 空间,则 \(K\)
空间中每一个像素表示一种二维正弦波,则 \(K_{u,v}\)
处的夹角表示该种正弦波的相位大小,\(|K_{u,v}|\) 表示该正弦波的幅度大小.

如果 \(K\)
空间的相位全部等于0,那么相当于平面波在相加的时候,都没有移动,所以图像一定会呈现一种周期性,而且中间的点一点很亮.
这是因为复平面波没有移动,那么所有的平面波在中心点相位为0,\(\exp(0)=1\),因此相当于所有的幅度叠加在一起了.

如果使得 \(K\)
空间的幅度全部等于1,那么相当于平面波在相加的时候,只有移动,而没有了各个波的大小信息,低频成分和高频成分全都一样了,所以图像应该比较嘈杂,变化比较剧烈,但是能看见大体的轮廓.
(由于相位图逆变换结果只有少数像素值,只能将像素值相对高一些的提高亮度,否则特征十分不明显)

\begin{figure}
\centering
\includegraphics{C:/Users/admin/Documents/GitHub/CVPR_homeworks/code/hw1/note.figure/幅度图和相位图做逆变换.png}
\caption{}
\end{figure}

\begin{figure}
\centering
\includegraphics{C:/Users/admin/Documents/GitHub/CVPR_homeworks/code/hw1/note.figure/交换相位图与幅度图的效果.png}
\caption{}
\end{figure}

\hypertarget{ux9891ux57dfux6ee4ux6ce2}{%
\paragraph{频域滤波}\label{ux9891ux57dfux6ee4ux6ce2}}

⾼斯滤波器进⾏图像的频率域滤波

我们分别取 \(K\)
空间中低频部分(也就是靠中间的部分)和高频部分(也就是靠近边缘的部分)分别进行傅里叶逆变换.

\begin{figure}
\centering
\includegraphics{C:/Users/admin/Documents/GitHub/CVPR_homeworks/code/hw1/note.figure/低频域与高频域分离.png}
\caption{}
\end{figure}

\begin{figure}
\centering
\includegraphics{C:/Users/admin/Documents/GitHub/CVPR_homeworks/code/hw1/note.figure/gauss处理频域1.png}
\caption{}
\end{figure}

\begin{figure}
\centering
\includegraphics{C:/Users/admin/Documents/GitHub/CVPR_homeworks/code/hw1/note.figure/gauss处理频域2.png}
\caption{}
\end{figure}

\end{document}
