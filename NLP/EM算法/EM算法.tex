\documentclass[UTF8]{ctexbeamer}
\usepackage{amsmath, amsthm, amssymb, bm, color, graphicx, geometry, mathrsfs,extarrows, braket, booktabs, array, xcolor, fontspec, appendix, float, subfigure, wrapfig}
\usetheme{Madrid}  % 使用主题颜色

%%%% 设置中文字体 %%%%
\setCJKmainfont{方正新书宋_GBK.ttf}[ BoldFont = 方正小标宋_GBK, ItalicFont = 方正楷体_GBK]
%%%% 设置英文字体 %%%%
\setmainfont{Times New Roman}
\setsansfont{Calibri}
\setmonofont{Consolas}

%%%% 设置代码块 %%%%
% 在vscode中使用minted需要先配置python解释器, Ctrl+Shift+P, 输入Python: Select Interpreter选择安装了Pygments的Python版本. 再在setting.json中xelatex和pdflatex的参数中加入 "--shell-escape", 即可
% TeXworks中配置方法参考: https://blog.csdn.net/RobertChenGuangzhi/article/details/108140093
\usepackage{minted}
\renewcommand{\theFancyVerbLine}{
    \sffamily\textcolor[rgb]{0.5,0.5,0.5}{\scriptsize\arabic{FancyVerbLine}}} % 修改代码前序号大小
\newmintinline{cpp}{fontsize=\small, linenos, breaklines, frame=lines}  % 使用\cppinline{代码}
\newminted{cpp}{fontsize=\small, linenos, breaklines, frame=lines}  % 使用\begin{cppcode}代码\end{cppcode}
\newmintedfile{cpp}{fontsize=\small, linenos, breaklines, frame=lines}  % 使用\pythonfile{代码地址}
\newmintinline{python}{fontsize=\small, linenos, breaklines, frame=lines, python3}  % 使用\pythoninline{代码}
\newminted{python}{fontsize=\small, linenos, breaklines, frame=lines, python3}  % 使用\begin{pythoncode}代码\end{pythoncode}
\newmintedfile{python}{fontsize=\small, linenos, breaklines, frame=lines, python3}  % 使用\pythonfile{代码地址}

%%%% 公式编号 %%%%
\setbeamertemplate{theorems}[numbered]

%%%% 图片相对路径 %%%%
\graphicspath{{figure/}} % 当前目录下的figure文件夹, {../figure/}则是父目录的figure文件夹
\setlength{\abovecaptionskip}{-0.2cm}  % 缩紧图片标题与图片之间的距离
\setlength{\belowcaptionskip}{0pt} 

\everymath{\displaystyle} % 默认全部行间公式, 想要变回行内公式使用\textstyle
\DeclareMathOperator*\uplim{\overline{lim}}     % 定义上极限 \uplim_{}
\DeclareMathOperator*\lowlim{\underline{lim}}   % 定义下极限 \lowlim_{}
\DeclareMathOperator*{\argmax}{arg\,max}  % 定义取最大值的参数 \argmax_{}
\DeclareMathOperator*{\argmin}{arg\,min}  % 定义取最小值的参数 \argmin_{}
\let\leq=\leqslant % 简写小于等于\leq (将全部leq变为leqslant)
\let\geq=\geqslant % 简写大于等于\geq (将全部geq变为geqslant)

%%%% 定理类环境的定义 %%%%
\newtheorem{remark}{注}
\newtheorem{condition}{条件}
\newtheorem{conclusion}{结论}
\newtheorem{assumption}{假设}
\numberwithin{equation}{section}  % 公式按section编号 (公式右端的小括号)
\newtheorem{algorithm}{算法}

%%%% 一些宏定义 %%%%
\def\bd{\boldsymbol}        % 加粗(向量) boldsymbol
\def\disp{\displaystyle}    % 使用行间公式 displaystyle(默认)
\def\tsty{\textstyle}       % 使用行内公式 textstyle
\def\sign{\text{sign}}      % sign function
\def\wtd{\widetilde}        % 宽波浪线 widetilde
\def\R{\mathbb{R}}          % Real number
\def\N{\mathbb{N}}          % Natural number
\def\Z{\mathbb{Z}}          % Integer number
\def\Q{\mathbb{Q}}          % Rational number
\def\C{\mathbb{C}}          % Complex number
\def\K{\mathbb{K}}          % Number Field
\def\P{\mathbb{P}}          % Polynomial
\def\N{\mathbb{N}}          % Natural number
\def\Z{\mathbb{Z}}          % Integer number
\def\E{\mathbb{E}}          % Exception
\def\var{\text{Var}}        % Variance
\def\bias{\text{bias}}      % bias
\def\d{\mathrm{d}}          % differential operator
\def\e{\mathrm{e}}          % Euler's number
\def\i{\mathrm{i}}          % imaginary number
\def\re{\mathrm{Re}}        % Real part
\def\im{\mathrm{Im}}        % Imaginary part
\def\res{\mathrm{Res}}      % Residue
\def\L{\mathcal{L}}         % Loss function
\def\wdh{\widehat}          % 宽帽子 widehat
\def\ol{\overline}          % 上横线 overline
\def\ul{\underline}         % 下横线 underline
\def\add{\vspace{1ex}}      % 增加行间距
\def\del{\vspace{-1.5ex}}   % 减少行间距
\def\red{\color{red}}       % 红色({\red 公式},用大括号包起来)

% 首页配置
\title{EM算法简介}
\author{吴天阳\and 张卓立}
\institute{XJTU\and 强基数学}
\date{\today}  % 显示日期
\titlegraphic{\hspace*{10cm}\includegraphics[height=1.5cm]{校徽蓝色.png}}

%%%% 正文开始 %%%%
\begin{document}
\frame{\titlepage}  % 首页

\begin{frame}
    \frametitle{目录}
    \tableofcontents
\end{frame}

\section{前置知识}
\begin{frame}
    \frametitle{前置知识}

    \begin{definition}[极大似然估计]
        设$X_1,\cdots,X_n$是来自密度函数为$f(x;\theta)$的独立随机样本,称$\L(\theta) = f(x_1,\cdots, x_n;\theta) = \prod_{i=1}^nf(x_i;\theta)$为关于$\theta$的\alert{似然函数},则$\theta$的极大似然估计为
        \begin{equation*}
            \hat{\theta} = \argmax_{\theta\in \Theta} L(\theta)
        \end{equation*}
        其中$\Theta$为参数空间.
    \end{definition}\pause
    \begin{remark}
        (1) $\theta$可以是一个参数向量$(\theta_1,\cdots, \theta_n)$,包含多个参数.

        (2) 计算时一般采用\alert{对数似然函数}$l(\theta) = \log\L(\theta) = \sum_{i=1}^n\log f(x_i;\theta)$.
    \end{remark}
\end{frame}

\begin{frame}
    \frametitle{前置知识}

    \begin{theorem}[Jensen不等式]
        设$f(x)$是$\R$上的实值凸函数,$X$为随机变量,若$\E X$存在,则
        \begin{equation*}
            \E[f(X)]\geq f(\E X)
        \end{equation*}
        上式取到等号,当且仅当,$X$为常量.
    \end{theorem}\pause
    \begin{remark}
        (1) 当$f(x)$为凹函数时,上述不等号反向,即$\E[f(X)]\leq f(\E X)$.
    \end{remark}
\end{frame}

\section{EM算法详解}
\subsection{问题引入}
\begin{frame}
    \frametitle{问题引入}

    假设有A和B两个硬币,每轮选择一个硬币,进行$N$次投掷得到一个样本集,包含$N$个样本$X_1,\cdots, X_n$,$X_i$表示每次的投掷结果(正面或反面),设硬币A,B分别服从不同的Bernoulli分布,记为$B(1,p_1),\ B(1,p_2)$. 则样本中每个样本$X_1,\cdots,X_n$来自这两个分布中的一个,但无法确定具体是哪一个分布,即不知道是投掷硬币A还是硬币B得到的$x_i$. 
    \begin{itemize}
        \item<2-> 可以将该问题分为以下两个:

        \begin{enumerate}
            \item 这轮投掷是使用硬币A还是硬币B?
            \item 硬币A正面的概率$p_1$和硬币B正面的概率$p_2$分别是多少?
        \end{enumerate}
        \item<3->已知:
        \begin{enumerate}
            \item 模型的分布(均满足二项分布)
            \item 观察到的样本(投掷结果)
        \end{enumerate}
        \item<3-> 未知:
        \begin{enumerate}
            \item 每个个体来自于哪个分布(投掷A还是B)
            \item 模型参数($p_1,p_2$)
        \end{enumerate}
    \end{itemize}
\end{frame}

\subsection{算法思路}
\begin{frame}
    \frametitle{算法思路}

    \begin{itemize}
        \item<1-> 通过引入隐变量$Z = (Z_1,\cdots ,Z_n)$来将描述未被观测到的隐含数据,表示上文中每个个体来自于哪个分布,是由隐变量$Z$控制的.
        \item<2-> 举一个例子:我们假设还有第三个硬币C,也服从Bernoulli分布$B(1,p_3)$,每次试验时,先投掷硬币C,若硬币C为正面,则投掷硬币A,反之,则投掷硬币B. 于是,隐变量就是硬币C的投掷结果,记为随机变量$Z$,且$Z\sim B(1,p_3)$.
        \item<3-> 通过这个例子,我们可以假设隐变量$Z_1,\cdots, Z_n$是来自密度函数为$g(z)$的分布.
        \item<3-> 由于隐变量是人为假定的,我们假设\textbf{隐变量独立同分布于某一种分布},从而能进一步对其研究. 在实际应用中,试验数据应该满足这一假设,才能使用EM算法.
        \item<3-> 这一假设分布也称为先验分布.
    \end{itemize}

\end{frame}

\subsection{算法大致流程}
\begin{frame}
    \frametitle{算法大致流程}
    以上文抛硬币问题为例
    \begin{enumerate}[<+->]
        \item 初始化参数:投掷硬币A,B正面的概率分别为$p_1,p_2$.
        \item 计算每个样本是来自于哪个分布:由A投掷出来的概率大,还是由B投掷出来的概率大.
        \item 重新估计参数:通过每个样本属于A的概率,从而得到硬币A正面的期望次数和反面的期望次数,通过极大似然得到对$p_1$的估计. 同理可以得到$p_2$的估计. 对它们进行更新.
        \item 若参数变化小于阈值,退出循环;否则返回步骤2.
    \end{enumerate}
    \uncover<5->{
    上述问题算法步骤2最难得到,这里就需要通过引入隐变量方法对其计算(利用Bayes公式).
    }
\end{frame}

\subsection{EM算法推导}
\begin{frame}
    \frametitle{EM算法推导}
    我们直接从最大似然估计开始推导
    \begin{align}
        \label{eq-1}\log \L(\theta) =&\ \sum_{i=1}^n\log f(x_i;\theta)\ {\red=} \sum_{i=1}^n\log\sum_{z_i}f(x_i,z_i;\theta)\\
        \nonumber =&\ \sum_{i=1}^n\log\sum_{z_i}g(z_i)\frac{f(x_i,z_i;\theta)}{g(z_i)}\\
        \label{eq-2}{\red\geq}&\ \sum_{i=1}^n\sum_{z_i}g(z_i)\log\frac{f(x_i,z_i;\theta)}{g(z_i)}
    \end{align}
    \pause
    \begin{itemize}[<+->]
        \item (\ref{eq-1})处引入了$z_i$,从而将边缘分布转化为联合分布的形式.
        \item (\ref{eq-2})处,由于$\log(x)$为凹函数,且$\sum_{z_i}g(z_i)\frac{f(x_i,z_i;\theta)}{g(z_i)}=\E\left[\frac{f(x_i,z_i;\theta)}{g(z_i)}\right]$,由Jensen不等式可得.
    \end{itemize}
\end{frame}

\begin{frame}
    \frametitle{EM算法推导}

    \begin{equation*}
        \log \L(\theta) \geq \sum_{i=1}^n\sum_{z_i}g(z_i)\log\frac{f(x_i,z_i;\theta)}{g(z_i)}
    \end{equation*}
    于是我们得到了对数似然函数的一个下界,这个下界是由期望$\E\left[\log\frac{f(x_i,z_i;\theta)}{g(z_i)}\right]$构成,这里就是EM算法中Exception部分. 我们希望将最大化问题转化为对右式最大化,也就是EM中的Maximum部分,先将$\theta$固定(使用上一次迭代的$\theta_{t}$),考虑当$g(z_i)$满足什么条件时,Jensen不等式取到等号.\pause
    \begin{align*}
        &\ \frac{f(x_i,z_i;\theta)}{g(z_i)} = c\Rightarrow f(x_i,z_i;\theta) = cg(z_i)\\
        \text{(对}z_i\text{求和)}\Rightarrow&\ f(x_i;\theta) =\sum_{z_i}f(x_i,z_i;\theta) = c\sum_{z_i}g(z_i) = c.
    \end{align*}
    于是$g(z_i) = \frac{f(x_i,z_i;\theta)}{c} = \frac{f(x_i,z_i;\theta)}{f(x_i;\theta)}=f(z_i|x_i;\theta)$.

\end{frame}

\section{EM算法流程}
\begin{frame}
    \frametitle{EM算法流程}

    \begin{enumerate}
        \item<1-> 初始化参数$\theta_0$.
        \item<2-> \begin{itemize}
            \item E步:计算条件概率期望$g_i(z_i) = f(z_i|x_i;\theta_t)$,\\对数似然函数$\log\L(\theta) =\sum_{i=1}^n\sum_{z_i}g_i(z_i)\log \frac{f(x_i,z_i;\theta)}{{\red g_i(z_i)}}$
            \item M步:极大化对数似然函数,更新参数$\theta$:
            \begin{equation*}
                \theta_{t+1} = \argmax_{\theta\in \Theta}\log\L(\theta)=\sum_{i=1}^n\sum_{z_i}g_i(z_i)\log f(x_i,z_i;\theta)
            \end{equation*}
        \end{itemize}
        \item<3-> 返回第2步,直到$\theta_{t+1}$收敛.
    \end{enumerate}

    \begin{itemize}
        \item<4-> M步中,由于对数似然函数中$g_i(z_i)$与$\theta$无关,所以可以去掉.
    \end{itemize}

\end{frame}

\definecolor{bg}{rgb}{0.95,0.95,0.95}

\defverbatim[colored]\exampleCode{
\begin{pythoncode}
prA, prB = 0.3, 0.7  # 初始化参数, 硬币A,B正面朝上的概率
samples = [4, 6, 0, 9, 5]  # 每个样本中正面朝上的个数
for _ in range(10):
    expectA, expectB = np.zeros(2), np.zeros(2)  # 硬币A,B的期望
    for i in range(len(samples)):
        tmp1 = np.power(prA, samples[i]) * np.power(1 - prA, 10 - samples[i])
        tmp2 = np.power(prB, samples[i]) * np.power(1 - prB, 10 - samples[i])
        chooseA = tmp1 / (tmp1 + tmp2)  # 选择硬币A的概率
        chooseB = 1 - chooseA  # 选择硬币B的概率
        expectA += np.array([samples[i] * chooseA, (10 - samples[i]) * chooseA])  # E步
        expectB += np.array([samples[i] * chooseB, (10 - samples[i]) * chooseB])
    prA = expectA[0] / np.sum(expectA)  # M步
    prB = expectB[0] / np.sum(expectB)
\end{pythoncode}
}

\subsection{一个例子}
\begin{frame}
    \frametitle{一个例子}
    \exampleCode
\end{frame}

\begin{frame}
    \frametitle{迭代结果}
\begin{table}[htbp]
    \zihao{6}
    \centering
\begin{tabular}{cllllll}
        \toprule
\textbf{迭代次数} & \multicolumn{1}{c}{\textbf{A正面期望}} & \multicolumn{1}{c}{\textbf{A背面期望}} & \multicolumn{1}{c}{\textbf{B正面期望}} & \multicolumn{1}{c}{\textbf{B背面期望}} & \multicolumn{1}{c}{\textbf{A正面概率}} & \multicolumn{1}{c}{\textbf{B正面概率}} \\
        \midrule
\textbf{1}    & 6.82                               & 18.19                              & 17.18                              & 7.81                               & 0.27                               & 0.69                               \\
        \midrule
\textbf{2}    & 5.82                               & 17.2                               & 18.18                              & 8.8                                & 0.25                               & 0.67                               \\
        \midrule
\textbf{3}    & 4.99                               & 16.32                              & 19.01                              & 9.68                               & 0.23                               & 0.66                               \\
        \midrule
\textbf{4}    & 4.27                               & 15.53                              & 19.73                              & 10.47                              & 0.22                               & 0.65                               \\
        \midrule
\textbf{5}    & 3.59                               & 14.75                              & 20.41                              & 11.25                              & 0.2                                & 0.64                               \\
        \midrule
\textbf{6}    & 2.92                               & 13.95                              & 21.08                              & 12.05                              & 0.17                               & 0.64                               \\
        \midrule
\textbf{7}    & 2.21                               & 13.04                              & 21.79                              & 12.96                              & 0.14                               & 0.63                               \\
        \midrule
\textbf{8}    & 1.39                               & 11.96                              & 22.61                              & 14.04                              & 0.1                                & 0.62                               \\
        \midrule
\textbf{9}    & 0.52                               & 10.75                              & 23.48                              & 15.25                              & 0.05                               & 0.61                               \\
        \midrule
\textbf{10}   & 0.03                               & 10.04                              & 23.97                              & 15.96                              & 0                                  & 0.6                               \\
        \bottomrule
\end{tabular}
\end{table}


\end{frame}
\end{document}
