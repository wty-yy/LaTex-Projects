\documentclass[12pt, a4paper, oneside]{ctexart}
\usepackage{amsmath, amsthm, amssymb, bm, color, graphicx, geometry, mathrsfs,extarrows, braket, booktabs, array, wrapfig, enumitem, subfigure, bbm}
\usepackage[colorlinks,linkcolor=red,anchorcolor=blue,citecolor=blue,urlcolor=blue,menucolor=black]{hyperref}
%%%% 设置中文字体 %%%%
% fc-list -f "%{family}\n" :lang=zh >d:zhfont.txt 命令查看已有字体
\setCJKmainfont{方正书宋.ttf}[BoldFont = 方正黑体_GBK.ttf, ItalicFont = simkai.ttf, BoldItalicFont = 方正粗楷简体.ttf]
%%%% 设置英文字体 %%%%
\setmainfont{Times New Roman}
\setsansfont{Calibri}
\setmonofont{Consolas}

%%%% 设置代码块 %%%%
% 在vscode中使用minted需要先配置python解释器, Ctrl+Shift+P, 输入Python: Select Interpreter选择安装了Pygments的Python版本. 再在setting.json中xelatex和pdflatex的参数中加入 "--shell-escape", 即可
% TeXworks中配置方法参考: https://blog.csdn.net/RobertChenGuangzhi/article/details/108140093
\usepackage{minted}
\renewcommand{\theFancyVerbLine}{
    \sffamily\textcolor[rgb]{0.5,0.5,0.5}{\scriptsize\arabic{FancyVerbLine}}} % 修改代码前序号大小
% 加入不同语言的代码块
\newmintinline{cpp}{fontsize=\small, linenos, breaklines, frame=lines}
\newminted{cpp}{fontsize=\small, baselinestretch=1, linenos, breaklines, frame=lines}
\newmintedfile{cpp}{fontsize=\small, baselinestretch=1, linenos, breaklines, frame=lines}
\newminted{r}{fontsize=\small, baselinestretch=1, linenos, breaklines, frame=lines}
\newmintinline{matlab}{fontsize=\small, linenos, breaklines, frame=lines}
\newminted{matlab}{fontsize=\small, baselinestretch=1, mathescape, linenos, breaklines, frame=lines}
\newmintedfile{matlab}{fontsize=\small, baselinestretch=1, linenos, breaklines, frame=lines}
\newmintinline{python}{fontsize=\small, linenos, breaklines, frame=lines, python3}  % 使用\pythoninline{代码}
\newminted{python}{fontsize=\small, baselinestretch=1, linenos, breaklines, frame=lines, python3}  % 使用\begin{pythoncode}代码\end{pythoncode}
\newmintedfile{python}{fontsize=\small, baselinestretch=1, linenos, breaklines, frame=lines, python3}  % 使用\pythonfile{代码地址}

%%%% 设置行间距与页边距 %%%%
\linespread{1.4}
%\geometry{left=2.54cm,right=2.54cm,top=3.18cm,bottom=3.18cm}
\geometry{left=1.84cm,right=1.84cm,top=2.18cm,bottom=2.18cm}

%%%% 图片相对路径 %%%%
\graphicspath{{figures/}} % 当前目录下的figures文件夹, {../figures/}则是父目录的figures文件夹
\setlength{\abovecaptionskip}{-0.2cm}  % 缩紧图片标题与图片之间的距离
\setlength{\belowcaptionskip}{0pt} 

%%%% 缩小item,enumerate,description两行间间距 %%%%
\setenumerate[1]{itemsep=0pt,partopsep=0pt,parsep=\parskip,topsep=5pt}
\setitemize[1]{itemsep=0pt,partopsep=0pt,parsep=\parskip,topsep=5pt}
\setdescription{itemsep=0pt,partopsep=0pt,parsep=\parskip,topsep=5pt}

%%%% 自定义公式 %%%%
\everymath{\displaystyle} % 默认全部行间公式
\DeclareMathOperator*\uplim{\overline{lim}} % 定义上极限 \uplim_{}
\DeclareMathOperator*\lowlim{\underline{lim}} % 定义下极限 \lowlim_{}
\DeclareMathOperator*{\argmax}{arg\,max}  % 定义取最大值的参数 \argmax_{}
\DeclareMathOperator*{\argmin}{arg\,min}  % 定义取最小值的参数 \argmin_{}
\let\leq=\leqslant % 将全部leq变为leqslant
\let\geq=\geqslant % geq同理
\DeclareRobustCommand{\rchi}{{\mathpalette\irchi\relax}}
\newcommand{\irchi}[2]{\raisebox{\depth}{$#1\chi$}} % 使用\rchi将\chi居中

%%%% 自定义环境配置 %%%%
\newcounter{problem}  % 问题序号计数器
\newenvironment{problem}[1][]{\stepcounter{problem}\par\noindent\textbf{题目\arabic{problem}. #1}}{\smallskip\par}
\newenvironment{solution}[1][]{\par\noindent\textbf{#1解答. }}{\smallskip\par}  % 可带一个参数表示题号\begin{solution}{题号}
\newenvironment{note}{\par\noindent\textbf{注记. }}{\smallskip\par}
\newenvironment{remark}{\begin{enumerate}[label=\textbf{注\arabic*.}]}{\end{enumerate}}
\BeforeBeginEnvironment{minted}{\vspace{-0.5cm}}  % 缩小minted环境距上文间距
\AfterEndEnvironment{minted}{\vspace{-0.2cm}}  % 缩小minted环境距下文间距

%%%% 一些宏定义 %%%%
\def\bd{\boldsymbol}        % 加粗(向量) boldsymbol
\def\disp{\displaystyle}    % 使用行间公式 displaystyle(默认)
\def\weekto{\rightharpoonup}% 右半箭头
\def\tsty{\textstyle}       % 使用行内公式 textstyle
\def\sign{\textrm{sign}}    % sign function
\def\cov{\textrm{Cov}}      % 
\def\var{\textrm{Var}}      % 
\def\E{\textrm{E}}          % 
\def\1{\bd{1}}
\def\wtd{\widetilde}        % 宽波浪线 widetilde
\def\R{\mathbb{R}}          % Real number
\def\N{\mathbb{N}}          % Natural number
\def\Z{\mathbb{Z}}          % Integer number
\def\Q{\mathbb{Q}}          % Rational number
\def\C{\mathbb{C}}          % Complex number
\def\K{\mathbb{K}}          % Number Field
\def\P{\textrm{P}}          % Possibility
\def\d{\mathrm{d}}          % differential operator
\def\e{\mathrm{e}}          % Euler's number
\def\i{\mathrm{i}}          % imaginary number
\def\re{\mathrm{Re}}        % Real part
\def\im{\mathrm{Im}}        % Imaginary part
\def\res{\mathrm{Res}}      % Residue
\def\ker{\mathrm{Ker}}      % Kernel
\def\vspan{\mathrm{vspan}}  % Span  \span与latex内核代码冲突改为\vspan
\def\L{\mathcal{L}}         % Loss function
\def\O{\mathcal{O}}         % big O notation
\def\wdh{\widehat}          % 宽帽子 widehat
\def\ol{\overline}          % 上横线 overline
\def\ul{\underline}         % 下横线 underline
\def\add{\vspace{1ex}}      % 增加行间距
\def\del{\vspace{-1.5ex}}   % 减少行间距

%%%% 定理类环境的定义 %%%%
\newtheorem{theorem}{定理}

%%%% 基本信息 %%%%
\newcommand{\RQ}{\today} % 日期
\newcommand{\km}{数据分析} % 科目
\newcommand{\bj}{强基数学002} % 班级
\newcommand{\xm}{吴天阳} % 姓名
\newcommand{\xh}{2204210460} % 学号

\begin{document}

%\pagestyle{empty}
\pagestyle{plain}
\vspace*{-15ex}
\centerline{\begin{tabular}{*5{c}}
    \parbox[t]{0.25\linewidth}{\begin{center}\textbf{日期}\\ \large \textcolor{blue}{\RQ}\end{center}} 
    & \parbox[t]{0.2\linewidth}{\begin{center}\textbf{科目}\\ \large \textcolor{blue}{\km}\end{center}}
    & \parbox[t]{0.2\linewidth}{\begin{center}\textbf{班级}\\ \large \textcolor{blue}{\bj}\end{center}}
    & \parbox[t]{0.1\linewidth}{\begin{center}\textbf{姓名}\\ \large \textcolor{blue}{\xm}\end{center}}
    & \parbox[t]{0.15\linewidth}{\begin{center}\textbf{学号}\\ \large \textcolor{blue}{\xh}\end{center}} \\ \hline
\end{tabular}}
\begin{center}
    \zihao{3}\textbf{第二次作业}
\end{center}\vspace{-0.2cm}
\begin{problem}
设$X^{(1)}$和$X^{(2)}$均为$p$维随机向量,已知
\begin{equation*}
    X = \begin{bmatrix}
        X^{(1)}\\X^{(2)}
    \end{bmatrix}\sim
    N_{2p}\left(\begin{bmatrix}
        \mu^{(1)}\\\mu^{(2)}
    \end{bmatrix},\begin{bmatrix}
        \Sigma_1&\Sigma_2\\
        \Sigma_2&\Sigma_1
    \end{bmatrix}
    \right)
\end{equation*}
其中$\mu^{(i)}(i=1,2)$为$p$维向量,$\Sigma_i(i=1,2)$是$p$阶矩阵.

1. 证明$X^{(1)}+X^{(2)}$和$X^{(1)}-X^{(2)}$相互独立;

2. 求$X^{(1)}+X^{(2)}$和$X^{(1)}-X^{(2)}$的分布.
\end{problem}
\begin{solution}
    1. 设$I_p$为$p$阶单位阵,由于
    \begin{align*}
        \begin{bmatrix}
            X^{(1)}+X^{(2)}\\
            X^{(1)}-X^{(2)}\\
        \end{bmatrix} = \begin{bmatrix}
            I_p&I_p\\ I_p&-I_p
        \end{bmatrix}X\sim N_{2p}\left(
            \begin{bmatrix}
                \mu_1+\mu_2\\\mu_1 - \mu_2
            \end{bmatrix},
            \begin{bmatrix}
                2(\Sigma_1+\Sigma_2)&0\\
                0&2(\Sigma_1-\Sigma_2)
            \end{bmatrix}
        \right)
    \end{align*}
    于是$X^{(1)}+X^{(2)}$与$X^{(1)} - X^{(2)}$独立.

    2. 由上一问可知
    \begin{align*}
        X^{(1)}+X^{(2)} = \begin{bmatrix}
            I_p&I_p
        \end{bmatrix}X &\ \sim N_p(\mu_1+\mu_2,2(\Sigma_1+\Sigma_2)),\\
        X^{(1)}-X^{(2)} = \begin{bmatrix}
            I_p&-I_p
        \end{bmatrix}X &\ \sim N_p(\mu_1-\mu_2,2(\Sigma_1-\Sigma_2)).\\
    \end{align*}
\end{solution}
\vspace{-1.3cm}
\begin{problem}
    设$X\sim N_3(\bd{\mu},\Sigma)$,其中
    \begin{align*}
        \bd{\mu} = (\mu_1,\mu_2,\mu_3)',\quad \Sigma = \begin{bmatrix}
            1&\rho&\rho\\
            \rho&1&\rho\\
            \rho&\rho&1
        \end{bmatrix}\quad(0<\rho<1)
    \end{align*}

    1. 求条件分布$(X_1,X_2|X_3)$和$(X_1|X_2,X_3)$.

    2. 给定$X_3=x_3$时,求出$X_1$和$X_2$的条件协方差.
\end{problem}
\begin{solution}
    1. 由条件分布计算公式可知
    \begin{align*}
        (X_1,X_2|X_3 = x_3)\sim&\ N_2\left(\begin{bmatrix}
            \mu_1\\\mu_2
        \end{bmatrix}+\begin{bmatrix}
            \rho\\\rho
        \end{bmatrix}(x_3-\mu_3)
        ,\begin{bmatrix}
            1&\rho\\\rho&1
        \end{bmatrix} - \begin{bmatrix}
            \rho\\\rho
        \end{bmatrix}\begin{bmatrix}
            \rho&\rho
        \end{bmatrix}\right) \\
        \sim&\ N_2\left(\begin{bmatrix}
            \mu_1+\rho(x_3-\mu_3)\\
            \mu_2+\rho(x_3-\mu_3)\\
        \end{bmatrix},\begin{bmatrix}
            1-\rho^2&\rho-\rho^2\\
            \rho-\rho^2&1-\rho^2
        \end{bmatrix}\right)
    \end{align*}
    \begin{align*}
        (X_1|X_2=x_2,X_3=x_3)\sim&\ N_1\left(
            \mu_1+\begin{bmatrix}
                \rho&\rho
            \end{bmatrix}\begin{bmatrix}
                1&\rho\\\rho&1
            \end{bmatrix}^{-1}\left(\begin{bmatrix}
                x_2\\x_3
            \end{bmatrix}-\begin{bmatrix}
                \mu_2\\\mu_3
            \end{bmatrix}\right), 1-\begin{bmatrix}
                \rho&\rho
            \end{bmatrix}\begin{bmatrix}
                1&\rho\\\rho&1
            \end{bmatrix}^{-1}\begin{bmatrix}
                \rho\\\rho
            \end{bmatrix}\right)\\
            \sim&\ N_1\left(\mu_1+\frac{\rho}{1+\rho}(x_2+x_3-\mu_2-\mu_3), 1-\frac{2\rho^2}{1+\rho}\right)
    \end{align*}
    
    2. 由第一问可知,$X_1,X_2$在给定$X_3=x_3$下的条件协方差均为$1-\rho^2$.
\end{solution}\clearpage
\begin{problem}
    设$X_1\sim N(0,1)$,$X_2 = \begin{cases}
        -X_1,&\quad -1\leq  X_1\leq 1,\\
        X_1,&\text{否则}.
    \end{cases}$
    
    1. 证明:$X_2\sim N(0,1)$.
    
    2. 证明$(X_1,X_2)$的联合分布不是正态分布.
\end{problem}
\begin{proof}
    1. 当$x\in [-1,1]$时,$f_{X_2}(x) = f_{X_1}(-x) = \frac{1}{\sqrt{2\pi}}\e^{-\frac{x^2}{2}}$;当$x\notin[-1,1]$时,
    $f_{X_2}(x) = f_{X_1}(x) = \frac{1}{\sqrt{2\pi}}\e^{-\frac{x^2}{2}}$. 综上$f_{X_1} = f_{X_2}$,所以$X_2\sim N(0,1)$.

    2. 当$x_2\in [-1,1]$时,$X_1,X_2$的联合分布函数满足
    \begin{equation*}
        F_{X_1,X_2}(x_1,x_2) = \P[X_1\leq x_1,X_2\leq x_2] = \P[X_1 \leq x_1,-X_1\leq x_2] = P[-x_2\leq X_1\leq x_1]
    \end{equation*}
    所以$F_{X_1,X_2}(x_1,x_2)$不是正态分布.
\end{proof}
\begin{problem}
    设$X\sim N_p(\mu,\Sigma)$,$A$为对称阵,证明:
    
    (1). $\E(XX') = \Sigma+\mu\mu'$;

    (2). $\E(X'AX) = \text{tr}(\Sigma A) + \mu'A\mu$;

    (3). 当$\mu = a\begin{bmatrix}
        1\\\vdots\\1
    \end{bmatrix}=: a\1_p,\ A = I_p - \frac{1}{p}\1_p\1_p',\ \Sigma = \sigma^2I_p$时,试利用(1)和(2)的结果证明$\E(X'AX) = \sigma^2(p-1)$.

    若记$X = (X_1,\cdots, X_p)'$此时$X'AX = \sum_{i=1}^p(X_i - \bar{X})^2$,则
    \begin{equation*}
        \E\left[\sum_{i=1}^p(X_i-\bar{X})^2\right] = \sigma^2(p-1).
    \end{equation*}
\end{problem}
\begin{solution}
    (1).  \del
    \begin{align*}
        &\ \int_{\R^p}\frac{xx'}{(2\pi)^{\frac{p}{2}}|\Sigma|^{\frac{1}{2}}}\exp\left\{-\frac{1}{2}(x-\mu)'\Sigma^{-1}(x-\mu)\right\}\,\d x\\
        \xlongequal{x\leftarrow x-\mu}&\ \frac{1}{(2\pi)^{\frac{p}{2}}|\Sigma|^{\frac{1}{2}}}\int_{\R^p}(x+\mu)(x+\mu)'\exp\left\{-\frac{1}{2}x'\Sigma^{-1}x\right\}\,\d x\\
        =&\ \frac{1}{(2\pi)^{\frac{p}{2}}|\Sigma|^{\frac{1}{2}}}\int_{\R^p}(xx'+2\mu x'+\mu\mu')\exp\left\{-\frac{1}{2}x'\Sigma^{-1}x\right\}\,\d x\\
        =&\ \mu\mu' - \frac{\Sigma}{(2\pi)^{\frac{p}{2}}|\Sigma|^{\frac{1}{2}}}\int_{\R^p}(x+2\mu)\,\d \exp\left\{-\frac{1}{2}x'\Sigma^{-1}x\right\}\\
        =&\ \mu\mu' + \frac{\Sigma}{(2\pi)^{\frac{p}{2}}|\Sigma|^{\frac{1}{2}}}\int_{\R^p}\exp\left\{-\frac{1}{2}x'\Sigma^{-1}x\right\}\,\d x\\
        =&\ \Sigma+\mu\mu'
    \end{align*} \vspace{0.5cm}

    (2). \del
    \begin{align*}
        &\ \int_{\R^p}\frac{x'Ax}{(2\pi)^{\frac{p}{2}}|\Sigma|^{\frac{1}{2}}}\exp\left\{-\frac{1}{2}(x-\mu)'\Sigma^{-1}(x-\mu)\right\}\,\d x\\
        \xlongequal{x\leftarrow x-\mu}&\ \frac{1}{(2\pi)^{\frac{p}{2}}|\Sigma|^{\frac{1}{2}}}\int_{\R^p}(x+\mu)'A(x+\mu)\exp\left\{-\frac{1}{2}x'\Sigma^{-1}x\right\}\,\d x\\
        \xlongequal{A\text{为对称阵}}&\ \frac{1}{(2\pi)^{\frac{p}{2}}|\Sigma|^{\frac{1}{2}}}\int_{\R^p}(x'Ax+2\mu'Ax+\mu'A\mu)\exp\left\{-\frac{1}{2}x'\Sigma^{-1}x\right\}\,\d x\\
        =&\ \mu'A\mu - \frac{\Sigma A}{(2\pi)^{\frac{p}{2}}|\Sigma|^{\frac{1}{2}}}\int_{\R^p}(x+2\mu)\,\d \exp\left\{-\frac{1}{2}x'\Sigma^{-1}x\right\}\\
        =&\ \mu\mu' + \frac{\Sigma A}{(2\pi)^{\frac{p}{2}}|\Sigma|^{\frac{1}{2}}}\int_{\R^p}\exp\left\{-\frac{1}{2}x'\Sigma^{-1}x\right\}\,\d x\\
        =&\ \text{tr}(\Sigma A) + \mu'A\mu
    \end{align*}

    (3). 由(2)可知:$\E(X'AX) = \text{tr}(\Sigma A) + \mu'A\mu = p\sigma^2(1-1/p) + a^2\begin{bmatrix}
        0&0&\cdots&0
    \end{bmatrix}\1_p = \sigma^2(p-1).$
    由于
    \begin{align*}
        X'AX =&\ \begin{bmatrix}
            X_1,\cdots,X_p
        \end{bmatrix}\begin{bmatrix}
            1-1/p&-1/p&\cdots&-1/p\\
            -1/p&1-1/p&\ddots&\vdots\\
            \vdots&\ddots&\ddots&-1/p\\
            -1/p&\cdots&-1/p&1-1/p
        \end{bmatrix}\begin{bmatrix}
            X_1\\\vdots\\X_p
        \end{bmatrix}\\
        =&\ \begin{bmatrix}
            X_1-\bar{X}&X_2-\bar{X}&\cdots&X_n-\bar{X}
        \end{bmatrix}\begin{bmatrix}
            X_1\\\vdots\\X_p
        \end{bmatrix}\\
        =&\ \sum_{i=1}^pX_i^2 - X_i\bar{X} = \sum_{i=1}^PX_i^2-2X_i\bar{X} + \sum_{i=1}^pX_i\bar{X}\\
        =&\ \sum_{i=1}^p(X_i^2-2X_i\bar{X}+\bar{X}^2) = \sum_{i=1}^p(X_i-\bar{X})^2
    \end{align*}
    故
    \begin{equation*}
        \E\left[\sum_{i=1}^p(X_i-\bar{X})^2\right] = \sigma^2(p-1).
    \end{equation*}
\end{solution}
\begin{problem}[2-3练习1]
    设$X\sim N_{n}(\bd{\mu},\sigma^2I_n)$,$A$为$n$阶对撑幂等矩阵,且$\text{rank}(A) = r(r\leq n)$,证明:
    $\frac{1}{\sigma^2}X'AX\sim \rchi^2(r,\delta)$,其中$\delta = \frac{1}{\sigma^2}\bd{\mu}'A\bd{\mu}$.
\end{problem}
\begin{proof}
    由于$X\sim N_n(\bd{\mu}, \sigma^2I_n)$,则$AX\sim N_n(A\bd{\mu}, \sigma^2A'A) = N_n(A\bd{\mu}, \sigma^2A)$,于是
    \begin{equation*}
        (AX)'(\sigma^2A)^{-1}(AX) = \frac{1}{\sigma^2}X'AX\sim \rchi^2(r, (A\bd{\mu})'(\sigma^2A)^{-1}(A\bd{\mu})) = \rchi^2(r,\frac{1}{\sigma^2}\bd{\mu}'A\bd{\mu})
    \end{equation*}
\end{proof}
\begin{problem}[2-3练习2]
    设$X\sim N_{n}(\bd{\mu},\sigma^2I_n)$,$A,B$为$n$阶对称矩阵,若$AB = O$,证明:$X'AX$与$X'BX$相互独立.
\end{problem}
\begin{proof}
    由于$AB = O$,由高代知识可知$r(AB)\geq r(A) + r(B) - n\Rightarrow r(A)+r(B) \leq n$,令$r(A) = p, r(B) = q$,于是$p + q\leq n$,
    又由于$A,B$为对称阵,\add 则存在正交阵$P,Q$使得$P'AP = \begin{bmatrix}
        A_\lambda&0\\
        0&0
    \end{bmatrix},Q'BQ = \begin{bmatrix}
        B_\lambda&0\\
        0&0
    \end{bmatrix}$\add 其中$A_\lambda =\text{diag}(\lambda_{11},\cdots,\lambda_{1p}),B_\lambda =\text{diag}(\lambda_{21},\cdots,\lambda_{2q})$,其中$\lambda_{1i},\lambda_{2j},\ (1\leq i\leq p,1\leq j\leq q)$分别为$A,B$的特征值.\add

    由于\add $AB = P\begin{bmatrix}
        A_\lambda&0\\
        0&0
    \end{bmatrix}P'Q\begin{bmatrix}
        B_\lambda&0\\
        0&0
    \end{bmatrix}Q'$,令$P'Q = \begin{bmatrix}
        C_{11}&C_{12}\\
        C_{21}&C_{22}
    \end{bmatrix}$其中$C_{11}\in\R^{p\times q},C_{12}\in\R^{p\times (n-q)},C_{21}\in\R^{(n-p)\times q},C_{22}\in\R^{(n-p)\times (n-q)}$,则
    \begin{equation*}
        AB = P\begin{bmatrix}
        A_\lambda&0\\
        0&0
    \end{bmatrix}\begin{bmatrix}
        C_{11}&C_{12}\\
        C_{21}&C_{22}
    \end{bmatrix}\begin{bmatrix}
        B_\lambda&0\\
        0&0
    \end{bmatrix}Q' = P\begin{bmatrix}
        A_{\lambda}C_{11}B_{\lambda}&0\\
        0&0
    \end{bmatrix}Q' = O
    \end{equation*}
    则$C_{11} = 0$. 令$Y_1 = P'X = (Y_{1i})_1^n, Y_2 = Q'X = (Y_{2i})_1^n$,于是
    \begin{align*}
        X'AX =&\ (P'X)'P'AP(P'X) = Y_1'\begin{bmatrix}
        A_\lambda&0\\
        0&0
    \end{bmatrix}Y_1 = \sum_{i=1}^p\lambda_{1i}Y_{1i}^2\\
        X'BX =&\ (Q'X)'Q'AQ(Q'X) = Y_2'\begin{bmatrix}
        B_\lambda&0\\
        0&0
    \end{bmatrix}Y_2 = \sum_{i=1}^q\lambda_{2i}Y_{2i}^2
    \end{align*}
    且\add $Y_1 = P'QY_2 = \begin{bmatrix}
        0&C_{12}\\
        C_{21}&C_{22}
    \end{bmatrix}\begin{bmatrix}
        Z_q\\Z_{n-q}
    \end{bmatrix} = \begin{bmatrix}
        C_{12}Z_{n-q}\\
        C_{21}Z_q+C_{22}Z_{n-q}
    \end{bmatrix}$,其中$Z_q = (Y_{21},\cdots,Y_{2q})',Z_{n-q} = (Y_{2,q+1},\cdots,Y_{2n})'$,
    由于$p\leq n-q$,于是$Y_{11},\cdots,Y_{1p}$是$Y_{2,q+1},\cdots,Y_{2n}$的线性组合,又由于$Y_2 = Q'X = N(Q'\bd{\mu},\sigma^2I_n)$,
    则$Y_{2i}$与$Y_{2j},\ (i\neq j)$独立,所以$\{Y_{11},\cdots,Y_{1p}\}$与$\{Y_{21},\cdots,Y_{2q}\}$独立,故$X'AX$与$X'BX$独立.
\end{proof}
\begin{problem}[2-3练习3]
    设$X\sim N_p(\bd{\mu},\Sigma),\Sigma > 0$,$A,B$为$p$阶对称阵,证明:$(X-\bd{\mu})'A(X-\bd{\mu})$与$(X-\bd{\mu})'B(X-\bd{\mu})$独立,当且仅当,
    $\Sigma A\Sigma B\Sigma = O_{p\times p}$.
\end{problem}
\begin{proof}
    下面证明充分性,令$r(\Sigma) = p$,特征值为$\lambda_1,\cdots,\lambda_p$,由于$\Sigma > 0$,则存在正交阵$P$,使得$P'\Sigma P = \text{diag}(\lambda_1,\cdots,\lambda_p)$.
    令$\Sigma^{\frac{1}{2}} = P\text{diag}(\sqrt{\lambda_1},\cdots,\sqrt{\lambda_2})P'$,则$\Sigma = \Sigma^{\frac{1}{2}}\Sigma^{\frac{1}{2}}$.

    令$Y = \Sigma^{\frac{1}{2}}(X-\mu)$,于是$Y\sim N_p(0, (\Sigma^{\frac{1}{2}})'\Sigma^{\frac{1}{2}}\Sigma^{\frac{1}{2}}) = N_p(0,I_p)$,于是
    \begin{align*}
        &\ (X-\bd{\mu})'A(X-\bd{\mu}) = Y'\Sigma^{\frac{1}{2}}A\Sigma^{\frac{1}{2}}Y = Y'C_1Y\\
        &\ (X-\bd{\mu})'B(X-\bd{\mu}) = Y'\Sigma^{\frac{1}{2}}B\Sigma^{\frac{1}{2}}Y = Y'C_2Y
    \end{align*}
    其中$C_1 = \Sigma^{\frac{1}{2}}A\Sigma^{\frac{1}{2}}, C_2 = \Sigma^{\frac{1}{2}}B\Sigma^{\frac{1}{2}}$且均为对称阵,由上题可知,若$C_1C_2 = O$,则$Y'C_1Y$与$Y'C_2Y$独立,
    于是$\Sigma A\Sigma B\Sigma = O\Leftrightarrow \Sigma^{\frac{1}{2}} A\Sigma B\Sigma^{\frac{1}{2}} = O\Leftrightarrow C_1C_2 = O\Rightarrow Y'C_1Y\text{ 与 }Y'C_2Y\text{ 独立 }$,
\end{proof}
\begin{problem}[2-3练习4]
    证明Wishart分布的性质4:设$X_{(a)}\sim N_p(\bd{0},\Sigma)(a=1,2,\cdots,n)$相互独立,其中$\Sigma = \begin{bmatrix}
        \Sigma_{11}&\Sigma_{12}\\
        \Sigma_{21}&\Sigma_{22}
    \end{bmatrix}$,其中$\Sigma_{11}\in\R^{r\times r},\Sigma_{22}\in\R^{(p-r)\times(p-r)}$,且已知
    \begin{equation*}
        W = \sum_{a=1}^nX_{(a)}X'_{(a)} = \begin{bmatrix}
            W_{11}&W_{12}\\
            W_{21}&W_{22}
        \end{bmatrix}\sim W_p(n,\Sigma)
    \end{equation*}
    其中$W_{11}\in\R^{r\times r},W_{22}\in\R^{(p-r)\times(p-r)}$,则有以下结论:

    (1) $W_{11}\sim W_r(n,\Sigma_{11}), W_{22}\sim W_{p-r}(n,\Sigma_{22})$.

    (2) 当$\Sigma_{12} = O$时,$W_{11}$与$W_{22}$相互独立.
\end{problem}
\begin{proof}
    (1) 令$X = (X_{(1)}, X_{(2)},\cdots, X_{(n)}) = (X_1, X_2)$,其中$X_1\in\R^{n\times r},X_2\in\R^{n\times (p-r)}$,且$X_1\sim N(\bd{0},\Sigma_{11}),X_2\sim N(\bd{0}, \Sigma_{22})$
    由于$W = X'X$,则$W_{11} = X_1'X_1,W_{22} = X_2'X_2$,于是$W_{11}\sim W_{r}(n,\Sigma_{11}), W_{22}\sim W_{p-r}(n,\Sigma_{22})$.

    (2) 由于$\Sigma_{12} = \Sigma_{21}' = 0$,则$X_1$与$X_2$独立,由于$W_{11},W_{22}$分别由$X_1$和$X_2$表出,所以$W_{11}$与$W_{22}$也独立.
\end{proof}
\begin{problem}[2-3练习5]
    对单个$p$元正态总体$N_p(\bd{\mu},\Sigma)$的均值向量的检验问题,试用似然比原理导出检验$H_0:\mu=\mu_0(\Sigma_0\text{已知})$的似然比统计量及其分布.
\end{problem}
\begin{solution}
    \add 似然比函数分子为$L_1 = L(\bd{\mu}_0, \Sigma_0) = (2\pi)^{-\frac{np}{2}}|\Sigma_0|^{-\frac{n}{2}}\exp\left\{-\frac{1}{2}\sum_{i=1}^n(X_i-\bd{\mu})'\Sigma_0^{-1}(X_i - \bd{\mu})\right\}$,
    分母为$L_2 = \max_{\bd{\mu}}L(\bd{\mu}, \Sigma_0) = L(\hat{\bd{\mu}}, \Sigma_0) \xlongequal{\hat{\bd{\mu}} = \bar{X}} (2\pi)^{-\frac{np}{2}}|\Sigma_0|^{-\frac{n}{2}}\exp\left\{-\frac{1}{2}\sum_{i=1}^n(X_i-\bar{X})'\Sigma_0^{-1}(X_i - \bar{X})\right\}$,
    于是广义似然比为
    \begin{equation*}
        \lambda = \frac{L_1}{L_2} = \exp\left\{-\frac{1}{2}\sum_{i=1}^n[(X_i-\bd{\mu})'\Sigma_0^{-1}(X_i - \bd{\mu}) - (X_i-\bar{X})'\Sigma_0^{-1}(X_i - \bar{X})]\right\}
    \end{equation*}
    由于
    \begin{align*}
        &\ \sum_{i=1}^n\bigg\{(X_i-\bd{\mu})'\Sigma_0^{-1}(X_i - \bd{\mu}) - (X_i-\bar{X})'\Sigma_0^{-1}(X_i - \bar{X})\bigg\}\\
        =&\ \sum_{i=1}^n\bigg\{X_i'\Sigma_0^{-1}X_i - 2X_i'\Sigma_0^{-1}\bd{\mu} + \bd{\mu}'\Sigma_0^{-1}\bd{\mu} - \left(X_i'\Sigma_0^{-1}X_i-2X_i'\Sigma_0^{-1}\bar{X}+\bar{X}'\Sigma_0^{-1}\bar{X}\right)\bigg\}\\
        =&\ \sum_{i=1}^n\bigg\{\bd{\mu}'\Sigma_0^{-1}\bd{\mu}+2(X_i'\Sigma_0^{-1}\bar{X}-X_i'\Sigma_0^{-1}\bd{\mu})-\bar{X}'\Sigma_0^{-1}\bar{X}\bigg\}\\
        =&\ n\bd{\mu}'\Sigma_0^{-1}\bd{\mu}+2n(\bar{X}'\Sigma_0^{-1}\bar{X}-\bar{X}'\Sigma_0^{-1}\bd{\mu})-n\bar{X}'\Sigma_0^{-1}\bar{X}\\
        =&\ n\bigg(\bd{\mu}'\Sigma_0^{-1}\bd{\mu}-2\bar{X}'\Sigma_0^{-1}\bd{\mu}+\bar{X}'\Sigma_0^{-1}\bar{X}\bigg)\\
        =&\ n(\bar{X}-\bd{\mu})'\Sigma_0^{-1}(\bar{X}-\bd{\mu})
    \end{align*}
    于是$\lambda = \exp\left\{-\frac{1}{2}(\bar{X}-\bd{\mu})'(\Sigma_0/n)^{-1}(\bar{X}-\bd{\mu})\right\}$,于是似然比统计量为
    \begin{equation*}
        \Lambda = (2\pi)^{-\frac{p}{2}}|\Sigma_0/n|^{-\frac{1}{2}}\exp\left\{-\frac{1}{2}(\bar{X}-\bd{\mu})(\Sigma_0/n)^{-1}(\bar{X}-\bd{\mu})\right\}\sim N_p(\bd{\mu}, \Sigma_0/n)
    \end{equation*}
    服从均值为$\bd{\mu}$,协方差矩阵为$\Sigma_0/n$的$p$元正态分布.
\end{solution}
\begin{problem}[2-3练习6]
    两个多元正态总体均值向量检验,样本量$n=10$,每个样本来自$X\sim N_4(\bd{\mu}_1,\Sigma_1),Y\sim N_4(\bd{\mu}_2,\Sigma_2)$,检验均值向量是否相同.
\end{problem}
\begin{solution}
    \begin{rcode}
setwd("C:/Users/99366/Documents/GitHub/LaTex-Projects/Data Analysis/3,4,5,6")
data <- read.table(file = "data.csv", sep = ",")
X <- data[1:10, ]
Y <- data[11:20, ]
print(t.test(X, Y, var.equal = FALSE, paired = FALSE, mu = 0))
# var.equal为FALSE表示假设两个总体方差不相等,paired为FALSE表示假设两个样本是独立的,mu为0表示检验两个总体均值是否相等
    \end{rcode}
    输出结果
    \begin{rcode}
        Welch Two Sample t-test

data:  X and Y
t = -0.96819, df = 74.917, p-value = 0.3361
alternative hypothesis: true difference in means is not equal to 0
95 percent confidence interval:
 -9.55497  3.30497
sample estimates:
mean of x mean of y
   50.125    53.250
    \end{rcode}
    由于p-value大于0.05,所以接受原假设,即均值向量相同.
\end{solution}
\end{document}