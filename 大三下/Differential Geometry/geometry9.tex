\documentclass[12pt, a4paper, oneside]{ctexart}
\usepackage{amsmath, amsthm, amssymb, bm, color, graphicx, geometry, mathrsfs,extarrows, braket, booktabs, array, wrapfig, enumitem}
\usepackage[colorlinks,linkcolor=red,anchorcolor=blue,citecolor=blue,urlcolor=blue,menucolor=black]{hyperref}
%%%% 设置中文字体 %%%%
% fc-list -f "%{family}\n" :lang=zh >d:zhfont.txt 命令查看已有字体
\setCJKmainfont[
    BoldFont=方正黑体_GBK,  % 黑体
    ItalicFont=方正楷体_GBK,  % 楷体
    BoldItalicFont=方正粗楷简体,  % 粗楷体
    Mapping = fullwidth-stop  % 将中文句号“.”全部转化为英文句号“.”,
]{方正书宋简体}  % !!! 注意在Windows中运行请改为“方正书宋简体.ttf” !!!
%%%% 设置英文字体 %%%%
\setmainfont{Minion Pro}
\setsansfont{Calibri}
\setmonofont{Consolas}

%%%% 设置行间距与页边距 %%%%
\linespread{1.4}
%\geometry{left=2.54cm,right=2.54cm,top=3.18cm,bottom=3.18cm}
\geometry{left=1.84cm,right=1.84cm,top=2.18cm,bottom=2.18cm}

%%%% 图片相对路径 %%%%
\graphicspath{{figures/}} % 当前目录下的figures文件夹, {../figures/}则是父目录的figures文件夹
\setlength{\abovecaptionskip}{-0.2cm}  % 缩紧图片标题与图片之间的距离
\setlength{\belowcaptionskip}{0pt} 

%%%% 缩小item,enumerate,description两行间间距 %%%%
\setenumerate[1]{itemsep=0pt,partopsep=0pt,parsep=\parskip,topsep=5pt}
\setitemize[1]{itemsep=0pt,partopsep=0pt,parsep=\parskip,topsep=5pt}
\setdescription{itemsep=0pt,partopsep=0pt,parsep=\parskip,topsep=5pt}

%%%% 自定义公式 %%%%
\everymath{\displaystyle} % 默认全部行间公式
\DeclareMathOperator*\uplim{\overline{lim}} % 定义上极限 \uplim_{}
\DeclareMathOperator*\lowlim{\underline{lim}} % 定义下极限 \lowlim_{}
\DeclareMathOperator*{\argmax}{arg\,max}  % 定义取最大值的参数 \argmax_{}
\DeclareMathOperator*{\argmin}{arg\,min}  % 定义取最小值的参数 \argmin_{}
\let\leq=\leqslant % 将全部leq变为leqslant
\let\geq=\geqslant % geq同理
\DeclareRobustCommand{\rchi}{{\mathpalette\irchi\relax}}
\newcommand{\irchi}[2]{\raisebox{\depth}{$#1\chi$}} % 使用\rchi将\chi居中

%%%% 自定义环境配置 %%%%
\newcounter{problem}  % 问题序号计数器
\newenvironment{problem}[1][]{\stepcounter{problem}\par\noindent\textbf{题目\arabic{problem}. #1}}{\smallskip\par}
\newenvironment{solution}[1][]{\par\noindent\textbf{#1解答. }}{\smallskip\par}  % 可带一个参数表示题号\begin{solution}{题号}
\newenvironment{note}{\par\noindent\textbf{注记. }}{\smallskip\par}
\newenvironment{remark}{\begin{enumerate}[label=\textbf{注\arabic*.}]}{\end{enumerate}}
\BeforeBeginEnvironment{minted}{\vspace{-0.5cm}}  % 缩小minted环境距上文间距
\AfterEndEnvironment{minted}{\vspace{-0.2cm}}  % 缩小minted环境距下文间距

%%%% 一些宏定义 %%%%
\def\bd{\boldsymbol}        % 加粗(向量) boldsymbol
\def\disp{\displaystyle}    % 使用行间公式 displaystyle(默认)
\def\weekto{\rightharpoonup}% 右半箭头
\def\tsty{\textstyle}       % 使用行内公式 textstyle
\def\sign{\text{sign}}      % sign function
\def\wtd{\widetilde}        % 宽波浪线 widetilde
\def\R{\mathbb{R}}          % Real number
\def\N{\mathbb{N}}          % Natural number
\def\Z{\mathbb{Z}}          % Integer number
\def\Q{\mathbb{Q}}          % Rational number
\def\C{\mathbb{C}}          % Complex number
\def\K{\mathbb{K}}          % Number Field
\def\P{\mathbb{P}}          % Polynomial
\def\E{\mathbb{E}}          % Polynomial
\def\d{\mathrm{d}}          % differential operator
\def\e{\mathrm{e}}          % Euler's number
\def\i{\mathrm{i}}          % imaginary number
\def\re{\mathrm{Re}}        % Real part
\def\im{\mathrm{Im}}        % Imaginary part
\def\res{\mathrm{Res}}      % Residue
\def\ker{\mathrm{Ker}}      % Kernel
\def\vspan{\mathrm{vspan}}  % Span  \span与latex内核代码冲突改为\vspan
\def\L{\mathcal{L}}         % Loss function
\def\O{\mathcal{O}}         % big O notation
\def\A{\mathcal{A}}         % 仿射坐标系
\def\sA{\mathscr{A}}        % 点空间
\def\F{\mathcal{F}}         % 光滑函数空间
\def\sF{\mathscr{F}}        % 光滑函数商空间
\def\wdh{\widehat}          % 宽帽子 widehat
\def\ol{\overline}          % 上横线 overline
\def\ul{\underline}         % 下横线 underline
\def\add{\vspace{1ex}}      % 增加行间距
\def\del{\vspace{-1.5ex}}   % 减少行间距

%%%% 定理类环境的定义 %%%%
\newtheorem{theorem}{定理}

%%%% 基本信息 %%%%
\newcommand{\RQ}{\today} % 日期
\newcommand{\km}{微分几何} % 科目
\newcommand{\bj}{强基数学002} % 班级
\newcommand{\xm}{吴天阳} % 姓名
\newcommand{\xh}{2204210460} % 学号

\begin{document}

%\pagestyle{empty}
\pagestyle{plain}
\vspace*{-15ex}
\centerline{\begin{tabular}{*5{c}}
    \parbox[t]{0.25\linewidth}{\begin{center}\textbf{日期}\\ \large \textcolor{blue}{\RQ}\end{center}} 
    & \parbox[t]{0.2\linewidth}{\begin{center}\textbf{科目}\\ \large \textcolor{blue}{\km}\end{center}}
    & \parbox[t]{0.2\linewidth}{\begin{center}\textbf{班级}\\ \large \textcolor{blue}{\bj}\end{center}}
    & \parbox[t]{0.1\linewidth}{\begin{center}\textbf{姓名}\\ \large \textcolor{blue}{\xm}\end{center}}
    & \parbox[t]{0.15\linewidth}{\begin{center}\textbf{学号}\\ \large \textcolor{blue}{\xh}\end{center}} \\ \hline
\end{tabular}}
\begin{center}
    \zihao{3}\textbf{第六次作业}
\end{center}\vspace{-0.2cm}
\begin{problem}[4.3练习1.]
    证明定义4.8中定义的$\partial_u(A)f:= \frac{\partial (f\circ \varphi)}{\partial u}\bigg|_{\varphi^{-1}(A)}$
    的确是$A$点的导算子。
\end{problem}
\begin{proof}
    1. 局部性:$f,g$为定义在$A$邻域中的两个光滑函数,且存在邻域$U$,使得在$U$上有$f=g$则
    \begin{equation*}
        \frac{\partial (f\circ \varphi)}{\partial u}(x) = \frac{\partial(g\circ \varphi)}{\partial u}(x),\quad(x\in\varphi^{-1}(U))
    \end{equation*}
    于是
    \begin{equation*}
        \partial_uf = \frac{\partial(f\circ \varphi)}{\partial u}\bigg|_{\varphi^{-1}(A)} = \frac{\partial(g\circ \varphi)}{\partial u}\bigg|_{\varphi^{-1}(A)} = \partial_ug
    \end{equation*}

    2. 线性性:$\forall \alpha,\beta\in\R$有
    \begin{equation*}
        \partial_u(\alpha f+\beta g) = \frac{\partial (\alpha f\circ \varphi+\beta g\circ\varphi)}{\partial u}\bigg|_{\varphi^{-1}(A)} = 
        \alpha\frac{\partial(f\circ\varphi)}{\partial u}\bigg|_{\varphi^{-1}(A)}+\beta\frac{\partial(g\circ\varphi)}{\partial u}\bigg|_{\varphi^{-1}(A)} = \alpha\partial_uf+\beta\partial_ug
    \end{equation*}

    3. Leibniz公式
    \begin{equation*}
        \partial_u(fg) = \frac{\partial(f\circ\varphi\cdot g\circ\varphi)}{\partial u}\bigg|_{\varphi^{-1}(A)}
        =\frac{\partial(f\circ \varphi)}{\partial u}\bigg|_{\varphi^{-1}(A)}g(A) + \frac{\partial(g\circ \varphi)}{\partial u}\bigg|_{\varphi^{-1}(A)}f(A)
        =g(A)\partial_uf+f(A)\partial_ug
    \end{equation*}
\end{proof}
\begin{problem}[4.3练习2.]
    考虑球面上的参数化:
    \begin{equation*}
        x^1 = \sin\theta\cos\varphi,\ x^2 = \sin\theta\sin\varphi,\ x^3 = \cos\theta
    \end{equation*}
    写出这个参数化的局部参数标架场,在$\R^3$中的$\{\bd{e}_1=(1,0,0),\bd{e}_2(0,1,0),\bd{e}_3(0,0,1)\}$下表出.
\end{problem}
\begin{solution}
    设任意的光滑函数$f(x_1,x_2,x_3) = f(\sin\theta\cos\varphi,\sin\theta\sin\varphi,\cos\theta)$,于是
    \begin{align*}
        \partial_\theta f =&\ \partial_1 f\cdot\cos\theta\cos\varphi + \partial_2f\cdot \cos\theta\sin\varphi - \partial_3 f\cdot \sin\theta\\
        \partial_\varphi f=&\ \partial_1 f\cdot (-\sin\theta\sin\varphi) + \partial_2f\cdot \sin\theta\cos\varphi
    \end{align*}
    于是$\partial_\theta = (\cos\theta\cos\varphi,\cos\theta\sin\varphi,-\sin\theta),\partial_\varphi = (-\sin\theta\sin\varphi,\sin\theta\cos\varphi,0)$.
\end{solution}
\begin{problem}[4.3练习3.]
    考虑$\R^3$中被表示成函数图像的一部分的曲面$S = \{(x^1,x^2,f(x^1,x^2)),-\varepsilon < x^1,x^2 < \varepsilon\}$,
    求$S$上任一点的切空间,在$\R^3$中的$\{\bd{e}_1=(1,0,0),\bd{e}_2(0,1,0),\bd{e}_3(0,0,1)\}$下表出.
\end{problem}
\begin{solution}
    设曲面上的任意光滑函数$g$,由参数$x_1,x_2$表出为$g(x_1,x_2,f(x_1,x_2))$,则有
    \begin{align*}
        \partial_{x^1}g = \partial_1g,\qquad \partial_{x^2}g = \partial_2g
    \end{align*}
    于是$\partial_{x^1} = (1,0,0),\partial_{x^2} = (0,1,0)$,故任意点的切空间为$\text{span}\{(1,0,0),(0,1,0)\}$.
\end{solution}
\begin{problem}[5.1练习1.]
    我们考虑$\E^3$上的标准正交坐标系$\{O,e_i\}$,考虑标准圆柱面
    \begin{equation*}
        x^1=\cos\theta,\ x^2=\sin\theta,\ x^3=x^3
    \end{equation*}
    考虑参数坐标下,质点$P$在$t=0$时位置是$\theta(0) = x^3(0) = 0$,初速度是$\partial_{\theta}+\partial_3$。
    设该自由质点只收到柱面的约束力,计算该质点的运动方程$(\theta(t),x^3(t))$.
\end{problem}
\begin{solution}
    设$f$为曲面上的光滑函数,则
    \begin{equation*}
        \partial_\theta f = \frac{\partial(f(\cos\theta,\sin\theta,x^3))}{\partial\theta} = -\sin\theta\cdot\partial_1f+\cos\theta\cdot\partial_2f+\partial_3f
    \end{equation*}
    于是$\partial_\theta = (-\sin\theta,\cos\theta,0),\partial_1 = (0,0,1)$,则
    \begin{equation*}
        \bd{g} = \begin{bmatrix}
            g_{\theta\theta}&g_{\theta 3}\\
            g_{3\theta}&g_{33}
        \end{bmatrix} = \begin{bmatrix}
            (\partial_\theta,\partial_\theta)&(\partial_\theta,\partial_3)\\
            (\partial_3,\partial_\theta)&(\partial_3,\partial_3)
        \end{bmatrix} = \begin{bmatrix}
            1&0\\ 0 &1
        \end{bmatrix}=I= \bd{g}^{-1}
    \end{equation*}
    则$\Gamma_{ab}^l = \frac{1}{2}g^{lc}(\partial_ag_{bc}+\partial_bg_{ac}-\partial_cg_{ab}) = 0$.\add

    设测地线为$y(\theta(t),x^3(t))$,由初值条件$y(0) = (\theta(0),x^3(0)) = (0,0),\ \hat{y}(0) = \partial_\theta + \partial_3 = (1,1)$,可知
    \begin{equation*}
        \begin{cases}
            \ddot{y}^1(t) = 0,\\
            \ddot{y}^2(t) = 0.
        \end{cases}\Rightarrow\begin{cases}
            \theta(t) = y^1(t) = y^1(0) + \dot{y}^1(0)t = t,\add\\
            x^3(t) = y^2(t) = y^2(0) + \dot{y}^2(0)t = t.
        \end{cases}
    \end{equation*}
    故测地线方程为$\left(t, t\right)$.
\end{solution}
\begin{problem}
    我们考虑$\E^3$上的标准正交坐标系$\{O,e_i\}$,考虑以$O$为球心的单位球面。考虑点$(1,0,0)$附近的参数化:
    \begin{equation*}
        x^1 = \cos\theta\sin(\pi/2+\varphi),\ x^2 = \sin\theta\sin(\pi/2+\varphi),\ x^3=\cos(\pi/2+\varphi)
    \end{equation*}
    参数$(\theta,\varphi)$的变化区域为$(-\pi/2,\pi/2)\times(-\pi/2,\pi/2)$. 设自由质点$P$的初始位置为$\theta(0) = \varphi(0) = 0$,
    初速度为$\partial\theta+\partial\varphi$,请计算质点的运动方程$(\theta(t),\varphi(t))$.
\end{problem}
\begin{solution}
    设曲面上的光滑函数为$f$,则
    \begin{align*}
        \partial_\theta f =&\ \frac{\partial f(\cos\theta\sin(\frac{\pi}{2}+\varphi),\sin\theta\sin(\frac{\pi}{2}+\varphi),\cos(\frac{\pi}{2}+\varphi))}{\partial \theta}\\
        =&\ \partial_1f\cdot(-\sin\theta\sin(\pi/2+\varphi))+\partial_2 f\cdot(\cos\theta\sin(\pi/2+\varphi))\\
        \partial_\varphi f =&\ \frac{\partial f(\cos\theta\sin(\frac{\pi}{2}+\varphi),\sin\theta\sin(\frac{\pi}{2}+\varphi),\cos(\frac{\pi}{2}+\varphi))}{\partial \varphi}\\
        =&\ \partial_1f\cdot(\cos\theta\cos(\pi/2+\varphi))+\partial_2f\cdot(\sin\theta\cos(\pi/2+\varphi))+\partial_3f\cdot(-\sin(\pi/2+\varphi))
    \end{align*}
    于是
    \begin{align*}
        \partial_\theta =&\ (-\sin\theta\sin(\pi/2+\varphi),\cos\theta\sin(\pi/2+\varphi),0),\\
        \partial\varphi=&\ (\cos\theta\cos(\pi/2+\varphi),\sin\theta\cos(\pi/2+\varphi),-\sin(\pi/2+\varphi))\\
        (\partial_\theta,\partial_\theta) =&\ \sin^2(\pi/2+\varphi),\quad(\partial\theta,\partial\varphi) = (\partial_\varphi,\partial_\theta) = 0,\quad (\partial\varphi,\partial\varphi) = 1
    \end{align*}
    则
    \begin{equation*}
        \bd{g} = \begin{bmatrix}
            \sin^2(\pi/2+\varphi)&0\\
            0&1
        \end{bmatrix},\quad\bd{g}^{-1}\begin{bmatrix}
            \frac{1}{\sin^2(\pi/2+\varphi)}&0\\
            0&1
        \end{bmatrix}
    \end{equation*}
    于是
    \begin{align*}
        &\ \Gamma_{ab}^1 = \frac{1}{2}g^{11}(\partial_a g{b1} + \partial_b g_{a1})\\
        \Rightarrow&\ \Gamma_{12}^1 = \Gamma_{21}^1 = \frac{1}{\tan(\pi/2+\varphi)}\ \text{且}\  \Gamma_{ab}^1 = 0,\ (a,b)\notin\{(1,2),(2,1)\}\\
        &\ \Gamma_{ab}^2 = \frac{1}{2}g^{22}(\partial_a g_{b2}+\partial_bg_{a2}-\partial_{\varphi}g_{ab})\\
        \Rightarrow&\ \Gamma_{11}^2 = -\sin(\pi/2+\varphi)\cos(\pi/2+\varphi) = -\frac{1}{2}\sin(\pi+2\varphi)\ \text{且}\ \Gamma_{ab}^2 = 0,\ (a, b) \neq (1,1)
    \end{align*}
    设测地线为$y(\theta(t), \varphi(t))$,满足一下初值条件:
    \begin{equation*}
        \begin{cases}
            \tan(\pi/2+\varphi)\ddot{y}^1 + 2\dot{y}^1\dot{y}^2 = 0\add\\
            \ddot{y}^2 -  \frac{1}{2}\sin(\pi+2\varphi)\dot{y}^1\dot{y}^1 = 0
        \end{cases},\quad\text{初值条件}\begin{cases}
            y(0) = (0, 0),\\
            \hat{y}(0) = (1,1).
        \end{cases}
    \end{equation*}
\end{solution}
\end{document}