\documentclass[12pt, a4paper, oneside]{ctexart}
\usepackage{amsmath, amsthm, amssymb, bm, color, graphicx, geometry, mathrsfs,extarrows, braket, booktabs, array, wrapfig, enumitem}
\usepackage[colorlinks,linkcolor=red,anchorcolor=blue,citecolor=blue,urlcolor=blue,menucolor=black]{hyperref}
%%%% 设置中文字体 %%%%
% fc-list -f "%{family}\n" :lang=zh >d:zhfont.txt 命令查看已有字体
\setCJKmainfont{方正书宋.ttf}[BoldFont = 方正黑体_GBK.ttf, ItalicFont = simkai.ttf, BoldItalicFont = 方正粗楷简体.ttf]
%%%% 设置英文字体 %%%%
\setmainfont{Times New Roman}
\setsansfont{Calibri}
\setmonofont{Consolas}

%%%% 设置行间距与页边距 %%%%
\linespread{1.4}
%\geometry{left=2.54cm,right=2.54cm,top=3.18cm,bottom=3.18cm}
\geometry{left=1.84cm,right=1.84cm,top=2.18cm,bottom=2.18cm}

%%%% 图片相对路径 %%%%
\graphicspath{{figures/}} % 当前目录下的figures文件夹, {../figures/}则是父目录的figures文件夹
\setlength{\abovecaptionskip}{-0.2cm}  % 缩紧图片标题与图片之间的距离
\setlength{\belowcaptionskip}{0pt} 

%%%% 缩小item,enumerate,description两行间间距 %%%%
\setenumerate[1]{itemsep=0pt,partopsep=0pt,parsep=\parskip,topsep=5pt}
\setitemize[1]{itemsep=0pt,partopsep=0pt,parsep=\parskip,topsep=5pt}
\setdescription{itemsep=0pt,partopsep=0pt,parsep=\parskip,topsep=5pt}

%%%% 自定义公式 %%%%
\everymath{\displaystyle} % 默认全部行间公式
\DeclareMathOperator*\uplim{\overline{lim}} % 定义上极限 \uplim_{}
\DeclareMathOperator*\lowlim{\underline{lim}} % 定义下极限 \lowlim_{}
\DeclareMathOperator*{\argmax}{arg\,max}  % 定义取最大值的参数 \argmax_{}
\DeclareMathOperator*{\argmin}{arg\,min}  % 定义取最小值的参数 \argmin_{}
\let\leq=\leqslant % 将全部leq变为leqslant
\let\geq=\geqslant % geq同理
\DeclareRobustCommand{\rchi}{{\mathpalette\irchi\relax}}
\newcommand{\irchi}[2]{\raisebox{\depth}{$#1\chi$}} % 使用\rchi将\chi居中

%%%% 自定义环境配置 %%%%
\newcounter{problem}  % 问题序号计数器
\newenvironment{problem}[1][]{\stepcounter{problem}\par\noindent\textbf{题目\arabic{problem}. #1}}{\smallskip\par}
\newenvironment{solution}[1][]{\par\noindent\textbf{#1解答. }}{\smallskip\par}  % 可带一个参数表示题号\begin{solution}{题号}
\newenvironment{note}{\par\noindent\textbf{注记. }}{\smallskip\par}
\newenvironment{remark}{\begin{enumerate}[label=\textbf{注\arabic*.}]}{\end{enumerate}}
\BeforeBeginEnvironment{minted}{\vspace{-0.5cm}}  % 缩小minted环境距上文间距
\AfterEndEnvironment{minted}{\vspace{-0.2cm}}  % 缩小minted环境距下文间距

%%%% 一些宏定义 %%%%
\def\bd{\boldsymbol}        % 加粗(向量) boldsymbol
\def\disp{\displaystyle}    % 使用行间公式 displaystyle(默认)
\def\weekto{\rightharpoonup}% 右半箭头
\def\tsty{\textstyle}       % 使用行内公式 textstyle
\def\sign{\text{sign}}      % sign function
\def\wtd{\widetilde}        % 宽波浪线 widetilde
\def\R{\mathbb{R}}          % Real number
\def\N{\mathbb{N}}          % Natural number
\def\Z{\mathbb{Z}}          % Integer number
\def\Q{\mathbb{Q}}          % Rational number
\def\C{\mathbb{C}}          % Complex number
\def\K{\mathbb{K}}          % Number Field
\def\P{\mathbb{P}}          % Polynomial
\def\d{\mathrm{d}}          % differential operator
\def\e{\mathrm{e}}          % Euler's number
\def\i{\mathrm{i}}          % imaginary number
\def\re{\mathrm{Re}}        % Real part
\def\im{\mathrm{Im}}        % Imaginary part
\def\res{\mathrm{Res}}      % Residue
\def\ker{\mathrm{Ker}}      % Kernel
\def\vspan{\mathrm{vspan}}  % Span  \span与latex内核代码冲突改为\vspan
\def\L{\mathcal{L}}         % Loss function
\def\O{\mathcal{O}}         % big O notation
\def\wdh{\widehat}          % 宽帽子 widehat
\def\ol{\overline}          % 上横线 overline
\def\ul{\underline}         % 下横线 underline
\def\add{\vspace{1ex}}      % 增加行间距
\def\del{\vspace{-1.5ex}}   % 减少行间距

%%%% 定理类环境的定义 %%%%
\newtheorem{theorem}{定理}

%%%% 基本信息 %%%%
\newcommand{\RQ}{\today} % 日期
\newcommand{\km}{微分几何} % 科目
\newcommand{\bj}{强基数学002} % 班级
\newcommand{\xm}{吴天阳} % 姓名
\newcommand{\xh}{2204210460} % 学号

\begin{document}

%\pagestyle{empty}
\pagestyle{plain}
\vspace*{-15ex}
\centerline{\begin{tabular}{*5{c}}
    \parbox[t]{0.25\linewidth}{\begin{center}\textbf{日期}\\ \large \textcolor{blue}{\RQ}\end{center}} 
    & \parbox[t]{0.2\linewidth}{\begin{center}\textbf{科目}\\ \large \textcolor{blue}{\km}\end{center}}
    & \parbox[t]{0.2\linewidth}{\begin{center}\textbf{班级}\\ \large \textcolor{blue}{\bj}\end{center}}
    & \parbox[t]{0.1\linewidth}{\begin{center}\textbf{姓名}\\ \large \textcolor{blue}{\xm}\end{center}}
    & \parbox[t]{0.15\linewidth}{\begin{center}\textbf{学号}\\ \large \textcolor{blue}{\xh}\end{center}} \\ \hline
\end{tabular}}
\begin{center}
    \zihao{3}\textbf{第三次作业}
\end{center}\vspace{-0.2cm}
\begin{problem}[练习3.1.1]
    从放射空间的定义出发,证明仿射空间中过不同两点有且仅有一条直线.
\end{problem}
\begin{proof}
    反设,存在两条不同的直线$l_1,l_2$分别过$A,B$两点,则$\forall k,p\in\R$,$\exists X\in l_1,Y\in l_2$,
    使得$\overrightarrow{AX} = k\overrightarrow{AB}, \overrightarrow{AY} = p\overrightarrow{AB}$,令$k=p$.
    则$\overrightarrow{AX} = k\overrightarrow{AB} = \overrightarrow{AY}$,由仿射空间的定义可知,$X=Y$,又由于$k$的任意性可知$l_1=l_2$.
\end{proof}
\begin{problem}[练习3.1.4]
    设三位仿射空间中有仿射坐标系$\mathcal{A} = \{O,\bd{e}_1,\bd{e}_2,\bd{e}_3\}$. 设某两点$P$的坐标为$(1,2,1)$,
    $O'$的坐标为$(2,1,-1)$. 计算$O$在坐标系$\{P, \bd{e}_1+\bd{e}_2,\bd{e}_2,\bd{e}_3\}$中的坐标和$P$在坐标系$\{O', \bd{e}_1-\bd{e}_2,\bd{e}_2,\bd{e}_3\}$中的坐标.
\end{problem}
\begin{solution}
    在$\mathcal{A}$中,$OP = (\bd{e}_1,\bd{e}_2,\bd{e}_3)\left(\begin{matrix}
        1\\2\\1
    \end{matrix}\right)$且$(\bd{e}_1,\bd{e}_2,\bd{e}_3) = (\bd{e}_1+\bd{e}_2,\bd{e}_2,\bd{e}_3)\begin{bmatrix}
        1&0&0\\
        -1&1&0\\
        0&0&1
    \end{bmatrix}$,则$O$在$\{P, \bd{e}_1+\bd{e}_2,\bd{e}_2,\bd{e}_3\}$中坐标为$(-1,-2,-1)\begin{bmatrix}
        1&-1&0\\
        0&1&0\\
        0&0&1
    \end{bmatrix} = (-1,-1,-1)$.

    由于$O'$在$\mathcal{A}$中表示为$OO' = (\bd{e}_1,\bd{e}_2,\bd{e}_3)\left(\begin{matrix}
        2\\1\\-1
    \end{matrix}\right)$且$(\bd{e}_1,\bd{e}_2,\bd{e}_3) = (\bd{e}_1-\bd{e}_2,\bd{e}_2,\bd{e}_3)\begin{bmatrix}
        1&0&0\\
        1&1&0\\
        0&0&1
    \end{bmatrix}$,则$P$在$\{O', \bd{e}_1-\bd{e}_2,\bd{e}_2,\bd{e}_3\}$中的坐标为
    \begin{equation*}
        \begin{bmatrix}
            (1,2,1) - (2,1,-1)
        \end{bmatrix}\begin{bmatrix}
            1&1&0\\
            0&1&0\\
            0&0&1
        \end{bmatrix} = (-1,1,2)\begin{bmatrix}
            1&1&0\\
            0&1&0\\
            0&0&1
        \end{bmatrix} = (-1,0,2)
    \end{equation*}
\end{solution}
\begin{problem}[练习3.2.1]
    设$\mathscr{A}^3$上有四个点$A_1,A_2,A_3,A_4$,其坐标分别为$(x_i^1,x_i^2,x_i^3),i=1,2,3,4$,定义
    \begin{equation*}
        \lambda = \frac{\sqrt{\sum_{j=1}^3|x_1^j-x_2^j|^2}}{\sqrt{\sum_{j=1}^3|x_3^j-x_4^j|^2}}
    \end{equation*}
    在什么条件下,$\lambda$是仿射几何量.
\end{problem}
\begin{solution}
    设任意一个仿射坐标系为$\mathcal{A} = \{O,\bd{e}_1,\bd{e}_2,\bd{e}_3\}$,条件为$A_1,A_2,A_3,A_4$在同一条直线$l:\bd{x}_0+t\bd{v}$上,其中$\bd{x}_0,\bd{v}\in\R^3$,
    设$A_i = \bd{x_0}+t_i\bd{v},\ (i=1,2,3,4)$,
    于是
    \begin{equation*}
        \lambda = \frac{\sqrt{\sum_{j=1}^3|x_1^j-x_2^j|^2}}{\sqrt{\sum_{j=1}^3|x_3^j-x_4^j|^2}}
        = \frac{||\bd{x}_1 - \bd{x}_2||_2}{||\bd{x}_3 - \bd{x}_4||_2} = \frac{|t_1 - t_2|\cdot||\bd{v}||_2}{|t_3-t_4|\cdot||\bd{v}||_2} = \frac{|t_1-t_2|}{|t_3-t_4|}
    \end{equation*}
    $\lambda$与仿射坐标系$\mathcal{A}$无关,所以$\lambda$是仿射几何量.
\end{solution}
\begin{problem}[练习3.3.1]
    设$X,Y$为两个集合,$f:X\to Y$为一个映射. 证明下述性质:

    (1) 设$\Lambda$为一个指标集,$A_\alpha\subset Y$,则
    \begin{equation*}
        f^{-1}\left(\bigcap_{\alpha\in \Lambda}A_{\alpha}\right) = \bigcap_{\alpha\in\Lambda}f^{-1}(A_\alpha)
    \end{equation*}

    (2)
    \begin{equation*}
        f^{-1}\left(\bigcup_{\alpha\in\Lambda}A_{\alpha}\right) = \bigcup_{\alpha\in\Lambda}f^{-1}(A_\alpha)
    \end{equation*}
\end{problem}
\begin{proof}
    (1)只需证明相互包含即可,如下可证
    \begin{equation*}
        x\in f^{-1}\left(\bigcap_{\alpha\in\Lambda}A_\alpha\right)\Leftrightarrow f(x)\in A_{\alpha},\ (\forall \alpha\in\Lambda)\Leftrightarrow x\in f^{-1}(A_\alpha),\ (\forall \alpha\in\Lambda)\Leftrightarrow x\in\bigcap_{\alpha\in\Lambda}f^{-1}(A_\alpha)
    \end{equation*}
    (2)只需证明相互包含即可,如下可证
    \begin{equation*}
        x\in f^{-1}\left(\bigcup_{\alpha\in\Lambda}A_\alpha\right)\Leftrightarrow \exists\alpha_0\in\Lambda,\ f(x)\in A_{\alpha_0}\Leftrightarrow \exists\alpha_0\in\Lambda,\ x\in f^{-1}(A_\alpha)\Leftrightarrow x\in\bigcup_{\alpha\in\Lambda}f^{-1}(A_\alpha)
    \end{equation*}
\end{proof}
\begin{problem}[练习3.3.2]
    根据上述性质证明在定义3.4中引入的开集族构成一个拓扑.
\end{problem}
\begin{proof}
    在拓扑空间$\mathscr{A}^n$中,任取一拓扑坐标系$\mathcal{A} = \{O,\bd{e}_i\}$,设开集族全体构成集合$\tau$,于是
    \begin{equation*}
        \tau = \{U\subset \mathscr{A}^n:\varphi_{\mathcal{A}}(U)\text{为开集}\}   
    \end{equation*}
    下面证明$(\mathscr{A}^n,\tau)$构成$\mathscr{A}^n$的一个拓扑空间:

    (1). 由于$\varphi_{\mathcal{A}}(\mathscr{A}^n) = \R^n,\ \varphi_{\mathcal{A}}(\varnothing) = \varnothing$,且$\R^n,\varnothing$为$\R^n$中的开集,故$\mathscr{A}^n,\varnothing\in\tau$.

    (2). $\forall \{U_1,\cdots,U_n,\cdots\}\subset\tau$,由于$\varphi_{\mathcal{A}}$为同胚映射,令上一题中$f^{-1} = \varphi_{\mathcal{A}}$,由于$\varphi_{\mathcal{A}}(U_n)$为开集和开集的性质可知
    \begin{equation*}
        \varphi_{\mathcal{A}}\left(\bigcup_{i=1}^\infty U_i\right) = \bigcup_{i=1}^\infty\varphi_{\mathcal{A}}(U_i)
    \end{equation*}
    为开集,则$\bigcup_{i=1}^\infty U_i\in \tau$.

    (3). $\forall n\in\N$,$\{U_1,\cdots,U_n\}\subset \tau$,类似(2)和上题结论,且$\varphi_{\mathcal{A}}(U_i)$为开集和开集的性质可知
    \begin{equation*}
        \varphi_{\mathcal{A}}\left(\bigcap_{i=1}^n U_i\right) = \bigcap_{i=1}^n\varphi_{\mathcal{A}}(U_i)
    \end{equation*}
    为开集,则$\bigcap_{i=1}^n U_i\in \tau$.

    综上,$(\mathscr{A}^n, \tau)$是$\mathscr{A}^n$的一个拓扑空间.
\end{proof}
\end{document}