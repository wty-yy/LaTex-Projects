\documentclass[12pt, a4paper, oneside]{ctexart}
\usepackage{amsmath, amsthm, amssymb, bm, color, graphicx, geometry, mathrsfs,extarrows, braket, booktabs, array}
\usepackage[colorlinks,linkcolor=red,anchorcolor=blue,citecolor=blue,urlcolor=blue,menucolor=black]{hyperref}
\setmainfont{Times New Roman}  % 设置英文字体
\setsansfont{Calibri}
\setmonofont{Consolas}

\linespread{1.4}
%\geometry{left=2.54cm,right=2.54cm,top=3.18cm,bottom=3.18cm}
\geometry{left=1.84cm,right=1.84cm,top=2.18cm,bottom=2.18cm}
\newcounter{problem}  % 问题序号计数器
\newenvironment{problem}{\stepcounter{problem}\par\noindent\textbf{题目\arabic{problem}. }}{\smallskip\par}
\newenvironment{solution}{\par\noindent\textbf{解答. }}{\smallskip\par}
\newenvironment{note}{\par\noindent\textbf{注记. }}{\smallskip\par}

%%%% 图片相对路径 %%%%
\graphicspath{{figure/}} % 当前目录下的figure文件夹, {../figure/}则是父目录的figure文件夹

\everymath{\displaystyle} % 默认全部行间公式
\DeclareMathOperator*\uplim{\overline{lim}} % 定义上极限 \uplim_{}
\DeclareMathOperator*\lowlim{\underline{lim}} % 定义下极限 \lowlim_{}
\let\leq=\leqslant % 将全部leq变为leqslant
\let\geq=\geqslant % geq同理

%%%% 一些宏定义 %%%%
\def\bd{\boldsymbol}        % 加粗(向量) boldsymbol
\def\disp{\displaystyle}    % 使用行间公式 displaystyle(默认)
\def\tsty{\textstyle}       % 使用行内公式 textstyle
\def\sign{\text{sign}}      % sign function
\def\wtd{\widetilde}        % 宽波浪线 widetilde
\def\R{\mathbb{R}}          % Real number
\def\N{\mathbb{N}}          % Natural number
\def\Z{\mathbb{Z}}          % Integer number
\def\Q{\mathbb{Q}}          % Rational number
\def\C{\mathbb{C}}          % Complex number
\def\d{\mathrm{d}}          % differential operator
\def\e{\mathrm{e}}          % Euler's number
\def\i{\mathrm{i}}          % imaginary number
\def\re{\mathrm{Re}}        % Real part
\def\im{\mathrm{Im}}        % Imaginary part
\def\res{\mathrm{Res}}      % Residue
\def\L{\mathcal{L}}         % Loss function
\def\wdh{\widehat}          % 宽帽子 widehat
\def\ol{\overline}          % 上横线 overline
\def\ul{\underline}         % 下横线 underline
\def\add{\vspace{1ex}}      % 增加行间距
\def\del{\vspace{-3.5ex}}   % 减少行间距

%%%% 定理类环境的定义 %%%%
\newtheorem{theorem}{定理}

%%%% 基本信息 %%%%
\newcommand{\RQ}{\today} % 日期
\newcommand{\km}{数学建模} % 科目
\newcommand{\bj}{强基数学002} % 班级
\newcommand{\xm}{吴天阳} % 姓名
\newcommand{\xh}{2204210460} % 学号

\begin{document}

%\pagestyle{empty}
\pagestyle{plain}
\vspace*{-15ex}
\centerline{\begin{tabular}{*5{c}}
    \parbox[t]{0.25\linewidth}{\begin{center}\textbf{日期}\\ \large \textcolor{blue}{\RQ}\end{center}} 
    & \parbox[t]{0.2\linewidth}{\begin{center}\textbf{科目}\\ \large \textcolor{blue}{\km}\end{center}}
    & \parbox[t]{0.2\linewidth}{\begin{center}\textbf{班级}\\ \large \textcolor{blue}{\bj}\end{center}}
    & \parbox[t]{0.1\linewidth}{\begin{center}\textbf{姓名}\\ \large \textcolor{blue}{\xm}\end{center}}
    & \parbox[t]{0.15\linewidth}{\begin{center}\textbf{学号}\\ \large \textcolor{blue}{\xh}\end{center}} \\ \hline
\end{tabular}}
\vspace*{4ex}

% 正文部分
\begin{problem}
    取$A=\left[\begin{matrix}
        1&2\\0&5
    \end{matrix}\right]$,用$A$加密\texttt{meet},求其逆矩阵对其解密.
\end{problem}
\begin{solution}
    将字符与数字进行对应:$a:1, b:2,\cdots, y:25, z:0$,原文矩阵为$P=\left[\begin{matrix}
        13&5\\5&20
    \end{matrix}\right]$,在模$26$意义下,密文矩阵为$Q = AP=  \left[\begin{matrix}
        23&19\\25&22
    \end{matrix}\right]$,于是对应的密文为\texttt{wysv}. 在模$26$意义下,$A^{-1} = \left[\begin{matrix}
        1&10\\0&21
    \end{matrix}\right]$,通过计算可得$A^{-1}Q = P$,从而实现解密.
\end{solution}
\begin{problem}
    有密文如下:\texttt{goqbxcbuglosnfal};根据英文的行文习惯以及获取密码的途径和背景,猜测是两个字母为一组的希尔密码,前四个明文字母是\texttt{dear},试破解这段密文.
\end{problem}
\begin{solution}
    密文前四个字母对应的密文矩阵为$Q = \left[\begin{matrix}
        7&17\\
        15&2
    \end{matrix}\right]$,在模$26$下其逆矩阵为$Q^{-1} = \left[\begin{matrix}
        22&21\\
        17&25
    \end{matrix}\right]$,密文对应的明文矩阵为$P=\left[\begin{matrix}
        4&1\\
        5&18
    \end{matrix}\right]$,则可得$A^{-1} = PQ^{-1} = \left[\begin{matrix}
        1&5\\
        0&9
    \end{matrix}\right]$.\add

    密文\texttt{goqbxcbuglosnfal}对应的密文矩阵为
    \begin{equation*}
        Q = \left[\begin{matrix}
            7& 17& 24&  2&  7& 15& 14&  1\\
            15&  2&  3& 21& 12& 19&  6& 12
        \end{matrix}\right]
    \end{equation*}
    通过加密矩阵可得\begin{equation*}
        P = A^{-1}Q = \left[\begin{matrix}
            4&  1& 13&  3& 15&  6& 18&  9\\
            5& 18&  1&  7&  4& 15&  2&  4
        \end{matrix}\right]
    \end{equation*}
    破解得到的明文为\texttt{dearmacgodforbid}.
\end{solution}

% 下面给一些功能的写法
\iffalse
% 图片模板
\centerline{
    \includegraphics[width=0.8\textwidth]{figure.png}
}
% 表格模板
\renewcommand\arraystretch{0.8} % 设置表格高度为原来的0.8倍
\begin{table}[!htbp] % table标准
    \centering % 表格居中
    \begin{tabular}{p{1cm}<{\centering}p{1cm}<{\centering}p{3cm}<{\centering}p{5cm}<{\centering}} % 设置表格宽度
    %\begin{tabular}{cccc}
        \toprule
        $x_i$ & $f[x_1]$ & $f[x_i,x_{i+1}]$ & $f[x_i,x_{i+1},x_{i+2}]$ \\
        \midrule
        $x_0$ & $f(x_0)$ &                  &                          \\
        $x_0$ & $f(x_0)$ & $f'(x_0)$        &                          \\
        $x_0$ & $f(x_1)$ & $\frac{f(x_1)-f(x_0)}{x_1-x_0}$ & $\frac{f(x_1)-f(x_0)}{(x_1-x_0)^2}-\frac{f'(x_0)}{x_1-x_0}$\\
        \bottomrule
    \end{tabular}
\end{table}

\def\Log{\text{Log}} % 一个简单的宏定义
$\Log$ % 调用方法
\fi

\end{document}