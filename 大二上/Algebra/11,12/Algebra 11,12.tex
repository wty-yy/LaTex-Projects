\documentclass[12pt, a4paper, oneside]{ctexart}
\usepackage{amsmath, amsfonts, amsthm, amssymb, bm, color, graphicx, geometry, hyperref, mathrsfs,extarrows, braket}

\linespread{1.5}
\geometry{left=2.54cm,right=2.54cm,top=3.18cm,bottom=3.18cm}
\newenvironment{problem}{\par\noindent\textbf{题目. }}{\bigskip\par}
\newenvironment{solution}{\par\noindent\textbf{解答. }}{\bigskip\par}
\newenvironment{note}{\par\noindent\textbf{注记. }}{\bigskip\par}

% 基本信息
\newcommand{\dt}{\today}
\newcommand{\sj}{近世代数}
\newcommand{\vt}{吴天阳 2204210460}

\begin{document}

\pagestyle{empty}
\vspace*{-20ex}
\centerline{\begin{tabular}{*3{c}}
    \parbox[t]{0.3\linewidth}{\begin{center}\textbf{日期}\\ \large \textcolor{blue}{\dt}\end{center}} 
    & \parbox[t]{0.3\linewidth}{\begin{center}\textbf{科目}\\ \large \textcolor{blue}{\sj}\end{center}}
    & \parbox[t]{0.3\linewidth}{\begin{center}\textbf{姓名,学号}\\ \large \textcolor{blue}{\vt}\end{center}} \\ \hline
\end{tabular}}
\vspace*{4ex}

\paragraph{习题 1.10}
\paragraph{2.}求下列$\text{Abel}$群的初等因子:

(1). $(\mathbb{Z}_{10},+)\oplus(\mathbb{Z}_{15},+)\oplus(\mathbb{Z}_{20},+)$;

(2). $(\mathbb{Z}_{28},+)\oplus(\mathbb{Z}_{42},+)$;

(3). $(\mathbb{Z}_{9},+)\oplus(\mathbb{Z}_{14},+)\oplus(\mathbb{Z}_{6},+)\oplus(\mathbb{Z}_{16},+)$.

\begin{solution}
    (1). 由于$10=2\cdot 5,15=3\cdot 5,20=2^2\cdot 5$,则其初等因子为:
    \begin{equation*}
        \{2,2^2,3,5,5,5\}
    \end{equation*}

    (2). 由于$28 = 2^2\cdot 7,42=2\cdot 3\cdot 7$,则其初等因子为:
    \begin{equation*}
        \{2,2^2,3,7,7\}
    \end{equation*}
    
    (3). 由于$9=3^2,14=2\cdot 7,6=2\cdot 3,16=2^4$,则其初等因子为:
    \begin{equation*}
        \{2,2,2^4,3,3^2,7\}
    \end{equation*}
\end{solution}
\paragraph{3.}设$G$为$100$阶$\text{Abel}$群。

(1). 证明$G$必含有$10$阶元;

(2). $G$的初等因子应当怎样才能使$G$不含阶大于$10$的元素?

\begin{solution}
(1). 由于$100=2^2\cdot 5$,且$2$的分拆为$2=1+1$,则$G$的初等因子当且仅有两种:
\begin{equation*}
    \{4,5\},\{2,2,5\}
\end{equation*}

I. 若$G\cong \mathbb{Z}_4\oplus\mathbb{Z}_5\cong \mathbb{Z}_{20}$,则$\overline{2}\in \mathbb{Z}_{20}$,且$|\overline{2}|=10$,故存在$10$阶元。

II. 若$G\cong\mathbb{Z}_2\oplus\mathbb{Z}_2\oplus\mathbb{Z}_5\cong\mathbb{Z}_2\oplus\mathbb{Z}_{10}$,则$(\overline{0},\tilde{1})\in \mathbb{Z}_2\oplus\mathbb{Z}_{10}$,且$|(\overline{0},\tilde{1})| = 10$,故存在$10$阶元。

综上,$G$必含有$10$阶元。

(2). 当$G$的初等因子为$\{4,5\}$时,$G\cong \mathbb{Z}_{20}$为循环群,存在$20$阶元素,舍去。

当$G$的初等因子为$\{2,2,5\}$时,$G\cong\mathbb{Z}_{2}\oplus\mathbb{Z}_{10}$,$\forall (a,b)\in\mathbb{Z}_{2}\oplus\mathbb{Z}_{10}$,且
\begin{equation*}
    (a,b)^{10} = (a^{10},b^{10}) = ((a^2)^5,b^{10}) = (\overline{0},\tilde{0})
\end{equation*}

所以,$|(a,b)|\biggl|10\Rightarrow |(a,b)|\leqslant 10$,满足题意。

综上,$G$的初等因子应当为$\{2,2,5\}$才能使$G$不含大于$10$的元素。
\end{solution}
\paragraph{5.}证明:如果一个$\text{Abel }p\text{-群}$恰好含有$p-1$个$p$阶元,那么它一定是循环群。
\begin{proof}
    设$G$为$\text{Abel }p\text{-群}$,$|G|=p^l$,它的初等因子为:
    \begin{equation*}
        \{p^{k_1},p^{k_2},\cdots,p^{k_r}\}
    \end{equation*}
    其中,$k_1\geqslant k_2\geqslant \cdots\geqslant k_r\geqslant 1$且$k_1+k_2+\cdots+k_r=l$,则
    \begin{equation*}
        G\cong \mathbb{Z}_{p^{k_1}}\oplus\mathbb{Z}_{p^{k_2}}\oplus\cdots\oplus\mathbb{Z}_{p^{k_r}}
    \end{equation*}
    设$a_1=(\overline{p},\overline{0},\cdots,\overline{0}), \cdots,a_i = (\overline{0},\cdots,\overline{p},\cdots,\overline{0}),\cdots,a_r=(\overline{0},\cdots,\overline{0},\overline{p})$,则$|a_i|=p,i=1,2\cdots,r$,且每个$a_i$都可以确定$p-1$个$p$阶元,所以一共有$r(p-1)$个$p$阶元,且它们两两不同,又由于$G$中恰好有$p-1$个$p$阶元,所以$r\equiv 1$,故$G\cong\mathbb{Z}_{p^{k_1}}=\mathbb{Z}_{p^l}$。
    
    综上,$G$为循环群。
\end{proof}

\paragraph{6.}设$p$是素数,$V$是域$\mathbb{Z}_p$上的$n$维线性空间,$V$的加法群$(V,+)$是不是初等$\text{Abel }p\text{-群}$?

\begin{solution}
    是的,因为$V$有$n$个正交的单位向量$\vec{e}_i,i=1,2,\cdots,n$,且由$\vec{e}_i$生成的子空间$W_i$同构于$\mathbb{Z}_{p}$,又由于$V\cong W_1\oplus W_2\oplus\cdots\oplus W_n$,于是
    \begin{equation*}
        V\cong\mathbb{Z}_{p}\oplus\mathbb{Z}_{p}\oplus\cdots\oplus\mathbb{Z}_{p}
    \end{equation*}
    所以,$V$的初等因子为$\{p,p,\cdots,p\}$,故$(V,+)$是初等$\text{Abel }p\text{-群}$。
\end{solution}
\paragraph{习题 2.1}
\paragraph{4.}设$R$是一个有单位元$1(\neq 0)$的交换环,证明:如果$R$没有非平凡的理想,那么$R$是一个域。
\begin{proof}
    反设,$R$不是一个域,则存在$a\in R$,使得$a$在$R$中无逆元,令$Ra:=\{ra:r\in R\}$,则
    \begin{equation*}
        \begin{aligned}
            r_1a-r_2a &= (r_1-r_2)a\in Ra\\
            r\cdot r_1a &= r_1a\cdot r = r_1ra\in Ra
        \end{aligned}
    \end{equation*}

    所以,$Ra$为$R$的一个理想,由于$a$无逆元,则$1\notin Ra$,故$Ra$为非平凡理想,与$R$无非平凡理想矛盾,则$R$中每个元素都有逆元,所以$(R,\cdot)$为$\text{Abel}$群,则$R$为域。
\end{proof}

\paragraph{6.}若$R$是有单位元$1(\neq 0)$的交换环,且$R$没有非零的零因子,则$R$称为整环。证明:有限整环一定是域。
\begin{proof}
    要证有限整环$R$一定是域,只需证明$R$中每一个元素都有逆元,由于$R$有限,令
    \begin{equation*}
        R = \{a_1,a_2,\cdots, a_n\}
    \end{equation*}

    $\forall a_i\in R\backslash\{0\},\ i=1,2,\cdots,n$,令
    \begin{equation*}
        J_i = \{a_ia_1,a_ia_2,\cdots,a_ia_n\}
    \end{equation*}

    假设$a_ia_j=a_ia_k,\ j\neq k$,则$a_ia_j-a_ia_k=0\Rightarrow a_i(a_j-a_k)=0$,由于$R$中无非零的零因子,且$a_i\neq 0$,所以$a_j-a_k=0\Rightarrow j = k$与$j\neq k$矛盾,所以$J_i$的基数为$n$。
    
    又由于$J_i\subset R$,则$J_i = R$,所以$J_i$为$R$的一个轮换,则$\exists j\in [1,n]$,使得$a_ia_j = 1$,则$a_j$为$a_i$的逆元。

    综上,有限整环一定是域。
\end{proof}
\paragraph{8.}若$R$是一个有单位元$1(\neq 0)$的环,且$R$的每一个非零元都可逆,则$R$称为一个除环或体,令
\begin{equation*}
    \mathcal{H} = \left\{\begin{pmatrix}
        \alpha&\beta\\-\bar{\beta}&\bar{\alpha}
    \end{pmatrix}
    :\alpha,\beta\in\mathbb{C}\right\}
\end{equation*}

证明:$\mathcal{H}$是一个除环,且$\mathcal{H}$与四元数体$\mathbf{H}$环同构。

\begin{proof}
    减法封闭:
    \begin{equation*}
        \begin{pmatrix}
            \alpha_1&\beta_1\\-\overline{\beta_1}&\overline{\alpha_1}
        \end{pmatrix}
        -
        \begin{pmatrix}
            \alpha_2&\beta_2\\-\overline{\beta_2}&\overline{\alpha_2}
        \end{pmatrix}
        =\begin{pmatrix}
            \alpha_1-\alpha_2&\beta_1-\beta_2\\-\overline{\beta_1}+\overline{\beta_2}&\overline{\alpha_1}-\overline{\alpha_2}
        \end{pmatrix}
        =\begin{pmatrix}
            \alpha_1-\alpha_2&\beta_1-\beta_2\\-\overline{\beta_1-\beta_2}&\overline{\alpha_1-\alpha}
        \end{pmatrix}
        \in \mathcal{H}
    \end{equation*}

    乘法封闭:
    \begin{equation*}
        \begin{pmatrix}
            \alpha_1&\beta_1\\-\overline{\beta_1}&\overline{\alpha_1}
        \end{pmatrix}
        \begin{pmatrix}
            \alpha_2&\beta_2\\-\overline{\beta_2}&\overline{\alpha_2}
        \end{pmatrix}
        =\begin{pmatrix}
            \alpha_1\alpha_2-\beta_1\overline{\beta_2}&\alpha_1\beta_2+\overline{\alpha_2}\beta_1\\
            -\overline{\alpha_1\beta_2+\overline{\alpha_2}\beta_1}&\overline{\alpha_1\alpha_2-\beta_1\overline{\beta_2}}
        \end{pmatrix}\in \mathcal{H}
    \end{equation*}

    则$\mathcal{H}$为$(M_2(\mathbb{C}),+,\cdot)$的一个含幺子环。

    逆元($\mathcal{H}\text{中非零元}\iff \alpha\neq 0\text{ 且 }\ \beta\neq 0\iff\alpha\bar{\alpha}+\beta\bar{\beta}\neq 0$):
    \begin{equation*}
        \begin{pmatrix}
            \alpha&\beta\\-\bar{\beta}&\bar{\alpha}
        \end{pmatrix}\cdot
        \frac{1}{\alpha\bar{\alpha}+\beta\bar{\beta}}\begin{pmatrix}
            \bar{\alpha}&-\beta\\
            \bar{\beta}&\alpha
        \end{pmatrix}=\begin{pmatrix}
            1&0\\0&1
        \end{pmatrix}
    \end{equation*}

    综上,$\mathcal{H}$为除环。

    令 $\alpha = a+bi, \beta = c+di$,则
    \begin{equation*}
        \begin{aligned}
            \begin{pmatrix}
                a+bi&c+di\\-c+di&a-bi
            \end{pmatrix}
            =&\ \begin{pmatrix}
                a&c\\-c&a
            \end{pmatrix}
            +\begin{pmatrix}
                b&d\\d&-b
            \end{pmatrix}
            i\\
            =&\ a\begin{pmatrix}
                1&0\\0&1
            \end{pmatrix}
            +b\begin{pmatrix}
                i&0\\0&-i
            \end{pmatrix}
            +c\begin{pmatrix}
                0&1\\-1&0
            \end{pmatrix}
            +d\begin{pmatrix}
                0&i\\i&0
            \end{pmatrix}
        \end{aligned}
    \end{equation*}

    构造映射:
    \begin{equation*}
        \begin{aligned}
            \psi:\mathbf{H}&\rightarrow \mathcal{H} \\
            a+bi+cj+dk&\mapsto 
            \ a\begin{pmatrix}
                1&0\\0&1
            \end{pmatrix}
            +b\begin{pmatrix}
                i&0\\0&-i
            \end{pmatrix}
            +c\begin{pmatrix}
                0&1\\-1&0
            \end{pmatrix}
            +d\begin{pmatrix}
                0&i\\i&0
            \end{pmatrix}
        \end{aligned}
    \end{equation*}

    下面验证$\psi$保乘法运算(只需要验证生成元之间的关系即可):

    \begin{equation*}
        \begin{aligned}
            &\begin{pmatrix}
                i&0\\0&-i
            \end{pmatrix}
            ^2=\begin{pmatrix}
                0&1\\-1&0
            \end{pmatrix}
            ^2=\begin{pmatrix}
                0&i\\i&0
            \end{pmatrix}
            ^2=-\begin{pmatrix}
                1&0\\0&1
            \end{pmatrix}\\
            \Rightarrow&\ \psi(i)^2=\psi(j)^2=\psi(k)^2=-\psi(1)=\psi(-1)=\psi(i^2)=\psi(j^2)=\psi(k^2)\\
            \Rightarrow&\ \psi(i)^2=\psi(i^2),\ \psi(j)^2=\psi(j^2),\ \psi(k)^2=\psi(k^2)
        \end{aligned}
    \end{equation*}
    \begin{equation*}
        \begin{aligned}
            &\begin{pmatrix}
                i&0\\0&-i
            \end{pmatrix}\begin{pmatrix}
                0&1\\-1&0
            \end{pmatrix}
            =\begin{pmatrix}
                0&i\\i&0
            \end{pmatrix}=-\begin{pmatrix}
                0&1\\-1&0
            \end{pmatrix}
            \begin{pmatrix}
                i&0\\0&-i
            \end{pmatrix}\\
            \Rightarrow&\ \psi(i)\psi(j)=\psi(k)=-\psi(j)\psi(i)\\
            \Rightarrow&\ \psi(i)\psi(j)=\psi(ij),\ -\psi(j)\psi(i)=\psi(-ji)
        \end{aligned}
    \end{equation*}
    \begin{equation*}
        \begin{aligned}
            &\begin{pmatrix}
                0&1\\-1&0
            \end{pmatrix}\begin{pmatrix}
                0&i\\i&0
            \end{pmatrix}
            =\begin{pmatrix}
                i&0\\0&-i
            \end{pmatrix}=-\begin{pmatrix}
                0&i\\i&0
            \end{pmatrix}
            \begin{pmatrix}
                0&1\\-1&0
            \end{pmatrix}\\
            \Rightarrow&\ \psi(j)\psi(k)=\psi(i)=-\psi(k)\psi(j)\\
            \Rightarrow&\ \psi(j)\psi(k)=\psi(jk),\ -\psi(k)\psi(j)=\psi(-kj)
        \end{aligned}
    \end{equation*}
    \begin{equation*}
        \begin{aligned}
            &\begin{pmatrix}
                0&i\\i&0
            \end{pmatrix}\begin{pmatrix}
                i&0\\0&-i
            \end{pmatrix}
            =\begin{pmatrix}
                0&1\\-1&0
            \end{pmatrix}=-\begin{pmatrix}
                i&0\\0&-i
            \end{pmatrix}
            \begin{pmatrix}
                0&i\\i&0
            \end{pmatrix}\\
            \Rightarrow&\ \psi(k)\psi(i)=\psi(j)=-\psi(i)\psi(k)\\
            \Rightarrow&\ \psi(k)\psi(i)=\psi(ki),\ -\psi(i)\psi(k)=\psi(-ik)
        \end{aligned}
    \end{equation*}

    综上,$\psi$保持乘法运算,不难看出,$\psi$保持加法运算,根据$\psi$的定义,知$\psi$为满射,所以$\psi$为环同态,故$\mathbf{H}\cong\mathcal{H}$。
\end{proof}
\end{document}
