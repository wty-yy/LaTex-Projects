\documentclass[12pt, a4paper, oneside]{ctexart}
\usepackage{amsmath, amsthm, amssymb, bm, color, graphicx, geometry, hyperref, mathrsfs,extarrows, braket}

\linespread{1.5}
\geometry{left=2.54cm,right=2.54cm,top=3.18cm,bottom=3.18cm}
\newenvironment{problem}{\par\noindent\textbf{题目. }}{\bigskip\par}
\newenvironment{solution}{\par\noindent\textbf{解答. }}{\bigskip\par}
\newenvironment{note}{\par\noindent\textbf{注记. }}{\bigskip\par}

% 基本信息
\newcommand{\dt}{\today}
\newcommand{\sj}{近世代数}
\newcommand{\vt}{吴天阳 2204210460}

\begin{document}

\pagestyle{empty}
\vspace*{-20ex}
\centerline{\begin{tabular}{*3{c}}
    \parbox[t]{0.3\linewidth}{\begin{center}\textbf{日期}\\ \large \textcolor{blue}{\dt}\end{center}} 
    & \parbox[t]{0.3\linewidth}{\begin{center}\textbf{科目}\\ \large \textcolor{blue}{\sj}\end{center}}
    & \parbox[t]{0.3\linewidth}{\begin{center}\textbf{姓名,学号}\\ \large \textcolor{blue}{\vt}\end{center}} \\ \hline
\end{tabular}}
\vspace*{4ex}
\paragraph{习题 2.2}
\paragraph{3.}如果环$R$中的元素$a$有一个正整数$n$,使得$a^n = 0$,那么称$a$是\textbf{幂零元}。证明:如果$a$是有单位元的环$R$中的一个幂零元,那么$1-a$可逆。
\begin{proof}
    由于
    \begin{equation*}
        \begin{aligned}
            &(1-a)(1+a+a^2+\cdots+a^{n-1}) = 1-a^n = 1\\
            &(1+a+a^2+\cdots+a^{n-1})(1-a) = 1-a^n = 1
        \end{aligned}
    \end{equation*}

    所以$1-a$的逆元为$1+a+a^2+\cdots+a^{n-1}$,故$1-a$可逆。
\end{proof}
\paragraph{5.}设$I_1,I_2,\cdots, I_s$都是环$R$的理想,并且
\begin{equation*}
    \begin{aligned}
        R &= I_1+I_2+\cdots+I_s,\\
        I_i\cap (\sum_{j\neq i} &\ I_j) = (0),\quad i = 1, 2, \cdots, s.
    \end{aligned}
\end{equation*}
证明:(1) 环$R$的每个元素$x$都可以唯一表示成
\begin{equation*}
    x = x_1+x_2+\cdots+x_s,\quad x_i\in I_i,i = 1, 2,\cdots, s;
\end{equation*}

(2) 有环同构
\begin{equation*}
    R\cong I_1\oplus I_2\oplus\cdots\oplus I_s,
\end{equation*}
此时称$R$是它的理想$I_1,I_2,\cdots,I_s$的\textbf{内直和}。
\begin{proof}
    (1). 由于$R = I_1+I_2+\cdots+I_s$,则$\forall x \in R,\ \exists x_i\in I_i,\  i=1,2,\cdots,s$,使得
    \begin{equation*}
        x = x_1+x_2+\cdots+x_s
    \end{equation*}
    下证唯一性,若$x$有两种表示法,$x = x_1+x_2+\cdots+x_s = x_1'+x_2'+\cdots+x_s'$,不妨令$x_1\neq x_1'$,则
    \begin{equation*}
        x_1-x_1' = (x_2'-x_2)+(x_3'+x_3)+\cdots+(x_s'-x_s)
    \end{equation*}
    又
    \begin{equation*}
        \begin{cases}
            x_1-x_1'\in I_1\\
            x_2'-x_2\in I_2\\
            \quad\quad\vdots\\
            x_s'-x_s\in I_s
        \end{cases}
    \end{equation*}

    若$x_i=x_i'\quad \forall i\geqslant 2$,则$x_1-x_1' = 0\Rightarrow x_1=x_1'$,矛盾。

    若$x_i\neq x_i'\quad \exists i\geqslant 2$,则
    \begin{equation*}
        (x_2'-x_2)+\cdots+(x_s'-x_s) = x_1-x_1'\in I_1\cap(\sum_{j\neq 1}I_j)\neq (0)
    \end{equation*}

    与$I_1\cap(\sum_{j\neq 1}I_j)= (0)$矛盾。

    综上,$\forall x\in R$,可以唯一表示成
    \begin{equation*}
        x = x_1+x_2+\cdots+x_s,\quad x_i\in I_i,i = 1, 2,\cdots, s;
    \end{equation*}

    (2). 由群直和性质知,
    \begin{equation*}
        (R,+)\cong(I_1,+)\oplus(I_2,+)\oplus\cdots\oplus(I_s,+)\cong(I_1\oplus I_2\oplus\cdots\oplus I_s,+)
    \end{equation*}

    其对应的群同构$\sigma$为:

    \begin{equation*}
        \begin{aligned}
            R&\rightarrow I_1\oplus I_2\oplus\cdots\oplus I_s\\
            x&\mapsto (x_1,x_2,\cdots,x_s)\quad \text{其中}x_i\in I_i
        \end{aligned}
    \end{equation*}

    下证$\sigma$对乘法保序,对$\forall i\neq j$,有
    \begin{equation*}
        x_ix_j\in I_iI_j\subset I_i\cap I_j\subset I_i\cap (\sum_{j\neq i}I_j)=(0)
    \end{equation*}

    则$x_ix_j = 0\ (\forall i\neq j)$,设$x,y\in R,\ x = x_1+x_2\cdots+x_s,\ y = y_1+y_2+\cdots+y_s$,则
    \begin{equation*}
        \begin{aligned}
            \sigma(xy) =&\ \sigma((x_1+\cdots+x_s)(y_1+\cdots+y_s))\\
            =&\ \sigma(x_1y_1+\cdots+x_sy_s)\\
            =&\ (x_1y_1,\cdots,x_sy_s)\\
            =&\ \sigma(x)\sigma(y)
        \end{aligned}
    \end{equation*}

    综上,$\sigma$为环同构,故$R\cong I_1\oplus I_2\oplus\cdots\oplus I_s$。
\end{proof}
\paragraph{7.}韩信点兵问题:“有一队士兵,三三数余二,五五数余一,七七数余四,问:这队士兵有多少人?”
\begin{solution}
    该问题等价于求解同余方程:
    \begin{equation*}
        \begin{cases}
            x\equiv 2\pmod 3\\
            x\equiv 1\pmod 5\\
            x\equiv 4\pmod 7
        \end{cases}
    \end{equation*}
    由于
    \begin{equation*}
        \begin{aligned}
            70\equiv 1\pmod 3,&\quad 70\equiv 0\pmod{35}\\
            21\equiv 1\pmod 5,&\quad 21\equiv 0\pmod{21}\\
            15\equiv 1\pmod 7,&\quad 15\equiv 0\pmod{15}
        \end{aligned}
    \end{equation*}
    则解为
    \begin{equation*}
        x\equiv 2\cdot 70 + 1\cdot 21 + 4\cdot 15\equiv 11\pmod{105}
    \end{equation*}

    综上,这队士兵人数为$11+105k\ (k\in \mathbb{Z}_{\geqslant 0})$。
\end{solution}
\paragraph{8.}在$\mathbb{Z}_{91}$中,求$\bar{1}$的全部平方根。
\begin{solution}
     由于$91 = 7\cdot 13$,该问题等价于求解如下同余方程:
     \begin{equation*}
         \begin{cases}
             x\equiv \pm 1\pmod{7}\\
             x\equiv \pm 1\pmod{13}
         \end{cases}
     \end{equation*}
     又由于
     \begin{equation*}
         \begin{aligned}
            &78\equiv 1\pmod 7,\ 78\equiv 0\pmod{13}\\
            &14\equiv 1\pmod{13},\ 14\equiv 0\pmod 7
         \end{aligned}
     \end{equation*}
     则
     \begin{equation*}
        \begin{aligned}
            &x\equiv 78+14\equiv 1\pmod{91}\\
            &x\equiv 14-78\equiv 27\pmod{91}\\
            &x\equiv 78-14\equiv 64\pmod{91}\\
            &x\equiv -78-14\equiv 90\pmod{91}\\
        \end{aligned}
     \end{equation*}

     综上,$\mathbb{Z}_{91}$中$\bar{1}$的全部平方根为$\bar{1},\overline{27},\overline{64},\overline{90}$。
\end{solution}
\paragraph{习题 2.3}
\paragraph{1.} 设$F$是一个代数封闭域(即$F[x]$中每一个不可约多项式都是一次多项式),求$F[x]$的全部素理想。
\begin{solution}
    由于$F[x]$为主理想整环,则
    \begin{equation*}
        P\text{为}F[x]\text{的素理想}\iff P = (p(x))\text{或}(0)
    \end{equation*}
    其中$p(x)$为$F[x]$中的不可约多项式,则$F[x]$的全部素理想为
    \begin{equation*}
        (0),\  x+c
    \end{equation*}
    其中$c\in F$。
\end{solution}
\paragraph{4.} 设$m=p_1^{r_1}p_2^{r_2}\cdots p_s^{r_s}$,其中$p_1,p_2,\cdots,p_s$是两两不等的素数,$r_i>0,i=1,2,\cdots,s$。求$\mathbb{Z}/(m)$的全部素理想。
\begin{solution}
    由理想对应定理知,
    \begin{equation*}
        \{I:I\text{为}\mathbb{Z}/(m)\text{的理想}\}\cong\{I\text{为}\mathbb{Z}\text{的理想}:(m)\subset I\}\cong\{(k):(m)\subset (k),\ k\in\mathbb{N}\}
    \end{equation*}
    设$\mathbb{Z}/(k)$为$\mathbb{Z}/(m)$的素理想,由环同构第二定理知
    \begin{equation*}
        (\mathbb{Z}/(m))/((k)/(m))\cong \mathbb{Z}/(k)
    \end{equation*}
    则$\mathbb{Z}/(k)$为整环$\iff$$(k)$为$\mathbb{Z}$的素理想$\iff$$k$为素数。

    所以$\mathbb{Z}/(k)$为$\mathbb{Z}/(m)$的素理想,当且仅当,$k$为素数且$(m)\subset (k)\iff k|m$。

    综上,$\mathbb{Z}/(m)$的所有素理想为
    \begin{equation*}
        \mathbb{Z}/(p_i)\quad i=1,2,\cdots,s
    \end{equation*}
\end{solution}
\paragraph{10.}设$R$是有单位元$1(\neq 0)$的交换环,证明:$R$的极大理想一定是素理想。
\begin{proof}
    \begin{equation*}
        M\text{为}R\text{的极大理想}\iff R/M\text{为域}\Rightarrow R/M\text{为整环}\iff M\text{为}R\text{的素理想}
    \end{equation*}
\end{proof}
\paragraph{12.}设$R$是偶数环$2\mathbb{Z}$,证明:$4\mathbb{Z}$是$R$的一个极大理想,但是$R/4\mathbb{Z}$不是域。
\begin{proof}
    存在$I$为$2\mathbb{Z}$的理想,且$4\mathbb{Z}\subsetneqq I$,则存在$k\in \mathbb{Z}$,使得$2 (2k+1)=4k+2\in I$,由于$4k\in4\mathbb{Z}\subset I$,则$2\in I$,所以$\forall k\in \mathbb{Z}$,有$2k\in I$,则$I= 2\mathbb{Z} = R$,故$4\mathbb{Z}$为$R$的极大理想。

    由于$R/4\mathbb{Z} = \{4\mathbb{Z},2+4\mathbb{Z}\}$,其中$4\mathbb{Z}$为$R/4\mathbb{Z}$中的零元,则$(2+4\mathbb{Z})(2+4\mathbb{Z}) = 4\mathbb{Z}$,则$2+4\mathbb{Z}$为$R/4\mathbb{Z}$中的非零的零因子,故$R/4\mathbb{Z}$不是域。
\end{proof}
\paragraph{习题 2.4}
\paragraph{1.}构造含$9$个元素的有限域,写出它的全部元素。
\begin{solution}
    $\mathbb{Z}_3$为含有$3$个元素的有限域,令$m(x) = x^2+1$,由于$m(\bar{0})=\bar{1},m(\bar{1})=\bar{2},m(\bar{2})=\bar{2}$,所以$m(x)$是$\mathbb{Z}_3[x]$上的不可约多项式,则$\mathbb{Z}_3[x]/(m(x))$为含有$3^2=9$个元素的有限域。

    如下定义$\mathbb{Z}_3$到$\mathbb{Z}_3[x]/(m(x))$上的映射$\sigma$:
    \begin{equation*}
        \begin{aligned}
            \sigma:\mathbb{Z}_3&\rightarrow \mathbb{Z}_3[x]/(m(x))\\
            \bar{a}&\mapsto \bar{a}+(m(x))
        \end{aligned}
    \end{equation*}
    不难验证,$\sigma$为单同态,所以可以在$\mathbb{Z}_3[x]/(m(x))$中将$\bar{a}$与$\bar{a}+(m(x))$视为相同的元素,记$u = x+(m(x))$,则
    \begin{equation*}
        \begin{aligned}
            \mathbb{Z}_3[x]/(m(x)) =&\ \{c_0+c_1u:c_0,c_i\in\mathbb{Z}_3\}\\
            =&\ \{\bar{0},\bar{1},\bar{2},u,\bar{1}+u,\bar{2}+u, \bar{2}u, \bar{1}+\bar{2}u,\bar{2}+\bar{2}u\}
        \end{aligned}
    \end{equation*}
\end{solution}
\paragraph{5.}证明$t=\sqrt{2}+\sqrt{3}$是一个代数数,并且求$t$在$\mathbb{Q}$上的极小多项式。
\begin{proof}
    由于
    \begin{equation*}
    t^2 = 5 + 2\sqrt{6}\Rightarrow (t^2-5)^2=24\Rightarrow t^4-10t^2+1=0
    \end{equation*}

    所以$t$为代数数,设$m(x) = x^4-10x^2+1 = (x^2)^2-10x^2+1$,将$m(x)$视为$\mathbb{R}$上的多项式,则$x^2=5\pm 2\sqrt{6}=(\sqrt{2}+\sqrt{3})^2\text{或}(\sqrt{2}-\sqrt{3})^2$,所以
    \begin{equation*}
        m(x) = (x-(\sqrt{2}+\sqrt{3}))(x-(-\sqrt{2}-\sqrt{3}))(x-(\sqrt{2}-\sqrt{3}))(x-(\sqrt{3}-\sqrt{2}))
    \end{equation*}

    则$m(x)$在$\mathbb{Q}$中没有一次或二次多项式作为因子,故$m(x)$不可约且是首一多项式,则$m(x)$为$t$在$\mathbb{Q}$上的极小多项式。
\end{proof}
\newpage
\paragraph{11.}证明:对于任意整数$m,n$,复数$m+ni$是代数整数,称这种形式的代数整数为\textbf{高斯整数}。
\begin{proof}
    设$t = m+ni$,则
    \begin{equation*}
        (t-m)^2=-n^2\Rightarrow t^2-2mt+m^2+n^2=0
    \end{equation*}
则复数$m+ni$为整系数多项式$x^2-2mx+m^2+n^2$的根,则$m+ni$为代数整数。
\end{proof}
\end{document}
