\documentclass[12pt, a4paper, oneside]{ctexart}
\usepackage{amsmath, amsthm, amssymb, bm, color, graphicx, geometry, hyperref, mathrsfs,extarrows, braket}

\linespread{1.5}
\geometry{left=2.54cm,right=2.54cm,top=3.18cm,bottom=3.18cm}
\newenvironment{problem}{\par\noindent\textbf{题目. }}{\bigskip\par}
\newenvironment{solution}{\par\noindent\textbf{解答. }}{\bigskip\par}
\newenvironment{note}{\par\noindent\textbf{注记. }}{\bigskip\par}

% 基本信息
\newcommand{\dt}{\today}
\newcommand{\sj}{近世代数}
\newcommand{\vt}{吴天阳 2204210460}

\begin{document}

\pagestyle{empty}
\vspace*{-20ex}
\centerline{\begin{tabular}{*3{c}}
    \parbox[t]{0.3\linewidth}{\begin{center}\textbf{日期}\\ \large \textcolor{blue}{\dt}\end{center}} 
    & \parbox[t]{0.3\linewidth}{\begin{center}\textbf{科目}\\ \large \textcolor{blue}{\sj}\end{center}}
    & \parbox[t]{0.3\linewidth}{\begin{center}\textbf{姓名,学号}\\ \large \textcolor{blue}{\vt}\end{center}} \\ \hline
\end{tabular}}
\vspace*{4ex}

\paragraph{习题 3.1}
\paragraph{1.}证明:$\mathbb{Z}[x]$是一个整环,并且$x^2+5$是$\mathbb{Z}[x]$的一个素元。
\begin{proof}
    由于$\mathbb{Z}[x]$为$\mathbb{Q}[x]$的一个子环,则$\mathbb{Z}[x]$对$+$和$\cdot$满足封闭性,分配律和交换律,$0$为$\mathbb{Z}[x]$的零元,且$1\in \mathbb{Z}[x]$,则$\mathbb{Z}[x]$有幺元,由于$\mathbb{Q}[x]$没有非零的零因子,所以$\mathbb{Z}[x]$中也没有非零的零因子,综上$\mathbb{Z}[x]$是一个整环。

    由于$x^2+5$中$(1,5)=1$,所以$x^2+5$是本原多项式,又由于方程$x^2+5=0$的根为$x=\pm \sqrt{5}\notin \mathbb{Z}$,所以$x^2+5$是不可约多项式,对于不可约的本原多项式,任意的$f(x),g(x)\in\mathbb{Z}(x)$,有
    \begin{equation*}
        (x^2+5)|f(x)\cdot g(x)\Rightarrow (x^2+5)|f(x)\text{或}(x^2+5)|g(x)
    \end{equation*}

    所以,$x^2+5$是$\mathbb{Z}[x]$的一个素元。
\end{proof}

\paragraph{8.}证明:$\mathbb{Z}[x]/(x^2+5)\cong\mathbb{Z}[\sqrt{5}i]$。

\begin{proof}
    证明该命题需要先证明如下的一个\textbf{引理}(整多项式环上首一多项式的带余除法):
    \paragraph{引理.}设$f(x),m(x)\in\mathbb{Z}[x]$,其中$m(x)$为首项系数为$1$的多项式且$\text{deg}\,m(x)\geqslant 1$,则存在唯一的一对$h(x),r(x)\in\mathbb{Z}[x]$,使得
    \begin{equation*}
        f(x)=h(x)m(x)+r(x)\quad\text{deg}\,r(x) < \text{deg}\,m(x)
    \end{equation*}
    下面对$f(x)$的阶用归纳法证明该引理:

    当$\text{deg}\,f(x)=0$时,由于$\text{deg}\,m(x) \geqslant 1$,则存在唯一的$h(x)=0,r(x)=f(x)$,满足命题。

    假设命题在$\text{deg}\,f(x)=n-1$时成立,则当$\text{deg}\,f(x)=n$时,令
    \begin{equation*}
        f(x)=a_nx^n+a_{n-1}x^{n-1}+\cdots+a_1x+a_0
    \end{equation*}

    若$\text{deg}\,m(x) > \text{deg}\,f(x)$,则存在唯一的$h(x) = 0,r(x)=f(x)$,满足命题。

    若$\text{deg}\,m(x)\leqslant \text{deg}\,f(x)$,记$\text{deg}\,m(x) = t$,则
    \begin{equation*}
        \text{deg}\,(f(x)-a_nx^tm(x))\leqslant n-1
    \end{equation*}
    由归纳假设知,存在唯一的一对$h(x),r(x)$,使得
    \begin{equation*}
        \begin{aligned}
            f(x)-a_nx^tm(x)=&\ h(x)m(x)+r(x)&\text{deg}\,r(x) < t\\
            f(x)=&\ (a_nx^t+h(x))m(x)+r(x)
        \end{aligned}
    \end{equation*}
    满足命题。综上,该引理得证。

    构造$\mathbb{Z}[x]$到$C$上的一个同态$\sigma$:

    \begin{equation*}
        \begin{aligned}
            \sigma:\mathbb{Z}[x]&\rightarrow \mathbb{C}\\
            f(x)=\sum_{k=0}^na_kx^k&\mapsto\sum_{k=0}^na_k(\sqrt{5}i)^k=:f(\sqrt{5}i)
        \end{aligned}
    \end{equation*}

    $\sigma$对$+$和$\cdot$保持运算,且$\sigma(1)=1$,所以$\sigma$为一个环同态。

    由于
    \begin{equation*}
        \mathbb{Z}[\sqrt{5}i]=\left\{\sum_{k=0}^na_k(\sqrt{5}i)^k:a_k\in\mathbb{Z},n\in\mathbb{N}\right\}
    \end{equation*}
    所以$\text{Im }\sigma = \mathbb{Z}[\sqrt{5}i]$,又由于
    \begin{equation*}
        \begin{aligned}
            \text{Ker }\sigma = &\{f(x)\in\mathbb{Z}[x]:f(\sqrt{5}i)=0\}\\
            =&\{f(x)\in\mathbb{Z}[x]:\sqrt{5}i\text{为}f(x)\text{的一个复根}\}
        \end{aligned}
    \end{equation*}

    下证$\text{Ker }\sigma = (x^2+5)$,假设存在$f(x)\in\text{Ker }\sigma$使得$(x^2+5)\nmid f(x)$,有\textbf{引理}知,存在唯一的一对$h(x),r(x)\in\mathbb{Z}[x]$,使得
    \begin{equation*}
        f(x) = (x^2+5)h(x)+r(x)\quad \text{deg }r(x) < 2
    \end{equation*}

    因为$\text{deg }r(x) <2$且$r(x)\neq 0$,令$r(x) = ax + b$,$a,b\in \mathbb{Z}$,由于
    \begin{equation*}
        r(x)=f(x)-(x^2+5)h(x)
    \end{equation*}

    则$r(\sqrt{5}i)=0\Rightarrow a\sqrt{5}i+b=0\Rightarrow i=-\dfrac{b}{a\sqrt{5}}\in\mathbb{R}$与$i=\sqrt{-1}\notin\mathbb{R}$矛盾。

    于是$\forall f(x)\in \text{Ker }\sigma$,都有$(x^2+5)|f(x)$,则$\text{Ker }\sigma\subset (x^2+5)$,又因为$\sqrt{5}i$为$x^2+5=0$的根,所以$x^2+5\in\text{Ker }\sigma\Rightarrow(x^2+5)\subset \text{Ker }\sigma$,故$\text{Ker }\sigma = (x^2+\sqrt{5})$。

    由环同态基本定理,知
    \begin{equation*}
        \begin{aligned}
            &\ \mathbb{Z}[x]/\text{Ker }\sigma\cong \text{Im }\sigma\\
            \Rightarrow&\ \mathbb{Z}[x]/(x^2+5)\cong\mathbb{Z}[\sqrt{5}i]
        \end{aligned}
    \end{equation*}
\end{proof}
\paragraph{习题 3.2}
\paragraph{3.}证明:$\mathbb{Z}[x]$不是主理想整环。

\begin{proof}
    反设$\mathbb{Z}[x]$为主理想整环,则
    \begin{equation*}
        a\text{为不可约元}\iff (a)\text{为极大理想}\iff \mathbb{Z}[x]/(a)\text{为域}
    \end{equation*}
    由于$x\in \mathbb{Z}[x]$,且$x$为不可约本原多项式,所以$x$为$\mathbb{Z}[x]$中的不可约元,则$\mathbb{Z}[x]/(x)$为域,但
    \begin{equation*}
        Z[x]/(x)=\{f(x)+(x):f(x)\in\mathbb{Z}[x]\}=\{a+(x):a\in\mathbb{Z}\}\cong \mathbb{Z}
    \end{equation*}
    则$\mathbb{Z}[x]/(x)$同构于整环$\mathbb{Z}$,与$Z[x]/(x)$为域矛盾,故$\mathbb{Z}[x]$不是主理想整环。
\end{proof}

\paragraph{7.}设$m$是一个不含平方因子的整数,且$m\neq 0,1$。证明:$\mathbb{Q}[\sqrt{m}]$是一个域,它的元素形如$a+b\sqrt{m},a,b\in\mathbb{Q}$。把$Q[\sqrt{m}]$,称它为$Q$上的一个\textbf{二次数域}。
\begin{proof}
    由于$m$不含平方因子,设它的标准分解式为
    \begin{equation*}
        m = p_1p_2\cdots p_s
    \end{equation*}
    其中$p_i\ (i=1,2,\cdots,s)$均为素数,由$\text{Eisenstein}$判别法知,素数$p_1$使得多项式$x^2-m$在$\mathbb{Q}[x]$中不可约,且$\sqrt{m}$为该多项式的一个根,所以$x^2-m$为$\sqrt{m}$的极小多项式,则$\mathbb{Q}[x]/(x^2-m)\cong \mathbb{Q}[\sqrt{m}]$是一个域,且
    \begin{equation*}
        \mathbb{Q}[\sqrt{m}] \cong \mathbb{Q}[x]/(x^2-m)=\{a+bu:a,b\in\mathbb{Q},u=x+(x^2-m)\}\cong\{a+b\sqrt{m}:a,b\in\mathbb{Q}\}
    \end{equation*}
\end{proof}

\end{document}
