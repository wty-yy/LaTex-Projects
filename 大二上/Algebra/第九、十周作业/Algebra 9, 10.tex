\documentclass[12pt, a4paper, oneside]{ctexart}
\usepackage{amsmath, amsthm, amssymb, bm, color, graphicx, geometry, hyperref, mathrsfs,extarrows, braket}

\linespread{1.5}
\geometry{left=2.54cm,right=2.54cm,top=3.18cm,bottom=3.18cm}
\newenvironment{problem}{\par\noindent\textbf{题目. }}{\bigskip\par}
\newenvironment{solution}{\par\noindent\textbf{解答. }}{\bigskip\par}
\newenvironment{note}{\par\noindent\textbf{注记. }}{\bigskip\par}

% 基本信息
\newcommand{\dt}{\today}
\newcommand{\sj}{近世代数}
\newcommand{\vt}{吴天阳 2204210460}

\begin{document}

\pagestyle{empty}
\vspace*{-20ex}
\centerline{\begin{tabular}{*3{c}}
    \parbox[t]{0.3\linewidth}{\begin{center}\textbf{日期}\\ \large \textcolor{blue}{\dt}\end{center}} 
    & \parbox[t]{0.3\linewidth}{\begin{center}\textbf{科目}\\ \large \textcolor{blue}{\sj}\end{center}}
    & \parbox[t]{0.3\linewidth}{\begin{center}\textbf{姓名,学号}\\ \large \textcolor{blue}{\vt}\end{center}} \\ \hline
\end{tabular}}
\vspace*{4ex}

\paragraph{习题1.8}
\paragraph{4.}设$F$是一个域,求一般线性群$GL_n(F)$的中心。
\begin{solution}

    $Z(GL_n(F)) = \{A\in GL_n(F):AB=BA,\forall B\in GL_n(F)\}$,即$A$能和所有的$n$阶可逆矩阵可交换。

    设$A=(a_{ij})_{n\times n}$,构造可逆矩阵$B$如下:
    \begin{equation*}
        B=E_n+\begin{bmatrix}
            0&1&0&\cdots&0\\
            0&0&0&\cdots&0\\
            \vdots&\vdots&\vdots&\ddots&\vdots\\
            0&0&0&\cdots&0
        \end{bmatrix}
         = E_n+C
    \end{equation*}
    
    则$AB=BA\Rightarrow A+AC=CA+A\Rightarrow AC=CA$,故$a_{11} = a_{22}$,同理可证$a_{ii}=a_{i+1,i+1}\ (i=1,2,\cdots,n-1)$,所以$A$的对角线上元素都相等。

    构造可逆矩阵$B$如下:
    \begin{equation*}
        B=E_n+\begin{bmatrix}
            1&0&\cdots&0\\0&0&\cdots&0\\
            \vdots&\vdots&\ddots&\vdots\\0&0&\cdots&0
        \end{bmatrix}
        =E_n+D
    \end{equation*}

    则$AB=BA\Rightarrow A+AD=DA+A\Rightarrow AD=DA$,于是
    \begin{equation*}
        AD=\begin{bmatrix}
            a_{11}&0&\cdots&0\\
            a_{21}&0&\cdots&0\\
            \vdots&\vdots&\ddots&\vdots\\
            a_{n1}&0&\cdots&0
        \end{bmatrix}=DA=\begin{bmatrix}
            a_{11}&a_{12}&\cdots&a_{1n}\\0&0&\cdots&0\\
            \vdots&\vdots&\ddots&\vdots\\0&0&\cdots&0
        \end{bmatrix}
    \end{equation*}

    故$a_{12}=\cdots=a_{1n}=a_{21}=\cdots=a_{n1}=0$,

    同理有$a_{k2}=\cdots=a_{kn}=a_{2k}=\cdots=a_{nk}=0\ (k=1,2\cdots,n)$。

    即$A$的非对角线上的元素都是$0$。

    综上,$A$只能为数量矩阵,即$Z(GL_n(F)) = \{kE_n:k\in F^*\}$,其中$E_n$为$GL_n(F)$上的单位阵。
\end{solution}

\paragraph{8.}求$S_n$的中心,其中$n\geqslant 3$。
\begin{solution}
    $\forall \sigma\in S_n$,设$\sigma = (a_1a_2\cdots a_r)$,其中$r\geqslant 2$,下面证明$\sigma\notin \text{Z}(S_n)$:

    当$r=2$,则$\exists a_3\in [1,n]$且$a_3\neq a_1,a_3\neq a_2$,则
    \begin{equation*}
        (a_1a_3)\sigma(a_1a_3)^{-1}=(a_1a_3)(a_1a_2)(a_1a_3)^{-1}=(a_3a_2)\neq \sigma
    \end{equation*}

    当$r\geqslant 3$,则
    \begin{equation*}
        (a_1a_2)\sigma(a_1a_2)^{-1} = (a_2a_1\cdots a_r)=(a_1a_3\cdots a_ra_2)\neq \sigma
    \end{equation*}

    则$\sigma\notin \text{Z}(S_n)$,故$\text{Z}(S_n) = \{(1)\}$。

\end{solution}
\paragraph{12.}设$H$是群$G$的一个子群,如果$G$的每一个自同构都把$H$映成自身,那么称$H$是$G$的\textbf{特征子群},证明:$G$的中心$\text{Z}(G)$和$G$的换位子群$G'$都是$G$的特征子群。
\begin{proof}
    设$\sigma\in\text{Aut}(G)$。
    
    $\forall z\in \text{Z}(G), \forall a\in G$,则
    \begin{equation*}
        \sigma(za)=\sigma(az) = \sigma(z)\sigma(a)=\sigma(a)\sigma(z)
    \end{equation*}

    由$\sigma$为自同构和$a$的任意性知,$\sigma(a)$能遍历$G$中的全体元素,所以$\sigma(z)\in \text{Z}(G)$,则$\sigma(\text{Z}(G)) \subset\text{Z}(G)$,又由于$\sigma$是双射,所以$\sigma(\text{Z}(G)) =\text{Z}(G)$,故$\text{Z}(G)$为$G$的特征子群。

    $\forall x, y\in G$,则
    \begin{equation*}
        \sigma(xyx^{-1}y^{-1}) = \sigma(x)\sigma(y)\sigma(x^{-1})\sigma(y^{-1}) = \sigma(x)\sigma(y)\sigma(x)^{-1}\sigma(y)^{-1}\in G'
    \end{equation*}

    由于$G' = \braket{xyx^{-1}y^{-1}:x,y\in G}$,所以$\sigma(G')\subset G'$,又由于$\sigma$是双射,则$\sigma(G') = G'$,故$G'$是$G$的特征子群。

\end{proof}
\paragraph{16.}求$S_4$的共轭类的个数,以及每个共轭类的代表和元素数目。

\begin{solution}
    
    先证明$S_n$的共轭类个数等于$n$的分拆数,设$\sigma\in S_n$的不相交的轮换分解式为:

    \begin{equation*}
        \sigma=(a_1a_2\cdots a_{l_1})(b_1b_2\cdots b_{l_2})\cdots(q_1q_2\cdots q_{l_t})
    \end{equation*}

    其中$l_1\geqslant l_2\geqslant \cdots\geqslant l_t$且$l_1+l_2+\cdots+l_t = n$,称有序数组$(l_1,l_2,\cdots, l_n)$为$n$的\textbf{分拆},也称为$\sigma$的\textbf{型},下证:如果$\sigma_1,\sigma_2$共轭当且仅当它们同型。

    设$\sigma_1 =(a_1a_2\cdots a_{l_1})\cdots(q_1q_2\cdots q_{l_t})$。

    $\Rightarrow$:由于$\sigma_1,\sigma_2$共轭,则$\exists \tau\in S_n$,使得$\tau\sigma_1\tau^{-1}=\sigma_2$,则
    \begin{equation*}
        \sigma_2 = \tau\sigma_1\tau^{-1}=(\tau(a_1)\tau(a_2)\cdots \tau(a_{l_1}))\cdots(\tau(q_1)\tau(q_2)\cdots \tau(q_{l_t}))
    \end{equation*}
    故$\sigma_2$与$\sigma_1$同型。

    $\Leftarrow$:由于$\sigma_1,\sigma_2$同型,设$\sigma_2 =(b_1b_2\cdots b_{l_1})\cdots(p_1p_2\cdots p_{l_t})$,构造置换$\tau$如下:
    \begin{equation*}
        \tau = \begin{pmatrix}
            a_1&a_2&\cdots&a_{l_1}&\cdots&q_1&q_2&\cdots&q_{l_t}\\
            b_1&b_2&\cdots&b_{l_1}&\cdots&p_1&p_2&\cdots&p_{l_t}\\
        \end{pmatrix}
    \end{equation*}
    则$\tau\sigma_1\tau^{-1} = \sigma_2$,故$\sigma_1,\sigma_2$共轭。

    所以,在$S_n$上的共轭作用下,$\sigma_1,\sigma_2$在同一个共轭类中当且仅当它们的同型,又由于一个同型对应$n$的一个分拆,所以$S_n$的共轭类个数等于$n$的分拆数。

    由于$4 = 3+1 = 2+2 = 2+1+1 = 1+1+1+1$,所以$4$的分拆数为$5$,故$S_4$的共轭类个数为$5$,代表元分别为:
    \begin{equation*}
        (1234)\quad(123)\quad(12)(34)\quad(12)\quad(1)
    \end{equation*}
    通过组合方法,得知其元素数目分别为:
    \begin{equation*}
        \begin{aligned}
            &|C((1234))| = \frac{A_4^4}{4} = 6\\
            &|C((123))| = \frac{A_4^3}{3} = 8\\
            &|C((12)(34))| = \frac{1}{2}\binom{4}{2} = 3\\
            &|C((12))| = \binom{4}{2} = 6\\
            &|C((1))| = 1
        \end{aligned}
    \end{equation*}
\end{solution}
\paragraph{23.}设$G$是一个群,$G$的所有子群组成的集合记做$\Omega$,令
\begin{equation*}
    \begin{aligned}
        G\times\Omega&\rightarrow \Omega\\
        (a,H)&\mapsto aHa^{-1}
    \end{aligned}
\end{equation*}
容易验证这给出了群$G$在$\Omega$上的一个作用,称它为群$G$在子群集合$\Omega$上的\textbf{共轭作用},$H$的$G-\text{轨道}\ G(H)$是由$H$的所有共轭子群组成的,$H$的稳定子群
\begin{equation*}
    G_H = \{g\in G:gHg^{-1} = H\}
\end{equation*}
称为$H$在$G$中的\textbf{正规化子},记做$N_G(H)$,容易看出$H\triangleleft N_G(H)$。证明:如果$G$为有限群,$H<G$,那么$H$的共轭子群的个数等于$[G:N_G(H)]$。
\begin{proof}
    设$G$在它的所有子群组成的集合$\Omega$上的共轭作用,下面证明这是一个作用
    \begin{equation*}
        \begin{aligned}
            &(ab)\circ H = (ab)H(ab)^{-1} = abHb^{-1}a^{-1} = a(b\circ H)a^{-1} = a\circ(b\circ H)\\
            &e\circ H = eHe^{-1} = H
        \end{aligned}
    \end{equation*}

    由$\text{轨道-稳定子定理}$知,$H < G, |O(H)| = [G: G_H] = [G:N_G(H)]$,又由于$|O(H)|$就是$H$的所有共轭子群集合,所以$H$的共轭子群的个数等于$[G:N_G(H)]$。
\end{proof}
\paragraph{28.}设$G$为一个有限群,$p$为$|G|$的最小素因子。证明:指数为$p$的子群(如果存在)必为正规子群。
\begin{proof}
    设$H < G$,且$[G:H] = p$,则$|(G/H)_l| = p$,所以$S_{(G/H)_l}\cong S_p$,构造$G$到$(G/H)_l$上的左平移,则其对应的$G$到$S_{(G/H)_l}$上的群同态$\psi$为:
    \begin{equation*}
        \begin{aligned}
            \psi:G&\rightarrow S_{(G/H)_l}\cong S_p\\
            g&\mapsto \psi(g)(xH):= gxH
        \end{aligned}
    \end{equation*}

    由$\text{群同态基本定理}$知,$G/\text{Ker }\psi \cong \text{Im }\psi\Rightarrow \dfrac{|G|}{|\text{Ker }\psi|}=|\text{Im }\psi|$,又由于$\text{Im }\psi< S_p$,由$\text{Lagrange定理}$知,$|\text{Im }\psi|\biggl||S_p|\Rightarrow |\text{Im }\psi|\biggl|p!$,于是
    \begin{equation*}
        \frac{|G|}{|\text{Ker }\psi|} \biggl | p!
    \end{equation*}
    
    设$|G|$的标准分解为:$|G| = p^{a_0}p_1^{a_1}\cdots p_r^{a_r}$,其中$p<p_1<\cdots<p_r$,且$\alpha_i\geqslant 1$,则$|\text{Ker }\psi|=p^{a_0-1}p_1^{a_1}\cdots p_r^{a_r}=\dfrac{|G|}{p}$,或者$|\text{Ker }\psi|=|G|$。

    由于
    \begin{equation*}
        \begin{aligned}
        \text{Ker }\psi =& \{a\in G:axH = xH,\forall x\in G\} \\
        =& \{a\in G:x^{-1}ax\in H,\forall x\in G\}\\
        =& \{a\in G:a \in xHx^{-1},\forall x\in G\} \\
        =& \bigcap_{x\in G}xHx^{-1}
        \end{aligned}
    \end{equation*}

    所以$\text{Ker }\psi\subset H\Rightarrow |\text{Ker }\psi| \leqslant |H|$,又由于$[G:H]=\dfrac{|G|}{|H|} = p$,所以$|\text{Ker }\psi|\leqslant \dfrac{|G|}{p}$,故$|\text{Ker }\psi| = \dfrac{|G|}{p} = |H|$,于是
    \begin{equation*}
        \begin{aligned}
            &\text{Ker }\psi = \bigcap_{x\in G}xHx^{-1} = H\\
            \Rightarrow &\forall x\in G,xHx^{-1}\in H\\
            \Rightarrow &H\text{为正规子群}
        \end{aligned}
    \end{equation*}
\end{proof}
\paragraph{29.}设群$G$在集合$\Omega$和$\Omega'$上分别有一个作用“$\circ$”和“$\cdot$”,如果$\Omega$到$\Omega'$有一个双射$\sigma$,使得
\begin{equation*}
    \sigma(a\circ x) = a\cdot(\sigma(x)) ,\quad a\in G,\forall x\in \Omega
\end{equation*}
那么称群$G$的这两个作用是\textbf{等价的}。证明:群$G$在任一集合$\Omega$上的传递作用等价于群$G$在左商集$(G/G_x)_l$上的左平移,其中$x\in G$。
\begin{proof}
    设$\Omega' = (G/G_x)_l$,“$\circ$”是$G$在$\Omega$上的传递作用,“$\cdot$”是$G$在$(G/G_x)_l$上的左平移,即$g\cdot aG_x = gaG_x$,由于$\circ$是传递作用,所以由“$\circ$”运算所定义的轨道$G(x)=\Omega$,利用证明$\text{轨道-稳定子定理}$时,所构造的双射$\sigma$:
    \begin{equation*}
        \begin{aligned}
            \sigma:G(x)&\leftrightarrow (G/G_x)_l\\
            a\circ x&\mapsto aG_x
        \end{aligned}
    \end{equation*}

    则$\sigma$是$\Omega$到$\Omega'$的双射,且$\forall y\in \Omega,\ \exists b\in G$,使得$b\circ x = y$,则
    \begin{equation*}
        \begin{aligned}
            a\cdot(\sigma(y)) =& a\cdot(\sigma(b\circ x))\\
            =& a\cdot bG_x\\
            =& abG_x\\
            =& \sigma((ab)\circ x)\\
            =& \sigma(a\circ(b\circ x))\\
            =& \sigma(a\circ y)
        \end{aligned}
    \end{equation*}
    由等价作用的定义知,“$\circ$”和“$\cdot$”等价。

\end{proof}
\paragraph{33.}证明:$n$阶循环群$G=\braket{a}$的自同构群同构于$\mathbb{Z}_n^*$。
\begin{proof}
    由于$\mathbb{Z}_n^*$中的元素都和$n$互素,所以$|\mathbb{Z}_n^*|=\varphi(n)$,设$\mathbb{Z}_n^* = \{\overline{b_1},\overline{b_2},\cdots,\overline{b_{\varphi(n)}}\}$。

    由于$a$为群$G$的生成元,则$|a| = n$,又由于
    \begin{equation*}
        |a^k| = \frac{|a|}{(k, |a|)} = \frac{n}{(k, n)}
    \end{equation*}
    所以$G$中的生成元$|a^k|=n$当且仅当$(k, n) = 1$,所以$G$中所有生成元为
    \begin{equation*}
        \{a^{b_1},a^{b_2},\cdots,a^{b_\varphi(n)}\}
    \end{equation*}
    由于自同构将同阶元映射到同阶元上,而且如果定义了$G$的生成元上的映射,相当于对$G$中每个元素都做出了定义,所以只需讨论$G$的同阶生成元上的置换即可,又由于$G$是循环群(可以通过一个元素生成),故$G$上的全体自同构个数有且仅有$\varphi(n)$个,分别记为$\sigma_i(a) = a^{b_i},\ (i = 1, 2,\cdots, \varphi(n))$,则
    \begin{equation*}
        \text{Aut}(G) = \{\sigma_1,\sigma_2,\cdots,\sigma_{\varphi(n)}\}
    \end{equation*}

    构造映射$\psi$如下:
    \begin{equation*}
        \begin{aligned}
            \psi:\text{Aut}(G)&\rightarrow \mathbb{Z}_n^*\\
            \sigma_i(a)&\mapsto \overline{b_i}\quad (i = 1, 2, \cdots, \varphi(n))
        \end{aligned}
    \end{equation*}

    由于$|\text{Aut}(G)| = \varphi(n) = |\mathbb{Z}_n^*|$,通过定义可以看出$\psi$为双射,下面证明$\psi$保运算

    $\forall i, j\in \{1, 2, \cdots, \varphi(n)\}$,有
    \begin{equation*}
        \sigma_i\sigma_j(a) = \sigma_i(a^{b_j}) = (\sigma_i(a))^{b_j} = a^{b_ib_j}
    \end{equation*}

    由于$b_i,b_j$和$n$互素,所以$b_ib_j$也和$n$互素,则$\exists k\in \{1, 2, \cdots, \varphi(n)\}$,使得
    \begin{equation*}
        \begin{aligned}
            &b_ib_j\equiv b_k\pmod{n}\\
            \Rightarrow\ &\overline{b_i}\overline{b_j}=\overline{b_k}
        \end{aligned}
    \end{equation*}
    则
    \begin{equation*}
        \sigma_i\sigma_j(a) = a^{b_ib_j} = a^{b_k} = \sigma_k(a)
    \end{equation*}
    故
    \begin{equation*}
        \psi(\sigma_i)\psi(\sigma_j) = \overline{b_i}\overline{b_j}=\overline{b_k}=\psi(\sigma_k)=\psi(\sigma_i\sigma_j)
    \end{equation*}

    综上,$\psi$是$\text{Aut}(G)$到$\mathbb{Z}_n^*$上的群同构映射。

\end{proof}

\paragraph{习题1.9}
\paragraph{3.} 证明:不存在阶为$56$的单群。

\begin{proof}
    设群$G$的阶为$56$,则存在$G$的$\text{Sylow 2-子群 }P$,设$P$的个数有$r$个,由$\text{Sylow第三定理}$知,$r\equiv 1\pmod 2,\ r|7$,则$r$的取值只有$1, 7$。

    1. 当$r=1$时,$P$唯一$\iff P\triangleleft G$,则$G$不为单群。

    2. 当$r=7$时,设它们为
    \begin{equation*}
        \Omega = \{P_i:i=1,2,\cdots,7\}
    \end{equation*}

    构造$G$在$\Omega$上的共轭作用$g\circ P_i = gP_ig^{-1}$,因为$\Omega$上的元素两两共轭,所以这是一个良作用,设其对应的$G$到$\Omega$的变换群上的群同态为$\psi$:
    \begin{equation*}
        \begin{aligned}
            \psi:G&\rightarrow S_{\Omega}\cong S_7\\
            g&\mapsto \psi(g)(P_i):=g\circ P_i
        \end{aligned}
    \end{equation*}

    由$\text{群同态基本定理}$知,
    \begin{equation*}
        G/\text{Ker }\psi\cong\text{Im }\psi\\
    \end{equation*}

    又由于$\text{Im }\psi < S_7\Rightarrow|\text{Im }\psi| \biggl| 7!$,于是
    \begin{equation*}
        \begin{aligned}
            &\frac{2^3\cdot 7}{|\text{Ker }\psi|}=|\text{Im }\psi|\biggl| 7!\\
            \Rightarrow\ &2^2\biggl | |\text{Ker }\psi|\\
            \Rightarrow\ &|\text{Ker }\psi|\neq 0
        \end{aligned}
    \end{equation*}

    又由于$\text{Ker }\psi\triangleleft G$,所以$\text{Ker }\psi$为$G$的非平凡正规子群。

    综上,不存在阶为$56$的单群。

\end{proof}
\paragraph{6.}确定$15$阶群的类型。

\begin{solution}
    设群$G$的阶数为$15$,则$|G|=15=3\cdot 5$,设$\text{Sylow 3-子群}$为$H$,$\text{Sylow 5-子群}$为$P$,下证$H,P$均唯一

    设$\text{Sylow 3-子群}$的个数为$r$个,$\text{Sylow 5-子群}$的个数为$s$个,则
    \begin{equation*}
        \left.\begin{aligned}
            &r\equiv 1\pmod 3\ \text{且}\ r|5\\
            &s\equiv 1\pmod 5\ \text{且}\ s|3
        \end{aligned}\right\}
        \Rightarrow
        r=s=1
    \end{equation*}

    所以$H, P$均唯一,故$H\triangleleft G, P\triangleleft G$,由于$H\cong \mathbb{Z}_3, P\cong \mathbb{Z}_5$,所以$H\cap P = \{e\}$,又由于
    \begin{equation*}
        |HP|=\frac{|H|\cdot|P|}{|H\cap P|} = 15 = |G|
    \end{equation*}

    所以,$G\cong \mathbb{Z}_3\oplus \mathbb{Z}_5\cong \mathbb{Z}_{15}$,故$G$一定为$15$阶循环群。
\end{solution}
\paragraph{10.}设$p,q$是不同的素数,证明:$p^2q$阶群必有一个正规的$\text{Sylow子群}$。

\begin{proof}分为以下两种情形:

    1. 若$p>q$,由于$G$的$\text{Sylow-p子群}$的指数为$q$,为$|G|$的最小的素因数,所以$\text{Sylow-p子群}$为正规子群。

    2. 若$p<q$,设$G$的$\text{Sylow-p子群}$个数为$r$,$\text{Sylow-q子群}$个数为$s$,则有
    \begin{equation*}
        \begin{aligned}
            &r\equiv 1\pmod p\ \text{且}\ r|q\\
            &s\equiv 1\pmod q\ \text{且}\ s|p^2
        \end{aligned}
    \end{equation*}

    若$s=1$,则$\text{Sylow-q子群}$为正规子群。

    若$s=p$,则$p\equiv 1\pmod{q}\Rightarrow p=1$,故$s$不能为$p$。

    若$s=p^2$,则$p^2\equiv 1\pmod{q}$,则$\exists k\in \mathbb{Z}_{\geqslant 1}$,使得$p^2 = kq+1$,则$kq = (p-1)(p+1)$,有$q|(p-1)$或$q|(p+1)$,由于$p\leqslant q-1$,所以$p-1\leqslant q-2,p+1\leqslant q$,故$q=p+1$,所以$p=q-1$,又由于$p,q$均为素数,所以$p=2,q=3$,则$s=4$,下面对$r$进行讨论

    若$r=1$,则$\text{Sylow-p子群}$为正规子群。
    
    若$r=3$,由于$4$阶群一定是循环群,所以$G$中至少有$3\cdot 3+4\cdot 2 + 1 = 18\geqslant |G| = 12$,矛盾,所以$r$不能为$3$。

    综上,$p^2q$阶群必有一个正规的$\text{Sylow子群}$。

\end{proof}
\paragraph{13.}设$G$为一个有限群,$N\triangleleft G$,$P$是$N$的一个$\text{Sylow p-子群}$,证明:
\begin{equation*}
    G=N\cdot \text{N}_G(P)
\end{equation*}
\begin{proof}
    $\forall g\in G$,由于$P < N$,则$gPg^{-1}\subset gNg^{-1} = N$,所以$gPg^{-1}$也是$N$的一个$\text{Sylow p-子群}$,由$\text{Sylow第二定理}$知,$\exists n\in N$,使得$gPg^{-1} = nPn^{-1}$,故
    \begin{equation*}
        \begin{aligned}
            &n^{-1}gPg^{-1}n = P\\
            \Rightarrow\ &n^{-1}gP(n^{-1}g)^{-1} = P\\
            \Rightarrow\ &n^{-1}g\in \text{N}_G(P)\\
            \Rightarrow\ &g\text{N}_G(P)=n\text{N}_G(P)\\
            \Rightarrow\ &g\in N\cdot \text{N}_G(P)\\
            \Rightarrow\ &G\subset N\cdot \text{N}_G(P)
        \end{aligned}
    \end{equation*}
    
    又由于$N\cdot\text{N}_G(P)\subset G$,所以$G=N\cdot\text{N}_G(P)$。
\end{proof}


\end{document}
