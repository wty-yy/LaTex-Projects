\documentclass[12pt, a4paper, oneside]{ctexart}
\usepackage{amsmath, amsthm, amssymb, bm, color, graphicx, geometry, hyperref, mathrsfs,extarrows, braket}

\linespread{1.5}
\geometry{left=2.54cm,right=2.54cm,top=3.18cm,bottom=3.18cm}
\newenvironment{problem}{\par\noindent\textbf{题目. }}{\bigskip\par}
\newenvironment{solution}{\par\noindent\textbf{解答. }}{\bigskip\par}
\newenvironment{note}{\par\noindent\textbf{注记. }}{\bigskip\par}

% 基本信息
\newcommand{\dt}{\today}
\newcommand{\sj}{近世代数}
\newcommand{\vt}{吴天阳 2204210460}

\begin{document}

\pagestyle{empty}
\vspace*{-20ex}
\centerline{\begin{tabular}{*3{c}}
    \parbox[t]{0.3\linewidth}{\begin{center}\textbf{日期}\\ \large \textcolor{blue}{\dt}\end{center}} 
    & \parbox[t]{0.3\linewidth}{\begin{center}\textbf{科目}\\ \large \textcolor{blue}{\sj}\end{center}}
    & \parbox[t]{0.3\linewidth}{\begin{center}\textbf{姓名,学号}\\ \large \textcolor{blue}{\vt}\end{center}} \\ \hline
\end{tabular}}
\vspace*{4ex}
\paragraph{习题1.5}
\paragraph{2.}设$G=\braket{a}$是6阶循环群,把$G$分解成它的两个非平凡子群的内直积。
\begin{solution}
    由$\text{Lagrange定理}$知,子群的阶一定是群的阶的因数,且$|G|=6=1\cdot 6=2\cdot 3$,所以$G$只能分解为两个阶分别为$2, 3$的非平凡子群。

    又由于$\braket{a^3} = \{e, a^3\}, \braket{a^2} = \{e, a^2, a^4\}$,且$\braket{a^3}\cap\braket{a^2} = \{e\}, G = \braket{a^3}\braket{a^2}$,由循环群的性质知,$G$是$\text{Abel群}$,群中元素两两可交换,则
    \begin{equation*}
        G = \braket{a^2}\times\braket{a^3}
    \end{equation*}
\end{solution}
\paragraph{5.}设$G$是$p^m$阶循环群,其中$p$为素数,$m$为正整数,证明:$G$不能分解成它的一些非平凡子群的内直和。
\begin{proof}
    若$G$能分解成它的一些非平凡子群的内直和,则
    \begin{equation*}
        G = H_1\oplus H_2\oplus\cdots\oplus H_n = H_1\oplus K,\ (K = H_2\oplus\cdots\oplus H_n)
    \end{equation*}
    所以$G$至少能被分解为它的两个非平凡子群的内直和。

    反设$G$可以分解为它的两个非平凡子群的内直和,由于$p^m = p^n\cdot p^{m-n}$,由循环群子群的性质,设$G = \braket{a}$,令
    \begin{equation*}
        G = \braket{a^{p^n}}\times\braket{a^{p^{m-n}}},\ (n=1,2,\cdots m-1)
    \end{equation*}
    不妨令$n < m-n$,则$a^{p^n}\in \braket{a^{p^{m-n}}}$,与$\braket{a^{p^n}}\cap\braket{a^{p^{m-n}}}=\{e\}$矛盾。

    故$G$不能分解为它的两个非平凡子群的内直和,由上述讨论知,$G$更不能分解为它的一些非平凡子群的内直和。
\end{proof}
\paragraph{8.}证明:$U_1\cong SO_2$。
\begin{proof}
    设$z\in U_1$,由$U_1$的定义知,$\bar{z}\cdot z = 1$,则$z$在复平面的单位圆上,由$\text{Euler公式}$知
    \begin{equation*}
        U_1 = \{e^{i\theta}:\theta\in[0,2\pi)\}
    \end{equation*}
    由于2阶特殊正交阵都可以写成
    \begin{equation*}
        \begin{bmatrix}
            cos\theta&-\sin\theta\\
            \sin\theta&\cos\theta
        \end{bmatrix}
    \end{equation*}
    所以
    \begin{equation*}
        SO_2 = \left\{
        \begin{bmatrix}
            cos\theta&-\sin\theta\\
            \sin\theta&\cos\theta
        \end{bmatrix}
        :\theta\in[0,2\pi)\right\}
    \end{equation*}
    构造$U_1$到$SO_2$上的映射:
    \begin{equation*}
        \begin{aligned}
            \sigma: U_1&\rightarrow SO_2\\
            e^{i\theta}&\mapsto
            \begin{bmatrix}
                cos\theta&-\sin\theta\\
                \sin\theta&\cos\theta
            \end{bmatrix}
        \end{aligned}
    \end{equation*}
    下证$\sigma$保运算
    \begin{equation*}
        \begin{aligned}
            \sigma(e^{i\theta_1})\sigma(e^{i\theta_2})&=\begin{bmatrix}
                \cos\theta_1&-\sin\theta_1\\
                \sin\theta_1&\cos\theta_1
            \end{bmatrix}\begin{bmatrix}
                \cos\theta_2&-\sin\theta_2\\
                \sin\theta_2&\cos\theta_2
            \end{bmatrix}\\
            &=\begin{bmatrix}
                \cos(\theta_1+\theta_2)&-\sin(\theta_1+\theta_2)\\
                \sin(\theta_1+\theta_2)&\cos(\theta_1+\theta_2)
            \end{bmatrix}\\
            &=\sigma(e^{i(\theta_1+\theta_2)})\\
            &=\sigma(e^{i\theta_1}\cdot e^{i\theta_2})
        \end{aligned}
    \end{equation*}
    由$\sigma$的定义看出,$\sigma$是满同态,所以$\text{Im}\sigma = SO_2$,$\text{Ker}\sigma = \{1\}$。

    由\textbf{群同态基本定理},知
    \begin{equation*}
        U_1\cong U_1/\{1\}\cong SO_2
    \end{equation*}
\end{proof}
\paragraph{习题1.6}
\paragraph{1.}设$f$是实数加法群$(\mathbb R, +)$到非零复数乘法群$\mathbb C^*$的一个映射:
\begin{equation*}
    f(x)=e^{2\pi ix}
\end{equation*}

(1).证明:$f$是一个同态。

(2).求$\text{Ker}f$和$\text{Im}f$。

\begin{solution}

    (1). $\forall x, y\in \mathbb R$,有
    \begin{equation*}
        f(x)\cdot f(y)=e^{2\pi ix}\cdot e^{2\pi iy}=e^{2\pi i(x+y)}=f(x+y)
    \end{equation*}

    则$f$是一个$(\mathbb R, +)\rightarrow (\mathbb C^*, \cdot)$上的同态。

    (2). $\mathbb C^*$中的幺元为$1$,由$\text{Euler}$公式知:$e^{2\pi ix} = \cos(2\pi x) + i\cdot \sin(2\pi x)$,则

    \begin{equation*}
        \begin{aligned}
            &1 = \cos(2\pi x) + i\cdot \sin(2\pi x)\\
            \Rightarrow&\begin{cases}
                \cos(2\pi x) = 1\\
                \sin(2\pi x) = 0
            \end{cases}
            \Rightarrow x\in \mathbb Z
        \end{aligned}
    \end{equation*}

    则 $\text{Ker}f=\mathbb Z$。

    由$\text{Euler}$公式知,$e^{2\pi ix} = cos(2\pi x) + i\cdot \sin(2\pi x)$的周期为$1$,故$\text{Im} f$为复平面上的单位元,$\text{Im}f = \{\cos(2\pi x) + i\cdot\sin(2\pi x):0\leqslant x < 1\}$。
\end{solution}

\paragraph{5.}设$F$是一个域,$\sigma$是$\text{GL}_n(F)$到$F^*$的行列式映射,即$\sigma(A)=|A|$。

(1). 证明:$\sigma$是$\text{GL}_n(F)$到$F^*$的一个群同态;

(2). 求$\text{Ker}\sigma$和$\text{Im}\sigma$;

(3). 证明:$\text{SL}_n(F)\triangleleft \text{GL}_n(F)$;

(4). 证明:$\text{GL}_n(F)/\text{SL}_n(F)\cong F^*$。

\begin{solution}
   
    (1). $\forall A, B\in \text{GL}_n(F)$,有
    \begin{equation*}
        \sigma(A)\sigma(B) = |A|\cdot |B| = |AB| = \sigma(AB)
    \end{equation*}
    则$\sigma$是一个同态。

    (2). $|A|=1$,则 $\{A\in \text{GL}_n(F):|A|=1\}=\text{SL}_n(F)$,则$\text{Ker}\sigma=\text{SL}_n(F)$。

    对$\forall a\in F^*$,则
    \begin{equation*}
        \sigma\left(\begin{bmatrix}
            a&0&\cdots&0\\
            0&1&\ddots&\vdots\\
            \vdots&\ddots&\ddots&0\\
            0&\cdots&0&1
        \end{bmatrix}\right)=\left|\begin{bmatrix}
            a&0&\cdots&0\\
            0&1&\ddots&\vdots\\
            \vdots&\ddots&\ddots&0\\
            0&\cdots&0&1
        \end{bmatrix}\right|=a
    \end{equation*}

    则$\text{Im}\sigma=F^*$。

    (3). 由(2)和\textbf{群同态基本定理}知:$\text{SL}_n(F) = \text{Ker}\sigma\triangleleft \text{GL}_n(F)$。

    (4). 由(2)和\textbf{群同态基本定理}知:$\text{GL}_n(F)/\text{SL}_n(F)\cong F^*$。
\end{solution}
\paragraph{10.}设$G$是一个群,$N\triangleleft G$,$H < G$,如果$G=NH$,且$N\cap H=\{e\}$,那么称$G$可分解成它的正规子群$N$与子群$H$的\textbf{半直积},记做$G=N\rtimes H$,证明:如果$G=N\rtimes H$,那么
\begin{equation*}
    G/H\cong H
\end{equation*}
\begin{proof}
    由于$N\triangleleft G$且$H < G$,由\textbf{第一群同构定理}知:
    \begin{equation*}
        H/H\cap N\cong HN/N\Rightarrow H/\{e\} \cong G/N
    \end{equation*}

    下证$H\cong H/\{e\}$。

    $H/\{e\}=\{h\{e\}:h\in H\}=\{\{h\}:h\in H\}$,构造映射:
    \begin{equation*}
        \begin{aligned}
            \sigma: H/\{e\}&\rightarrow H\\
            \{h\}&\mapsto h
        \end{aligned}
    \end{equation*}
    由于
    \begin{equation*}
        \sigma(\{h_1\}) = \sigma(\{h_2\})\iff h_1 = h_2\iff \{h_1\} = \{h_2\}
    \end{equation*}
    则$\sigma$是单射,从$\sigma$定义看出它是满射,所以$\sigma$是双射,又由于
    \begin{equation*}
        \sigma(\{h_1\})\sigma(\{h_2\}) = h_1h_2 = \sigma(\{h_1h_2\}) = \sigma(\{h_1\}\{h_2\})
    \end{equation*}
    则$\sigma$是$H/\{e\}$到$H$的群同构映射,所以
    \begin{equation*}
        H\cong H/\{e\}\cong G/N
    \end{equation*}
\end{proof}
\paragraph{11.}证明:$S_n$可分解成$A_n$与$\braket{(12)}$的半直积,其中$n\geqslant 3$。
\begin{proof}
    由于$A_n\triangleleft S_n, \braket{(12)} < S_n$,$A_n\cap \braket{(12)} = \{(1)\}$,且
    \begin{equation*}
        A_n\braket{(12)} = A_n\sqcup A_n(12)
    \end{equation*}
    $A_n(12)$表示$S_n$中的所有奇置换组成的集合,由于$S_n$中的置换要么是偶置换要么是奇置换,所以$S_n\subset A_n\braket{(12)}$,又因为$A_n\braket{(12)} < S_n$,所以$S_n = A_n\braket{(12)}$,由半直积定义知
    \begin{equation*}
        S_n = A_n\rtimes \braket{(12)}
    \end{equation*}
\end{proof}
\paragraph{12.}证明:如果置换群$G$含有奇置换,那么$G$必有指数为$2$的子群。
\begin{proof}
    由于$G$可能为无限群,所以不能直接使用11题结论,模仿\textbf{例一},构造$G$到$\{-1, 1\}$对于复数乘法构成的群上的一个映射:
    \begin{equation*}
        \begin{aligned}
           \sigma: G&\rightarrow \{-1, 1\}\\ 
           \text{奇置换}&\mapsto -1\\
           \text{偶置换}&\mapsto 1
        \end{aligned}
    \end{equation*}

    由\textbf{例一}知,$\sigma$是$G$到$\{-1, 1\}$上的满同态,且$\text{Ker}\sigma = \{\text{偶置换}\} = A$由\textbf{群同态基本定理}知
    \begin{equation*}
        G/A \cong \{-1, 1\}
    \end{equation*}
    所以,$|G/A| = [G:A] = |\{-1, 1\}| = 2$,$G$上的所有偶置换构成的集合指数为$2$。
\end{proof}

\paragraph{习题1.7}
\paragraph{1.}分别求$D_3,D_4$的换位子群。
\begin{solution}

    $D_3 = \{\text{I}, \sigma, \sigma^2, \tau, \sigma\tau, \sigma^2\tau\}$,$|D_3| = 6$,其非平凡子群的阶数只能为$2,3$。
    
    设$H=\{\text{I}, \sigma, \sigma^2\}=\braket{\sigma}$,则$\forall \tau\in D_3$,有$\tau\sigma\tau^{-1} = \sigma^{-1} = \sigma^{2}\in H$,则$H\triangleleft D_3$,且$|D_3/H|=|D_3|/|H|=6/3=2$,则$D_3/H\cong \mathbb Z_2$为$\text{Abel群}$,于是$D_3'\subset H$,又由于$|H| = 3=1\cdot 3$,则$H$没有非平凡正规子群,故$D_3' = H$。

    $D_4 = \{\text{I},\sigma,\sigma^2,\sigma^3,\tau,\sigma\tau,\sigma^2\tau,\sigma^3\tau\}=\braket{\sigma}$,$|D_4|=8$,其非平凡子群的阶数只能为$2,4$。

    设$H=\{\text{I}, \sigma,\sigma^2,\sigma^3\}$,则$\forall \tau\in D_4$,有$\tau\sigma\tau^{-1}=\sigma^{-1}=\sigma^3$,则$H\triangleleft D_4$,且$|D_4/H|=|D_4|/|H|=8/4=2$,则$D_4/H\cong\mathbb Z_2$为$\text{Abel群}$,于是$D_4'\subset H$。

    又由于$|H|=4=2\cdot 2$,则$H$的非平凡子群的阶数只能为$2$,设$N=\{\text{I}, \sigma^2\}=\braket{2}$,则$N$是$H$的唯一的非平凡子群,由于
    \begin{equation*}
        \begin{aligned}
            &\sigma\sigma^2\sigma^{-1}=\sigma^3\sigma^2\sigma^{-3}=\sigma^{2}\in N\\
            &\tau\sigma^2\tau^{-1}=\sigma^{-2}=\sigma^2\in N
        \end{aligned}
    \end{equation*}
    则$N\triangleleft D_4$,且$|D_4/H|=|D_4|/|H|=8/2=4$,由于$4$阶群只能为$\text{Abel群}$,故$D_4/H$为$\text{Abel群}$,又由于$N$是$D_4$中最小的正规子群,所以$D_4'=N$。
\end{solution}

\paragraph{3.}求$S_n$的换位子群,其中$n\geqslant 3$。
\begin{solution}
    
    由于$A_n\triangleleft S_n$,且$|S_n/A_n| = |S_n|/|A_n|=2$,则$S_n/A_n\cong \mathbb Z_2$为$\text{Abel群}$,则$S_n'\subset A_n$,由于$A_n$中的元素可以由$3-\text{轮换}$生成,设$(i j k)\in A_n$,则
    \begin{equation*}
        (ijk)=(ikj)^{-1}=(ikj)^2=((ij)(ik))^2=(ij)(ik)(ij)(ik)=(ij)(ik)(ij)^{-1}(ik)^{-1}\subset S_n'
    \end{equation*}
    则$A_n\subset S_n'$,综上$S_n'=A_n$。
\end{solution}

\paragraph{5.}$S_4$是不是可解群?
\begin{solution}
    $S_4$是可解群。
    
    因为$S_4'=A_4,S_4''=V_4,S_4^{(3)}=\{(1)\}$,所以$S_4$是可解群。
\end{solution}

\paragraph{6.}$n\geqslant 5$时,$S_n$是不可解群么?
\begin{solution}
    是的。

    因为$S_n'=A_n$,下面证明,当$n\geqslant 5$时,$A_n'=A_n$。

    由于$A_n'\triangleleft A_n$,只需证明$A_n\subset A_n'$,由于$A_n$可以由$3-\text{轮换}$生成,所以只需证明$3-\text{轮换}$都属于$A_n'$。

    设$(a_1a_2a_3)\in A_n$,则
    \begin{equation*}
        \begin{aligned}
            (a_1a_2a_3) &= (a_1a_4a_3)(a_1a_2a_4) = (a_1a_4a_3)(a_1a_4a_2)^{-1} = (a_1a_4a_3)(a_2a_1a_4)^{-1}\\
            &= (a_5a_2a_1a_4a_3)(a_2a_1a_4)(a_5a_2a_1a_4a_3)^{-1}(a_2a_1a_4)^{-1}\in A_n'
        \end{aligned}
    \end{equation*}
    所以$A_n'=A_n$,则$S_n^{(m)}=A_n\ (m\in\mathbb Z_{\geqslant 1})$,故$n\geqslant 5$时,$S_n$是不可解群。
\end{solution}

\end{document}
