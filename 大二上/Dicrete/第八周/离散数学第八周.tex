\documentclass[12pt, a4paper, oneside]{ctexart}
\usepackage{amsmath, amsthm, amssymb, bm, color, graphicx, geometry, hyperref, mathrsfs,extarrows, braket}

\linespread{1.5}
\geometry{left=2.54cm,right=2.54cm,top=3.18cm,bottom=3.18cm}
\newenvironment{problem}{\par\noindent\textbf{题目. }}{\bigskip\par}
\newenvironment{solution}{\par\noindent\textbf{解答. }}{\bigskip\par}
\newenvironment{note}{\par\noindent\textbf{注记. }}{\bigskip\par}

% 基本信息
\newcommand{\dt}{\today}
\newcommand{\sj}{离散数学}
\newcommand{\vt}{吴天阳 2204210460}

\begin{document}

\pagestyle{empty}
\vspace*{-20ex}
\centerline{\begin{tabular}{*3{c}}
    \parbox[t]{0.3\linewidth}{\begin{center}\textbf{日期}\\ \large \textcolor{blue}{\dt}\end{center}} 
    & \parbox[t]{0.3\linewidth}{\begin{center}\textbf{科目}\\ \large \textcolor{blue}{\sj}\end{center}}
    & \parbox[t]{0.3\linewidth}{\begin{center}\textbf{姓名,学号}\\ \large \textcolor{blue}{\vt}\end{center}} \\ \hline
\end{tabular}}
\vspace*{4ex}

\paragraph{习题六}
\paragraph{14.}\begin{solution}

    (1).不是,因为$(-1,-2),(2,1)\in R$,但$(-1+2,1-2)=(1,-1)\notin R$,则$R$不是同余关系。

    (2).不是,因为$(-5,4)\in R$,但$(-5-5,4+4)=(-10,8)\notin R$,则$R$不是同余关系。

    (3).不是,因为$(1,1),(-1,1)\in R$,但$(1-1,1+1)=(0,2)\notin R$,则$R$不是同余关系。

    (4).不是,因为$(2,1)\in R$,但$(1,2)\notin R$,则$R$不是对称的,所以$R$不是等价关系,更不是同余关系。
\end{solution}
\paragraph{17.}\begin{solution}
    构造$X\rightarrow Y$上的映射$\sigma$如下:
    \begin{equation*}
        \begin{aligned}
            \sigma:\mathbb N&\rightarrow \{0,1\}\\
            n&\mapsto \begin{cases}
                0,&n\text{为偶数};\\
                1,&n\text{为奇数}.
            \end{cases}
        \end{aligned}
    \end{equation*}
    下面证明$\sigma$是$X\rightarrow Y$上的满同态。

    设$n_1,n_2,m_1,m_2\in \mathbb N$,其中$n_1,n_2$为偶数,$m_1,m_2$为奇数,所以
    \begin{equation*}
        \begin{aligned}
            &\sigma(n_1)\sigma(n_2) = 0\times 0 = 0 = \sigma(n_1n_2)\\
            &\sigma(n_1)\sigma(m_1) = 0\times 1 = 0 = \sigma(n_1m_1)\\
            &\sigma(m_1)\sigma(m_2) = 1\times 1 = 1 = \sigma(m_1m_2)
        \end{aligned}
    \end{equation*}
    故$\sigma$保运算,又由于$\sigma(0) = 0,\sigma(1) = 1$,所以$\sigma$是满同态。
    
    综上,$Y$是$X$的同态像。
\end{solution}

\newpage
\paragraph{19.}\begin{proof}
    
    由于$f_1,f_2$是$\braket{X,*}\rightarrow\braket{Y,\oplus}$的同态函数,且$*$和$\oplus$满足交换律和结合律,则
    \begin{equation*}
        \begin{aligned}
            h(x_1)\oplus h(x_2)=&f_1(x_1)\oplus f_2(x_1)\oplus f_1(x_2)\oplus f_2(x_2)\\
            \xlongequal{\text{交换律}}& f_1(x_1)\oplus f_1(x_2)\oplus f_2(x_1)\oplus f_2(x_2)\\
            \xlongequal{\text{结合律}}& (f_1(x_1)\oplus f_1(x_2))\oplus (f_2(x_1)\oplus f_2(x_2))\\
            \xlongequal{\text{保运算}}& f_1(x_1x_2)\oplus f_2(x_1x_2)\\
            =&h(x_1x_2)
        \end{aligned}
    \end{equation*}
\end{proof}
\paragraph{补充题:}

\begin{solution}
    
    由二元运算表知,$\braket{2^A,\cap,\cup},\braket{B,\wedge,\vee}$都是$\text{交换群}$,则
    \begin{equation*}
        \begin{aligned}
            &h(\varnothing)\wedge h(\varnothing)=1\wedge 1=1=h(\varnothing)=h(\varnothing\cup \varnothing)\\
            &h(\varnothing)\vee h(\varnothing) = 1\vee 1 = 1 =h(\varnothing)= h(\varnothing\cap \varnothing)\\
            &h(\varnothing)\wedge h(A) = 1\wedge 0 = 0 = h(A) = h(\varnothing\cup A)\\
            &h(\varnothing)\vee h(A) = 1\vee 0 = 1 = h(\varnothing) = h(\varnothing \cap A)\\
            &h(A)\wedge h(A) = 0\wedge 0 = 0 = h(A)=h(A\cup A)\\
            &h(A)\vee h(A) = 0\vee 0 = 0 = h(A)=h(A\cap A)
        \end{aligned}
    \end{equation*}

    综上,$h$是$\braket{2^A,\cap,\cup}\rightarrow \braket{B,\wedge,\vee}$上的同态函数,并且将运算$\cap$映射为$\vee$,$\cup$映射为$\wedge$。
\end{solution}
\paragraph{23.}\begin{proof}
    设$x,y,z\in S$,则
    \begin{equation*}
        \begin{aligned}
            (x\oplus y)\oplus z=&(x*a*y)\oplus z\\
            =&(x*a*y)*a*z\\
            \xlongequal{*\text{ 运算具有结合律}}&x*a*(y*a*z)\\
            =&x\oplus(y*a*z)\\
            =&x\oplus(y\oplus z)
        \end{aligned}
    \end{equation*}

    故$\oplus$运算满足结合律,所以$\braket{S,\oplus}$是半群。
\end{proof}
\paragraph{27.}\begin{proof}
    (1). 反设$x*x=y$且$y\neq x$,则$x*y\neq y*x$,但
    \begin{equation*}
        y*x=(x*x)*x=x*(x*x)=x*y
    \end{equation*}

    与$x*y\neq y*x$矛盾,所以$x*x=x$。

    (2). 反设$x*y*x=z$且$z\neq x$,则$x*z\neq z*x$,但
    \begin{equation*}
        x*z=x*(x*y*x)=x*y*x=(x*y*x)*x=z*x
    \end{equation*}

    与$x*z\neq z*x$矛盾,所以$x*y*x=x$。

    (3). 反设$x*y*z=a$且$a\neq x*z$,则$x*z*a\neq a*x*z$,但
    \begin{equation*}
        \begin{aligned}
            x*z*a=&x*z*(x*y*z)\\
            =&(x*z*x)*y*z=x*y*z=x*y*(z*x*z)\\
            =&(x*y*z)*x*z=a*x*z
        \end{aligned}
    \end{equation*}

    与$x*z*a\neq a*x*z$矛盾,所以$x*y*z=x*z$。
\end{proof}
\paragraph{29.}\begin{proof}
    (1). 反设$x*y\neq y*x$,不妨令$x*y=x,y*x=y$,则
    \begin{equation*}
        y=x*x=(x*y)*x=x*(y*x)=x*y=x
    \end{equation*}
    
    与$x\neq y$矛盾,另一种情况$x*y=y,y*x=x$,则
    \begin{equation*}
        y=x*x=x*(y*x)=(x*y)*x=y*x=x
    \end{equation*}

    也与$x\neq y$矛盾。

    综上$x*y=y*x$。

    (2). 反设$y*y=x$,不妨令$x*y=y*x=x$,则
    \begin{equation*}
        \begin{aligned}
            &y*y=x\\
            \Rightarrow&x*y*y*x=x*x*x\\
            \Rightarrow&(x*y)*(y*x)=(x*x)*x\\
            \Rightarrow&x*x=y*x\\
            \Rightarrow&y=x
        \end{aligned}
    \end{equation*}
    
    与$x\neq y$矛盾,若$x*y=y*x=y$同理可得出$x=y$矛盾。

    综上$y*y=y$。
\end{proof}
\paragraph{30.}\begin{proof}
    
    令$x=a$,则$\exists u_0,v_0\in S$,使得$a*u_0=v_0*a=a$。

    当$x\neq a$,则$\exists u_1,v_1\in S$,使得$a*u_1=v_1*a=x$,又有
    \begin{equation*}
        \begin{aligned}
            &x*u_0=(v_1*a)*u_0=v_1*(a*u_0)=v_1*a=x\\
            &v_0*x=v_0*(a*u_1)=(v_0*a)*u_1=a*u_1=x
        \end{aligned}
    \end{equation*}
    故,对于$\forall x\in S$,都有$x=x*u_0=v_0*x$,分别令$x=u_0,x=v_0$,得
    \begin{equation*}
        u_0=v_0*u_0=v_0
    \end{equation*}
    令$e=u_0$,则$\forall x\in S$,有$x*e=e*x=x$,所以$\braket{S,*}$是含幺半群。
\end{proof}

\end{document}
