\documentclass[12pt, a4paper, oneside]{ctexart}
\usepackage{amsmath, amsthm, amssymb, bm, cases, color, graphicx, geometry, hyperref, mathrsfs,extarrows, braket}

\linespread{1.5}
\geometry{left=2.54cm,right=2.54cm,top=3.18cm,bottom=3.18cm}
\newenvironment{problem}{\par\noindent\textbf{题目. }}{\bigskip\par}
\newenvironment{solution}{\par\noindent\textbf{解答. }}{\bigskip\par}
\newenvironment{note}{\par\noindent\textbf{注记. }}{\bigskip\par}

% 基本信息
\newcommand{\dt}{\today}
\newcommand{\sj}{离散数学}
\newcommand{\vt}{吴天阳 2204210460}

\begin{document}

\pagestyle{empty}
\vspace*{-20ex}
\centerline{\begin{tabular}{*3{c}}
    \parbox[t]{0.3\linewidth}{\begin{center}\textbf{日期}\\ \large \textcolor{blue}{\dt}\end{center}} 
    & \parbox[t]{0.3\linewidth}{\begin{center}\textbf{科目}\\ \large \textcolor{blue}{\sj}\end{center}}
    & \parbox[t]{0.3\linewidth}{\begin{center}\textbf{姓名,学号}\\ \large \textcolor{blue}{\vt}\end{center}} \\ \hline
\end{tabular}}
\vspace*{4ex}

\paragraph{第六章}

\paragraph{66.}\begin{solution}
    方程组:
    \begin{numcases}{}
        x\oplus(c\otimes y) = a\\
        (c\otimes x) \oplus y = b
    \end{numcases}

    对$(1)$进行变化:

    \begin{equation}
        x = a \oplus(-c\otimes y)
    \end{equation}

    将$(3)$代入$(2)$中,得

    \begin{equation*}
        \begin{aligned}
            (c\otimes(a\oplus(-c\otimes y)))\oplus y =& b\\
            c\otimes a\oplus c\otimes(-c\otimes y)\oplus y =& b\\
            c\otimes a\oplus (-c^2\otimes y)\oplus y =& b\\
            (-c^2\oplus 1)\otimes y =& b\oplus (-c\otimes a)
        \end{aligned}
    \end{equation*}

    当$c=1$时,$a = b$,则$x\oplus y=a$,无唯一解。

    当$c=-1$时,$a=-b$,则$x\oplus(-y) = a$,无唯一解。

    当$c\neq\pm 1$时,则$y = (b\oplus(-c\otimes a))\otimes(-c^2\oplus 1)^{-1}$,则有
    \begin{equation*}
        \begin{aligned}
            x =&\ a\oplus(-c\otimes(b\oplus(-c\otimes a))\otimes(-c^2\oplus 1)^{-1})\\
            =&\ a\oplus(-(b\otimes c\oplus(-c^2\otimes a))\otimes(-c^2\oplus 1)^{-1})\\
            =&\ (a\oplus(-a\otimes c^2)\oplus(-b\otimes c)\oplus(c^2\otimes a))\otimes(-c^2\oplus 1)^{-1}\\
            =&\ (a\oplus(-b\otimes c))\otimes(-c^2\oplus 1)^{-1}
        \end{aligned}
    \end{equation*}

    故
    \begin{equation*}
        \begin{cases}
            x = (a\oplus(-b\otimes c))\otimes(-c^2\oplus 1 )^{-1}\\
            y = (b\oplus(-c\otimes a))\otimes(-c^2\oplus 1)^{-1}
        \end{cases}
    \end{equation*}

\end{solution}
\paragraph{67.}\begin{solution}
    不一定。设域为实数域,$\braket{\mathbb{R}, +, \times}$,取
    \begin{equation*}
        R = \{\text{所有偶数}\} = \{\cdots, -6,-4,-2,0,2,4,6,\cdots\}
    \end{equation*}
    则$\braket{R,+,\times}$构成环,所以$\braket{R,+\times}$是$\braket{\mathbb{R},+,\times}$的子环,但$1\notin R$,所以$R$中没有幺元,故$R$不是整环。
\end{solution}
\paragraph{68.}\begin{solution}
    (1). 不是,因为$-1\notin X$,所以$1\in X$没有关于$\oplus$的逆元,故$\braket{X,+,\times}$不是域。

    (2). 是,由上次作业$60.(5)$题知,$\braket{X,+,\times}$构成整环,则只需证明$\forall a+b\sqrt{3}\in X$都有逆元即可,由于
    \begin{equation*}
        \begin{aligned}
            (a+b\sqrt{3})\frac{(a-b\sqrt{3})}{a^2-3b^2} =& 1\\
            \frac{(a-b\sqrt{3})}{a^2-3b^2}(a+b\sqrt{3}) =& 1
        \end{aligned}
    \end{equation*}

    假设$a^2-3b^2 = 0$,则$\dfrac{a}{b} = \sqrt{3}$,由于$a, b\in \mathbb Q$,则$\sqrt{3}$是有理数,与$\sqrt{3}$为无理数矛盾,所以$\forall a,b\in \mathbb{Q}$,$a+b\sqrt{3}$都有逆元,故$\braket{X,+,\times}$是域。

    (3). 不是,由于$\sqrt[3]{5}\times \sqrt[3]{5} = \sqrt[3]{25}\notin X$,所以$\braket{X,+,\times}$不是域。

    (4). 是,证明方式同本题的$(2)$。

    (5). 不是,因为$a\neq kb$,所以$1\notin X$,则$X$无幺元,故$\braket{X,+,\times}$不是域。
\end{solution}
\paragraph{69.} \begin{proof}
    由于:

    1. $S_1\subset F\text{ 且 }S_2\subset F$,则$S_1\cap S_2\subset F$

    2. $\braket{S_1, \oplus},\braket{S_2, \oplus}$为$\text{Abel}$群,则$\braket{S_1\cap S_2, \oplus}$为$\text{Abel}$群

    3. $\braket{S_1\backslash\{0\}, \otimes},\braket{S_2\backslash\{0\}, \otimes}$为$\text{Abel}$群,则$\braket{S_1\cap S_2\backslash\{0\}, \otimes}$为$\text{Abel}$群

    4. $\otimes, \oplus$为域$F$中的运算,则满足分配律,所以$S_1\cap S_2$中的$\otimes$对$\oplus$运算满足分配律。

    综上,$\braket{S_1\cap S_2,\oplus,\otimes}$是$\braket{F,\oplus,\otimes}$的子域。
\end{proof}
\paragraph{70.}\begin{solution}
    存在,取素多项式 $f(x)=1+x+x^2$ 为模数,可以构造出多项式模环 $\braket{R,+,\times}$,$R=\{0,1,x,1+x\},\ x\in\mathbb{Z}_{2}$,其中的 $+,\times$ 运算定义如下:
    \begin{center}
    \begin{tabular}{|c|c|c|c|c|}
    \hline
    +   & 0   & 1   & x   & 1+x \\ \hline
    0   & 0   & 1   & x   & 1+x \\ \hline
    1   & 1   & 0   & 1+x & x   \\ \hline
    x   & x   & 1+x & 0   & 1   \\ \hline
    1+x & 1+x & x   & 1   & 0   \\ \hline
    \end{tabular}
    \quad\quad
    \begin{tabular}{|c|c|c|c|c|}
    \hline
    $\times$   & 0 & 1   & x   & 1+x \\ \hline
    0   & 0 & 0   & 0   & 0   \\ \hline
    1   & 0 & 1   & x   & 1+x \\ \hline
    x   & 0 & x   & 1+x & 1   \\ \hline
    1+x & 0 & 1+x & 1   & x   \\ \hline
    \end{tabular}
    \end{center}

\end{solution}

\end{document}
