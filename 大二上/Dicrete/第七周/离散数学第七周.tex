\documentclass[12pt, a4paper, oneside]{ctexart}
\usepackage{amsmath, amsthm, amssymb, bm, color, graphicx, geometry, hyperref, mathrsfs,extarrows, braket}

\linespread{1.5}
\geometry{left=2.54cm,right=2.54cm,top=3.18cm,bottom=3.18cm}
\newenvironment{problem}{\par\noindent\textbf{题目. }}{\bigskip\par}
\newenvironment{solution}{\par\noindent\textbf{解答. }}{\bigskip\par}
\newenvironment{note}{\par\noindent\textbf{注记. }}{\bigskip\par}

% 基本信息
\newcommand{\dt}{\today}
\newcommand{\sj}{离散数学}
\newcommand{\vt}{吴天阳 2204210460}

\begin{document}

\pagestyle{empty}
\vspace*{-20ex}
\centerline{\begin{tabular}{*3{c}}
    \parbox[t]{0.3\linewidth}{\begin{center}\textbf{日期}\\ \large \textcolor{blue}{\dt}\end{center}} 
    & \parbox[t]{0.3\linewidth}{\begin{center}\textbf{科目}\\ \large \textcolor{blue}{\sj}\end{center}}
    & \parbox[t]{0.3\linewidth}{\begin{center}\textbf{姓名,学号}\\ \large \textcolor{blue}{\vt}\end{center}} \\ \hline
\end{tabular}}
\vspace*{4ex}
\paragraph{习题六}
\paragraph{1.}\begin{solution}
    
    (1). 是

    (2). 是

    (3). 是

    (4). 不是,因为$1 / 2\notin\mathbb Z$,所以$(1, 2)$没有定义。

    (5). 不是,因为$2^{-1}\notin\mathbb Z$,所以$(2,-1)$没有定义。

    (6). 不是,因为$\sqrt[1]{2}\notin\mathbb Z$,所以$(2, 1)$没有定义。

    (7). 是

    (8). 是

    (9). 是

    (10). 是

\end{solution}
\paragraph{6.}\begin{solution}
    观察表格知,$a$是$X$上关于$*$二元运算的幺元,则
    \begin{center}
        \begin{tabular}{|c|c|c|}
            \hline
            元素&左逆元&右逆元\\
            \hline
            a&a&a\\
            \hline
            b&c, d&c\\
            \hline
            c&b, e&b, d\\
            \hline
            d&c&b\\
            \hline
            e&\text{无}&c\\
            \hline
        \end{tabular}
    \end{center}
\end{solution}
\paragraph{8.}\begin{solution}
    
    满足结合律,先证明$\text{LCM}(x,y,z) = \text{LCM}(\text{LCM}(x,y),z)$,设$d$为$x, y,z$的公倍数,则$x|d,y|d,z|d$,特别的,$\text{LCM}(x, y)|d$,于是$d$为$\text{LCM}(x,y),z$的公倍数,因此,$x, y, z$的公倍数是$\text{LCM}(x, y),z$的公倍数。反之亦然,比较两组数的公倍数中的最小者即可得证,所以
    \begin{equation*}
        \begin{aligned}
            x*y*z&=\text{LCM}(x,y)*z=\text{LCM}(\text{LCM}(x, y), z)=\text{LCM}(x, y, z)\\
            &=\text{LCM}(x,\text{LCM}(y,z))=x*\text{LCM}(y,z)=x*(y*z)
        \end{aligned}
    \end{equation*}

    满足交换律,因为$x, y$的最小公倍数和$y,x$的最小公倍数相同,所以
    \begin{equation*}
        x*y=\text{LCM}(x, y)=\text{LCM}(y, x)=y*x
    \end{equation*}

    存在幺元,且幺元为$0$,因为$0$是任何正整数的因数,所以
    \begin{equation*}
        x*0=0*x=\text{LCM}(0, x) = x
    \end{equation*}

    不存在零元,反设存在零元$z$,则$z=(z+1)*z=\text{LCM}(z+1,z)\geqslant z+1$,矛盾。

    不存在逆元,$x\in \mathbb N^*$,对于$\forall y\in \mathbb N$,$\text{LCM}(x, y) \geqslant x > 0$,所以不存在逆元。
\end{solution}
\paragraph{12.}\begin{solution}

    对应于$S_1$的子关系:
    \begin{center}
        \begin{tabular}{c|cc}
            $\oplus$ & b & d \\ \hline
            b     & b & b \\
            d     & b & d
        \end{tabular}
        \quad\quad
        \begin{tabular}{c|cc}
            $\otimes$ & b & d \\ \hline
            b     & b & d \\
            d     & d & d
        \end{tabular}
    \end{center}

    则$\braket{S_1,\oplus,\otimes}$是$\braket{X,\oplus,\otimes}$的子代数系统。

    对应于$S_2$的子关系:
    \begin{center}
        \begin{tabular}{c|cc}
            $\oplus$ & a & d \\ \hline
            a     & a & a \\
            d     & a & d
        \end{tabular}
        \quad\quad
        \begin{tabular}{c|cc}
            $\oplus$ & a & d \\ \hline
            a     & a & d \\
            d     & d & d
        \end{tabular}
    \end{center}

    则$\braket{S_2,\oplus,\otimes}$是$\braket{X,\oplus,\otimes}$的子代数系统。

    对应于$S_3$的子关系:
    \begin{center}
        \begin{tabular}{c|cc}
            $\oplus$ & b & c \\ \hline
            b     & b & a \\
            c     & a & c
        \end{tabular}
        \quad\quad
        \begin{tabular}{c|cc}
            $\oplus$ & b & c \\ \hline
            b     & b & d \\
            c     & d & c
        \end{tabular}
    \end{center}

    则$\braket{S_3,\oplus,\otimes}$不是$\braket{X,\oplus,\otimes}$的子代数系统,因为$b\oplus c\notin S_3$。
\end{solution}
\end{document}
