\documentclass[12pt, a4paper, oneside]{ctexart}
\usepackage{amsmath, amsthm, amssymb, bm, color, graphicx, geometry, hyperref, mathrsfs,extarrows, braket}

\linespread{1.5}
\geometry{left=2.54cm,right=2.54cm,top=3.18cm,bottom=3.18cm}
\newenvironment{problem}{\par\noindent\textbf{题目. }}{\bigskip\par}
\newenvironment{solution}{\par\noindent\textbf{解答. }}{\bigskip\par}
\newenvironment{note}{\par\noindent\textbf{注记. }}{\bigskip\par}

% 基本信息
\newcommand{\dt}{\today}
\newcommand{\sj}{离散数学}
\newcommand{\vt}{吴天阳 2204210460}

\begin{document}

\pagestyle{empty}
\vspace*{-20ex}
\centerline{\begin{tabular}{*3{c}}
    \parbox[t]{0.3\linewidth}{\begin{center}\textbf{日期}\\ \large \textcolor{blue}{\dt}\end{center}} 
    & \parbox[t]{0.3\linewidth}{\begin{center}\textbf{科目}\\ \large \textcolor{blue}{\sj}\end{center}}
    & \parbox[t]{0.3\linewidth}{\begin{center}\textbf{姓名,学号}\\ \large \textcolor{blue}{\vt}\end{center}} \\ \hline
\end{tabular}}
\vspace*{4ex}

\paragraph{47.}证明:$\braket{H_1H_2,*}$是$\braket{G,*}$的子群$\iff H_1H_2 = H_2H_1$。

\begin{proof}
    $\Rightarrow$:由于$H_1H_2$为$G$的子群。
    
    $\forall h_1*h_2\in H_1H_2$,有$(h_1*h_2)^{-1}\in H_1H_2$,则$\exists a_1*a_2\in H_1H_2$,使得$a_1*a_2 = (h_1h_2)^{-1}$,故
    \begin{equation*}
        \begin{aligned}
            &h_1*h_2 = (a_1*a_2)^{-1} = a_2^{-1}*a_1^{-1}\in H_2H_1\\
            \Rightarrow\ &H_1H_2\subset H_2H_1
        \end{aligned}
    \end{equation*}

    $\forall h_2*h_1\in H_2H_1$,由于$(h_2*h_1)^{-1}=h_1^{-1}*h_2^{-1}\subset H_1H_2$,则
    \begin{equation*}
        \begin{aligned}
            &(h_1^{-1}*h_2^{-1})^{-1}\in H_1H_2\\
            \Rightarrow\ &h_2*h_1\in H_1H_2\\
            \Rightarrow\ &H_2H_1\subset H_1H_2
        \end{aligned}
    \end{equation*}

    综上,$H_1H_2=H_2H_1$。

    $\Leftarrow$:$\forall a_1*a_2, b_1*b_2\in H_1H_2$,则$a_1*a_2*(b_1*b_2)^{-1}=a_1*a_2*b_2^{-1}*b_1^{-1}$,由于$a_2*b_2^{-1}*b_1^{-1}\in H_2H_1$,且$H_2H_1=H_1H_2$,则$\exists c_1c_2\in H_1H_2$,使得$c_1c_2=a_2*b_2^{-1}*b_1^{-1}$,故
    \begin{equation*}
        a_1*a_2*(b_1*b_2)^{-1} = a_1*c_1*c_2\in H_1H_2
    \end{equation*}

    所以$\braket{H_1H_2,*}$是$\braket{G,*}$的子群。
\end{proof}
\paragraph{55.}\begin{proof}
    单射:
    \begin{equation*}
        \begin{aligned}
            &f(x)=f(y)\\
            \Rightarrow\ &a*x*a^{-1}=a*y*a^{-1}\\
            \Rightarrow\ &a^{-1}*a*x*a^{-1}*a=a^{-1}*a*y*a^{-1}*a\\
            \Rightarrow\ &x=y
        \end{aligned}
    \end{equation*}

    满射:$\forall x\in G$,有
    \begin{equation*}
            f(a^{-1}*x*a)=a*(a^{-1}*x*a)*a^{-1}=x
    \end{equation*}

    保运算:$\forall x,y\in G$,有
    \begin{equation*}
        \begin{aligned}
            f(xy)=&a*x*y*a^{-1}\\
            =&a*x*a^{-1}*a*y*a^{-1}\\
            =&f(x)*f(y)
        \end{aligned}
    \end{equation*}

    综上,$f$是$\braket{G,*}$上的自同构函数。
\end{proof}
\paragraph{57.}\begin{proof}
    自反性:$\forall a\in G$,有
    \begin{equation*}
        e*a*e^{-1}=a\Rightarrow (a,a)\in R
    \end{equation*}

    对称性:$\forall a, b\in G$,若$(a,b)\in R$,则$\exists x\in G$,使得$b=x*a*x^{-1}$,则$a=x^{-1}*b*x$,故$(b,a)\in R$。

    传递性:$\forall a,b,c\in G$,若$(a,b),(b,c)\in R$,则$\exists x, y\in G$,使得$b=x*a*x^{-1},c=y*b*y^{-1}$,则
    \begin{equation*}
        c=y*(x*a*x^{-1})*y^{-1}=(y*x)*a*(y*x)^{-1}
    \end{equation*}

    故$(a,c)\in R$。

    综上,$R$是$G$上的等价关系。
\end{proof}
\paragraph{59.}\begin{proof}
    先证明$\braket{\mathbb{Z}, \oplus}$是交换群:

    1. $a\oplus b = a+b-1\in \mathbb{Z}$,则$\oplus$是$\mathbb{Z}$上的二元运算。

    2. 结合律:
    \begin{equation*}
        \begin{aligned}
            a\oplus b\oplus c =& (a+b-1)\oplus c\\
             =& a+b-1+c-1 \\
             =& a+(b+c-1)-1 \\
             =& a + (b\oplus c) - 1 \\
             =& a\oplus(b\oplus c)
        \end{aligned}
    \end{equation*}

    3. 左幺元:$\forall a\in \mathbb{Z}$,则$1\oplus a = 1+a-1 = a$,所以$1$为$\braket{\mathbb{Z},\oplus}$的左幺元。

    4. 左逆元:$\forall a\in \mathbb{Z}$,则$(-a+2)\oplus a = -a+2+a-1 = 1$,所以$a$的左逆元为$-a+2$。

    5. 交换律:$a\oplus b = a+b-1 = b+a-1 = b\oplus a$。

    再证明$\braket{\mathbb{Z},\otimes}$是含幺半群:

    1. $a\otimes b = a+b-ab\in \mathbb{Z}$,则$\otimes$是$\mathbb{Z}$上的二元运算。
    
    2. 结合律:
    \begin{equation*}
        \begin{aligned}
            a\otimes b\otimes c = & (a+b-ab)\otimes c\\
            = & (a+b-ab)+c-(a+b-ab)c\\
            = & a+b-ab + c-ac-bc+abc\\
            = & a+(b+c-bc)-a(b+c-bc)\\
            = & a\otimes (b+c-bc)\\
            = & a\otimes (b\otimes c)
        \end{aligned}
    \end{equation*}

    3. 幺元:
    \begin{equation*}
        \begin{aligned}
            &0\otimes b = 0+b-0\cdot b = b\\
            &b\otimes 0 = b + 0 - b\cdot 0 = b
        \end{aligned}
    \end{equation*}

    所以,$0$是$\braket{\mathbb{Z},\otimes}$上的幺元。

    4. 交换律:$a\otimes b = a+b-ab = b+a-ba = b\otimes a$。

    最后证明$\otimes$对$\oplus$满足分配律,$\forall a,b, c\in \mathbb{Z}$,有
    \begin{equation*}
        \begin{aligned}
            a\otimes(b\oplus c) = &\ a\otimes(b+c-1)\\ 
            =&\ a+b+c-1-a(b+c-1)\\
            =&\ (a+b-ab)+(a+c-ac)-1\\
            =&\ (a+b-ab)\oplus(a+c-ac)\\
            =&\ (a\otimes b)\oplus(a\otimes c)
        \end{aligned}
    \end{equation*}
    \begin{equation*}
        \begin{aligned}
            (b\oplus c)\otimes a=\ &(b+c-1)\otimes a\\
            =\ &(b+c-1)+a-(b+c-1)a\\
            =\ &(b+a-ba)+(c+a-ca)-1\\
            =\ &(b+a-ba)\oplus(c+a-ca)\\
            =\ &(b\otimes a)\oplus(c\otimes a)
        \end{aligned}
    \end{equation*}

    综上,$\braket{\mathbb{Z},\oplus,\otimes}$是有幺元的交换环。

\end{proof}
\paragraph{60.}\begin{solution}

    (1). 不是,因为$1\notin X$,所以$\times$运算没有幺元,故$\braket{X,+,\times}$不是整环。

    (2). 不是,因为$0\notin X$,所以$+$运算没有幺元(也就是环没有零元),故$\braket{X,+,\times}$不是整环。

    (3). 不是,因为$1\in X$,但是$-1\notin X$,所以$1$在$+$的运算中没有负元,故$\braket{X,+,\times}$不是整环。

    (4). 不是,因为$\sqrt[4]{5}\in X$,但是$\sqrt[4]{5}\times\sqrt[4]{5}=\sqrt{5}\notin X$,所以$\times$不是$X$上的二元运算,故$\braket{X,+,\times}$不是整环。

    (5). 是,容易验证$+,\times$满足结合律、交换律,$0$是环的零元,$1$是环的幺元,$a+b\sqrt{3}\in X$,则$-a-b\sqrt{3}$是其负元,$\dfrac{1}{a^2-3b^2}(a-b\sqrt{3})$是其逆元。
    
    下面证明$+,\times$满足封闭性,结合律:
    \begin{equation*}
        \begin{aligned}
            a+b\sqrt{3}+c+d\sqrt{3} =&\ (a+c)+(b+d)\sqrt{3}\\
            (a+b\sqrt{3})(c+d\sqrt{3}) =&\  (ac+3bd)+(ad+bc)\sqrt{3}\\
            (a+b\sqrt{3})(c+d\sqrt{3}+u+v\sqrt{3})=\ &ac+au+3(bd+bv)+(ad+av+bc+bu)\sqrt{3}\\
            =\ &(ac+3bd+(ad+bc)\sqrt{3})+(au+3bv+(av+bu)\sqrt{3})\\
            =\ &(a+b\sqrt{3})(c+d\sqrt{3})+(a+b\sqrt{3})(u+v\sqrt{3})
        \end{aligned}
    \end{equation*}

    假设$a+b\sqrt{3}$没有关于$\times$运算的逆元,也就是$a^2-3b^2=0\Rightarrow\dfrac{a}{b}=\sqrt{3}$,由于$\sqrt{3}$是无理数,不能表示为既约分数的形式,矛盾,则$X$中每个元素都有逆元,所以满足消去律,故$\times$运算无零因子。

    综上,$\braket{X,+,\times}$是整环。
\end{solution}
\paragraph{61.}\begin{proof}
    (1). 由于$\forall x\in R$,有$x\otimes x = x$,所以
    \begin{equation*}
        \begin{aligned}
            &(x\oplus x)\otimes (x\oplus x) = x\oplus x\\
            \Rightarrow\ &(x\otimes x)\oplus(x\otimes x)\oplus(x\otimes x)\oplus(x\otimes x)=x\oplus x\\
            \Rightarrow\ &x\oplus x\oplus x\oplus x=x\oplus x\\
            \Rightarrow\ &x\oplus x = 0
        \end{aligned}
    \end{equation*}
    
    (2). $\forall a, b\in R$,则
    \begin{equation*}
        \begin{aligned}
            (a\oplus b)\otimes(a\oplus b)=\ &a\oplus b\\
            (a\otimes a)\oplus (a\otimes b) \oplus (b\otimes a)\oplus (b\otimes b) = \ &a\oplus b\\
            a\oplus (a\otimes b) \oplus (b\otimes a)\oplus b = \ &a\oplus b\\
            (a\otimes b) \oplus (b\otimes a)=\ &0
        \end{aligned}
    \end{equation*}

    由于$x\oplus x=0$,所以$x$关于$\oplus$的逆元为$x$,由于逆元的唯一性,所以$a\otimes b = b\otimes a$,故$\braket{R,\oplus,\otimes}$是交换环。
\end{proof}

\paragraph{62.}\begin{solution}不难验证$\oplus,\otimes$都具有封闭性。

    关于$\oplus$运算:

    1. 结合律:\begin{equation*}
        \begin{aligned}
            (x_1,y_1)\oplus(x_2,y_2)\oplus(x_3,y_3)=\ &(x_1+x_2,y_1+y_2)\oplus(x_3,y_3)\\
            =\ &(x_1+x_2+x_3,y_1+y_2+y_3)\\
            =\ &(x_1,y_1)\oplus(x_2+x_3,y_2+y_3)\\
            =\ &(x_1,y_1)\oplus((x_2,y_2)\oplus (x_3,y_3))
        \end{aligned}
    \end{equation*}

    2. 左幺元:$(0,0)\oplus(x, y) = (0+x,0+y)=(x, y)$。

    3. 左逆元:$(-x,-y)\oplus(x,y) = (-x+x,-y+y) = (0,0)$。

    4. 交换律:$(x_1,y_1)\oplus(x_2,y_2) = (x_1+x_2,y_1+y_2) = (x_2+x_1,y_2+y_1)=(x_2,y_2)\oplus (x_1,y_1)$。

    综上,$\braket{X, \oplus}$构成交换群。

    关于$\otimes$运算:

    1. 结合律:\begin{equation*}
        \begin{aligned}
            (x_1,y_1)\otimes (x_2,y_2)\otimes(x_3,y_3)=\ &(x_1x_2,y_1y_2)\otimes(x_3,y_3)\\
            =\ &(x_1x_2x_3,y_1y_2y_3)\\
            =\ &(x_1,y_1)\otimes(x_2x_3,y_2y_3)\\
            =\ &(x_1,y_1)\otimes((x_2,y_2)\otimes(x_3,y_3))
        \end{aligned}
    \end{equation*}

    2. 幺元:
    \begin{equation*}
        \begin{aligned}
            (1,1)\otimes(x,y) =& (1\cdot x, 1\cdot y) = (x, y)\\
            (x,y)\otimes(1,1) =& (x\cdot 1, y\cdot 1) = (x, y)
        \end{aligned}
    \end{equation*}

    所以$(1,1)$是$\otimes$运算的幺元。

    3. 分配律:
    \begin{equation*}
        \begin{aligned}
            (x_1,y_1)\otimes((x_2,y_2)\oplus(x_3,y_3)) =\ &(x_1,y_1)\otimes(x_2+x_3,y_2+y_3)\\
            =\ &(x_1(x_2+x_3),y_1(y_2+y_3))\\
            =\ &(x_1x_2+x_1x_3,y_1y_2+y_1y_3)\\
            =\ &(x_1x_2,y_1y_2)\oplus(x_1x_3,y_1y_3)\\
            =\ &((x_1,y_1)\otimes(x_2,y_2))\oplus((x_1,y_1)\otimes(x_3,y_3))
        \end{aligned}
    \end{equation*}

    所以,$\braket{X,\oplus,\otimes}$构成环,且$\otimes$运算有幺元$(1,1)$。

    有零因子:$(1,0),(0,1)\in X$,且$(1,0)\neq (0,0),(0,1)\neq (0, 0)$,但
    \begin{equation*}
        (1,0)\otimes(0,1) = (1\cdot 0,0\cdot 1) = (0, 0)
    \end{equation*}
    则$\otimes$运算有零因子。

    逆元:$\forall (x, y)\in X$,$x\neq 0$且$y\neq 0$时,$(x, y)$有逆元,只需证明左逆元,因为$\otimes$运算具有幺元,
    \begin{equation*}
        (\frac{1}{x},\frac{1}{y})\otimes(x, y)=(\frac{1}{x}\cdot x,\frac{1}{y}\cdot y)=(1,1)
    \end{equation*}
    则$(\frac{1}{x},\frac{1}{y})$为$(x, y)$的逆元,且当$x=0$或$y=0$时,$(x,y)$不存在逆元。



\end{solution}


\end{document}
