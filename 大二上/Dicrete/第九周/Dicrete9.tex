\documentclass[12pt, a4paper, oneside]{ctexart}
\usepackage{amsmath, amsthm, amssymb, bm, color, graphicx, geometry, hyperref, mathrsfs,extarrows, braket}

\linespread{1.5}
\geometry{left=2.54cm,right=2.54cm,top=3.18cm,bottom=3.18cm}
\newenvironment{problem}{\par\noindent\textbf{题目. }}{\bigskip\par}
\newenvironment{solution}{\par\noindent\textbf{解答. }}{\bigskip\par}
\newenvironment{note}{\par\noindent\textbf{注记. }}{\bigskip\par}

% 基本信息
\newcommand{\dt}{\today}
\newcommand{\sj}{离散数学}
\newcommand{\vt}{吴天阳 2204210460}

\begin{document}

\pagestyle{empty}
\vspace*{-20ex}
\centerline{\begin{tabular}{*3{c}}
    \parbox[t]{0.3\linewidth}{\begin{center}\textbf{日期}\\ \large \textcolor{blue}{\dt}\end{center}} 
    & \parbox[t]{0.3\linewidth}{\begin{center}\textbf{科目}\\ \large \textcolor{blue}{\sj}\end{center}}
    & \parbox[t]{0.3\linewidth}{\begin{center}\textbf{姓名,学号}\\ \large \textcolor{blue}{\vt}\end{center}} \\ \hline
\end{tabular}}
\vspace*{4ex}

\paragraph{37.}\begin{proof}
    $S_1$的封闭性:$\forall a_1, a_2\in S_1,\ \exists y_1, y_2\in S$,使得$y_1*a_1 = e, y_2*a_2 = e$,则
    \begin{equation*}
        (y_2*y_1)*(a_1*a_2) = y_2*(y_1*a_1)*a_2=y_2*e*a_2=y_2*a_2=e
    \end{equation*}
    所以$a_1*a_2\in S_1$,故$S_1$满足封闭性。

    结合律,由于$S_1$是$S$的子集,所以$S_1$具有结合律。

    幺元,由于$e*e = e$,所以$e\in S_1$,故$S_1$中含有幺元$e$。

    综上,$\braket{S_1,*}$是$\braket{S,*}$的子含幺半群。
\end{proof}

\paragraph{39.}\begin{proof}
    (1). 由于$\braket{G,*}$为群,所以$G$中每个元素都存在唯一的逆元,则$x = a^{-1}*b$,故$x$是唯一的。

    (2). 由于$\braket{G,*}$为群,所以$G$中每个元素都存在唯一的逆元,则$y = b * a^{-1}$,故$y$是唯一的。
\end{proof}

\paragraph{40.}\begin{proof}
    (1). $\exists d\in S$,使得$d*a = e$,则
    \begin{equation*}
        \begin{aligned}
            a*b&=a*c\\
            \Rightarrow  d*a*b&=d*a*c\\
            \Rightarrow  b&=c
        \end{aligned}
    \end{equation*}

    (2). $\braket{S,*}$是半群,所以具有结合律。

    下证$S$含有幺元,由于$e*e=e$,则$\forall a \in G$,有
    \begin{equation*}
        a*e*e=a*e\Rightarrow a*e = a
    \end{equation*}
    所以$e$也是右幺元,故$e$为$S$中的幺元。

    下证$S$中元素都含有逆元,$\forall x\in G,\ \exists y\in G$,使得$y*x = e$,则
    \begin{equation*}
        y*x*y=e*y=y=y*e=y*y*x
    \end{equation*}
    对上式同时左乘$y$的逆元,得
    \begin{equation*}
        x*y=y*x=e
    \end{equation*}
    所以$y$也是$x$的右逆元,故$y$是$x$的逆元。

    综上,$\braket{S,*}$是群。
\end{proof}
\paragraph{41.}\begin{proof}
    反设,$G$中不含有二阶元,则$\forall a\in G, a\neq e$都有$a^{-1}\neq a$,所以$G$中所有的非零元都可以两两配对,设这样的配对一共有$m$对,则$|G| = 2m+1$,与$|G| = 2n$矛盾,故$G$中至少有一个二阶元素。
\end{proof}

\end{document}
