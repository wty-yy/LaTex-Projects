% !Mode:: "TeX:UTF-8" 

%-----------------------------------------------------------------------------------------
% 中文摘要
\clearpage
\titlespacing{\chapter}{0pt}{0mm}{5mm}
% \Abstract{摘\quad 要}{Abstract (In Chinese)}
\BiAppendixChapter{摘\quad 要}{Abstract (In Chinese)}
% \chapter{摘\quad 要}


\defaultfont
本文首次提出了一种基于非嵌入式离线强化学习的训练策略,应用于游戏皇室战争(Clash Royale)。
结合当前目标识别与光学文本识别的顶尖算法,使用离线强化学习算法制定策略,成功实现了智能体在非嵌入条件下的实时对局:
移动设备上实时图像获取、感知融合、智能模型决策及控制移动设备执行动作,从而能够与对手进行实时对局,并且能够战胜游戏中的内置AI。

首先,本文设计了一种高效制作切片数据集的方法,基于该方法制作了包含$4654$张切片、共150个类别的切片数据集,
以及包含$116878$个目标框、共$6939$张图像的目标识别数据集\footnote{图像数据集:\url{https://github.com/wty-yy/Clash-Royale-Detection-Dataset}\hfill},
并提出了一种可行的生成式目标识别数据集算法,使其能够模拟真实对局场景生成含有任意数量部队及种类的带标签图像。
通过生成式图像训练出的模型在真实视频流中表现出良好的泛化性,具有很高的识别准确率。

其次,本文基于计算机视觉模型的输出结果设计了感知融合算法,该算法结合视频数据中的上下文信息优化特征结果,从而进一步提升识别的准确率。

最后,在决策模型方面,本文从架构及预测目标两个方面对传统模型进行了改进,将难以学习的离散动作序列转化为连续动作序列,
大幅提高了模型性能。本文制作了包含$105$回合、总共$113981$帧的专家数据集\footnote{专家数据集:\url{https://github.com/wty-yy/Clash-Royale-Replay-Dataset}\hfill},
并基于该离线数据集在不与真实环境交互的条件下,训练出了能够战胜游戏内置AI的智能体。

本文的全部代码均已开源\footnote{全部代码:\url{https://github.com/wty-yy/katacr}}。

\vspace{\baselineskip}
\noindent{\textbf{关键词:}} 目标识别;光学文本识别;强化学习


%-----------------------------------------------------------------------------------------
% 英文摘要
\clearpage
%\phantomsection
\markboth{Abstract}{Abstract}

\titlespacing{\chapter}{0pt}{0mm}{5mm}
% \chapter*{ABSTRACT}
\BiAppendixChapter{ABSTRACT}{}

\noindent This paper introduces, for the first time, a non-embedded offline reinforcement learning training strategy applied to the game Clash Royale. By combining state-of-the-art algorithms in object recognition and optical character recognition, we use an offline reinforcement learning algorithm to develop strategies, successfully enabling real-time gameplay by the agent under non-embedded conditions: real-time image capture on a mobile device, perception fusion, intelligent model decision-making, and control of the mobile device to execute actions. As a result, the agent can engage in real-time matches against opponents and defeat the game's built-in AI.

\noindent First, we designed an efficient method for creating slice datasets. Using this method, we created a slice dataset containing $4654$ slices across $150$ categories, and an object recognition dataset containing $116878$ target boxes across $6939$ images\footnote{Image dataset: \url{https://github.com/wty-yy/Clash-Royale-Detection-Dataset}\hfill}. We also proposed a feasible generative algorithm for creating object recognition datasets, allowing the generation of labeled images with any number and type of units that mimic real match scenarios. Models trained on these generative images demonstrated excellent generalization performance in real video streams, achieving high recognition accuracy.

\noindent Additionally, we designed a perception fusion algorithm based on the output results of computer vision models, which optimizes feature results by incorporating contextual information from video data, further improving recognition accuracy.

\noindent Finally, in terms of decision-making models, we improved traditional models from both the architecture and prediction target perspectives. By transforming difficult-to-learn discrete action sequences into continuous action sequences, this method significantly enhanced model performance. We created an expert dataset containing $105$ rounds and a total of $113981$ frames\footnote{Expert dataset: \url{https://github.com/wty-yy/Clash-Royale-Replay-Dataset}\hfill}. Based on this offline dataset, the agent was trained to defeat the built-in AI without interacting with the real environment.

\noindent All the code used in this paper has been open-sourced\footnote{All code: \url{https://github.com/wty-yy/katacr}}.

\vspace{\baselineskip}
\noindent{\textbf{KEY WORDS:}} Object Recognition; Optical Character Recognition; Reinforcement Learning


\titlespacing{\chapter}{0pt}{-6mm}{5mm}
\clearpage{\pagestyle{empty}\cleardoublepage}
