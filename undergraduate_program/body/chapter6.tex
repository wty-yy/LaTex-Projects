% !Mode:: "TeX:UTF-8" 

\BiChapter{总结与展望}{6}

% 本文首次提出了一种基于非嵌入式离线强化学习的训练策略,应用于游戏皇室战争(Clash Royale)。
% 结合当前目标识别与光学文本识别的顶尖算法,使用离线强化学习算法制定策略,成功实现了智能体在非嵌入条件下的实时对局:
% 移动设备上实时图像获取、感知融合、智能模型决策及控制移动设备执行动作,从而能够与对手进行实时对局,并且能够战胜游戏中的内置AI。
% 
% 首先,本文设计了一种高效制作切片数据集的方法,基于该方法制作了包含$4654$张切片、共150个类别的切片数据集,
% 以及包含$116878$个目标框、共$6939$张图像的目标识别数据集\footnote{图像数据集:\url{https://github.com/wty-yy/Clash-Royale-Detection-Dataset}\hfill},
% 并提出了一种可行的生成式目标识别数据集算法,使其能够模拟真实对局场景生成含有任意数量部队及种类的带标签图像。
% 通过生成式图像训练出的模型在真实视频流中表现出良好的泛化性,具有很高的识别准确率。
% 
% 其次,本文基于计算机视觉模型的输出结果设计了感知融合算法,该算法结合视频数据中的上下文信息优化特征结果,从而进一步提升识别的准确率。
% 
% 最后,在决策模型方面,本文从架构及预测目标两个方面对传统模型进行了改进,将难以学习的离散动作序列转化为连续动作序列,
% 大幅提高了模型性能。本文制作了包含$105$回合、总共$113981$帧的专家数据集\footnote{专家数据集:\url{https://github.com/wty-yy/Clash-Royale-Replay-Dataset}\hfill},
% 并基于该离线数据集在不与真实环境交互的条件下,训练出了能够战胜游戏内置AI的智能体。
% 
本文基于游戏皇室战争(Clash Royale),首次提出了一种基于非嵌入式的离线强化学习训练策略。
结合目标识别和光学文本识别的顶尖算法,成功实现了智能体在移动设备上进行实时对局,并且战胜了游戏中的内置AI。

主要贡献包含以下三点:
\begin{enumerate}
  \item 数据集制作:本文设计了一种高效制作切片数据集的方法,制作了包含4654张切片、共150个类别的切片数据集,
  以及包含116878个目标框、共6939张图像的目标识别数据集。提出的生成式目标识别数据集算法可以模拟真实对局场景生成带标签的图像,
  通过生成式图像训练的模型在真实视频流中表现出良好的泛化性,具有很高的识别准确率。
  \item 感知融合算法:基于计算机视觉模型的输出结果,设计了感知融合算法,该算法结合视频数据中的上下文信息优化特征结果,进一步提升了识别的准确率。
  \item 决策模型改进:在决策模型方面,从架构及预测目标两个方面对传统模型进行改进,
  将难以学习的离散动作序列转化为连续动作序列,大幅提高了模型性能。制作了包含105回合、总共113981帧的专家数据集,并基于该离线数据集训练出能够战胜游戏内置AI的智能体。
\end{enumerate}

本文为非嵌入式强化学习在移动设备上的应用提供了新的思路。未来的工作可以从以下几个方面进行扩展:
\begin{enumerate}
  \item 数据集的扩展与优化:增加数据集的规模和多样性,进一步提升模型的泛化能力和识别准确率。
  \item 算法的改进:当前在固定卡组下进行训练,仍然无法$100\%$战胜游戏中的内置AI,因此远无法达到人类平均水平,
  而且制作离线强化学习数据集需要花费大量的人力,
  若要进一步提升智能体能力,应该需要采用在线强化学习算法,与此同时需要使用更加高效的感知融合算法和决策模型架构,才有可能进一步提高智能体的实时决策能力和对局胜率。
  \item 实际应用的拓展:将本文的方法应用于更多的游戏和实际场景中,验证其通用性和实用价值。
\end{enumerate}

本文的全部代码均已开源,期望能够为相关领域的研究者提供有价值的参考和借鉴。