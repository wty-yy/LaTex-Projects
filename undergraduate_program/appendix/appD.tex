% -*-coding: utf-8 -*-

\BiAppChapter{论文相关代码}{}

本论文全部代码均已开源~\url{https://github.com/wty-yy/katacr}~,核心代码总计1.4万行左右,架构设计如下:

\setitemize{leftmargin=2em,itemsep=0em,partopsep=0em,parsep=0em,topsep=-0em}
\begin{itemize}
  \item \texttt{build\_dataset}
  \begin{itemize}
    \item 对视频文件进行预处理(划分episode,逐帧提取,图像不同部分提取)
    \item 目标识别数据集搭建工具(辅助标记数据集,数据集版本管理,生成式目标识别,标签转化及识别标签生成,图像切片提取)
  \end{itemize}
  \item \texttt{classification}:用ResNet进行手牌及圣水分类
  \item \texttt{constants}:常量存储(卡牌名称及对应圣水花费,目标识别类别名称)
  \item \texttt{detection}:自行用JAX复现的YOLOv5模型(后弃用)
  \item \texttt{interact}:测试与手机进行实时交互,包括目标识别,文本识别,GUI
  \item \texttt{ocr\_text}:包括用JAX复现的CRNN(后弃用)和PaddleOCR的接口转化
  \item \texttt{policy}:
  \begin{itemize}
    \item \texttt{env}:两种测试环境:
    \begin{itemize}
      \item \texttt{VideoEnv}:将视频数据集作为输入,仅用于调试模型的输入是否与预测相对应
      \item \texttt{InteractEnv}:与手机进行实时交互,使用多进程方式执行感知融合
    \end{itemize}
    \item \texttt{offline}:包含了决策模型StARformer和DT的训练,验证的功能,并包含三种CNN测试结构ResNet, CSPDarkNet, CNNBlocks
    \item \texttt{perceptron}:感知融合,包含了state,action,reward三种特征生成器,并整合到SARBuilder中(感知基于YOLOv8, PaddleOCR, ResNet Classifier)
    \item \texttt{replay\_data}:提取专家视频中的感知特征,制作并测试离线数据集
    \item \texttt{visualization}:实时监测手机图像,可视化感知融合特征
  \end{itemize}
  \item \texttt{utils}:用于目标检测相关的工具(绘图、坐标转化、图像数据增强),用于视频处理的ffmpeg相关工具
  \item \texttt{yolov8}:重构YOLOv8源码,包括数据读取、模型训练、验证、目标检测、跟踪,模型识别类型设置以及参数配置
\end{itemize}
下面将展示论文中提到的部分代码架构。
\BiSection{生成式数据集}{}\label{app-generator}
\pythonfile{coding/generator.py}
\BiSection{特征融合}{}
\BiSubsection{状态特征提取部分代码}{}\label{app-state-feature}
\pythonfile{coding/state_builder.py}
\BiSubsection{动作特征提取部分代码}{}\label{app-action-feature}
\pythonfile{coding/action_builder.py}
\BiSubsection{奖励特征提取部分代码}{}\label{app-reward-feature}
\pythonfile{coding/reward_builder.py}