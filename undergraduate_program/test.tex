\documentclass{article}
\usepackage[newfloat]{minted}
\usepackage{caption}

\newenvironment{codetitle}{\captionsetup{type=listing}}{}
\SetupFloatingEnvironment{listing}{name=Source Code}
%%%% 设置代码块 %%%%
% 在vscode中使用minted需要先配置python解释器, Ctrl+Shift+P, 输入Python: Select Interpreter选择安装了Pygments的Python版本. 再在setting.json中xelatex和pdflatex的参数中加入 "--shell-escape", 即可
% TeXworks中配置方法参考: https://blog.csdn.net/RobertChenGuangzhi/article/details/108140093
\usepackage{minted}
\renewcommand{\theFancyVerbLine}{
    \sffamily\textcolor[rgb]{0.5,0.5,0.5}{\scriptsize\arabic{FancyVerbLine}}} % 修改代码前序号大小
% 加入不同语言的代码块
\newmintinline{cpp}{fontsize=\small, linenos, breaklines, frame=lines}
\newminted{cpp}{fontsize=\small, baselinestretch=1, linenos, breaklines, frame=lines}
\newmintedfile{cpp}{fontsize=\small, baselinestretch=1, linenos, breaklines, frame=lines}
\newmintinline{matlab}{fontsize=\small, linenos, breaklines, frame=lines}
\newminted{matlab}{fontsize=\small, baselinestretch=1, mathescape, linenos, breaklines, frame=lines}
\newmintedfile{matlab}{fontsize=\small, baselinestretch=1, linenos, breaklines, frame=lines}
\newmintinline{python}{fontsize=\small, linenos, breaklines, frame=lines, python3}  % 使用\pythoninline{代码}
\newminted{python}{fontsize=\small, baselinestretch=1, linenos, breaklines, frame=lines, python3}  % 使用\begin{pythoncode}代码\end{pythoncode}
\newmintedfile{python}{fontsize=\small, baselinestretch=1, linenos, breaklines, frame=lines, python3}  % 使用\pythonfile{代码地址}

\begin{document}
\begin{codetitle}
\captionof{listing}{My C-Code}
\label{code:c-code}
\begin{cppcode}
int main() {
printf("hello, world");
return 0;
actiona
actiona
action
}
int main() {
int main() {
int main() {
int main() {
int main() {
int main() {
int main() {
int main() {
int main() {
printf("hello, world");
return 0;
actiona
actiona
action
}
printf("hello, world");
return 0;
actiona
actiona
action
}
printf("hello, world");
return 0;
actiona
actiona
action
}
printf("hello, world");
return 0;
actiona
actiona
action
}
printf("hello, world");
return 0;
actiona
actiona
action
}
printf("hello, world");
return 0;
actiona
actiona
action
}
printf("hello, world");
return 0;
actiona
actiona
action
}
printf("hello, world");
return 0;
actiona
actiona
action
}
printf("hello, world");
return 0;
actiona
actiona
action
}
\end{cppcode}
\end{codetitle}

Reference to \ref{code:c-code}.  

\end{document}