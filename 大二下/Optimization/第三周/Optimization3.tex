\documentclass[12pt, a4paper, oneside]{ctexart}
\usepackage{amsmath, amsthm, amssymb, bm, color, graphicx, geometry, hyperref, mathrsfs,extarrows, braket}

\linespread{1.5}
%\geometry{left=2.54cm,right=2.54cm,top=3.18cm,bottom=3.18cm}
\geometry{left=1.84cm,right=1.84cm,top=2.18cm,bottom=2.18cm}
\newenvironment{problem}{\par\noindent\textbf{题目. }}{\bigskip\par}
\newenvironment{solution}{\par\noindent\textbf{解答. }}{\bigskip\par}
\newenvironment{note}{\par\noindent\textbf{注记. }}{\bigskip\par}

% 基本信息
\newcommand{\dt}{\today}
\newcommand{\sj}{最优化方法}
\newcommand{\vt}{吴天阳 2204210460}

\begin{document}

%\pagestyle{empty}
\pagestyle{plain}
\vspace*{-15ex}
\centerline{\begin{tabular}{*3{c}}
    \parbox[t]{0.3\linewidth}{\begin{center}\textbf{日期}\\ \large \textcolor{blue}{\dt}\end{center}} 
    & \parbox[t]{0.3\linewidth}{\begin{center}\textbf{科目}\\ \large \textcolor{blue}{\sj}\end{center}}
    & \parbox[t]{0.3\linewidth}{\begin{center}\textbf{姓名,学号}\\ \large \textcolor{blue}{\vt}\end{center}} \\ \hline
\end{tabular}}
\vspace*{4ex}
\paragraph{4.} (1) 证明有限个凸集的交集仍然是凸集。

(2) 设$D_1=\{x:x_1+x_2\leqslant 1, x_1\geqslant 0\}, D_2=\{x:x_1-x_2\geqslant 0, x_1\leqslant 0\}$。
令$D=D_1\cup D_2$。证明$D_1,D_2$均为凸集,但$D$却不是凸的,由此得出凸集的并集未必是凸集。
\begin{proof}
    (1) 设$n\in \mathbb{N}$,集合类$\{D_1,D_2,\cdots, D_n\}$,其中$D_i\ (1\leqslant i\leqslant n)$均为凸集,下面利用数学归纳法证明命题:$\bigcap\limits_{i=1}^nD_i$为凸集。

    当$n=1$时,命题显然成立。

    假设命题在$n$时成立,下面讨论$n+1$的情况。
    
    $\forall x, y\in\bigcap\limits_{i=1}^{n+1}D_i, \lambda \in [0,1]$,由归纳假设知,$\bigcap\limits_{i=1}^nD_i$为凸集,则
    \begin{equation*}
        \begin{aligned}
            &\ \lambda x+(1-\lambda)y\in\bigcap_{i=1}^n D_i\ \text{且}\ \lambda x+(1-\lambda)y\in D_{n+1}\\
            \Rightarrow &\ \lambda x+(1-\lambda)y\in\bigcap_{i=1}^{n+1} D_i\\
            \Rightarrow &\ \bigcap_{i=1}^{n+1}D_i\ \text{为凸集}
        \end{aligned}
    \end{equation*}

    由数学归纳法知,对$n \in \mathbb{N}$该命题成立。

    (2) $\forall x, y\in D_1, \lambda\in[0, 1]$,记$x=(x_1,x_2), y=(y_1,y_2)$,则
    \begin{equation*}
        \begin{cases}
            x_1+x_2\leqslant 1,\ x_1\geqslant 0\\
            y_1+y_2\leqslant 1,\ y_1\geqslant 0\\
        \end{cases}
        \text{且}\ \lambda x + (1-\lambda) y = (\lambda x_1+(1-\lambda)y_1, \lambda x_2+(1-\lambda) y_2)
    \end{equation*}

    所以
    \begin{equation*}
        \begin{aligned}
            &\begin{cases}
                \lambda x_1+(1-\lambda)y_1+ \lambda x_2+(1-\lambda) y_2=\lambda(x_1+x_2)+(1-\lambda)(y_1+y_2)\leqslant \lambda+(1-\lambda) = 1\\
                \lambda x_1+(1-\lambda) y_1\geqslant 0
            \end{cases}\\
            \Rightarrow &\ \lambda x+(1-\lambda)y\in D_1\\
            \Rightarrow &\ D_1\ \text{为凸集}
        \end{aligned}
    \end{equation*}

    $\forall x, y\in D_2, \lambda\in[0, 1]$,记$x=(x_1,x_2), y=(y_1,y_2)$,则
    \begin{equation*}
        \begin{cases}
            x_1-x_2\geqslant 1,\ x_1\leqslant 0\\
            y_1-y_2\geqslant 1,\ y_1\leqslant 0\\
        \end{cases}
        \text{且}\ \lambda x + (1-\lambda) y = (\lambda x_1+(1-\lambda)y_1, \lambda x_2+(1-\lambda) y_2)
    \end{equation*}

    所以
    \begin{equation*}
        \begin{aligned}
            &\begin{cases}
                \lambda x_1+(1-\lambda)y_1- (\lambda x_2+(1-\lambda) y_2)=\lambda(x_1-x_2)+(1-\lambda)(y_1-y_2)\geqslant 0\\
                \lambda x_1+(1-\lambda) y_1\leqslant 0
            \end{cases}\\
            \Rightarrow &\ \lambda x+(1-\lambda)y\in D_2\\
            \Rightarrow &\ D_2\ \text{为凸集}
        \end{aligned}
    \end{equation*}

    取$x=(0, -1)\in D_1, y=(-1,-1)\in D_2, \lambda = \dfrac{1}{2}$,则
    \begin{equation*}
        \begin{aligned}
            &\ z = \lambda x + (1-\lambda)y = (-\frac{1}{2}, -1)\\
            \Rightarrow &\ z\notin D_1\ \text{且}\ z\notin D_2\\
            \Rightarrow &\ z\notin D\\
            \Rightarrow &\ D\ \text{不是凸集}
        \end{aligned}
    \end{equation*}
\end{proof}
\paragraph{6.}设$f(x)$为定义在凸集$D\subset \mathbb{R}^n$上的凸函数,$\alpha$为一个给定的实数,称集合
\begin{equation*}
    \mathcal{T} =\{x:f(x)\leqslant \alpha\}
\end{equation*}
为函数$f(x)$关于实数$\alpha$的水平集,证明对任意实数$\alpha$,集合$\mathcal{T}$是凸集。
\begin{proof}
    $\forall x, y\in \mathcal{T},\lambda\in [0,1]$,由$f$的凸性知,
    \begin{equation*}
        \begin{aligned}
            &\ f(\lambda x+(1-\lambda)y)\leqslant \lambda f(x)+(1-\lambda)f(y)\leqslant \lambda \alpha + (1-\lambda)\alpha= \alpha\\
            \Rightarrow&\ \lambda x+(1-\lambda)y\in \mathcal{T}\\
            \Rightarrow&\ \mathcal{T}\ \text{为凸集}
        \end{aligned}
    \end{equation*}
\end{proof}
\paragraph{14.}求出函数
\begin{equation*}
    f(x)=2x_1^3-3x_1^2-6x_1x_2(x_1-x_2-1)
\end{equation*}
的所有稳定点,其中哪一个点是极小值点?哪一个点是极大值点?有没有既不是极大又不是极小的点?
\begin{solution}
    $\nabla f = 6((x_1-x_2)(x_1-x_2-1),x_1(-x_1+2x_2+1))^T$,稳定点为所有$\nabla f=\boldsymbol{0}$的点,即
    \begin{equation*}
        \begin{aligned}
            &\begin{cases}
                (x_1-x_2)(x_1-x_2-1) = 0\\
                x_1(-x_1+2x_2+1) = 0
            \end{cases}\\
            \Rightarrow&
            \begin{cases}
                x_1 = 0\\
                x_2 = 0
            \end{cases}\text{或}
            \begin{cases}
                x_1 = -1\\
                x_2 = -1
            \end{cases}\text{或}
            \begin{cases}
                x_1 = 1\\
                x_2 = 0
            \end{cases}\text{或}
            \begin{cases}
                x_1 = 0\\
                x_2 = -1
            \end{cases}
        \end{aligned}
    \end{equation*}

    由于
    \begin{equation*}
        \begin{aligned}
            \nabla^2f =&\ \left[\begin{matrix}
                \frac{\partial^2 f}{\partial x_1^2}&\frac{\partial^2 f}{\partial x_1\partial x_2}\\
                \frac{\partial^2 f}{\partial x_2\partial x_1}&\frac{\partial^2 f}{\partial x_2^2}
            \end{matrix}\right] = 6\left[\begin{matrix}
                2x_1-2x_2-1&-2x_1+2x_2+1\\
                -2x_1+2x_2+1&2x_1
            \end{matrix}\right]
        \end{aligned}
    \end{equation*}

    令$x^*$为稳定点,当$x^*= (0, 0)$时,$\nabla^2f(x^*)=6\left[\begin{matrix}
        -1&1\\1&0
    \end{matrix}\right]$为负定矩阵,则$x^*$为极大值点。

    当$x^*= (-1, -1)$时,$\nabla^2f(x^*)=6\left[\begin{matrix}
        -1&1\\1&-2
    \end{matrix}\right]$为不定矩阵,则$x^*$既不是极大值又不是极小值。

    当$x^*= (1, 0)$时,$\nabla^2f(x^*)=6\left[\begin{matrix}
        1&-1\\-1&2
    \end{matrix}\right]$为正定矩阵,则$x^*$为极小值点。

    当$x^*= (0, -1)$时,$\nabla^2f(x^*)=6\left[\begin{matrix}
        1&-1\\-1&0
    \end{matrix}\right]$为不定矩阵,则$x^*$既不是极大值又不是极小值。
\end{solution}
\paragraph{15.}确定线性函数$f(x)=2x_1-x_2+3x_3$的所有下降方向。请问这样的下降方向是否同所在点的位置有关?
\begin{solution}
    $\forall x_0\in \mathbb{R}^3$,设$s\in\mathbb{R}^3,\alpha > 0$,$s$为$f(x)$在$x_0$处的下降方向,$\alpha$为充分小量,由带$\text{Lagrange}$余项的$\text{Taylor}$展式,知
    \begin{equation*}
        \begin{aligned}
            &\ f(x+\alpha s) = f(x)+\alpha\nabla f(\xi)^Ts\\
            \Rightarrow&\ \alpha\nabla f(\xi)^Ts = f(x+\alpha s)-f(x) < 0\\
            \Rightarrow&\ \nabla f(\xi)^Ts < 0
        \end{aligned}
    \end{equation*}
    其中$\xi = x_0+\lambda \alpha s$。当$\alpha\rightarrow 0^+$时,$\xi\rightarrow x_0$,由$\nabla f$的连续性知,$f(\xi)\rightarrow f(x_0)$,则$\nabla f(x_0)^T s < 0$,故
    \begin{equation*}
        \mathcal{D}(x_0) = \{s:\nabla f(x_0)^Ts < 0\} = \{s:(2,-1,3)^Ts < 0\}
    \end{equation*}
    由$x_0$的任意性知,下降方向与所在点的位置无关。
\end{solution}

\end{document}