\documentclass[12pt, a4paper, oneside]{ctexart}
\usepackage{amsmath, amsthm, amssymb, bm, color, graphicx, geometry, hyperref, mathrsfs,extarrows, braket}

\linespread{1.5}
\geometry{left=2.54cm,right=2.54cm,top=3.18cm,bottom=3.18cm}
\newenvironment{problem}{\par\noindent\textbf{题目. }}{\bigskip\par}
\newenvironment{solution}{\par\noindent\textbf{解答. }}{\bigskip\par}
\newenvironment{note}{\par\noindent\textbf{注记. }}{\bigskip\par}

% 基本信息
\newcommand{\dt}{\today}
\newcommand{\sj}{数值分析}
\newcommand{\vt}{吴天阳 2204210460}

\begin{document}

\pagestyle{empty}
\vspace*{-20ex}
\centerline{\begin{tabular}{*3{c}}
    \parbox[t]{0.3\linewidth}{\begin{center}\textbf{日期}\\ \large \textcolor{blue}{\dt}\end{center}} 
    & \parbox[t]{0.3\linewidth}{\begin{center}\textbf{科目}\\ \large \textcolor{blue}{\sj}\end{center}}
    & \parbox[t]{0.3\linewidth}{\begin{center}\textbf{姓名,学号}\\ \large \textcolor{blue}{\vt}\end{center}} \\ \hline
\end{tabular}}
\vspace*{4ex}

1. $e = 2.718281828\cdots$,则近似值:$x_1=2.71828325$有$6$位有效数字,$x_2=2.71828225$有$7$位有效数字;

2. 在浮点数系中求解方程$x^2-16x+1=0$,应如何计算,才能获得较准确的根$x_1,x_2$?请写出计算式:较大的正根$x_1=8+3\sqrt{7}$,较小的正根$x_2 = \frac{1}{8+3\sqrt{7}}$;

3. $e = 2.718281828\cdots,e^10 = 22026.46579\cdots$,它们在浮点数系$F(10,8,-8,8)$中浮点化数$fl(e)=0.27182818\times 10^1,fl(e^{10})=0.22026466\times10^5$,在浮点数系$F(10,8,-8,8)$中计算$fl(e)+fl(e^10)=0.22029184\times10^5$;

4. 在浮点数系$F(2,8,-7,8)$中,共有$4097$个数(包括0),实数$3.625$和$59.6$在系数中的浮点化数$fl(3.625) = 0.11101000\times2^2,fl(59.6)=0.11101110\times2^{6}$,在浮点数系$F(2,8,-7,8)$中计算$fl(3.625)+fl(59.6)=0.11111101\times10^6$。

\end{document}
