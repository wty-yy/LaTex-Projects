\documentclass[12pt, a4paper, oneside]{ctexart}
\usepackage{amsmath, amsthm, amssymb, bm, color, graphicx, geometry, hyperref, mathrsfs,extarrows, braket, booktabs, array}

\linespread{1.5}
%\geometry{left=2.54cm,right=2.54cm,top=3.18cm,bottom=3.18cm}
\geometry{left=1.84cm,right=1.84cm,top=2.18cm,bottom=2.18cm}
\newenvironment{problem}{\par\noindent\textbf{题目. }}{\bigskip\par}
\newenvironment{solution}{\par\noindent\textbf{解答. }}{\bigskip\par}
\newenvironment{note}{\par\noindent\textbf{注记. }}{\bigskip\par}

% 基本信息
\newcommand{\dt}{\today}
\newcommand{\sj}{数值分析}
\newcommand{\vt}{吴天阳 2204210460}

\begin{document}

%\pagestyle{empty}
\pagestyle{plain}
\vspace*{-15ex}
\centerline{\begin{tabular}{*3{c}}
    \parbox[t]{0.3\linewidth}{\begin{center}\textbf{日期}\\ \large \textcolor{blue}{\dt}\end{center}} 
    & \parbox[t]{0.3\linewidth}{\begin{center}\textbf{科目}\\ \large \textcolor{blue}{\sj}\end{center}}
    & \parbox[t]{0.3\linewidth}{\begin{center}\textbf{姓名,学号}\\ \large \textcolor{blue}{\vt}\end{center}} \\ \hline
\end{tabular}}
\vspace*{4ex}

\paragraph{4.7}$f(x) = x^7+x^4+3x+1$,求$f[2^0,2^1,\cdots,2^7]$和$f[2^0,2^1,\cdots, 2^8]$。

\begin{solution}
    设插值区间为$[a,b]$,对于插值点$\{x_0,x_1,\cdots,x_n\}$,由差商的性质知,存在$\xi$介于所有插值点之间,满足
    \begin{equation*}
        f[x_0,x_1,\cdots,x_n] = \frac{f^{n}(\xi)}{n!}
    \end{equation*}

    由于$f(x)$为$7$阶首一多项式,则$f^{7}(x) = 7!, f^{8}(x) = 0$,所以
    \begin{equation*}
        f[2^0,2^1,\cdots,2^6, 2^7] = \frac{f^7(\xi)}{7!} = 1
    \end{equation*}
    \begin{equation*}
        f[2^0,2^1,\cdots, 2^7, 2^8] = \frac{f^8(\xi')}{8!} = 0
    \end{equation*}
\end{solution}
\paragraph{4.8}设$f(x)$在区间$[a,b]$上三阶导数连续,$x_0,x_1\in[a,b]$,构造满足如下条件:
\begin{equation*}
    p(x_0)=f(x_0),\quad p'(x_0)=f'(x_0),\quad p(x_1)=f(x_1)
\end{equation*}
的二次插值多项式$p(x)$,并写出截断误差表达式。
\begin{solution}
    通过计算得到如下的差商表:
    \renewcommand\arraystretch{0.8}
    \begin{table}[!htbp]
        \centering
        \begin{tabular}{cccc}
            \toprule
            $x_i$ & $f[x_1]$ & $f[x_i,x_{i+1}]$ & $f[x_i,x_{i+1},x_{i+2}]$ \\
            \midrule
            $x_0$ & $f(x_0)$ &                  &                          \\
            $x_0$ & $f(x_0)$ & $f'(x_0)$        &                          \\
            $x_1$ & $f(x_1)$ & $\frac{f(x_1)-f(x_0)}{x_1-x_0}$ & $\frac{f(x_1)-f(x_0)}{(x_1-x_0)^2}-\frac{f'(x_0)}{x_1-x_0}$\\
            \bottomrule
        \end{tabular}
    \end{table}

    通过差商表可得插值多项式和截断误差表达式:
    \begin{equation*}
        p(x) = f(x_0)+f'(x_0)(x-x_0)+\left(\frac{f(x_1)-f(x_0)}{(x_1-x_0)^2}-\frac{f'(x_0)}{x_1-x_0}\right)(x-x_0)^2
    \end{equation*}
    \begin{equation*}
        R_n(x)=f[x_0,x_0,x_1,x](x-x_0)^2(x-x_1)
    \end{equation*}
\end{solution}
\paragraph{4.9}已知函数$y=f(x)$在若干点处的函数值、导数值如下表所示,求埃尔米特插值多项式和截断误差表达式:


(1)
\renewcommand\arraystretch{0.5}
\begin{tabular}{p{1cm}<{\centering}p{1cm}<{\centering}p{1cm}<{\centering}p{1cm}<{\centering}}
    \specialrule{0em}{1pt}{1pt}
    \toprule
    $x_i$  & $-1$ & $0$ & $1$ \\
    \midrule
    $y_i$  & $-1$ & $0$ & $1$ \\
    \midrule
    $y_i'$ & $0$  & $0$ & $0$ \\
    \bottomrule
\end{tabular};\quad\quad
(2)
\begin{tabular}{p{1cm}<{\centering}p{1cm}<{\centering}p{1cm}<{\centering}p{1cm}<{\centering}}
    \toprule
    $x_i$  & $0$ & $1$ & $2$ \\
    \midrule
    $y_i$  & $1$ & $-1$ & $0$ \\
    \midrule
    $y_i'$ & $0$  & $0$ &  \\
    \midrule
    $y_i''$ & $2$  & & \\
    \bottomrule
\end{tabular}.
\newpage
\begin{solution} (1) 计算差商表如下:
    \renewcommand\arraystretch{0.8}
    \begin{table}[!htbp]
        \centering
        \begin{tabular}{p{1cm}<{\centering}p{1cm}<{\centering}p{2cm}<{\centering}p{3cm}<{\centering}p{1cm}<{\centering}p{1cm}<{\centering}p{1cm}<{\centering}}
            \toprule
            $x_i$ & $f[x_1]$ & $f[x_i,x_{i+1}]$ & $f[x_i,x_{i+1},x_{i+2}]$ & $\cdots$ & $\cdots$ & $\cdots$\\
            \midrule
            $-1$ & $-1$ & & & & & \\
            $-1$ & $-1$ & $0$ & & & & \\
            $0$ & $0$ & $1$ & $1$ & & & \\
            $0$ & $0$ & $0$ & $-1$ & $-2$ & & \\
            $1$ & $1$ & $1$ & $1$ & $1$ & $\frac{3}{2}$ & \\
            $1$ & $1$ & $0$ & $-1$ & $-2$ & $-\frac{3}{2}$ & $-\frac{3}{2}$\\
            \bottomrule
        \end{tabular}
    \end{table}

    通过差商表可得插值多项式和截断误差表达式:
    \begin{equation*}
        \begin{aligned}
            H_5(x) =&\ -1+(x+1)^2-2(x+1)^2x+\frac{3}{2}(x+1)^2x^2-\frac{3}{2}(x+1)^2x^2(x-1)\\
            =&\ -\frac{x^3}{2}(3x^2-5)
        \end{aligned}
    \end{equation*}
    \begin{equation*}
        R_5(x)=f[-1,-1,0,0,1,1,x](x+1)^2x^2(x-1)^2
    \end{equation*}

    (2) 计算差商表如下:
    \renewcommand\arraystretch{0.8}
    \begin{table}[!htbp]
        \centering
        \begin{tabular}{p{1cm}<{\centering}p{1cm}<{\centering}p{2cm}<{\centering}p{3cm}<{\centering}p{1cm}<{\centering}p{1cm}<{\centering}p{1cm}<{\centering}}
            \toprule
            $x_i$ & $f[x_1]$ & $f[x_i,x_{i+1}]$ & $f[x_i,x_{i+1},x_{i+2}]$ & $\cdots$ & $\cdots$ & $\cdots$\\
            \midrule
            $0$ & $1$ & & & & & \\
            $0$ & $1$ & $0$ & & & & \\
            $0$ & $1$ & $0$ & $1$ & & & \\
            $1$ & $-1$ & $-2$ & $-2$ & $-3$ & & \\
            $1$ & $-1$ & $0$ & $2$ & $4$ & $7$ & \\
            $2$ & $0$ & $1$ & $1$ & $-\frac{1}{2}$ & $-\frac{9}{4}$ & $-\frac{37}{8}$\\
            \bottomrule
        \end{tabular}
    \end{table}

    通过差商表可得插值多项式和截断误差表达式:
    \begin{equation*}
        H_5(x)=1+x^2-3x^3+7x^3(x-1)-\frac{37}{8}x^3(x-1)^2
    \end{equation*}
    \begin{equation*}
        R_5(x)=f[0,0,0,1,1,2,x]x^3(x-1)^2(x-2)
    \end{equation*}
\end{solution}
\paragraph{4.10}设$x_i\ (i=0,1,\cdots, n)$是互不相同的插值节点,$l_i(x)\ (i = 0, 1,\cdots, n)$是拉格朗日插值基函数。证明:

(1) $\sum\limits_{i=0}^nl_i(x)=1$;

(2) $\sum\limits_{i=0}^nl_i(x)x_i^k=x^k\ (k=1,2,\cdots, n)$;

(3) $\sum\limits_{i=0}^nl_i(x)(x_i-x)^k=0\ (k=1,2,\cdots, n)$;

(4) $\sum\limits_{i=0}^nl_i(0)x_i^k=
\begin{cases}
    1,&k=0,\\0,&k=1,2,\cdots,n,\\(-1)^nx_0x_1\cdots x_n,&k=n+1.
\end{cases}$
\begin{proof}
    (1) 构造常值函数$f(x) = 1$,则$f^{(k)}(x)=0\ (\forall k\in\mathbb{N}^*)$。设插值区间为$(-\infty, +\infty)$,对于$\mathbb{R}$上任意$n+1$个插值点,由$\text{Lagrange插值法}$知
    \begin{equation*}
        L_n(x) = l_0(x)+l_1(x)+\cdots+l_n(x) = \sum_{i=0}^nl_i(x)
    \end{equation*}
    \begin{equation*}
        R_n(x) = \frac{f^{(n+1)}(\xi)}{(n+1)!}\pi_{n+1}(x)\equiv 0
    \end{equation*}
    由于$R_n(x)\equiv 0$,则$\forall x \in \mathbb{R}$,有$L_n(x) \equiv f(x) = 1$,故
    \begin{equation*}
        \sum_{i=0}^nl_i(x) = 1
    \end{equation*}

    (2) 构造函数$f(x)=  x^k\ (k = 1,2,\cdots, n)$,则$f^{(n+1)}=0$。设插值区间为$(-\infty, +\infty)$,对于$\mathbb{R}$上任意$n+1$个插值点,由$\text{Lagrange插值法}$知
    \begin{equation*}
        L_n(x) = l_0(x)x_0^k+l_1(x)x_1^k+\cdots+l_n(x)x_n^k=\sum_{i=0}^nl_i(x)x_i^k
    \end{equation*}
    \begin{equation*}
        R_n(x)=\frac{f^{(n+1)}(\xi)}{(n+1)!}\pi_{n+1}(x)\equiv 0
    \end{equation*}
    由于$R_n(x)\equiv 0$,则$\forall x \in \mathbb{R}$,有$L_n(x) \equiv f(x) = x^k$,故
    \begin{equation*}
        \sum_{i=0}^nl_i(x)x_i^k = x^k
    \end{equation*}

    (3) $\forall t\in\mathbb{R}$,构造函数$f(x)=  (x-t)^k\ (k = 1,2,\cdots, n)$,则$f^{(n+1)}=0$。设插值区间为$(-\infty, +\infty)$,对于$\mathbb{R}$上任意$n+1$个插值点,由$\text{Lagrange插值法}$知
    \begin{equation*}
        L_n(x) = l_0(x)(x_0-t)^k+l_1(x)(x_1-t)^k+\cdots+l_n(x)(x_n-t)^k=\sum_{i=0}^nl_i(x)(x_i-t)^k
    \end{equation*}
    \begin{equation*}
        R_n(x)=\frac{f^{(n+1)}(\xi)}{(n+1)!}\pi_{n+1}(x)\equiv 0
    \end{equation*}
    由于$R_n(x)\equiv 0$,则$\forall x \in \mathbb{R}$,有$L_n(x) \equiv f(x) = (x-t)^k$,所以
    \begin{equation*}
        \sum_{i=0}^nl_i(x)(x_i-t)^k = (x-t)^k
    \end{equation*}
    又由于$t$的任意性,取$t=x$,得
    \begin{equation*}
        \sum_{i=0}^nl_i(x)(x_i-x)^k = 0
    \end{equation*}

    (4) 由 (1),(2) 分别可知$k=0, k=1,2,\cdots, n$的两种情况,下面证明$k=n+1$的情况:

    构造函数$f(x)=  x^{n+1}$,则$f^{(n+1)}=(n+1)!$。设插值区间为$(-\infty, +\infty)$,对于$\mathbb{R}$上任意$n+1$个插值点,由$\text{Lagrange插值法}$知
    \begin{equation*}
        L_n(x) = l_0(x)x_0^{n+1}+l_1(x)x_1^{n+1}+\cdots+l_n(x)x_n^{n+1}=\sum_{i=0}^nl_i(x)x_i^{n+1}
    \end{equation*}
    \begin{equation*}
        R_n(x)=\frac{f^{(n+1)}(\xi)}{(n+1)!}\pi_{n+1}(x) = \pi_{n+1}(x)
    \end{equation*}
    则
    \begin{equation*}
        \begin{aligned}
            &\ f(x)-L_n(x) = R_n(x) = \pi_{n+1}(x)\\
            \Rightarrow &\ \sum_{i=0}^nl_i(x)x_i^{n+1} = x^{n+1}-(x-x_0)(x-x_1)\cdots(x-x_n)\\
            (\text{令}x = 0)\ \Rightarrow&\ \sum_{i=0}^nl_i(0)x_i^{n+1} = (-1)^nx_0x_1\cdots x_n
        \end{aligned}
    \end{equation*}
\end{proof}
\paragraph{4.11}设$f(x) = a_0+a_1x+a_2x^2+\cdots+a_{n-1}x^{n-1}+a_nx^n$,方程$f(x)=0$有$n$个不同的实根$x_1,x_2,\cdots,x_n$,证明

(1)\begin{equation*}
    \sum_{i=1}^n\frac{x_i^k}{f'(x_i)}=\begin{cases}
        0,&0\leqslant k\leqslant n-2,\\
        1/a_n,&k=n-1.
    \end{cases}
\end{equation*}

(2) 设$x_i\neq 0,-1\ (i=1,2,\cdots, n)$,则
\begin{equation*}
    \sum_{i=1}^n\frac{x_i^nf(x_i^{-1})}{f'(x_i)(1+x_i)}=(-1)^n(x_1x_2\cdots x_n-1);
\end{equation*}

(3)\begin{equation*}
    \sum_{i=1}^n\frac{i^k}{(i-1)\cdots(i-i+1)(i-i-1)\cdots(i-n)}=0,\quad k=0,1,\cdots, n-2.
\end{equation*}
\begin{proof}
    由代数基本定理知,$f(x) = a_n(x-x_1)(x-x_2)\cdots(x-x_n) = a_n\pi_n(x)$,

    (1) 令$g(x) = x^k$,根据差商的性质可知,

    \begin{equation*}
        \sum_{i=1}^n\frac{x_i^k}{f'(x_i)}=\sum_{i=1}^n\frac{x_i^k}{a_n\pi'_n(x_i)} = \frac{1}{a_n}g[x_1,x_2,\cdots,x_n]=\frac{g^{(n-1)}(\xi)}{a_n(n-1)!}
    \end{equation*}

    其中$\xi$在$x_1,x_2,\cdots,x_n$之间。当$k\leqslant n-2$时$g^{(n-1)}\equiv 0$,当$k=n-1$时$g^{(n-1)}=(n-1)!$,则
    \begin{equation*}
        \sum_{i=1}^n\frac{x_i^k}{f'(x_i)}=\begin{cases}
            0,&0\leqslant k\leqslant n-2,\\
            1/a_n,&k=n-1.
        \end{cases}
    \end{equation*}
    
    (2) 令$g(x) = x^nf(x^{-1})=a_0x^n+a_1x^{n-1}+\cdots+a_{n-1}x+a_n$,做$g(x)$对$(1+x)$的带余数除法,存在唯一的$q(x),r(x)$,使得
    \begin{equation*}
            g(x) = (1+x)q(x) + r(x)
    \end{equation*}
    其中$\text{deg}\ q(x) = n-1$且首项为$a_n$,由多项式除法可知
    \begin{equation*}
        r(x) = a_0(-1)^n+a_1(-1)^{n-1}+\cdots+a_{n-1}(-1)+a_n=g(-1)=(-1)^nf(-1)
    \end{equation*}

    则

    \begin{equation}
        \begin{aligned}
            \sum_{i=1}^n\frac{x_i^nf(x_i^{-1})}{f'(x_i)(1+x_i)} = &\sum_{i=1}^n\frac{g(x_i)}{f'(x_i)(1+x_i)}\\
            =&\sum_{i=1}^n\left(\frac{q(x_i)}{f'(x_i)}+\frac{r(x_i)}{f'(x_i)(1+x_i)}\right)\\
            \xlongequal[\text{利用}(1)\text{结论}]{\text{将}q(x)\text{首项提出来}}&\ \frac{a_0}{a_n}+\sum_{i=1}^n\frac{(-1)^nf(-1)}{f'(x_i)(1+x_i)}
        \end{aligned}
    \end{equation}

    令$l_i = \dfrac{f(x)}{f'(x_i)(x-x_i)}$,由\textbf{4.10(1)}可知

    \begin{equation}
        \begin{aligned}
            &\sum_{i=1}^nl_i = \sum_{i=1}^n\frac{f(x)}{f'(x_i)(x-x_i)} = 1\\
            (\text{令}x=-1) \Rightarrow& \sum_{i=1}^n\frac{f(-1)}{f'(x_i)(-1-x_i)} = 1\\
            \Rightarrow& \sum_{i=1}^n\frac{f(-1)}{f'(x_i)(1+x_i)} = -1
        \end{aligned}
    \end{equation}

    又由于$f(x) = a_nx^n+a_{n-1}x^{n-1}+\cdots+a_0=a_n(x-x_1)\cdots(x-x_n)$,通过对比常数项系数可以发现:
    \begin{equation}
        a_0=(-1)^na_nx_1x_2\cdots x_n
    \end{equation}

    将$(2),(3)$带回到$(1)$式中,可得
    \begin{equation*}
        \begin{aligned}
            \sum_{i=1}^n\frac{x_i^nf(x_i^{-1})}{f'(x_i)(1+x_i)} =&\ (-1)^nx_1x_2\cdots x_n+(-1)^{n+1}\\
            =&\ (-1)^n(x_1x_2\cdots x_n-1)
        \end{aligned}
    \end{equation*}

    (3) 令$f(x) = (x-1)(x-2)\cdots(x-n)$,$x_i=i\ (i=1,2,\cdots, n)$,由 (1) 知,

    \begin{equation*}
        \sum_{i=1}^n\frac{i^k}{(i-1)\cdots(i-i+1)(i-i-1)\cdots(i-n)}=0,\quad k=0,1,\cdots,n-2.
    \end{equation*}
\end{proof}
\end{document}
