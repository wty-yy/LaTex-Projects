\documentclass[12pt, a4paper, oneside]{ctexart}
\usepackage{amsmath, amsthm, amssymb, bm, color, graphicx, geometry, hyperref, mathrsfs,extarrows, braket}

\linespread{1.5}
\geometry{left=2.54cm,right=2.54cm,top=3.18cm,bottom=3.18cm}
\newenvironment{problem}{\par\noindent\textbf{题目. }}{\bigskip\par}
\newenvironment{solution}{\par\noindent\textbf{解答. }}{\bigskip\par}
\newenvironment{note}{\par\noindent\textbf{注记. }}{\bigskip\par}

% 基本信息
\newcommand{\dt}{\today}
\newcommand{\sj}{概率论}
\newcommand{\vt}{吴天阳 2204210460}

\begin{document}

\pagestyle{empty}
\vspace*{-20ex}
\centerline{\begin{tabular}{*3{c}}
    \parbox[t]{0.3\linewidth}{\begin{center}\textbf{日期}\\ \large \textcolor{blue}{\dt}\end{center}} 
    & \parbox[t]{0.3\linewidth}{\begin{center}\textbf{科目}\\ \large \textcolor{blue}{\sj}\end{center}}
    & \parbox[t]{0.3\linewidth}{\begin{center}\textbf{姓名,学号}\\ \large \textcolor{blue}{\vt}\end{center}} \\ \hline
\end{tabular}}
\vspace*{4ex}

\paragraph{习题1.1}
\paragraph{3.} 考虑正方形$6$个面的中心,从中任意选择$3$个点连成三角形,把剩下的$3$个点也连成三角形,以$A$表示所得到的两个三角形相互全等的事件,则$A$是一个什么样的事件?
\begin{solution}
    必然事件。
\end{solution}

\paragraph{4.} 连续抛掷一枚均匀的骰子,直到六个面都出现为止。以$A$表示所需的抛掷次数不超过$100$的事件,则$A$是一个什么样的事件?
\begin{solution}
    随机事件。
\end{solution}

\paragraph{习题1.2}
\paragraph{4.} 设$A$和$B$为两个事件,试求出所有的事件$X$,使
\begin{equation*}
    (X\cup A)^C\cup(X\cup A^C)^C=B
\end{equation*}
\begin{solution}
    \begin{equation*}
        \begin{aligned}
            B=(X\cup A)^C\cup(X\cup A^C)^C&=((X\cup A)(X\cup A^C))^C\\
            &=(X\cup(XA^C)\cup(AX)\cup\varnothing)^C\\
            &=X^C
        \end{aligned}
    \end{equation*}
    所以,
    \begin{equation*}
        X=B^C
    \end{equation*}
\end{solution}

\paragraph{9.}市场调查员报道了如下数据:在被询问的$1000$名顾客中,有$811$人喜欢巧克力糖,$752$人喜欢夹心糖,$418$人喜欢大白兔糖,$570$人喜欢巧克力糖和夹心糖,$356$人喜欢巧克力糖和大白兔糖,$348$人喜欢夹心糖和大白兔糖,以及$297$人喜欢全部三种糖果。证明这一信息有误。
\begin{proof}
    设事件$A$为喜欢巧克力糖的人,事件$B$为喜欢夹心糖的人,事件$C$为喜欢大白兔糖的人,则
    \begin{equation*}
        |A|=811,|B|=752,|C|=418,|AB|=570,|AC|=356,|BC|=348,|ABC| = 297
    \end{equation*}
    由容斥原理知,
    \begin{equation*}
        \begin{aligned}
            |A\cup B\cup C| &= |A|+|B|+|C|-(|AB|+|AC|+|BC|)+|ABC|\\
            &= 811+752+418-(570+356+348)+297\\
            &= 1004\neq 1000
        \end{aligned}
    \end{equation*}
    所以,该调查员信息有误。
\end{proof}
\paragraph{11.}设事件$\{A_n\}$单调上升,即对任何$n\in \mathbb{N}$,有$A_n\subset A_{n+1}$,试用概率论语言证明
\begin{equation*}
    \limsup_{n\rightarrow \infty}A_n=\bigcup_{n=1}^{\infty}A_n=\liminf_{n\rightarrow\infty}A_n
\end{equation*}
\begin{proof}
    由于$A_n\subset A_{n+1}$,即事件$A_n$蕴涵事件$A_{n+1}$,则$\lim_{n\rightarrow \infty}A_n = \bigcup_{n=1}^{\infty}A_n$,又由于
    \begin{equation*}
        \begin{aligned}
            \limsup_{n\rightarrow\infty}A_n=\lim_{n\rightarrow \infty}\bigcup_{k=n}^{\infty}A_k=\lim_{n\rightarrow \infty}A_n\\
            \liminf_{n\rightarrow\infty}A_n=\lim_{n\rightarrow \infty}\bigcap_{k=n}^{\infty}A_k=\lim_{n\rightarrow \infty}A_n
        \end{aligned}
    \end{equation*}
    综上,
    \begin{equation*}
        \lim_{n\rightarrow \infty}A_n=\bigcup_{n=1}^{\infty}A_n=\limsup_{n\rightarrow\infty}A_n=\liminf_{n\rightarrow\infty}A_n
    \end{equation*}

\end{proof}
\paragraph{12.}进行独立重复的$\text{Bernoulli}$试验。以事件$A_n$表示“事件$A$在第$n$次试验时出现”,事件$B_{n,m}$为“事件$A$在前$n$次试验中出现$m$次”。

(1). 试以$A_i$表示$B_{4,2}$;

(2). 试解释事件$B_m=\bigcap_{n=m}^{\infty}\left(\bigcup_{k=0}^mB_{n,k}\right)$;

(3). 记$B=\bigcup_{m=1}^{\infty}B_m$。试问关系式$\bigcap_{n=1}^{\infty}A_n\subset B^C$与$\bigcap_{n=1}^{\infty}A_n^C\subset B$是否成立?
\begin{solution}
   (1). 通过枚举法,知
   \begin{equation*}
       B_{4,2}=A_1A_2\cup A_1A_3\cup A_1A_4\cup A_2A_3\cup A_2A_4\cup A_3A_4
   \end{equation*} 

   (2). 记
   \begin{equation*}
       C_n = \bigcup_{k=0}^mB_{n, k}
   \end{equation*}

   $C_n$的含义是:“事件$A$在前$n$次中出现的次数小于等于$m$次”,由于
   \begin{equation*}
       \begin{aligned}
           B_{n+1,0}&\subset B_{n,0}\\
           B_{n+1,1}&\subset B_{n,1}\\
           &\vdots\\
           B_{n+1,m}&\subset B_{n, m}
       \end{aligned}
   \end{equation*}

   所以,
   \begin{equation*}
       C_{n+1}\subset C_n
   \end{equation*}

   则$\{C_n\}$是一个不增的集合列,于是
   \begin{equation*}
       \lim_{n\rightarrow\infty}C_n = \bigcap_{n=m}^{\infty}C_n = \bigcap_{n=m}^{\infty}\left(\bigcup_{k=0}^mB_{n, k}\right) = B_m
   \end{equation*}

   故,$B_m$的含义是:“事件$A$在前无穷多次试验中,出现的次数小于等于$m$次”。

   (3). 由$B_m$的含义知,
   \begin{equation*}
       B_m\subset B_{m+1}
   \end{equation*}

   所以,$\{B_m\}$是一个不降的集合列,则
   \begin{equation*}
       \lim_{m\rightarrow\infty}B_m = \bigcup_{m=1}^{\infty}B_m = B
   \end{equation*}

   $B$的含义是:“事件$A$在前无穷多次试验中,出现了有穷多次”。
   
   假设,$\bigcap_{n=1}^{\infty}A_n\subset B$,则存在$m$使得,
   \begin{equation*}
       \bigcap_{n=1}^{\infty}A_n\subset B_m
   \end{equation*}

   但$\bigcap_{n=1}^{m+1}A_n$表示:“事件$A$在前$m+1$次中都出现”,所以$\bigcap_{n=1}^{m+1}A_n\nsubseteq B_m$,矛盾。
   
   故
   \begin{equation*}
       \bigcap_{n=1}^{\infty}A_n\subset B^C
   \end{equation*}

   由于$\bigcap_{n=1}^{\infty}A_n^C$表示:“事件$A$在无穷多次中没有出现过”,再通过$B$的含义,可知第二个关系式成立。

   综上,两个关系式均成立。
\end{solution}
\paragraph{13.}盒中盛有许多黑球和白球,从中相继取出$n$个球,以$A_i$表示第$i$个被取出的球是白球的事件$(1\leqslant i\leqslant n)$,试用$A_i$表示如下各事件:

(1). 所有$n$个球都是白球;(2). 至少有一个白球;(3). 恰有一个白球;(4). 不多于$k$个白球;(5). 不少于$k$个白球;(6). 恰有$k$个白球;(7). 所有$n$个球同色。
\begin{solution}
    (1). “所有$n$个球都是白球”:\begin{equation*}
        \bigcap_{i=1}^nA_i
    \end{equation*}
    (2). “至少有一个白球”:\begin{equation*}
        \bigcup_{i=1}^nA_i
    \end{equation*}
    (3). “恰有一个白球”:\begin{equation*}
        \bigcup_{i=1}^nA_1^CA_2^C\cdots A_i\cdots A_n^C
    \end{equation*}
    (4). 构造双射 \begin{equation*}
        \begin{aligned}
            \sigma: \{0, 1\}^n&\rightarrow \Omega\\
            (a_1,a_2,\cdots,a_n)&\mapsto A_1^{a_1}A_2^{a_2}\cdots A_n^{a_n}
        \end{aligned}
    \end{equation*}
    
    其中,$\Omega$为样本空间,\begin{equation*}
        A_i^j=
        \begin{cases}
            A_i, &j=0;\\
            A_i^C, &j=1.
        \end{cases}
    \end{equation*}

    即,$A_i^0$代表第$i$位取到白球的情况,$A_i^1$代表第$i$位取到黑球的情况。

    于是,“不多于$k$个白球”可以表示为\begin{equation*}
        \bigcup\sigma\left(\left\{(a_1,a_2,\cdots,a_n):a_i\in\{0,1\},\sum_{i=1}^na_i\geqslant n-k\right\}\right)
    \end{equation*}

    (5). 沿用(4)给出的$\sigma$定义,“不少于$k$个白球”可以表示为\begin{equation*}
        \bigcup\sigma\left(\left\{(a_1,a_2,\cdots,a_n):a_i\in\{0,1\},\sum_{i=1}^na_i\leqslant n-k\right\}\right)
    \end{equation*}

    (6). 沿用(4)给出的$\sigma$定义,“恰有$k$个白球”可以表示为\begin{equation*}
        \bigcup\sigma\left(\left\{(a_1,a_2,\cdots,a_n):a_i\in\{0,1\},\sum_{i=1}^na_i= n-k\right\}\right)
    \end{equation*}

    (7). “所有$n$个球同色”:\begin{equation*}
        \left(\bigcap_{i=1}^nA_i\right)\cup\left(\bigcap_{i=1}^nA_i^C\right)
    \end{equation*}
\end{solution}
\paragraph{习题1.3}
\paragraph{4.}考察正方体各个面的中心(一共$6$个点)。从中任意选择$3$个点连成三角形,试求:
(1). 所得的三角形为等边三角形的概率;(2). 所得的三角形为直角等腰三角形的概率。
\begin{solution}
    (1). 设$A$为所得的三角形为等边三角形的事件,由于等边三角形可以通过正方形的一个顶点唯一确定,所以
    \begin{equation*}
        |A| = 8
    \end{equation*}

    总的样本空间为
    \begin{equation*}
        |\Omega| = \binom{6}{3} = 20
    \end{equation*}

    所以
    \begin{equation*}
        \textbf{P}(A) = \frac{8}{20} = \frac{2}{5}
    \end{equation*}

    (2). 设$B$为所得三角形为直角三角形的事件,由于从正方形各个面中心连接所得的三角形,要么是直角三角形,要么是等边三角形,所以
    \begin{equation*}
        \textbf{P}(B) = 1-\textbf{P}(A) = \frac{3}{5}
    \end{equation*}
\end{solution}
\paragraph{8.}一学生宿舍有$6$名学生,试求如下各事件的概率:(1). $6$个人生日都是在星期天;(2). $6$个人的生日都不在星期天;(3). $6$个人生日不都在星期天。
\begin{solution}
    (1). 设事件$A$为$6$个人生日都在星期天,则$|A| = 1^6 = 1$,设样本空间为$\Omega$,则$|\Omega| = 7^6 = 117649$,则
    \begin{equation*}
        \textbf{P}(A) = \frac{1}{117649}
    \end{equation*}

    (2). 设事件$B$为$6$个人的生日都不在星期天,则$|B| = 6^6 = 46656$,则
    \begin{equation*}
        \textbf{P}(B) = \frac{46656}{117649}
    \end{equation*}

    (3). 设事件$C$为$6$个人的生日不都在星期天,由于事件$C$与事件$A$成对立事件,则
    \begin{equation*}
        \textbf{P}(C) = 1-\textbf{P}(A) = \frac{117648}{117649}
    \end{equation*}
\end{solution}
\paragraph{10.}从$0,1,2,\cdots,9$共$10$个数字中不重复地任取$4$个,求它们能排成一个$4$位偶数的概率。
\begin{solution}
    设事件$A$为排成一个$4$位偶数,先排个位数,再排其他数,且$0$不能在最高位,则$|A| = 4\cdot8\cdot8\cdot7+9\cdot8\cdot7 = 2296$,样本空间$|\Omega| = 10\cdot9\cdot8\cdot7=5040$,则
    \begin{equation*}
        \textbf{P}(A) = \frac{2296}{5040} = \frac{287}{630}
    \end{equation*}
\end{solution}
\paragraph{12.}扔一枚均匀的硬币,直到它连续出现两次相同的结果为止,试描述此样本空间,并求下列事件的概率:(1). 试验在第六次之前结束;(2). 必须扔偶数次才能结束。
\begin{solution}
    样本空间$\Omega = \{\{0, 1\}^n: n\in\mathbb{N}_{\geqslant 2}\}$。

    (1). 设事件$A$为试验在第六次之前结束,则
    \begin{equation*}
        \textbf{P}(A) = 2\left((\frac{1}{2})^2+(\frac{1}{2})^3+(\frac{1}{2})^4+(\frac{1}{2})^5\right) = \frac{15}{16}
    \end{equation*}

    (2). 设事件$B$为扔偶数次结束,则
    \begin{equation*}
        \textbf{P}(B) = 2\left((\frac{1}{2})^2+(\frac{1}{2})^4+\cdots+(\frac{1}{2})^{2n}+\cdots\right) = \frac{2}{3}
    \end{equation*}
\end{solution}
\paragraph{18.} 在一个装有$n$个白球、$n$个黑球、$n$个红球的袋中,不放回地任取$m$个球。求其中白、黑、红球分别为$m_1,m_2,m_3(m_1+m_2+m_3=m)$个的概率。
\begin{solution}
    设样本空间为$\Omega$,满足题意的情况为$A$,则
    \begin{equation*}
        \begin{aligned}
            |\Omega| &= \binom{3n}{n}\\
            |A| &= \binom{n}{m_1}\binom{n}{m_2}\binom{n}{m_3}
        \end{aligned}
    \end{equation*}

    综上,
    \begin{equation*}
        \textbf{P}(A) = \frac{\binom{n}{m_1}\binom{n}{m_2}\binom{n}{m_3}}{\binom{3n}{n}}
    \end{equation*}
\end{solution}
\paragraph{习题1.4}
\paragraph{3.}大厅里共有$n+k$个座位,$n$个人随意入座,试求某给定的$m(m\leqslant n)$个座位有人入座的概率。
\begin{solution}
    设$\Omega$为所有“入座方式”的集合,$A$为“给定的座位有人入座”的情况,则
    \begin{equation*}
        \begin{aligned}
            |\Omega| &= (n+k)(n+k-1)\cdots(k+1)\\
            |A| &= \binom{n}{m}m!(n+k-m)\cdots(k+1)
        \end{aligned}
    \end{equation*}
    
    则
    \begin{equation*}
        \textbf{P}(A) = \frac{\binom{n}{m}m!}{(n+k)\cdots(n+k-m+1)}
    \end{equation*}
\end{solution}
\paragraph{6.}罐中有$a$个白球和$b$个黑球,从中无放回地随意抽取两个球。试求如下事件的概率:(1). 两个球的颜色相同;(2). 两个球的颜色不同。
\begin{solution}
    设$\Omega$为“无放回地取出两个球”的情况,$A$为“两个球的颜色相同”的情况,$B$为“两个球的颜色不同”的情况,则
    \begin{equation*}
        \begin{aligned}
            |\Omega|&=\binom{a+b}{2}\\
            |A|&=\binom{a}{2}+\binom{b}{2}\\
            |B|&=\binom{a}{1}\binom{b}{1}
        \end{aligned}
    \end{equation*}
    
    则
    \begin{equation*}
        \begin{aligned}
            \textbf{P}(A) &= \frac{\binom{a}{2}+\binom{b}{2}}{\binom{a+b}{2}}=\frac{a(a-1)+b(b-1)}{(a+b)(a+b-1)}\\
            \textbf{P}(B) &= \frac{ab}{\binom{a+b}{2}} = \frac{2ab}{(a+b)(a+b-1)}
        \end{aligned}
    \end{equation*}
\end{solution}
\paragraph{9.}$n$个人随机地坐成一排,试求出两个指定的人相邻而坐的概率。如果$n$个人坐成一圈,再求该概率。
\begin{solution}
    设$\Omega$为“随机坐成一排”的情况,$A$为“坐成一排,两个指定的人相邻”的情况,则
    \begin{equation*}
        \begin{aligned}
            |\Omega| &= n!\\
            |A| &= 2!\cdot 2^{n-2}\\
            \textbf{P}(A) &= \frac{2^{n-1}}{n!}
        \end{aligned}
    \end{equation*}
    设$\Omega$为“坐成一圈”的情况,$B$为“坐成一圈,两个指定的人相邻”的情况,则
    \begin{equation*}
        \begin{aligned}
            |\Omega| &= \frac{n!}{n}\\
            |A| &= 2!(n-2)!\\
            \textbf{P}(A) &= \frac{2}{n-1}
        \end{aligned}
    \end{equation*}
\end{solution}
\paragraph{15.}罐中有$m$个白球和$n$个黑球$(m>n)$,从中无放回地逐个取出所有的球。试求在某一时刻罐中剩下的白球数目与黑球数目相等的概率。
\begin{solution}
    设$E$为“剩下的白球和黑球数目相等”,$A$为“第一次取出白球”,$B$为“第一次取出黑球”,则$A\subset E$,且$\textbf{P}(A) = \frac{m}{n+m}$
    \begin{equation*}
        \textbf{P}(E) = \textbf{P}(A)+\textbf{P}(EB)
    \end{equation*}
    
    由\textbf{例1.4.7}知,$EB$的折线和$A$的折线一一对应,所以
    \begin{equation*}
        \textbf{P}(E) = \frac{2m}{n+m}
    \end{equation*}
\end{solution}
\paragraph{18.}每一页书都有$N$个符号可能误印,现知全书共有$n$页,$r$个印错的符号。证明:第$1,2,\cdots,n$页分别含有$r_1,r_2,\cdots,r_n$个印错的符号$\left(\sum_{j=1}^nr_j=r\right)$的概率为
\begin{equation*}
    \frac{\binom{N}{r_1}\binom{N}{r_2}\cdots\binom{N}{r_n}}{\binom{nN}{r}}
\end{equation*}
\begin{proof}
    设$\Omega$为“印错$r$个符号”的所有情况,$A$为题目要求的情况,则
    \begin{equation*}
        \begin{aligned}
            |\Omega| &= \binom{nN}{r}\\
            |A| &= \binom{N}{r_1}\binom{N}{r_2}\cdots\binom{N}{r_n}\\
            \textbf{P}(A) &= \frac{\binom{N}{r_1}\binom{N}{r_2}\cdots\binom{N}{r_n}}{\binom{nN}{r}}
        \end{aligned}
    \end{equation*}
\end{proof}
\paragraph{习题1.5}
\paragraph{2.}甲、乙两船欲停靠在同一码头。假设它们都有可能在某天的一昼夜内任何时刻到达,且甲船与乙船到达后各需在码头停留$3$小时与$4$小时。求有船到达时需等待空出码头的概率。
\begin{solution}
    设甲到达码头的时间为$x$,乙到达码头的时间为$y$,则甲需要等待的时间条件为$0<x-y\leqslant 4$记为事件$A$,乙需要等待的时间条件为$0<y-x\leqslant 3$记为事件$B$,由几何概型知,需要空出码头的概率为
    \begin{equation*}
        \begin{aligned}
            \textbf{P}(A)+\textbf{P}(B) = 1-\frac{\frac{21\cdot21+20\cdot20}{2}}{24\cdot24} = \frac{311}{1152}
        \end{aligned}
    \end{equation*}
\end{solution}
\paragraph{7.}把长为$l$的线段任意折成$3$段,试求如下各事件的概率:(1). 它们可构成一个三角形;(2). 它们中最长的不超过$\frac{2l}{3}$。
\begin{solution}
    设三段长度分别为$x,y,z$,则构成三角形所需条件为:
    \begin{equation*}
        \begin{cases}
            x+y+z=l\\
            x+y>\frac{l}{2}\\
            y+z>\frac{l}{2}\\
            x+z>\frac{l}{2}\\
        \end{cases}
    \end{equation*}
    
    由\textbf{例1.5.1}的图像知,构成三角形的概率为$\frac{1}{4}$。

    最大长度不超过$\frac{2l}{3}$所需条件为:
    \begin{equation*}
        \begin{cases}
            x+y+z=l\\
            x\leqslant \frac{2l}{3}\\
            y\leqslant \frac{2l}{3}\\
            z\leqslant \frac{2l}{3}
        \end{cases}
    \end{equation*}

    由几何概型知,最大长度不超过$\frac{2l}{3}$的概率为:
    \begin{equation*}
        1-\frac{3\cdot\left(\frac{\sqrt{2}}{3}l\right)^2\cdot\frac{\sqrt{3}}{4}}{(\sqrt{2}l)^2\cdot\frac{\sqrt{3}}{4}}=\frac{2}{3}
    \end{equation*}
\end{solution}
\paragraph{10.}在平面上画有一些间隔距离均为$a$的平行直线,向该平面投掷一枚直径为$R(R<a/2)$的硬币。试求硬币与任何直线相交的概率。
\begin{solution}
    考虑两条相邻的直线,由于硬币的半径为$R$,所以能覆盖直线的面积为$2R\cdot x$,总面积为$a\cdot x$,则相交概率为$\frac{2R}{a}$。
\end{solution}
\paragraph{14.}向一个正方形中随机抛掷$3$个点,试求它们形成下述三角形的顶点的概率:(1). 任一三角形;(2). 正三角形;(3). 直角三角形。
\begin{solution}
    (1). 由于只有三点共线的情况下,不能形成三角形,而线在二维的测度下为$0$,由几何概型知,形成任一三角形的概率为$1$。

    (2). 由于当确定两点后,要满足正三角形的条件下,第三个点的取值只能在一条线上,线在二维测度下为$0$,由几何概型知,形成正三角形的概率为$0$。

    (3). 同理(2),形成直角三角形的概率也为$0$。
\end{solution}
\end{document}
