\documentclass[12pt, a4paper, oneside]{ctexart}
\usepackage{amsmath, amsthm, amssymb, bm, color, graphicx, geometry, hyperref, mathrsfs,extarrows, braket}

\linespread{1.5}
%\geometry{left=2.54cm,right=2.54cm,top=3.18cm,bottom=3.18cm}
\geometry{left=1.84cm,right=1.84cm,top=2.18cm,bottom=2.18cm}
\newenvironment{problem}{\par\noindent\textbf{题目. }}{\bigskip\par}
\newenvironment{solution}{\par\noindent\textbf{解答. }}{\bigskip\par}
\newenvironment{note}{\par\noindent\textbf{注记. }}{\bigskip\par}

% 基本信息
\newcommand{\dt}{\today}
\newcommand{\sj}{实变函数}
\newcommand{\vt}{吴天阳 2204210460}

\begin{document}

%\pagestyle{empty}
\pagestyle{plain}
\vspace*{-15ex}
\centerline{\begin{tabular}{*3{c}}
    \parbox[t]{0.3\linewidth}{\begin{center}\textbf{日期}\\ \large \textcolor{blue}{\dt}\end{center}} 
    & \parbox[t]{0.3\linewidth}{\begin{center}\textbf{科目}\\ \large \textcolor{blue}{\sj}\end{center}}
    & \parbox[t]{0.3\linewidth}{\begin{center}\textbf{姓名,学号}\\ \large \textcolor{blue}{\vt}\end{center}} \\ \hline
\end{tabular}}
\vspace*{4ex}

\paragraph{习题1.1}
\paragraph{3.}(i)等式$(A-B)\cup C=A-(B-C)$成立的充要条件是什么?

(ii)证明:\begin{equation*}
    \begin{aligned}
        &(A\cup B)-C=(A-C)\cup(B-C)\\
        &A-(B\cup C)=(A-B)\cap(A-C)
    \end{aligned}
\end{equation*}
\begin{solution}
    记$A\cap B = AB$。

    (i). \begin{equation*}
        \begin{aligned}
            &\begin{aligned}
                (A-B)\cup C &= A-(B-C)\\
                (AB^c)\cup C &= A(BC^c)^c\\
                (A\cup C)(B^c\cup C) &= A(B^c\cup C)
            \end{aligned}\\
            \iff& \begin{cases}
                ((A\cup C)-A)(B^c\cup C) = \varnothing\\
                (A-(A\cup C))(B^c\cup C) = \varnothing\Rightarrow \varnothing = \varnothing\\
            \end{cases}\\
            \iff&\begin{aligned}
                \varnothing &= ((A\cup C)-A)(B^c\cup C)= ((A\cup C)A^c)(B^c\cup C)\\
                &= (A^cC)(B^c\cup C)= A^cB^cC\cup A^cC \\
            \end{aligned}\\
            \iff&\begin{cases}
                A^cB^cC = \varnothing\\
                A^cC = \varnothing
            \end{cases}\\
            \iff& C\subset A
        \end{aligned}
    \end{equation*}

    (ii). \begin{equation*}
        \begin{aligned}
            (A-C)\cup(B-C)&=(AC^c)\cup(BC^c)\\
            &= (A\cup B)C^c\\
            &= (A\cup B)-C
        \end{aligned}
    \end{equation*}
    \begin{equation*}
        \begin{aligned}
            A-(B\cup C) &= A(B\cup C)^c\\
            &= A(B^c\cap C^c)\\
            &= (AB^c)\cap(AC^c)\\
            &= (A-B)\cap(A-C)
        \end{aligned}
    \end{equation*}
\end{solution}
\paragraph{5.}设$\{A_n\}$是一列集,

(i) 作$B_1=A_1,B_n=A_n-\left(\bigcup_{i=1}^{n-1}A_i\right)\ (n > 1)$。证明$\{B_n\}$是一列互不相交的集,而且
\begin{equation*}
    \bigcup_{i=1}^nA_i=\bigcup_{i=1}^nB_i,\quad n=1,2,3,\cdots.
\end{equation*}

(ii) 如果$\{A_n\}$是单调减少的集列,那么
\begin{equation*}
    A_1=(A_1-A_2)\cup(A_2-A_3)\cup\cdots\cup(A_n-A_{n+1})\cup\cdots\cup(\bigcap_{i=1}^{\infty}A_i),
\end{equation*}

并且其中各项互不相交。
\begin{solution}
    (i) 对于$\forall n \geqslant 1$,根据$B_n$的定义知,$B_n\subset A_n$,且\begin{equation*}
        \begin{aligned}
            &B_n\cap\left(\bigcup_{i=1}^{n-1}A_i\right) = \varnothing\\
            \Rightarrow&B_n\cap\left(\bigcup_{i=1}^{n-1}B_i\right)=\varnothing
        \end{aligned}
    \end{equation*}

    则$\{B_n\}$是一列互不相交的集。

    下面用归纳法证明
    \begin{equation*}
        \bigcup_{i=1}^{n}A_i=\bigcup_{i=1}^{n}B_i,\quad n=1,2,3,\cdots
    \end{equation*}
    
    当$n=1$时,$A_1=B_1$ 成立。

    假设命题在$k$时成立,下面讨论$k+1$的情况,

    \begin{equation*}
        \begin{aligned}
            &B_{k+1} = A_{k+1}-\left(\bigcup_{i=1}^kA_i\right)=A_{k+1}-\bigcup_{i=1}^kB_i\\
            \Rightarrow&A_{k+1}\subset \bigcup_{i=1}^{k+1}B_i\\
            \Rightarrow&\bigcup_{i=1}^{k+1}A_i\subset\bigcup_{i=1}^{k+1}B_i
        \end{aligned}
    \end{equation*}
    
    又由于$B_i\subset A_i$,则
    \begin{equation*}
        \bigcup_{i=1}^{k+1}B_i\subset\bigcup_{i=1}^{k+1}A_i
    \end{equation*}

    得,
    \begin{equation*}
        \bigcup_{i=1}^{k+1}A_i=\bigcup_{i=1}^{k+1}B_i
    \end{equation*}

    由数学归纳法知,
    \begin{equation*}
        \bigcup_{i=1}^{n}A_i=\bigcup_{i=1}^{n}B_i,\quad n=1,2,3,\cdots
    \end{equation*}

    (ii) 记$B = (A_1-A_2)\cup(A_2-A_3)\cup\cdots\cup(A_n-A_{n+1})\cup\cdots\cup(\bigcap_{i=1}^{\infty}A_i)$,
    
    $\forall x\in A_1$,若$\exists i \geqslant 1$,使得$x\in A_i$,且$x\notin A_{i+1}$,则$x\in A_i-A_{i+1}\subset B$;否则$\forall i\geqslant 1$,有$x\in A_i$,则$x\in \bigcap_{i=1}^{\infty}A_i\subset B$。故$A\subset B$。

    由于$\{A_n\}$单调减小,则$B\subset A$。综上,$A=B$。

    对于任意的$i, j\ (i< j)$,由于$A_j\subset A_{i+1}$,则$A_j\cap A_iA_{i+1}^c = \varnothing$,则
    \begin{equation*}
        (A_i-A_{i+1})\cap(A_j-A_{j+1})\subset(A_iA_{i+1}^c)\cap A_j=\varnothing
    \end{equation*}

    由于$A_{i+2}\subset A_{i+1}$,则
    \begin{equation*}
        \begin{aligned}
            &A_{i+2}\cap(A_i-A_{i+1}) = \varnothing\\
            \Rightarrow&\left(\bigcap_{k=1}^{\infty}A_k\right)\cap(A_i-A_{i+1})=  \varnothing
        \end{aligned}
    \end{equation*}

    所以,$B$中各项互不相交。
\end{solution}
\paragraph{6.}设$A_{2n-1}=\left(0,\frac{1}{n}\right), A_{2n}=(0,n),\ n=1,2,3,\cdots$,求出集列$\{A_n\}$的上限集和下限集。
\begin{solution}
    通过上下极限定义可知,\begin{equation*}
        \begin{aligned}
            &\limsup_{n\rightarrow \infty}A_n = (0, +\infty)\\
            &\liminf_{n\rightarrow \infty}A_n = \varnothing
        \end{aligned}
    \end{equation*}
\end{solution}
\paragraph{8.}证明:(i) $A\triangle B = (A\cap B^c)\cup(A^c\cap B)$;

(ii) $A\cup B=(A\triangle B)\cup(A\cap B)$;

(iii) $\chi_{A\triangle B}(x) = |\chi_{A}(x)-\chi_B(x)|$;

(iv) $A\triangle B = \{x:\chi_A(x)\neq\chi_B(x)\}$.

\begin{solution}
    记$A\cap B = AB$,全集为$X$。

    (i)
    \begin{equation*}
        \begin{aligned}
            A\triangle B &= (A-B)\cup (B-A)\\
            &= (A B^c)\cup(B A^c)\\
            &= (A B^c)\cup(A^c B)
        \end{aligned}
    \end{equation*}

    (ii) 
    \begin{equation*}
        \begin{aligned}
            (A\triangle B)\cup(AB) &= (AB^c)\cup(A^cB)\cup(AB)\\
            &= (AB^c)\cup\left((A^cB)\cup(AB)\right)\\
            &= (AB^c)\cup\left(X(A^c\cup B)(A\cup B)B\right)\\
            &= (AB^c)\cup B\\
            &= (A\cup B)X\\
            &= A\cup B\\
        \end{aligned}
    \end{equation*}

    (iii) \begin{equation*}
        \begin{aligned}
            &x\in AB^c, \text{则} \chi_{A\triangle B}(x) = 1 = |1-0| = |\chi_A(x)-\chi_B(x)|\\
            &x\in A^cB, \text{则} \chi_{A\triangle B}(x) = 1 = |0-1| = |\chi_A(x)-\chi_B(x)|\\
            &x\in A\cap B, \text{则} \chi_{A\triangle B}(x) = 0 = |1-1| = |\chi_A(x)-\chi_B(x)|\\
            &x\in (A\cup B)^c, \text{则} \chi_{A\triangle B}(x) = 0 = |0-0| = |\chi_A(x)-\chi_B(x)|\\
            \Rightarrow & 
            \chi_{A\triangle B}(x) = |\chi_A(x)-\chi_B(x)|
        \end{aligned}
    \end{equation*}
    
    (iv) \begin{equation*}
        \begin{aligned}
            A\triangle B &= \{x:\chi_{A\triangle B}(x) = 1\}\\
            &= \{x:|\chi_A(x)-\chi_B(x)| = 1\}\\
            &= \{x:\chi_A(x)\neq \chi_B(x)\}
        \end{aligned}
    \end{equation*}
\end{solution}
\paragraph{10.}设集$E$上的实函数列$\{f_n\}$及$f$具有性质$f_1(x)\leqslant f_2(x)\leqslant \cdots \leqslant f_n(x)\leqslant \cdots$,并且$\lim\limits_{n\rightarrow \infty}f_n(x)=f(x)$。证明
\begin{equation*}
    E(f\leqslant c)=\bigcap_{n=1}^{\infty}E(f_n\leqslant c)=\lim_{n\rightarrow \infty}E(f_n\leqslant c).
\end{equation*}
\begin{proof}
    令$A_n = E(f_n\leqslant c)$,由于$f_1(x)\leqslant f_2(x)\leqslant \cdots \leqslant f_n(x)\leqslant \cdots$,则$A_{n+1}\subset A_n$,$\{A_n\}$为单调下降集列,所以
    \begin{equation*}
        \begin{aligned}
            &\lim_{n\rightarrow\infty}A_n=\bigcap_{n=1}^{\infty}A_n\\
            \Rightarrow&\lim_{n\rightarrow\infty}E(f_n\leqslant c)=\bigcap_{n=1}^{\infty}E(f_n\leqslant c)
        \end{aligned}
    \end{equation*}
    
    由于
    \begin{equation*}
        \lim_{n\rightarrow \infty}E(f_n\leqslant c) = \lim_{n\rightarrow \infty}\{x\in E:f_n(x)\leqslant c\}=\{x\in E:f(x)\leqslant c)=E(f\leqslant c)
    \end{equation*}

    故
    \begin{equation*}
        E(f\leqslant c)=\bigcap_{n=1}^{\infty}E(f_n\leqslant c)=\lim_{n\rightarrow \infty}E(f_n\leqslant c).
    \end{equation*}
\end{proof}

\paragraph{12.}设$X$是固定的集,$A\subset X$,$\chi_A(x)$是集$A$的特征函数,证明:

(i) $A=X$等价于$\chi_A(x)\equiv 1,A=\varnothing$等价于$\chi_A(x)\equiv 0$;

(ii) $A\subset B$等价于$\chi_A(x)\leqslant \chi_B(x)$;$A=B$等价于$\chi_A(x)=\chi_B(x)$;

(iii) $\chi_{\bigcup\limits_{\alpha\in N}A_{\alpha}}(x) = \max\limits_{\alpha\in N}\chi_{A_{\alpha}}(x)$;
$\chi_{\bigcap\limits_{\alpha\in N}A_{\alpha}}(x) = \min\limits_{\alpha\in N}\chi_{A_{\alpha}}(x)$;

(iv) 设$\{A_n\}$是一列集,那么极限$\lim\limits_{n\rightarrow\infty}A_n$存在的充要条件是$\lim\limits_{n\rightarrow\infty}\chi_{A_n}(x)$存在,而且当极限存在时,有
\begin{equation*}
    \chi_{\lim\limits_{n\rightarrow\infty}A_n}(x) = \lim_{n\rightarrow \infty}\chi_{A_n}(x).
\end{equation*}
\begin{proof}
    (i) $\forall x\in X$,由于$A=X$,则$x\in A\iff \chi_A(x) = 1$,由$x$的任意性知,$\chi_A(x)\equiv 1$。

    $\forall x\in X$,由于$A=\varnothing$,则$x\notin A\iff \chi_A(x) = 0$,由$x$的任意性知,$\chi_A(x)\equiv 0$。

    (ii) $A\subset B\iff\forall x\in A, x\in B\iff \forall x\in A, \chi_A(x)= \chi_B(x)\iff\forall x\in X, \chi_A(x)\leqslant \chi_B(x)$。

    $A=B\iff A\subset B \text{且} B\subset A\iff \chi_A(x)\leqslant \chi_B(x) \text{且} \chi_A(x)\geqslant \chi_B(x)\iff \chi_A(x)=\chi_B(x)$。

    (iii) 设$\{A_\alpha\}_{\alpha\in N}=\{A_1,A_2,\cdots,A_n\}$,当$n=2$时,$\forall x\in A_1\cup A_2$,有$\chi_{A_1}(x)=1$或$\chi_{A_2}(x)=1$,则$\chi_{A_1\cup A_2} = \max(\chi_{A_1}(x), \chi_{A_2}(x))$,由数学归纳法知,
    \begin{equation*}
        \chi_{\bigcup\limits_{\alpha\in N}A_{\alpha}}(x) = \max\limits_{\alpha\in N}\chi_{A_{\alpha}}(x)
    \end{equation*}

    同理可得,
    \begin{equation*}
        \chi_{\bigcap\limits_{\alpha\in N}A_{\alpha}}(x) = \min\limits_{\alpha\in N}\chi_{A_{\alpha}}(x)
    \end{equation*}

    (iv). $\forall x\in \overline{\lim}A_n$,当且仅当,$x$属于无穷多项$A_{n_1},A_{n_2},\cdots,A_{n_k},\cdots$中,所以有
    \begin{equation*}
        \chi_{\overline{\lim}A_n}=\overline{\lim}\chi_{A_n}
    \end{equation*}

    同理可得,
    \begin{equation*}
        \chi_{\underline{\lim}A_n}=\underline{\lim}\chi_{A_n}
    \end{equation*}

    所以,
    \begin{equation*}
        \begin{aligned}
        \lim A_n\text{存在}&\iff \overline{\lim}A_n=\underline{\lim}A_n\iff \chi_{\overline{\lim}A_n}=\chi_{\overline{\lim}A_n}\\
        &\iff\overline{\lim}\chi_{A_n}=\underline{\lim}\chi_{A_n}\iff\lim\chi_{A_n}\text{存在}
        \end{aligned}
    \end{equation*}

    由$\underline{\lim}A_n\subset\lim A_n\subset \overline{\lim}A_n$,知
    \begin{equation*}
        \underline{\lim}\chi_{A_n} = \chi_{\underline{\lim}A_n}\leqslant\chi_{\lim A_n}\leqslant\chi_{\overline{\lim}A_n}=\overline{\lim}\chi_{A_n}
    \end{equation*}

    由夹逼定理知,当极限存在时,有
    \begin{equation*}
        \chi_{\lim\limits_{n\rightarrow \infty} A_n} =  \lim\limits_{n\rightarrow\infty}\chi_{A_n}
    \end{equation*}

\end{proof}
\paragraph{14.}设$F,E_1$及$E_2$是$X$的任意三个子集,记$F_1=F\cap(E_1\cap E_2^c)^c$,证明:

(i) $F_1\cap E_1\cap E_2=F\cap E_1\cap E_2$;

(ii) $F_1\cap E_1\cap E_2^c=\varnothing$;

(iii) $F_1\cap E_1^c\cap E_2=F\cap E_1^c\cap E_2$;

(iv) $F_1\cap E_1^c\cap E_2^c=F\cap E_1^c\cap E_2^c$.

\begin{proof}
    $F_1=F\cap(E_1\cap E_2^c)^c = F\cap(E_1^c\cup E_2)$

    (i)\begin{equation*}
        \begin{aligned}
            F_1\cap E_1\cap E_2=&F\cap(E_1^c\cup E_2)\cap E_1\cap E_2\\
            \xlongequal{E_2\subset (E_1^c\cup E_2)}&F\cap E_1\cap E_2
        \end{aligned}
    \end{equation*}

    (ii) \begin{equation*}
        \begin{aligned}
            F_1\cap E_1\cap E_2^c=&F\cap(E_1^c\cup E_2)\cap E_1\cap E_2^c\\
            =&F\cap(E_1^c\cup E_2)\cap(E_1^c\cup E_2)^c\\
            =& F\cap \varnothing = \varnothing
        \end{aligned}
    \end{equation*}

    (iii) \begin{equation*}
        \begin{aligned}
            F_1\cap E_1^c\cap E_2=&F\cap(E_1^c\cup E_2)\cap E_1^c\cap E_2\\
            \xlongequal{E_2\subset (E_1^c\cup E_2)}&F\cap E_1^c\cap E_2
        \end{aligned}
    \end{equation*}

    (iv) \begin{equation*}
        \begin{aligned}
            F_1\cap E_1^c\cap E_2^c=&F\cap(E_1^c\cup E_2)\cap E_1^c\cap E_2^c\\
            \xlongequal{E_1^c\subset (E_1^c\cup E_2)}&F\cap E_1^c\cap E_2^c
        \end{aligned}
    \end{equation*}
\end{proof}

\end{document}
