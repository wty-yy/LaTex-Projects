\documentclass[12pt, a4paper, oneside]{ctexart}
\usepackage{amsmath, amsthm, amssymb, bm, color, graphicx, geometry, mathrsfs,extarrows, braket, booktabs, array, wrapfig, enumitem}
\usepackage[colorlinks,linkcolor=red,anchorcolor=blue,citecolor=blue,urlcolor=blue,menucolor=black]{hyperref}
%%%% 设置中文字体 %%%%
% fc-list -f "%{family}\n" :lang=zh >d:zhfont.txt 命令查看已有字体
\setCJKmainfont[
    BoldFont=方正黑体_GBK,  % 黑体
    ItalicFont=方正楷体_GBK,  % 楷体
    BoldItalicFont=方正粗楷简体,  % 粗楷体
    Mapping = fullwidth-stop  % 将中文句号“.”全部转化为英文句号“.”,
]{方正书宋简体}  % !!! 注意在Windows中运行请改为“方正书宋简体.ttf” !!!
%%%% 设置英文字体 %%%%
\setmainfont{Minion Pro}
\setsansfont{Calibri}
\setmonofont{Consolas}

%%%% 设置行间距与页边距 %%%%
\linespread{1.4}
%\geometry{left=2.54cm,right=2.54cm,top=3.18cm,bottom=3.18cm}
\geometry{left=1.84cm,right=1.84cm,top=2.18cm,bottom=2.18cm}

%%%% 图片相对路径 %%%%
\graphicspath{{figures/}} % 当前目录下的figures文件夹, {../figures/}则是父目录的figures文件夹
\setlength{\abovecaptionskip}{-0.2cm}  % 缩紧图片标题与图片之间的距离
\setlength{\belowcaptionskip}{0pt} 

%%%% 缩小item,enumerate,description两行间间距 %%%%
\setenumerate[1]{itemsep=0pt,partopsep=0pt,parsep=\parskip,topsep=5pt}
\setitemize[1]{itemsep=0pt,partopsep=0pt,parsep=\parskip,topsep=5pt}
\setdescription{itemsep=0pt,partopsep=0pt,parsep=\parskip,topsep=5pt}

%%%% 自定义公式 %%%%
\everymath{\displaystyle} % 默认全部行间公式
\DeclareMathOperator*\uplim{\overline{lim}} % 定义上极限 \uplim_{}
\DeclareMathOperator*\lowlim{\underline{lim}} % 定义下极限 \lowlim_{}
\DeclareMathOperator*{\argmax}{arg\,max}  % 定义取最大值的参数 \argmax_{}
\DeclareMathOperator*{\argmin}{arg\,min}  % 定义取最小值的参数 \argmin_{}
\let\leq=\leqslant % 将全部leq变为leqslant
\let\geq=\geqslant % geq同理
\DeclareRobustCommand{\rchi}{{\mathpalette\irchi\relax}}
\newcommand{\irchi}[2]{\raisebox{\depth}{$#1\chi$}} % 使用\rchi将\chi居中

%%%% 自定义环境配置 %%%%
\newcounter{problem}  % 问题序号计数器
\newenvironment{problem}[1][]{\stepcounter{problem}\par\noindent\textbf{题目\arabic{problem}. #1}}{\smallskip\par}
\newenvironment{solution}[1][]{\par\noindent\textbf{#1解答. }}{\smallskip\par}  % 可带一个参数表示题号\begin{solution}{题号}
\newenvironment{note}{\par\noindent\textbf{注记. }}{\smallskip\par}
\newenvironment{remark}{\begin{enumerate}[label=\textbf{注\arabic*.}]}{\end{enumerate}}
\BeforeBeginEnvironment{minted}{\vspace{-0.5cm}}  % 缩小minted环境距上文间距
\AfterEndEnvironment{minted}{\vspace{-0.2cm}}  % 缩小minted环境距下文间距

%%%% 一些宏定义 %%%%
\def\bd{\boldsymbol}        % 加粗(向量) boldsymbol
\def\disp{\displaystyle}    % 使用行间公式 displaystyle(默认)
\def\weekto{\rightharpoonup}% 右半箭头
\def\tsty{\textstyle}       % 使用行内公式 textstyle
\def\sign{\text{sign}}      % sign function
\def\wtd{\widetilde}        % 宽波浪线 widetilde
\def\R{\mathbb{R}}          % Real number
\def\N{\mathbb{N}}          % Natural number
\def\Z{\mathbb{Z}}          % Integer number
\def\Q{\mathbb{Q}}          % Rational number
\def\C{\mathbb{C}}          % Complex number
\def\K{\mathbb{K}}          % Number Field
\def\P{\mathbb{P}}          % Polynomial
\def\d{\mathrm{d}}          % differential operator
\def\e{\mathrm{e}}          % Euler's number
\def\i{\mathrm{i}}          % imaginary number
\def\re{\mathrm{Re}}        % Real part
\def\im{\mathrm{Im}}        % Imaginary part
\def\res{\mathrm{Res}}      % Residue
\def\ker{\mathrm{Ker}}      % Kernel
\def\vspan{\mathrm{vspan}}  % Span  \span与latex内核代码冲突改为\vspan
\def\L{\mathcal{L}}         % Loss function
\def\O{\mathcal{O}}         % big O notation
\def\A{\mathcal{A}}         % 仿射坐标系
\def\sA{\mathscr{A}}        % 点空间
\def\F{\mathcal{F}}         % 光滑函数空间
\def\sF{\mathscr{F}}        % 光滑函数商空间
\def\wdh{\widehat}          % 宽帽子 widehat
\def\ol{\overline}          % 上横线 overline
\def\ul{\underline}         % 下横线 underline
\def\add{\vspace{1ex}}      % 增加行间距
\def\del{\vspace{-1.5ex}}   % 减少行间距

%%%% 定理类环境的定义 %%%%
\newtheorem{theorem}{定理}

%%%% 基本信息 %%%%
\newcommand{\RQ}{\today} % 日期
\newcommand{\km}{微分几何} % 科目
\newcommand{\bj}{强基数学002} % 班级
\newcommand{\xm}{吴天阳} % 姓名
\newcommand{\xh}{2204210460} % 学号

\begin{document}

%\pagestyle{empty}
\pagestyle{plain}
\vspace*{-15ex}
\centerline{\begin{tabular}{*5{c}}
    \parbox[t]{0.25\linewidth}{\begin{center}\textbf{日期}\\ \large \textcolor{blue}{\RQ}\end{center}} 
    & \parbox[t]{0.2\linewidth}{\begin{center}\textbf{科目}\\ \large \textcolor{blue}{\km}\end{center}}
    & \parbox[t]{0.2\linewidth}{\begin{center}\textbf{班级}\\ \large \textcolor{blue}{\bj}\end{center}}
    & \parbox[t]{0.1\linewidth}{\begin{center}\textbf{姓名}\\ \large \textcolor{blue}{\xm}\end{center}}
    & \parbox[t]{0.15\linewidth}{\begin{center}\textbf{学号}\\ \large \textcolor{blue}{\xh}\end{center}} \\ \hline
\end{tabular}}
\begin{center}
    \zihao{3}\textbf{第五次作业}
\end{center}\vspace{-0.2cm}
\begin{problem}[3.7练习1.]证明命题3.6的逆命题也成立,即如果一条仿射空间中的曲线在其每个点附近都可以被某个广义坐标系的映射映射为$\R^3$中的
    连续/可微/光滑/正则曲线段,那么该曲线本身也是连续/可微/光滑/正则。
\end{problem}
\begin{proof}
    设$\gamma:(-\varepsilon, \varepsilon)\to \sA^3$为$\sA^3$中的曲线,$A = \gamma(t_0),\ t_0\in (-\varepsilon, \varepsilon)$,
    $U$为$A$在$\sA^3$中的邻域,存在广义坐标系$\{U, \varphi_U\}$,坐标映射为$\varphi_U:\sA^3\to\{y^i\}$,则存在$\varepsilon'$使得
    \begin{equation*}
        (t_0-\varepsilon',t_0+\varepsilon')\xrightarrow{\gamma}U\xrightarrow{\varphi_U}\varphi_U(U)\subset\R^3
    \end{equation*}
    连续/可微/光滑,且$\varphi_U\circ\gamma$是$\R^3$上的正则曲线。

    设$\A = \{O, \bd{e}_i\}$为仿射坐标系,坐标映射为$\varphi_{\A}$,则
    \begin{equation*}
        \varphi_{\A}\circ\gamma = (\varphi_{\A}\circ\varphi_{U}^{-1})\circ(\varphi_U\circ\gamma)
    \end{equation*}
    由于$\varphi_{\A}\circ\varphi_{U}^{-1}$是光滑映射,且$\varphi_U\circ\gamma$连续/可微/光滑,于是$\varphi_{\A}\circ\gamma$
    是连续/可微/光滑映射,所以$\gamma$在$\sA^3$上连续。

    由于
    \begin{equation*}
        \frac{\d(\varphi_{\A}\circ\gamma)}{\d t}(t) = \frac{\partial x^j}{\partial y^i}\bigg|_{\varphi_U(\gamma(t))}\frac{\d (\varphi_U\circ\gamma)}{\d t}(t),\quad
        t\in(t_0-\varepsilon', t_0+\varepsilon')
    \end{equation*}
    其中矩阵$\frac{\partial x^j}{\partial y^i}\bigg|_{\varphi_U(\gamma(t))}$非退化,且$\frac{\d (\varphi_U\circ\gamma)}{\d t}(t)\neq 0$,于是
    $\frac{\d(\varphi_{\A}\circ\gamma)}{\d t}(t)\neq 0$,故$\gamma$为正则曲线。
\end{proof}
\begin{problem}[3.8练习1.]
    证明\textbf{定义3.14}中引入的关系$\sim$是$\F_A$上的一个等价关系,并证明\textbf{定义3.15}在$\F_A/\sim$上定义的线性运算的合理性。
\end{problem}
\begin{proof}
    设$A$为$\sA^n$中的点,记$\F_A$为定义在$A$附近的光滑函数全体,$\forall f,g,h\in\F_A$,设$D$为定义在$A$的导算子$D$,定义关系如下
    \begin{equation*}
        f\sim g\iff Df = Dg,\quad(\forall D\in \F_A)
    \end{equation*}
    下证其是一个等价关系:

    1. 自反性:由于$Df = Df$,则$f\sim f$。

    2. 对称性:由于$Df = Dg = Df$,则$f\sim g\iff g\sim f$。

    3. 传递性:由于$Df = Dg = Dh$,则$f\sim g,g\sim h\iff f\sim h$。

    综上,$f\sim g$是一个等价关系。

    记$\sF_A = \F_A/\sim$为$A$点的余向量空间,函数$f$所在的等价类记为$\bar{f}$,定义如下线性运算
    \begin{equation}
        \alpha\bar{f}+\beta\bar{g} = \ol{\alpha f+\beta g},\quad(\forall \alpha,\beta\in\R)
    \end{equation}
    下证明上式与代表元选取无关:设$\bar{f}_1 = \bar{f}_2,f_1\neq f_2$,$\bar{g}_1 = \bar{g}_2,g_1\neq g_2$,则
    \begin{equation*}
        D(\alpha f_1 + \beta g_1) = \alpha Df_1 + \beta Dg_1 = \alpha Df_2 + \beta Dg_2 = D(\alpha f_2 + \beta g_2)
    \end{equation*}
    于是$\ol{\alpha f_1 + \beta g_1} = \ol{\alpha f_2 + \beta g_2}$,故(1)式结果与代表元选取无关。
\end{proof}
\begin{problem}[3.8练习2.]
    证明$\ol{fg} = g(A)\bar{f} + f(A)\bar{g}$.
\end{problem}
\begin{proof}
    由于$D(fg) = f(A)Dg + Df\cdot g(A) = D(f(A)g + g(A)f)$,于是$\ol{fg} = \ol{f(A)g+g(A)f} = f(A)\bar{g} + g(A)\bar{f}$.
\end{proof}
\begin{problem}[3.8练习3.]
    考虑$\R^3\backslash r = 0$或$x^1 = x^2 = 0$上的标准柱面坐标系:
    \begin{equation*}
        x^1 = r\cos\theta,\ x^2 = r\sin\theta,\ x^3 = z
    \end{equation*}
    在每个点写出自然余标架场,表出在$(\d x^1, \d x^2,\d x^3)$这组基下。
\end{problem}
\begin{solution}
    由于
    \begin{equation*}
        \begin{bmatrix}
            \partial r\\
            \partial \theta\\
            \partial z
        \end{bmatrix} = \begin{bmatrix}
            \tsty\frac{\partial x^1}{\partial r}&\tsty\frac{\partial x^2}{\partial r}&\tsty\frac{\partial x^3}{\partial r}\\
            \tsty\frac{\partial x^1}{\partial \theta}&\tsty\frac{\partial x^2}{\partial \theta}&\tsty\frac{\partial x^3}{\partial \theta}\\
            \tsty\frac{\partial x^1}{\partial z}&\tsty\frac{\partial x^2}{\partial z}&\tsty\frac{\partial x^3}{\partial z}
        \end{bmatrix}\begin{bmatrix}
            \partial_1\\
            \partial_2\\
            \partial_3
        \end{bmatrix} = \begin{bmatrix}
            \cos\theta\partial_1+\sin\theta\partial_2\\
            -r\sin\theta\partial_1+r\cos\theta\partial_2\\
            \partial_3
        \end{bmatrix}
    \end{equation*}
    由于$r,\theta,z$可以视为$\sA^3$中的光滑函数,所以$\d r,\d \theta,\d z$是余向量空间中的元素,
    可以在$\d x^1,\d x^2,\d x^3$中线性表出,令$\d r = a\d x^1+b\d x^2 + c\d x^3$,由对偶基的性质可知
    \begin{equation*}
        \begin{cases}
            \langle\partial r,\d r\rangle = 1,\\
            \langle\partial \theta,\d r\rangle = 0,\\
            \langle\partial z,\d r\rangle = 0.
        \end{cases}\Rightarrow\begin{cases}
            \cos\theta a + \sin\theta b = 1,\\
            -r\sin\theta a + r\cos\theta b = 0,\\
            c = 0.
        \end{cases}\Rightarrow\begin{cases}
            a = \cos\theta,\\
            b = \sin\theta,\\
            c = 0.
        \end{cases}\Rightarrow \d r = \cos\theta\d x^1 + \sin\theta\d x^2.
    \end{equation*}

    类似地,可以令$\d \theta = a\d x^1+b\d x^2 + c\d x^3$,由对偶基的性质可知
    \begin{equation*}
        \begin{cases}
            \langle\partial r,\d r\rangle = 0,\\
            \langle\partial \theta,\d r\rangle = 1,\\
            \langle\partial z,\d r\rangle = 0.
        \end{cases}\Rightarrow\begin{cases}
            \cos\theta a + \sin\theta b = 0,\\
            -r\sin\theta a + r\cos\theta b = 1,\\
            c = 0.
        \end{cases}\Rightarrow\begin{cases}
            a = -\sin\theta/r,\\
            b = \cos\theta/,\\
            c = 0.
        \end{cases}\Rightarrow \d \theta = \frac{1}{r}(-sin\theta\d x^1 + \cos\theta\d x^2).
    \end{equation*}
    令$\d z = a\d x^1+b\d x^2 + c\d x^3$,由对偶基的性质可知
    \begin{equation*}
        \begin{cases}
            \langle\partial r,\d r\rangle = 0,\\
            \langle\partial \theta,\d r\rangle = 0,\\
            \langle\partial z,\d r\rangle = 1.
        \end{cases}\Rightarrow\begin{cases}
            \cos\theta a + \sin\theta b = 0,\\
            -r\sin\theta a + r\cos\theta b = 0,\\
            c = 1.
        \end{cases}\Rightarrow\begin{cases}
            a = 0,\\
            b = 0,\\
            c = 1.
        \end{cases}\Rightarrow \d z = \d x^3.
    \end{equation*}
\end{solution}
\begin{problem}[3.9练习1.]
    严格证明\textbf{引理3.11}:设$V$为$n$维实线性空间,则$\dim \text{Sym}_2(V^*) = \frac{n(n+1)}{2}$,如果$\{e_i\}$为$V$的一组基,
    那么
    \begin{equation*}
        e^{*i}e^{*j} = \frac{1}{2}(e^{*i}\otimes e^{*j}+e^{*j}\otimes e^{*i}),\quad i\leq j
    \end{equation*}
    为$\text{Sym}_2(V^*)$的一组基。
\end{problem}
\begin{proof}
    设$V$为有限维实线性空间,$\{e_1,\cdots e_n\}$为$V$的一组基,$\forall g\in\text{Sym}_2(V^*)$,
    则
    \begin{equation*}
        g = \sum_{i\leq j}g_{ij}e^{*i}e^{*j},\quad\text{其中}g_{ij} = g(e_i,e_j)
    \end{equation*}
    所以$e^{*i}e^{*j}$能够线性表出$\text{Sym}_2(V^*)$上的元素。令$\lambda_{ij}\in \R,\ (i\leq j)$满足
    \begin{equation*}
        \sum_{i\leq j}\lambda_{ij}e^{*i}e^{*j} = 0
    \end{equation*}
    代入$(e_i, e_j),\ (i\leq j)$可得
    \begin{align*}
        &\ \sum_{i\leq j}\lambda_{ij}e^{*i}e^{*j}(e_i,e_j) = \sum_{i\leq j}\frac{\lambda_{ij}}{2}(e^{*i}\otimes e^{*j}(e_i,e_j)+e^{*j}\otimes e^{*i}(e_i,e_j)) = 0\\
        \Rightarrow&\ \frac{\lambda_{ij}}{2} = 0\Rightarrow \lambda_{ij} = 0,\quad(i\leq j)
    \end{align*}
    所以$e^{*i}e^{*j}$在$V^*\otimes V^*$上线性无关。

    综上,$\{e^{*i}e^{*j}:i\leq j\}$是$\text{Sym}_2(V^*)$的一组基。
\end{proof}
\begin{problem}[3.9练习3.]
    求$\R^3\backslash\{0\}$上的球坐标系
    \begin{equation*}
        x^1 = r\sin\theta\cos\varphi,\ x^2 = r\sin\theta\sin\varphi,\ x^3 = r\cos\theta
    \end{equation*}
    在每个点写出自然余标架场,在$(\d x^1,\d x^2,\d x^3)$下表出。
\end{problem}
\begin{solution}
    由于
    \begin{equation*}
        \begin{bmatrix}
            \partial r\\
            \partial \theta\\
            \partial \varphi
        \end{bmatrix} = \begin{bmatrix}
            \sin\theta\cos\varphi&\sin\theta\sin\varphi&\cos\theta\\
            r\cos\theta\cos\varphi&r\cos\theta\sin\varphi&-r\sin\theta\\
            -r\cos\theta\sin\varphi&r\sin\theta\cos\varphi&0
        \end{bmatrix}\begin{bmatrix}
            \partial_1\\
            \partial_2\\
            \partial_3
        \end{bmatrix}
    \end{equation*}
    由于$r,\theta,\varphi$可以视为$\sA^3$中的光滑函数,所以$\d r,\d \theta,\d z$是余向量空间中的元素,
    可以在$\d x^1,\d x^2,\d x^3$中线性表出,令$\d r = a\d x^1+b\d x^2+c\d x^3$,由对偶基性质可知
    \begin{equation*}
        \begin{cases}
            \langle\partial r,\d r\rangle = 1,\\
            \langle\partial \theta,\d r\rangle = 0,\\
            \langle\partial \varphi,\d r\rangle = 0.
        \end{cases}
        \hspace{-0.35cm}\Rightarrow\begin{cases}
            \sin\theta\cos\varphi a+\sin\theta\sin\varphi b + \cos\theta c = 1,\\
            r\cos\theta\cos\varphi a+r\cos\theta\sin\varphi b - r\sin\theta c = 0,\\
            -r\sin\theta\sin\varphi + r\sin\theta\cos\varphi b = 0.
        \end{cases}\hspace{-0.35cm}\Rightarrow\begin{cases}
            a = \sin\theta\cos\varphi,\\
            b = \sin\theta\sin\varphi,\\
            c = \cos\theta.
        \end{cases}
    \end{equation*}
    于是$\d r = \sin\theta\cos\varphi \d x^1 + \sin\theta\sin\varphi\d x^2 + \cos\theta\d x^3$;
    同理,令$\d \theta = a\d x^1 + b\d x^2 + c\d x^3$,则
    \begin{equation*}
        \begin{cases}
            \langle\partial r,\d r\rangle = 0,\\
            \langle\partial \theta,\d r\rangle = 1,\\
            \langle\partial \varphi,\d r\rangle = 0.
        \end{cases}
        \hspace{-0.35cm}\Rightarrow\begin{cases}
            \sin\theta\cos\varphi a+\sin\theta\sin\varphi b + \cos\theta c = 0,\\
            r\cos\theta\cos\varphi a+r\cos\theta\sin\varphi b - r\sin\theta c = 1,\\
            -r\sin\theta\sin\varphi + r\sin\theta\cos\varphi b = 0.
        \end{cases}\hspace{-0.35cm}\Rightarrow\begin{cases}
            a = \cos\theta\cos\varphi/r,\\
            b = \cos\theta\sin\varphi/r,\\
            c = -\sin\theta/r.
        \end{cases}
    \end{equation*}
    于是$\d \theta = \frac{1}{r}(\cos\theta\cos\varphi\d x^1 + \cos\theta\sin\varphi\d x^2 - \sin\theta\d x^3)$;
    令$\d \varphi = a\d x^1 + b\d x^2 + c\d x^3$,则
    \begin{equation*}
        \begin{cases}
            \langle\partial r,\d r\rangle = 0,\\
            \langle\partial \theta,\d r\rangle = 0,\\
            \langle\partial \varphi,\d r\rangle = 1.
        \end{cases}
        \hspace{-0.35cm}\Rightarrow\begin{cases}
            \sin\theta\cos\varphi a+\sin\theta\sin\varphi b + \cos\theta c = 0,\\
            r\cos\theta\cos\varphi a+r\cos\theta\sin\varphi b - r\sin\theta c = 0,\\
            -r\sin\theta\sin\varphi + r\sin\theta\cos\varphi b = 1.
        \end{cases}\hspace{-0.35cm}\Rightarrow\begin{cases}
            a = -\frac{\sin\varphi}{r\sin\theta},\add\\
            b = \frac{\cos\varphi}{r\sin\theta},\\
            c = 0.
        \end{cases}
    \end{equation*}
    于是$\d \varphi = \frac{1}{r\sin\theta}(-\sin\varphi\d x^1 + \cos\varphi\d x^2)$.\add
\end{solution}
\begin{problem}[3.9练习4.]
    将$\R^3$上的欧几里德度量表出在球坐标系的自然余标架场下。
\end{problem}
\begin{solution}
    由于在$\R^3$的欧式度量可以表示为$g = (\d x^1)^2 + (\d x^2)^2 + (\d x^3)^2 = g_{rr}\d r\d r + g_{\theta\theta}\d \theta\d \theta + 
    g_{\varphi\varphi}\d\varphi\d\varphi + 2g_{r\theta}\d r\d\theta+2g_{r\varphi}\d r\d\varphi+2g_{\theta\varphi}\d\theta\d\varphi$,
    又由于
    \begin{align*}
        g_{rr} = g(\partial_r,\partial_r) =&\ (\d x^1(\partial_r))^2 + (\d x^2(\partial_r))^2 + (\d x^3(\partial_r))^2\\
        =&\ \sin^2\theta\cos^2\varphi+\sin^2\theta\sin^2\varphi+\cos^2\theta = 1
    \end{align*}
    同理可得$g_{\theta\theta} = r^2, g_{\varphi\varphi} = r^2\sin^2\theta, g_{r\theta}=g_{r\varphi}=g_{\theta\varphi} = 0$。

    综上,欧式空间中的度量在球坐标下的表出为
    \begin{equation*}
        g = (\d r)^2 + r^2(\d \theta)^2 + r^2\sin^2\theta(\d \varphi)^2
    \end{equation*}
\end{solution}
\end{document}