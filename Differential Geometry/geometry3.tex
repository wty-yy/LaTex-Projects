\documentclass[12pt, a4paper, oneside]{ctexart}
\usepackage{amsmath, amsthm, amssymb, bm, color, graphicx, geometry, mathrsfs,extarrows, braket, booktabs, array, wrapfig, enumitem}
\usepackage[colorlinks,linkcolor=red,anchorcolor=blue,citecolor=blue,urlcolor=blue,menucolor=black]{hyperref}
%%%% 设置中文字体 %%%%
% fc-list -f "%{family}\n" :lang=zh >d:zhfont.txt 命令查看已有字体
\setCJKmainfont{方正书宋.ttf}[BoldFont = 方正黑体_GBK.ttf, ItalicFont = simkai.ttf, BoldItalicFont = 方正粗楷简体.ttf]
%%%% 设置英文字体 %%%%
\setmainfont{Times New Roman}
\setsansfont{Calibri}
\setmonofont{Consolas}

%%%% 设置行间距与页边距 %%%%
\linespread{1.4}
%\geometry{left=2.54cm,right=2.54cm,top=3.18cm,bottom=3.18cm}
\geometry{left=1.84cm,right=1.84cm,top=2.18cm,bottom=2.18cm}

%%%% 图片相对路径 %%%%
\graphicspath{{figures/}} % 当前目录下的figures文件夹, {../figures/}则是父目录的figures文件夹
\setlength{\abovecaptionskip}{-0.2cm}  % 缩紧图片标题与图片之间的距离
\setlength{\belowcaptionskip}{0pt} 

%%%% 缩小item,enumerate,description两行间间距 %%%%
\setenumerate[1]{itemsep=0pt,partopsep=0pt,parsep=\parskip,topsep=5pt}
\setitemize[1]{itemsep=0pt,partopsep=0pt,parsep=\parskip,topsep=5pt}
\setdescription{itemsep=0pt,partopsep=0pt,parsep=\parskip,topsep=5pt}

%%%% 自定义公式 %%%%
\everymath{\displaystyle} % 默认全部行间公式
\DeclareMathOperator*\uplim{\overline{lim}} % 定义上极限 \uplim_{}
\DeclareMathOperator*\lowlim{\underline{lim}} % 定义下极限 \lowlim_{}
\DeclareMathOperator*{\argmax}{arg\,max}  % 定义取最大值的参数 \argmax_{}
\DeclareMathOperator*{\argmin}{arg\,min}  % 定义取最小值的参数 \argmin_{}
\let\leq=\leqslant % 将全部leq变为leqslant
\let\geq=\geqslant % geq同理
\DeclareRobustCommand{\rchi}{{\mathpalette\irchi\relax}}
\newcommand{\irchi}[2]{\raisebox{\depth}{$#1\chi$}} % 使用\rchi将\chi居中

%%%% 自定义环境配置 %%%%
\newcounter{problem}  % 问题序号计数器
\newenvironment{problem}[1][]{\stepcounter{problem}\par\noindent\textbf{题目\arabic{problem}. #1}}{\smallskip\par}
\newenvironment{solution}[1][]{\par\noindent\textbf{#1解答. }}{\smallskip\par}  % 可带一个参数表示题号\begin{solution}{题号}
\newenvironment{note}{\par\noindent\textbf{注记. }}{\smallskip\par}
\newenvironment{remark}{\begin{enumerate}[label=\textbf{注\arabic*.}]}{\end{enumerate}}
\BeforeBeginEnvironment{minted}{\vspace{-0.5cm}}  % 缩小minted环境距上文间距
\AfterEndEnvironment{minted}{\vspace{-0.2cm}}  % 缩小minted环境距下文间距

%%%% 一些宏定义 %%%%
\def\bd{\boldsymbol}        % 加粗(向量) boldsymbol
\def\disp{\displaystyle}    % 使用行间公式 displaystyle(默认)
\def\weekto{\rightharpoonup}% 右半箭头
\def\tsty{\textstyle}       % 使用行内公式 textstyle
\def\sign{\text{sign}}      % sign function
\def\wtd{\widetilde}        % 宽波浪线 widetilde
\def\R{\mathbb{R}}          % Real number
\def\N{\mathbb{N}}          % Natural number
\def\Z{\mathbb{Z}}          % Integer number
\def\Q{\mathbb{Q}}          % Rational number
\def\C{\mathbb{C}}          % Complex number
\def\K{\mathbb{K}}          % Number Field
\def\P{\mathbb{P}}          % Polynomial
\def\d{\mathrm{d}}          % differential operator
\def\e{\mathrm{e}}          % Euler's number
\def\i{\mathrm{i}}          % imaginary number
\def\re{\mathrm{Re}}        % Real part
\def\im{\mathrm{Im}}        % Imaginary part
\def\res{\mathrm{Res}}      % Residue
\def\ker{\mathrm{Ker}}      % Kernel
\def\vspan{\mathrm{vspan}}  % Span  \span与latex内核代码冲突改为\vspan
\def\L{\mathcal{L}}         % Loss function
\def\O{\mathcal{O}}         % big O notation
\def\wdh{\widehat}          % 宽帽子 widehat
\def\ol{\overline}          % 上横线 overline
\def\ul{\underline}         % 下横线 underline
\def\add{\vspace{1ex}}      % 增加行间距
\def\del{\vspace{-1.5ex}}   % 减少行间距

%%%% 定理类环境的定义 %%%%
\newtheorem{theorem}{定理}

%%%% 基本信息 %%%%
\newcommand{\RQ}{\today} % 日期
\newcommand{\km}{微分几何} % 科目
\newcommand{\bj}{强基数学002} % 班级
\newcommand{\xm}{吴天阳} % 姓名
\newcommand{\xh}{2204210460} % 学号

\begin{document}

%\pagestyle{empty}
\pagestyle{plain}
\vspace*{-15ex}
\centerline{\begin{tabular}{*5{c}}
    \parbox[t]{0.25\linewidth}{\begin{center}\textbf{日期}\\ \large \textcolor{blue}{\RQ}\end{center}} 
    & \parbox[t]{0.2\linewidth}{\begin{center}\textbf{科目}\\ \large \textcolor{blue}{\km}\end{center}}
    & \parbox[t]{0.2\linewidth}{\begin{center}\textbf{班级}\\ \large \textcolor{blue}{\bj}\end{center}}
    & \parbox[t]{0.1\linewidth}{\begin{center}\textbf{姓名}\\ \large \textcolor{blue}{\xm}\end{center}}
    & \parbox[t]{0.15\linewidth}{\begin{center}\textbf{学号}\\ \large \textcolor{blue}{\xh}\end{center}} \\ \hline
\end{tabular}}
\begin{center}
    \zihao{3}\textbf{第二次作业}
\end{center}\vspace{-0.2cm}
\begin{problem}[2.2习题1]
求证本节映射$\eta$定义合理,即$\forall s \in (c,d)$,$\exists! t =\eta(s) \in(a,b)$使得
\begin{equation*}
    \int_a^{\eta(s)}||r'(\eta)||_2\,\d \tau = s-c,
\end{equation*}
并且该映射是$C^1$正则参数变换,并且$\eta'(s) = \frac{1}{||r'(t)||_2}$,从而$||\bar{\bd{r}}'(s)||_2\equiv 1$.
\end{problem}
\begin{proof}
    由于曲线的弧长定义为$s(t) = \int_a^t||\bd{r}(\tau)||\,\d\tau$,则$s'(t) = ||\bd{r}'(t)|| > 0$,
    由反函数定理,则$\exists! t=\eta(s)$,且$\eta\in C^1$,$\eta'(s) = \frac{1}{s'(t)} = \frac{1}{||\bd{r}'(t)||}$,
    而且
    \begin{equation*}
        \int_a^{\eta(s)}||\bd{r}'(\tau)||\,\d \tau = \int_a^ts'(\tau)\,\d \tau = s(\tau)|_a^t = s-c
    \end{equation*}
\end{proof}
\begin{problem}[2.2练习4]
    设$a,b,w > 0$,求螺线
    \begin{align*}
        \bd{r}:(t_0,t_1)&\ \to\R^3\\
        t&\ \mapsto(a\sin\omega t,a\cos\omega t, bt)
    \end{align*}
    的切向量,并给出一个弧长参数化.
\end{problem}
\begin{solution}
切向量为$\bd{r}' = (a\omega \cos\omega t, -a\omega\sin\omega t, b)$,则
\begin{equation*}
    s(t) = \int_{t_0}^t||r'(\tau)||_2\,\d \tau = \int_{t_0}^t\sqrt{a^2\omega^2+b^2}\,\d\tau = \sqrt{a\omega^2+b^2}(t-t_0)
\end{equation*}
于是$t = \frac{s}{\sqrt{a^2w^2+b^2}}+t_0 = \eta(s)$,故弧长参数化为
\begin{equation*}
    \tilde{\bd{r}} = \bd{r}'\circ\eta = (a\sin\omega\eta(s), a\omega\cos\omega\eta(s),b\eta(s))
\end{equation*}
\end{solution}
\begin{problem}[2.3练习1]
    计算半径为$r$的平面圆周曲率.
\end{problem}
\begin{solution}
    二维平面中圆形在原点,半径为$r$的圆周可以有以下参数化表示方法:
    \begin{align*}
        \bd{r}:[0,2\pi)&\ \to\R^2\\
        \theta&\ \mapsto (r\cos\theta,r\sin\theta)
    \end{align*}
    则\add $s(\theta) = \int_0^\theta||\bd{r}(\tau)||\,\d\tau = \int_0^\theta r\,\d\tau = r\theta$,则$\eta(s) = s/r = \theta$,
    对应的弧长参数化为$\tilde{\bd{r}} = \bd{r}\,\circ\,\bd{\eta} = (r\sin s/r,r\sin s/r)$,则曲率为
    \begin{equation*}
        \kappa(t) = ||\tilde{\bd{r}}''|| = ||(-r\cos t,-r\sin t)|| = \left|\left|-\frac{1}{r}(\cos s/r, \sin s/r)\right|\right| = \frac{1}{r}
    \end{equation*}
\end{solution}
\begin{problem}[2.3练习2]
    计算螺线$\bd{r}(t) = (a\cos \omega t, a\sin \omega t, bt)$的曲率和挠率($a,\omega,b>0$).
\end{problem}
\begin{solution}
    $\bd{r}'(t) = (-a\omega\sin\omega t, a\omega\cos\omega t, b), \bd{r}''(t) = -a\omega^2(\cos\omega t,\sin\omega t, 0),\bd{r}'''(t) = a\omega^3(\sin\omega t,-\cos\omega t, 0)$,
    则$|\bd{r}'\times \bd{r}''| = a\omega\sqrt{a^2\omega^2+b^2},\ |\bd{r}'| = \sqrt{a^2w^2+b^2},\ (r',r'',r''') = -a^4b\omega^{10}$,于是
    \begin{equation*}
        \kappa(t) = \frac{|\bd{r}'\times \bd{r}''|}{|\bd{r}'|^2} = \frac{a\omega}{\sqrt{a^2\omega^2+b^2}},\quad \tau(t) = \frac{(r',r'',r''')}{|r'\times r''|^2} = -\frac{a^2b\omega^8}{a^2\omega^2+b^2}
    \end{equation*}
\end{solution}
\begin{problem}[2.3练习3]
    证明:如果一条平面曲线的挠率恒为零,且曲率为常数($\neq 0$),则该曲线是一段圆弧.
\end{problem}
\begin{proof}
    由于$\dot{\alpha}(s) = \kappa(s)\beta(s),\ \dot{\beta} = -\kappa(s)\alpha(s) + \tau(s)\gamma(s)$,由于$\tau(s) = 0,\ \kappa(s) = c$,其中$c > 0$为常数(曲率非负),则
    \begin{equation*}
        \dot{\alpha}(s) = c\cdot\beta(s),\ \dot{\beta}(s) = -c\cdot\alpha(s)\quad \Rightarrow\quad \ddot{\alpha}(s) = -c^2\cdot\alpha(s),
    \end{equation*}
    上述线性常微分方程对应的特征方程为$a^2 = -c^2\Rightarrow a = \pm\i c$,于是$\alpha(s) = \bd{a}\sin cs + \bd{b}\cos cs,\ (s \geq 0)$,其中$\bd{a},\bd{b}\in\R^3$为待定系数,
    由弧长参数化性质可知
    \begin{equation*}
        |\alpha(s)| = |\bd{a}|^2\sin^2 cs + |\bd{b}|^2\cos^2 cs + 2(\bd{a},\bd{b})\sin cs\cos cs = 1
    \end{equation*}
    \add 取$s = 0,\ \frac{\pi}{2c}$,可得$|\bd{a}|^2=1,\ |\bd{b}|^2 = 1$,代入上式可得$(\bd{a},\bd{b})\sin cs\cos cs = 0$,由$s$的任意性可知$(\bd{a},\bd{b}) = 0$. 
    设$A$为任一正交阵,$\bd{a}_0 = (1, 0, 0)^T, \bd{b}_0 = (0, 1, 0)^T$,取$\bd{a} = A\bd{a_0}, \bd{b} = -A\bd{b_0}$,于是
    \begin{equation*}
        \bd{r}(s) = A(\sin cs, \cos cs, 0) + \xi
    \end{equation*}
    其中$\xi\in\R^3$为常向量,故$\bd{r}$是一段圆弧.
\end{proof}
\begin{problem}[2.3练习6]
    求
    \begin{equation*}
        \begin{cases}
            x^2+y^2+z^2 = 1,\\
            x^2+y^2=x
        \end{cases}
    \end{equation*}
    在$(0,0,1)$处的曲率和挠率.
\end{problem}
\begin{solution}
    令$\bd{r}(x,y,z) = \bd{r}(\sin\varphi\cos\theta, \sin\varphi\sin\theta, \cos\varphi)$,代入$x^2+y^2=x$中可得$\sin^2\varphi = \sin\varphi\cos\theta\Rightarrow \sin\varphi = \cos\theta\ (\varphi\neq 0)$,则
    \begin{align*}
        \bd{r}(\varphi) =&\ (\sin^2\varphi, \sin\varphi\cos\varphi, \cos\varphi),\\
        \bd{r}'(\varphi) =&\ (-2\sin\varphi\cos\varphi, \cos^2\varphi-\sin^2\varphi, -\sin\varphi),\\
        \bd{r}''(\varphi) =&\ (4\sin^2\varphi-2, -4\sin\varphi\cos\varphi, -\cos\varphi),\\
        \bd{r}'''(\varphi) =&\ (8\cos\varphi, 4(\sin^2\varphi - \cos^2\varphi), \sin\varphi).
    \end{align*}
    取$\varphi\to 0^+$有$\bd{r}' = (0,1,0),\ \bd{r}'' = (-2, 0, 1),\ \bd{r}''' = (8, -4, 0),\ |\bd{r}'\times \bd{r}''| = \sqrt{5},\ (\bd{r}',\bd{r}'',\bd{r}''') = -8$则
    \begin{equation*}
        \kappa(0,0,1) = \frac{|\bd{r}'\times \bd{r}''|}{|\bd{r}'|} = \sqrt{5},\quad \tau(0,0,1) = -\frac{8\sqrt{5}}{5}.
    \end{equation*}
\end{solution}
\begin{problem}[2.4练习3]
    证明:若曲线段曲率$\kappa$处处不为$0$,每个点的密切平面都过一个固定点,则这个曲线段在一个平面内.
\end{problem}
\begin{proof}
    令弧长参数化为
    \begin{equation*}
    \begin{aligned}
        \bd{r}:(a,b)&\ \to \R^3\\
        s&\ \mapsto \bd{r}(s)
    \end{aligned}
    \end{equation*}
    设密切平面均过点$\bd{x}_0$,则存在实函数$a(s),b(s)$使得$\bd{r}(s) - \bd{x}_0 = a(s)\bd{\alpha}(s)+b(s)\bd{\beta}(s)$,两边对$s$求导可得
    \begin{equation*}
        \bd{\alpha}(s) = a'(s)\bd{\alpha}(s) + a(s)\dot{\bd{\alpha}}(s) + b'(s)\bd{\beta}(s) + b(s)\dot{\bd{\beta}}(s)
    \end{equation*}
    由标架运动公式可知
    \begin{equation*}
        \frac{\d}{\d s}\begin{bmatrix}
            \bd{\alpha}(s)\\
            \bd{\beta}(s)\\
            \bd{\gamma}(s)
        \end{bmatrix} = \begin{bmatrix}
            0&\kappa(s)&0\\
            -\kappa(s)&0&0\\
            0&-\tau(s)&0
        \end{bmatrix}\begin{bmatrix}
            \bd{\alpha}(s)\\
            \bd{\beta}(s)\\
            \bd{\gamma}(s)
        \end{bmatrix}
    \end{equation*}
    于是
    \begin{equation*}
        \bd{\alpha}(s) = (a'(s) - b(s)\kappa(s)-1)\bd{\alpha}(s) + (b'(s)+a(s)\kappa(s))\beta(s) + b(s)\tau(s)\bd{\gamma}(s) = 0
    \end{equation*}
    由于$\alpha(s), \beta(s), \gamma(s)$两两正交,于是上式系数恒为$0$,若$b(s)\equiv 0$,则$a(s)\kappa(s) = 0$,由于$\kappa(s)$几乎处处不为零,
    则$a(s)$几乎处处为零,于是$\bd{r}(s)$几乎处处为$\bd{x}_0$,与$\kappa(s)$几乎处处不为零矛盾. 所以$\tau(s)$几乎处处为零,由\textbf{引理2.3}可知
    该曲线段是平面曲线段.
\end{proof}
\begin{problem}[2.4练习4]
    设$\{r(s),\alpha_1(s),\alpha_2(s),\alpha_3(s)\}$是定义在弧长参数曲线$r(s)$上的单位正交标架. 令
    \begin{equation*}
        \dot{\alpha_i}(s) = \sum_{j=1}^3\lambda_i^j(s)\alpha_j(s)
    \end{equation*}
    求证:$\lambda_i^j + \lambda_j^i = 0$.
\end{problem}
\begin{solution}
    取$s\in(a,b)$,由于$\{\bd{\alpha}_1(s),\bd{\alpha}_2(s),\bd{\alpha}_3(s)\}$是单位正交标架,设$\bd{r}(s)$的Frenet标架为$\{\bd{\alpha}(s),\bd{\beta}(s),\bd{\gamma}(s)\}$,
    于是存在正交阵$A$,使得$A(\bd{\alpha}_1(s),\bd{\alpha}_2(s),\bd{\alpha}_3(s))^T = (\bd{\alpha}(s),\bd{\beta}(s),\bd{\gamma}(s))^T$,且$A^T = A^{-1}$,
    由题目可知,只需证下述矩阵$C$为反对称矩阵
    \begin{equation}
        \frac{\d}{\d s}\begin{bmatrix}
            \bd{\alpha}_1(s)\\
            \bd{\alpha}_2(s)\\
            \bd{\alpha}_3(s)
        \end{bmatrix} = \begin{bmatrix}
            \lambda_1^1&\lambda_1^2&\lambda_1^3\\
            \lambda_2^1&\lambda_2^2&\lambda_2^3\\
            \lambda_3^1&\lambda_3^2&\lambda_3^3
        \end{bmatrix}\begin{bmatrix}
            \bd{\alpha}_1(s)\\
            \bd{\alpha}_2(s)\\
            \bd{\alpha}_3(s)
        \end{bmatrix}=: C\begin{bmatrix}
            \bd{\alpha}_1(s)\\
            \bd{\alpha}_2(s)\\
            \bd{\alpha}_3(s)
        \end{bmatrix}
    \end{equation}
    由标架运动公式可知
    \begin{equation*}
        \frac{\d}{\d s}\begin{bmatrix}
            \bd{\alpha}(s)\\
            \bd{\beta}(s)\\
            \bd{\gamma}(s)
        \end{bmatrix} = \begin{bmatrix}
            0&\kappa(s)&0\\
            -\kappa(s)&0&0\\
            0&-\tau(s)&0
        \end{bmatrix}\begin{bmatrix}
            \bd{\alpha}(s)\\
            \bd{\beta}(s)\\
            \bd{\gamma}(s)
        \end{bmatrix}=: B\begin{bmatrix}
            \bd{\alpha}(s)\\
            \bd{\beta}(s)\\
            \bd{\gamma}(s)
        \end{bmatrix}
    \end{equation*}
    于是对式(1)两侧同时左乘正交阵$A$可得
    \begin{equation*}
        \frac{\d}{\d s}\begin{bmatrix}
            \bd{\alpha}(s)\\
            \bd{\beta}(s)\\
            \bd{\gamma}(s)
        \end{bmatrix} = ACA^{-1}A\begin{bmatrix}
            \bd{\alpha}_1(s)\\
            \bd{\alpha}_2(s)\\
            \bd{\alpha}_3(s)
        \end{bmatrix} = ACA^{-1}\begin{bmatrix}
            \bd{\alpha}(s)\\
            \bd{\beta}(s)\\
            \bd{\gamma}(s)
        \end{bmatrix}
    \end{equation*}
    于是$ACA^{-1} = B\Rightarrow C = A^{-1}BA = A^TBA$,其中$B$为反对称矩阵,下面证明$C$是反对称矩阵:

    令$AB = D = [d_{ij}]$,则$d_{ij} = \sum_{k}a_{ik}b_{kj}$,于是
    \begin{align*}
        c_{ij} =&\ \sum_{l}d_{il}a_{jl} = \sum_{k,l}a_{ik}b_{kl}a_{jl} = \sum_{l > k}a_{ik}a_{jl}b_{kl} + \sum_{l < k}a_{ik}a_{jl}b_{kl}\\
        \xlongequal{b_{kl}=-b_{lk}}&\ \sum_{l > k}a_{ik}a_{jl}b_{kl} - \sum_{k > l}a_{ik}a_{jl}b_{lk}
        = \sum_{l > k}(a_{ik}a_{jl} - a_{il}a_{jk})b_{kl}
    \end{align*}
    则
    \begin{align*}
        c_{ii} =&\ \sum_{l > k}(a_{ik}a_{il}-a_{ik}a_{il})b_{kl} = 0,\quad(\forall i = 1,2,3)\\
        c_{ij} =&\ -\sum_{l > k}(a_{jk}a_{il} - a_{jl}a_{ik})b_{kl} = -c_{ji}\quad(\forall i,j = 1,2,3,\ i\neq j).
    \end{align*}
    故矩阵$C$为反对称矩阵.
\end{solution}
\end{document}