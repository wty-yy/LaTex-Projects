\documentclass[12pt, a4paper, oneside]{ctexart}
\usepackage{amsmath, amsthm, amssymb, bm, color, graphicx, geometry, mathrsfs,extarrows, braket, booktabs, array, wrapfig, enumitem}
\usepackage[colorlinks,linkcolor=red,anchorcolor=blue,citecolor=blue,urlcolor=blue,menucolor=black]{hyperref}
%%%% 设置中文字体 %%%%
% fc-list -f "%{family}\n" :lang=zh >d:zhfont.txt 命令查看已有字体
\setCJKmainfont[
    BoldFont=方正黑体_GBK,  % 黑体
    ItalicFont=方正楷体_GBK,  % 楷体
    BoldItalicFont=方正粗楷简体,  % 粗楷体
    Mapping = fullwidth-stop  % 将中文句号“.”全部转化为英文句号“.”,
]{方正书宋简体}  % !!! 注意在Windows中运行请改为“方正书宋简体.ttf” !!!
%%%% 设置英文字体 %%%%
\setmainfont{Minion Pro}
\setsansfont{Calibri}
\setmonofont{Consolas}

%%%% 设置行间距与页边距 %%%%
\linespread{1.4}
%\geometry{left=2.54cm,right=2.54cm,top=3.18cm,bottom=3.18cm}
\geometry{left=1.84cm,right=1.84cm,top=2.18cm,bottom=2.18cm}

%%%% 图片相对路径 %%%%
\graphicspath{{figures/}} % 当前目录下的figures文件夹, {../figures/}则是父目录的figures文件夹
\setlength{\abovecaptionskip}{-0.2cm}  % 缩紧图片标题与图片之间的距离
\setlength{\belowcaptionskip}{0pt} 

%%%% 缩小item,enumerate,description两行间间距 %%%%
\setenumerate[1]{itemsep=0pt,partopsep=0pt,parsep=\parskip,topsep=5pt}
\setitemize[1]{itemsep=0pt,partopsep=0pt,parsep=\parskip,topsep=5pt}
\setdescription{itemsep=0pt,partopsep=0pt,parsep=\parskip,topsep=5pt}

%%%% 自定义公式 %%%%
\everymath{\displaystyle} % 默认全部行间公式
\DeclareMathOperator*\uplim{\overline{lim}} % 定义上极限 \uplim_{}
\DeclareMathOperator*\lowlim{\underline{lim}} % 定义下极限 \lowlim_{}
\DeclareMathOperator*{\argmax}{arg\,max}  % 定义取最大值的参数 \argmax_{}
\DeclareMathOperator*{\argmin}{arg\,min}  % 定义取最小值的参数 \argmin_{}
\let\leq=\leqslant % 将全部leq变为leqslant
\let\geq=\geqslant % geq同理
\DeclareRobustCommand{\rchi}{{\mathpalette\irchi\relax}}
\newcommand{\irchi}[2]{\raisebox{\depth}{$#1\chi$}} % 使用\rchi将\chi居中

%%%% 自定义环境配置 %%%%
\newcounter{problem}  % 问题序号计数器
\newenvironment{problem}[1][]{\stepcounter{problem}\par\noindent\textbf{题目\arabic{problem}. #1}}{\smallskip\par}
\newenvironment{solution}[1][]{\par\noindent\textbf{#1解答. }}{\smallskip\par}  % 可带一个参数表示题号\begin{solution}{题号}
\newenvironment{note}{\par\noindent\textbf{注记. }}{\smallskip\par}
\newenvironment{remark}{\begin{enumerate}[label=\textbf{注\arabic*.}]}{\end{enumerate}}
\BeforeBeginEnvironment{minted}{\vspace{-0.5cm}}  % 缩小minted环境距上文间距
\AfterEndEnvironment{minted}{\vspace{-0.2cm}}  % 缩小minted环境距下文间距

%%%% 一些宏定义 %%%%
\def\bd{\boldsymbol}        % 加粗(向量) boldsymbol
\def\disp{\displaystyle}    % 使用行间公式 displaystyle(默认)
\def\weekto{\rightharpoonup}% 右半箭头
\def\tsty{\textstyle}       % 使用行内公式 textstyle
\def\sign{\text{sign}}      % sign function
\def\wtd{\widetilde}        % 宽波浪线 widetilde
\def\R{\mathbb{R}}          % Real number
\def\N{\mathbb{N}}          % Natural number
\def\Z{\mathbb{Z}}          % Integer number
\def\Q{\mathbb{Q}}          % Rational number
\def\C{\mathbb{C}}          % Complex number
\def\K{\mathbb{K}}          % Number Field
\def\P{\mathbb{P}}          % Polynomial
\def\d{\mathrm{d}}          % differential operator
\def\e{\mathrm{e}}          % Euler's number
\def\i{\mathrm{i}}          % imaginary number
\def\re{\mathrm{Re}}        % Real part
\def\im{\mathrm{Im}}        % Imaginary part
\def\res{\mathrm{Res}}      % Residue
\def\ker{\mathrm{Ker}}      % Kernel
\def\vspan{\mathrm{vspan}}  % Span  \span与latex内核代码冲突改为\vspan
\def\L{\mathcal{L}}         % Loss function
\def\O{\mathcal{O}}         % big O notation
\def\A{\mathcal{A}}         % 仿射坐标系
\def\wdh{\widehat}          % 宽帽子 widehat
\def\ol{\overline}          % 上横线 overline
\def\ul{\underline}         % 下横线 underline
\def\add{\vspace{1ex}}      % 增加行间距
\def\del{\vspace{-1.5ex}}   % 减少行间距

%%%% 定理类环境的定义 %%%%
\newtheorem{theorem}{定理}

%%%% 基本信息 %%%%
\newcommand{\RQ}{\today} % 日期
\newcommand{\km}{微分几何} % 科目
\newcommand{\bj}{强基数学002} % 班级
\newcommand{\xm}{吴天阳} % 姓名
\newcommand{\xh}{2204210460} % 学号

\begin{document}

%\pagestyle{empty}
\pagestyle{plain}
\vspace*{-15ex}
\centerline{\begin{tabular}{*5{c}}
    \parbox[t]{0.25\linewidth}{\begin{center}\textbf{日期}\\ \large \textcolor{blue}{\RQ}\end{center}} 
    & \parbox[t]{0.2\linewidth}{\begin{center}\textbf{科目}\\ \large \textcolor{blue}{\km}\end{center}}
    & \parbox[t]{0.2\linewidth}{\begin{center}\textbf{班级}\\ \large \textcolor{blue}{\bj}\end{center}}
    & \parbox[t]{0.1\linewidth}{\begin{center}\textbf{姓名}\\ \large \textcolor{blue}{\xm}\end{center}}
    & \parbox[t]{0.15\linewidth}{\begin{center}\textbf{学号}\\ \large \textcolor{blue}{\xh}\end{center}} \\ \hline
\end{tabular}}
\begin{center}
    \zihao{3}\textbf{第四次作业}
\end{center}\vspace{-0.2cm}
\begin{problem}[3.4练习1]证明\textbf{定义3.7}中Jacobi矩阵可逆这个条件不依赖于仿射坐标系的选取,也就是说如果$\varphi_U\circ\varphi_\A^{-1}$
    在$\varphi_\A(U)$上逐点可逆,那么在另一个仿射坐标系$\A'$中,$\varphi_U\circ\varphi_{\A'}^{-1}$在$\varphi_{\A'}(U)$上也逐点可逆.
\end{problem}
\begin{proof}
    设$(\varphi_U\circ\varphi_{\A}^{-1})' = J$可逆,由于$\A,\A'$均为仿射坐标系,则存在正交阵$T$和常向量$\bd{a}\in\R^n$使得$\varphi_{\A}\circ\varphi_{\A'}^{-1}(\bd{x}) = T\bd{x} + \bd{a}$,
    于是$\forall \bd{x}\in\varphi_{\A'}(U)$有
    \begin{align*}
        \varphi_{U}\circ\varphi_{\A'}^{-1}(\bd{x}) =&\ (\varphi_U\circ\varphi_{\A}^{-1})(\varphi_{\A}\circ\varphi_{\A'}^{-1})(\bd{x})\\
        \Rightarrow (\varphi_{U}\circ\varphi_{\A'}^{-1})'(\bd{x}) =&\ ((\varphi_{\A}\circ\varphi_{\A'}^{-1})')^T[(\varphi_U\circ\varphi_{\A}^{-1})'(\varphi_\A\circ\varphi_{\A'}^{-1})](\bd{x})\\
        \xlongequal{T^{-1} = T^T}&\ T^{-1}J(T\bd{x}+\bd{a})
    \end{align*}
    令$F(\bd{x}) = T^{-1}(J^{-1}T\bd{x}-\bd{a}) = T^{-1}J^{-1}T\bd{x} - T^{-1}\bd{a}$,于是$F[(\varphi_U\circ\varphi_{\A'}^{-1})'(\bd{x})] = I$,
    则$\varphi_U\circ\varphi_{\A'}^{-1}$的Jacobi矩阵在$\bd{x}$处可逆,逆变换为$F$,由$\bd{x}$的任意性可知$\varphi_U\circ\varphi_{\A'}^{-1}$在$\varphi_{\A'}(U)$上逐点可逆.
\end{proof}
\begin{problem}[3.4练习4.]证明\textbf{命题3.2}:设$U$为$\mathscr{A}^n$中的开区域,带有广义坐标系$\{U,\varphi_U\}$,则:

    (1) $U$中的开子集在$\varphi_U$下的像是$\R^n$中的开子集. 反之,$\R^n$中的开子集在$\varphi_U$下的原像是$U$中的开子集.

    (2) 设$f$为$U$上定义的标量场,则$f$连续等价于$f\circ\varphi_U^{-1}$是$\varphi_U(U)$上的连续函数.
\end{problem}
\begin{proof}
    (1) 由于\del\del
    \begin{align*}
        \varphi_U\circ\varphi_{\A}^{-1}:\R^n&\ \to \R^n\\
        \bd{x} = (x^1,\cdots,x^n)&\ \mapsto (y^1(\bd{x}),\cdots,y^n(\bd{x}))
    \end{align*}
    于是$\varphi_U\circ\varphi_{\A}^{-1}$对应的Jacobi矩阵为$J = (\varphi_U\circ\varphi_{\A}^{-1})'= [\partial_jy^i(x^1,\cdots,x^n)]_{ij}$,
    下证多元函数可微可推出连续:$\forall \varphi_\A(\bd{x})\in U,\ \bd{h}\in\R^n$,由$J$的连续性和多元函数微分的定义可知:
    \begin{equation*}
        |\varphi_U\circ\varphi_{\A}^{-1}(\bd{x}+\bd{h}) - \varphi_U\circ\varphi_{\A}^{-1}(\bd{x})| = 
        \left||\bd{h}|\frac{\varphi_U\circ\varphi_{\A}^{-1}(\bd{x}+\bd{h}) - \varphi_U\circ\varphi_{\A}^{-1}(\bd{x})-J\bd{h}}{|\bd{h}|}+J\bd{h}\right|\to 0,\ (\bd{h}\to 0)
    \end{equation*}
    由$\bd{x}$的任意性可知$\varphi_U\circ\varphi_{\A}^{-1}$在$U$上连续.

    由于$\varphi_U\circ\varphi_{\A}^{-1}$的Jacobi矩阵可逆,等价于,逆映射$\varphi_{\A}\circ\varphi_U^{-1}$的Jacobi矩阵可逆,
    所以$\varphi_{\A}\circ\varphi_U^{-1}$连续可微. 设$V$为$U$中的开集,由于$\varphi_U(V) = (\varphi_{\A}\circ\varphi_U^{-1})^{-1}(\varphi_\A(V))$,
    且$\varphi_\A$是同胚映射,则$\varphi_\A(V)$是开集,$(\varphi_{\A}\circ\varphi_U^{-1})^{-1}$将开集映射为开集,所以$\varphi(V)$是开集,于是$\varphi_U^{-1}$连续.

    由于$\varphi_U^{-1}$连续,又由广义坐标系性质可知$\varphi_U$可逆,于是$\varphi_U$连续,所以$\varphi_U^{-1}$将开集映射为开集.
    综上,$\varphi_U$是同胚映射.

    (2) 设$W$为$\varphi_U(U)$上的开子集.
    
    “$\Rightarrow$”由于$(f\circ \varphi_U^{-1})^{-1}(W) = \varphi_U(f^{-1}(W))$,
    由于$f$连续,则$f^{-1}(W)$为开集,又由于$\varphi_U$为同胚映射,所以$\varphi_U(f^{-1}(W))$为开集,所以$f\circ\varphi_U^{-1}$连续.

    “$\Leftarrow$”由于$\varphi_U^{-1}((f\circ \varphi_U^{-1})^{-1}W) = f^{-1}(W)$,由于$\varphi_U$为同胚映射,$(f\circ\varphi_U^{-1})^{-1}(W)$为开集,
    所以$f^{-1}(W)$为开集,故$f$连续.

    下证明,上述命题中与$\varphi_U$的选取无关,任取广义坐标系$\{U,\varphi_U'\}$,假设$f\circ \varphi_U$连续,于是$(f\circ\varphi_U^{-1})^{-1}(W)$为开集,由于
    \begin{equation*}
        (f\circ\varphi_{U'}^{-1})^{-1}(W) = (f\circ\varphi_U^{-1}\circ\varphi_U\circ\varphi_{U'}^{-1})^{-1}(W) = (\varphi_U\circ\varphi_{U'}^{-1})^{-1}(f\circ\varphi_U^{-1})^{-1}(W)
    \end{equation*}
    由于$(\varphi_U\circ\varphi_{U'}^{-1})$是同胚映射,所以$(f\circ\varphi_{U'}^{-1})^{-1}(W)$为开集,故$f\circ\varphi_{U'}^{-1}$连续.
\end{proof}
\begin{problem}[3.5练习1.]
    仍然考虑$\R^3$上的柱面坐标系,
    \begin{equation*}
        x^1 = r\cos\theta,\ x^2 = r\sin\theta,\ x^3 = z
    \end{equation*}
    其中,$r > 0,\theta\in(0,2\pi),z\in\R$,计算柱面坐标系的自然标架场(在$R^3$的自然坐标系$\mathcal{A} = \{O = (0,0,0),\bd{e}_1 = (1,0,0),
    \bd{e}_2 = (0,1,0),\bd{e}_3 = (0,0,1)\}$上表出)
\end{problem}
\begin{solution}
    由于
    \begin{equation*}
        T = \left[\frac{\partial x^i}{\partial r}\quad\frac{\partial x^i}{\partial \theta}\quad\frac{\partial x^i}{\partial z}\right]_{i=1}^3 = 
        \begin{bmatrix}
            \cos\theta&-r\sin\theta&0\\
            \sin\theta&r\cos\theta&0\\
            0&0&1
        \end{bmatrix}
    \end{equation*}
    又由自然标架$(\bd{\sigma}_1,\bd{\sigma}_2,\bd{\sigma}_3)$的定义可得
    \begin{equation*}
        \begin{bmatrix}
            \bd{\sigma}_1\\
            \bd{\sigma}_2\\
            \bd{\sigma}_3
        \end{bmatrix} = T\begin{bmatrix}
            \bd{e}_1\\
            \bd{e}_2\\
            \bd{e}_3
        \end{bmatrix} = \begin{bmatrix}
            \cos\theta\bd{e}_1-r\sin\theta\bd{e}_2\\
            \sin\theta\bd{e}_1+r\cos\theta\bd{e}_2\\
            \bd{e}_3
        \end{bmatrix}\Rightarrow\begin{cases}
            \bd{\sigma}_1 = (\cos\theta,-r\sin\theta,0),\\
            \bd{\sigma}_2 = (\sin\theta,r\cos\theta,0),\\
            \bd{\sigma}_3 = (0,0,1)
        \end{cases}
    \end{equation*}
\end{solution}
\begin{problem}
    考虑$\R^3$上的球坐标系:
    \begin{equation*}
        x^1 = r\cos\theta\sin\phi,\ x^2 = r\sin\theta\sin\phi,\ x^3 = r\cos\theta
    \end{equation*}
    计算该坐标系的自然标架(在$\R^3$的自然坐标系$\mathcal{A} = \{O = (0,0,0),\bd{e}_1 = (1,0,0),
    \bd{e}_2 = (0,1,0),\bd{e}_3 = (0,0,1)\}$上表出)
\end{problem}
\begin{solution}
    由于
    \begin{equation*}
        T = \left[\frac{\partial x^i}{\partial r}\quad\frac{\partial x^i}{\partial \theta}\quad\frac{\partial x^i}{\partial z}\right]_{i=1}^3 = 
        \begin{bmatrix}
            -r\sin\theta\sin\phi&r\cos\theta\cos\phi&\cos\theta\sin\phi\\
            r\cos\theta\sin\phi&r\sin\theta\cos\phi&\sin\theta\sin\phi\\
            -r\sin\theta&0&\cos\theta
        \end{bmatrix}
    \end{equation*}
    又由自然标架$(\bd{\sigma}_1,\bd{\sigma}_2,\bd{\sigma}_3)$的定义可得
    \begin{equation*}
        \hspace{-1.8cm}
        \begin{bmatrix}
            \bd{\sigma}_1\\
            \bd{\sigma}_2\\
            \bd{\sigma}_3
        \end{bmatrix} = T\begin{bmatrix}
            \bd{e}_1\\
            \bd{e}_2\\
            \bd{e}_3
        \end{bmatrix} = 
        \begin{bmatrix}
            -r\sin\theta\sin\phi\bd{e}_1+r\cos\theta\cos\phi\bd{e}_2+\cos\theta\sin\phi\bd{e}_3\\
            r\cos\theta\sin\phi\bd{e}_1+r\sin\theta\cos\phi\bd{e}_2+\sin\theta\sin\phi\bd{e}_3\\
            -r\sin\theta\bd{e}_1+\cos\theta\bd{e}_3
        \end{bmatrix}
        \Rightarrow\begin{cases}
            \bd{\sigma}_1 = (-r\sin\theta\sin\phi,r\cos\theta\cos\phi,\cos\theta\sin\phi),\\
            \bd{\sigma}_2 = (r\cos\theta\sin\phi,r\sin\theta\cos\phi,\sin\theta\sin\phi),\\
            \bd{\sigma}_3 = (-r\sin\theta,0,\cos\theta)
        \end{cases}
    \end{equation*}
\end{solution}
\begin{problem}[3.6练习1.]
    证明引理3.7
\end{problem}
\begin{solution} 设$V$是$A\in\mathscr{A}$的一个邻域,$\A = \{O,\bd{e}_i\}$是一个仿射坐标系,$A = v^i\bd{e}_i$,于是\\
    (1). 局部性,若在邻域$V$上有$f_{\A} = g_{\A}$,于是$\nabla f_{\A} = \nabla g_{\A}$,所以
    \begin{equation*}
        \partial_{\bd{v}}f(A) = \frac{\partial f_{\A}}{\partial x^i}\bigg|_{\varphi_\A}v^i = \nabla f_{\A}\cdot \bd{v} =  \nabla g_{\A}\cdot \bd{v} = \frac{\partial g_{\A}}{\partial x^i}\bigg|_{\varphi_\A}v^i = \partial_{\bd{v}}g(A)
    \end{equation*}
    (2). 线性性,由$f_{\A}$和$g_{\A}$的线性性可知
    \begin{align*}
        \partial_{\bd{v}}(\alpha f+\beta g)(A) =&\ \frac{\partial(\alpha f + \beta g)_\A}{\partial x^i}\bigg|_{\varphi_\A}v^i = \frac{\alpha\partial f_\A + \beta \partial g_\A}{\partial x^i}\bigg|_{\varphi_\A}v^i\\
        =&\ \alpha\frac{f_\A}{\partial x^i}\bigg|_{\varphi_\A}v^i + \beta\frac{g_\A}{\partial x^i}\bigg|_{\varphi_\A}v^i = \alpha\partial_{\bd{v}}f(A)+\beta\partial_{\bd{v}}g(A)
    \end{align*}
    (3). Leibniz公式,由$\partial(fg)_{\A} = f_{\A}\partial g_{\A} + g_{\A}\partial f_{\A}$可知
    \begin{equation*}
        \partial_{\bd{v}}(fg)(A) = \frac{\partial_{\bd{v}}(fg)_\A}{\partial x^i}\bigg|_{\varphi_{\A}}v^i = \frac{f_{\A}\partial g_{\A} + g_{\A}\partial f_{\A}}{\partial x^i}\bigg|_{\varphi_\A}v^i = f(A)\partial_{\bd{v}}g(A)+g(A)\partial_{\bd{v}}f(A)
    \end{equation*}
\end{solution}
\begin{problem}[3.7练习2.]
    设$D$是$A\in\mathscr{A}^n$上的导算子,$f$为定义在$A$的一个邻域上的可微函数,并且在$A$的某个邻域$U$上为常数。求证$Df = 0$。
\end{problem}
\begin{solution}
    由导算子的Leibniz公式$D(fg) = f(A)Dg+g(A)Df$可知,只需令$f = g = 1$,于是$D1 = D1 + D1\Rightarrow D1 = 0$,又由于导算子具有线性性,所以$DC = 0$,
    由于$f$在$U$的邻域上是常数,不妨令$f = C$,于是$Df = DC = 0$.
\end{solution}
\end{document}