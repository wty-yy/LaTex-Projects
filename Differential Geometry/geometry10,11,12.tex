\documentclass[12pt, a4paper, oneside]{ctexart}
\usepackage{amsmath, amsthm, amssymb, bm, color, graphicx, geometry, mathrsfs,extarrows, braket, booktabs, array, wrapfig, enumitem}
\usepackage[colorlinks,linkcolor=red,anchorcolor=blue,citecolor=blue,urlcolor=blue,menucolor=black]{hyperref}
%%%% 设置中文字体 %%%%
% fc-list -f "%{family}\n" :lang=zh >d:zhfont.txt 命令查看已有字体
\setCJKmainfont[
    BoldFont=方正黑体_GBK,  % 黑体
    ItalicFont=方正楷体_GBK,  % 楷体
    BoldItalicFont=方正粗楷简体,  % 粗楷体
    Mapping = fullwidth-stop  % 将中文句号“.”全部转化为英文句号“.”,
]{方正书宋简体}  % !!! 注意在Windows中运行请改为“方正书宋简体.ttf” !!!
%%%% 设置英文字体 %%%%
\setmainfont{Minion Pro}
\setsansfont{Calibri}
\setmonofont{Consolas}

%%%% 设置行间距与页边距 %%%%
\linespread{1.4}
%\geometry{left=2.54cm,right=2.54cm,top=3.18cm,bottom=3.18cm}
\geometry{left=1.84cm,right=1.84cm,top=2.18cm,bottom=2.18cm}

%%%% 图片相对路径 %%%%
\graphicspath{{figures/}} % 当前目录下的figures文件夹, {../figures/}则是父目录的figures文件夹
\setlength{\abovecaptionskip}{-0.2cm}  % 缩紧图片标题与图片之间的距离
\setlength{\belowcaptionskip}{0pt} 

%%%% 缩小item,enumerate,description两行间间距 %%%%
\setenumerate[1]{itemsep=0pt,partopsep=0pt,parsep=\parskip,topsep=5pt}
\setitemize[1]{itemsep=0pt,partopsep=0pt,parsep=\parskip,topsep=5pt}
\setdescription{itemsep=0pt,partopsep=0pt,parsep=\parskip,topsep=5pt}

%%%% 自定义公式 %%%%
\everymath{\displaystyle} % 默认全部行间公式
\DeclareMathOperator*\uplim{\overline{lim}} % 定义上极限 \uplim_{}
\DeclareMathOperator*\lowlim{\underline{lim}} % 定义下极限 \lowlim_{}
\DeclareMathOperator*{\argmax}{arg\,max}  % 定义取最大值的参数 \argmax_{}
\DeclareMathOperator*{\argmin}{arg\,min}  % 定义取最小值的参数 \argmin_{}
\let\leq=\leqslant % 将全部leq变为leqslant
\let\geq=\geqslant % geq同理
\DeclareRobustCommand{\rchi}{{\mathpalette\irchi\relax}}
\newcommand{\irchi}[2]{\raisebox{\depth}{$#1\chi$}} % 使用\rchi将\chi居中

%%%% 自定义环境配置 %%%%
\newcounter{problem}  % 问题序号计数器
\newenvironment{problem}[1][]{\stepcounter{problem}\par\noindent\textbf{题目\arabic{problem}. #1}}{\smallskip\par}
\newenvironment{solution}[1][]{\par\noindent\textbf{#1解答. }}{\smallskip\par}  % 可带一个参数表示题号\begin{solution}{题号}
\newenvironment{note}{\par\noindent\textbf{注记. }}{\smallskip\par}
\newenvironment{remark}{\begin{enumerate}[label=\textbf{注\arabic*.}]}{\end{enumerate}}
\BeforeBeginEnvironment{minted}{\vspace{-0.5cm}}  % 缩小minted环境距上文间距
\AfterEndEnvironment{minted}{\vspace{-0.2cm}}  % 缩小minted环境距下文间距

%%%% 一些宏定义 %%%%
\def\bd{\boldsymbol}        % 加粗(向量) boldsymbol
\def\disp{\displaystyle}    % 使用行间公式 displaystyle(默认)
\def\weekto{\rightharpoonup}% 右半箭头
\def\tsty{\textstyle}       % 使用行内公式 textstyle
\def\sign{\text{sign}}      % sign function
\def\wtd{\widetilde}        % 宽波浪线 widetilde
\def\R{\mathbb{R}}          % Real number
\def\N{\mathbb{N}}          % Natural number
\def\Z{\mathbb{Z}}          % Integer number
\def\Q{\mathbb{Q}}          % Rational number
\def\C{\mathbb{C}}          % Complex number
\def\K{\mathbb{K}}          % Number Field
\def\P{\mathbb{P}}          % Polynomial
\def\E{\mathbb{E}}          % Polynomial
\def\d{\mathrm{d}}          % differential operator
\def\e{\mathrm{e}}          % Euler's number
\def\i{\mathrm{i}}          % imaginary number
\def\re{\mathrm{Re}}        % Real part
\def\im{\mathrm{Im}}        % Imaginary part
\def\res{\mathrm{Res}}      % Residue
\def\ker{\mathrm{Ker}}      % Kernel
\def\vspan{\mathrm{vspan}}  % Span  \span与latex内核代码冲突改为\vspan
\def\L{\mathcal{L}}         % Loss function
\def\O{\mathcal{O}}         % big O notation
\def\A{\mathcal{A}}         % 仿射坐标系
\def\sA{\mathscr{A}}        % 点空间
\def\F{\mathcal{F}}         % 光滑函数空间
\def\sF{\mathscr{F}}        % 光滑函数商空间
\def\wdh{\widehat}          % 宽帽子 widehat
\def\ol{\overline}          % 上横线 overline
\def\ul{\underline}         % 下横线 underline
\def\add{\vspace{1ex}}      % 增加行间距
\def\del{\vspace{-1.5ex}}   % 减少行间距

%%%% 定理类环境的定义 %%%%
\newtheorem{theorem}{定理}

%%%% 基本信息 %%%%
\newcommand{\RQ}{\today} % 日期
\newcommand{\km}{微分几何} % 科目
\newcommand{\bj}{强基数学002} % 班级
\newcommand{\xm}{吴天阳} % 姓名
\newcommand{\xh}{2204210460} % 学号

\begin{document}

%\pagestyle{empty}
\pagestyle{plain}
\vspace*{-15ex}
\centerline{\begin{tabular}{*5{c}}
    \parbox[t]{0.25\linewidth}{\begin{center}\textbf{日期}\\ \large \textcolor{blue}{\RQ}\end{center}} 
    & \parbox[t]{0.2\linewidth}{\begin{center}\textbf{科目}\\ \large \textcolor{blue}{\km}\end{center}}
    & \parbox[t]{0.2\linewidth}{\begin{center}\textbf{班级}\\ \large \textcolor{blue}{\bj}\end{center}}
    & \parbox[t]{0.1\linewidth}{\begin{center}\textbf{姓名}\\ \large \textcolor{blue}{\xm}\end{center}}
    & \parbox[t]{0.15\linewidth}{\begin{center}\textbf{学号}\\ \large \textcolor{blue}{\xh}\end{center}} \\ \hline
\end{tabular}}
\begin{center}
    \zihao{3}\textbf{第七次作业}
\end{center}\vspace{-0.2cm}
\begin{problem}[5.3 练习1.]
    设$a_ix^i = 0$为$\E^3$上标准坐标系$\{O,e_i\}$的一个平面。当$a_1\neq 0$时,平面有参数化
    \begin{equation*}
        x^1 = -\frac{1}{a_1}(a_2x^2+a_3x^3),\ x^2=x^2,\ x^3=x^3
    \end{equation*}
    其中$(x^2,x^3)$为参数。求证在这个在参数标架下$\Gamma_{ij}^k = 0$。
\end{problem}
\begin{proof}
    由于$\partial_2 = (-a_2/a_1,1,0),\ \partial_3 = (-a_3/a_1,0,1)$,于是
    \begin{equation*}
        \bd{g} = \begin{bmatrix}
            (\partial_2,\partial_2)&(\partial_2,\partial_3)\\
            (\partial_3,\partial_2)&(\partial_3,\partial_3)\\
        \end{bmatrix} = \frac{1}{a_1^2}\begin{bmatrix}
            a_2^2&a_2a_3\\
            a_2a_3&a_3^2
        \end{bmatrix}
    \end{equation*}
    于是$\partial_k\bd{g}_{ij} = 0$,由$\Gamma_{ij}^k$的计算式可知$\Gamma_{ij}^k = 0$。
\end{proof}
\begin{problem}[5.11 练习1.]
    求证,直线段是所在平面的测地线,反之平面的测地线一定是直线段。
\end{problem}
\begin{proof}
    设三维空间中的平面为$a_ix^i = 0$,于是可以在$x^2,x^3$参数标架下表出$x^1 = -\frac{1}{a_1}(a_2x^2+a_3x^3)$,
    由上题可知,$\Gamma_{ij}^k = 0$。假设任意一条从原点$(0,0)$开始,初始速度为$(\partial_2,\partial_3)$的测地线
    $y(x_2(t),x_3(t))$,即初值条件为$y(0) = (0,0),\dot{y}(0) = (1,1)$,于是求解测地线方程可得
    \begin{equation*}
        \begin{cases}
            \ddot{y}^1 = 0,\\
            \ddot{y}^2 = 0.
        \end{cases}\Rightarrow\begin{cases}
            y^1(t) = t,\\
            y^2(t) = t.
        \end{cases}
    \end{equation*}
    则$y(x^2(t),x^3(t)) = (t,t)\Rightarrow y(t) = (-\frac{a_2+a_3}{a_1}t,t,t)$\add ,就是三维空间中的直线,
    通过平移可将原点平移到三维空间中任意一点,给定测地线终点,上式说明三维平面的测地线一定是一条线段。

    由弧长变分可知,三维平面上的测地线一定是线段。
\end{proof}
\begin{problem}[5.11 练习2.]
    求证,大圆是球面的测地线,反之,球面上的测地线一定是大圆。
\end{problem}
\begin{proof}
    由5.1的练习2可知,考虑三维空间中的单位球在$(1,0,0)$附近的参数化:
    \begin{equation*}
        x^1 = \cos\theta\sin(\pi/2+\varphi),\ x^2 = \sin\theta\sin(\pi/2+\varphi),\ x^3 = \cos(\pi/2+\varphi)
    \end{equation*}
    设测地线为$y(\theta(t),\varphi(t))$,在参数$(\theta,\varphi)$标架下可得测地线方程
    \begin{equation*}
        \begin{cases}
            \tan(\pi/2+\varphi(t))\ddot{y}^1 + 2\dot{y}^1\dot{y}^2 = 0,\\
            \ddot{y}^2-\frac{1}{2}\sin(\pi+2\varphi)\dot{y}^1\dot{y}^1 = 0.
        \end{cases}
    \end{equation*}
    给定初始条件$y(0) = (0,0),\ \dot{y}(0) = (1,0)$,可解得$y=(t,0) = (\cos t,\sin t, 0)$. 这表明在单位球上从$(1,0,0)$开始,
    初速度为$(\partial_\theta,0)$的质点,运动轨迹就是$xOy$平面上的单位圆。可以通过平移旋转使得初始状态达到单位单位球上任意一点,
    通过等比例缩放从单位球得到任意球,从而可以说明大圆就是球的测地线。

    由弧长变分可知,球上的测地线一定是大圆。
\end{proof}
\begin{problem}[6.2 练习1.]
    通过计算说明圆柱面的第二基本形式不为零,这就是它不是外蕴意义下的平面的原因。
\end{problem}
\begin{solution}
    设三维空间中的圆柱面方程为$x = \cos\theta, y = \sin\theta, z = z$,法向量为$\bd{n} = \frac{\partial_\theta\times \partial_z}{||\partial_\theta\times\partial_z||} = (\cos\theta,\sin\theta,0)$
    于是$(1,0,0)$在$(\theta,z)$的局部参数化下的第二基本形式为
    \begin{equation*}
        \Pi_{\theta\theta} = \left(\frac{\partial^2\cos\theta}{\partial\theta^2},\frac{\partial^2\sin\theta}{\partial\theta^2},\frac{\partial^2z}{\partial\theta^2}\right)\cdot\bd{n} = (-\cos\theta,-\sin\theta,0)\cdot(\cos\theta,\sin\theta,0) = -1\neq 0
    \end{equation*}
    所以圆柱面不是外蕴意义下的平面。
\end{solution}
\begin{problem}
    证明:曲面的第二基本形式与具体的标准正交坐标系无关。
\end{problem}
\begin{proof}
    设曲面上切向量场$X,Y$在标准坐标系$\{O,e_i\}$,令$Y=Y^ie_i$,假设$\{O',e_i'\}$为另一个标准正交坐标系,则存在可逆矩阵$T$且$|T| = 1$,使得
    $Z^ie'_i = T(Y^ie_i)$,于是
    \begin{equation*}
        \Pi(X,Z) = (\partial_XZ^ie'_i)_{\bot} = [\partial_X T(Y^ie_i)]_\bot = |T|\cdot (\partial_XY^ie_i)_\bot = \Pi(X,Y)
    \end{equation*}
    说明曲面的第二基本形式与具体的标准正交坐标系选取无关。
\end{proof}
\end{document}