% -*-coding: utf-8 -*-

\BiAppChapter{算法与代码}{Algorithms and Codes}

对于数学、计算机和电子信息专业,算法和代码也是经常用到的排版技巧。

%=========================================================================================
\BiSection{算法}{Algorithms}

算法描述使用 algorithm2e 宏包,效果如算法~\ref{alg_appB_lms}~所示。

\begin{algorithm}[ht]
	\caption{LMS~算法详细描述 \label{alg_appB_lms}}
	\IncMargin{2em}
	\DontPrintSemicolon
	\KwIn{$\mbf{x}(k), \quad \mu, \quad \mbf{w}(0)$}
	\KwOut{$y(k), \quad \varepsilon(k)$}
	%
	\For{$k=0,1,\cdots$}{
		$y(k) = \mbf{w}^H(k)\mbf{x}(k)$ \tcp*{output signal}
		$\varepsilon(k) = d(k)-y(k)$ \tcp*{error signal}
		$\mbf{w}(k+1) = \mbf{w}(k)+\mu\varepsilon^{\ast}(k)\mbf{x}(k)$ \tcp*{weight vector update}
	}
\end{algorithm}


%=========================================================================================
\BiSection{代码}{Codes} 

源代码使用 listings 宏包,LMS~算法的~Verilog~模块端口声明如代码~\ref{code_appB_lms}~所示。

{\fontsize{10pt}{0.5\baselineskip}\selectfont
	\begin{lstlisting}[caption={空时~LMS~算法~Verilog~模块端口声明},label={code_appB_lms}]
	module stap_lms
	#(
	parameter      M                = 4,    // number of antennas
	               L                = 5,    // length of FIR filter
	               W_IN             = 18,   // wordlength of input data
	               W_OUT            = 18,   // wordlength of output data
	               W_COEF           = 20    // wordlength of weights
	)(
	output  signed [W_OUT-1:0]      y_i,    // in-phase component of STAP output
	output  signed [W_OUT-1:0]      y_q,    // quadrature component of STAP output
	output                          vout,   // data valid flag of output (high)
	input          [M*W_IN-1:0]     u_i,    // in-phase component of M antennas
	input          [M*W_IN-1:0]     u_q,    // quadrature component of M antennas
	input                           vin,    // data valid flag for input (high)
	input                           clk,    // clock signal
	input                           rst     // reset signal (high)
	);
	\end{lstlisting}
}

% 在最后一个附录 tex 文件末尾添加以下命令,使研究成果从奇数页开始
\clearpage{\pagestyle{empty}\cleardoublepage}
