% !Mode:: "TeX:UTF-8" 

\BiAppendixChapter{致\quad 谢}{Acknowledgements}

感谢西安交通大学数学学院能够给我这次自拟毕设题目的机会,如果错过这次机会,之后可能很难再有充足的时间与精力完成本毕设内容;
感谢数学学院强基计划,通过本科课程学习,使我有了扎实的数学基础,能够比较轻松地看懂计算机视觉、强化学习等领域中的论文,
理解并对其中的公式进行推导加深理解,最后使用代码进行复现。

感谢兰旭光老师对本毕设的支持,即使本毕设内容充满各种未知挑战,但老师仍给我提供了必要的算力支持,如果没有充足算力,本文进展将相当缓慢,完全无法完成任务。
感谢课题组中各位师兄师姐为本次毕设提供思路,其中使用SAM辅助切片制作和多目标识别模型进行目标检测思路来自王宇航博士,
决策模型中将离散预测转化为连续预测思路来自万里鹏博士,如果没有这些改进,精准的识别模型可能无法实现,决策模型也没有任何性能,
战胜游戏中的内置AI将成为幻想。感谢我身边朋友的支持,他们的支持使我愈发坚定将不可能变为可能的决心。
感谢所有开源社区中贡献者们,他们无私奉献出各种领域的顶尖模型(YOLO系列,PaddleOCR,GPT,DT等)也让本次毕设成为可能,
从他们的源代码中我总能学到非常多新知识、新架构设计方案,从而能不断提升自身的代码水平,向他们看齐。

感谢我的父母对我一直以来的支持,让我能在高中参加算法竞赛,打下扎实的编程基础,
并支持我去追随自己儿时的梦想——“做出一个能够与现实进行直接交互并改善生活的智能体”。
如果没有这样坚定的梦想,我可能早在制作数据集的枯燥过程中放弃,但通过大学的各种知识与技术的学习,
让我看到了前进的道路,坚定下自己的信念,将全部代码从零实现了出来,从而初步完成了本毕设开题时所设定的目标,成功地走出了第一步。
