\documentclass[12pt, a4paper, oneside]{ctexart}
\usepackage{amsmath, amsthm, amssymb, bm, color, graphicx, geometry, mathrsfs,extarrows, braket, booktabs, array, wrapfig, enumitem}
\usepackage[colorlinks,linkcolor=red,anchorcolor=blue,citecolor=blue,urlcolor=blue,menucolor=black]{hyperref}
%%%% 设置中文字体 %%%%
% fc-list -f "%{family}\n" :lang=zh >d:zhfont.txt 命令查看已有字体
\setCJKmainfont[
    BoldFont=方正黑体_GBK,  % 黑体
    ItalicFont=方正楷体_GBK,  % 楷体
    BoldItalicFont=方正粗楷简体,  % 粗楷体
    Mapping = fullwidth-stop  % 将中文句号“.”全部转化为英文句号“.”
]{方正书宋简体}  % !!! 注意在Windows中运行请改为“方正书宋简体.ttf” !!!
%%%% 设置英文字体 %%%%
\setmainfont{Minion Pro}
\setsansfont{Calibri}
\setmonofont{Consolas}

%%%% 设置行间距与页边距 %%%%
\linespread{1.4}
%\geometry{left=2.54cm,right=2.54cm,top=3.18cm,bottom=3.18cm}
\geometry{left=1.84cm,right=1.84cm,top=2.18cm,bottom=2.18cm}

%%%% 图片相对路径 %%%%
\graphicspath{{figures/}} % 当前目录下的figures文件夹, {../figures/}则是父目录的figures文件夹
\setlength{\abovecaptionskip}{-0.2cm}  % 缩紧图片标题与图片之间的距离
\setlength{\belowcaptionskip}{0pt} 

%%%% 缩小item,enumerate,description两行间间距 %%%%
\setenumerate[1]{itemsep=0pt,partopsep=0pt,parsep=\parskip,topsep=5pt}
\setitemize[1]{itemsep=0pt,partopsep=0pt,parsep=\parskip,topsep=5pt}
\setdescription{itemsep=0pt,partopsep=0pt,parsep=\parskip,topsep=5pt}

%%%% 自定义公式 %%%%
\everymath{\displaystyle} % 默认全部行间公式
\DeclareMathOperator*\uplim{\overline{lim}} % 定义上极限 \uplim_{}
\DeclareMathOperator*\lowlim{\underline{lim}} % 定义下极限 \lowlim_{}
\DeclareMathOperator*{\argmax}{arg\,max}  % 定义取最大值的参数 \argmax_{}
\DeclareMathOperator*{\argmin}{arg\,min}  % 定义取最小值的参数 \argmin_{}
\let\leq=\leqslant % 将全部leq变为leqslant
\let\geq=\geqslant % geq同理
\DeclareRobustCommand{\rchi}{{\mathpalette\irchi\relax}}
\newcommand{\irchi}[2]{\raisebox{\depth}{$#1\chi$}} % 使用\rchi将\chi居中

%%%% 自定义环境配置 %%%%
\newcounter{problem}  % 问题序号计数器
\newenvironment{problem}[1][]{\stepcounter{problem}\par\noindent\textbf{题目\arabic{problem}. #1}}{\smallskip\par}
\newenvironment{solution}[1][]{\par\noindent\textbf{#1解答. }}{\smallskip\par}  % 可带一个参数表示题号\begin{solution}{题号}
\newenvironment{note}{\par\noindent\textbf{注记. }}{\smallskip\par}
\newenvironment{remark}{\begin{enumerate}[label=\textbf{注\arabic*.}]}{\end{enumerate}}

%%%% 一些宏定义 %%%%
\def\bd{\boldsymbol}        % 加粗(向量) boldsymbol
\def\disp{\displaystyle}    % 使用行间公式 displaystyle(默认)
\def\weekto{\rightharpoonup}% 右半箭头
\def\tsty{\textstyle}       % 使用行内公式 textstyle
\def\sign{\text{sign}}      % sign function
\def\wtd{\widetilde}        % 宽波浪线 widetilde
\def\R{\mathbb{R}}          % Real number
\def\N{\mathbb{N}}          % Natural number
\def\Z{\mathbb{Z}}          % Integer number
\def\Q{\mathbb{Q}}          % Rational number
\def\C{\mathbb{C}}          % Complex number
\def\K{\mathbb{K}}          % Number Field
\def\P{\mathbb{P}}          % Polynomial
\def\E{\mathbb{E}}          % Exception
\def\d{\mathrm{d}}          % differential operator
\def\e{\mathrm{e}}          % Euler's number
\def\i{\mathrm{i}}          % imaginary number
\def\re{\mathrm{Re}}        % Real part
\def\im{\mathrm{Im}}        % Imaginary part
\def\res{\mathrm{Res}}      % Residue
\def\ker{\mathrm{Ker}}      % Kernel
\def\vspan{\mathrm{vspan}}  % Span  \span与latex内核代码冲突改为\vspan
\def\L{\mathcal{L}}         % Loss function
\def\O{\mathcal{O}}         % big O notation
\def\wdh{\widehat}          % 宽帽子 widehat
\def\ol{\overline}          % 上横线 overline
\def\ul{\underline}         % 下横线 underline
\def\add{\vspace{1ex}}      % 增加行间距
\def\del{\vspace{-1.5ex}}   % 减少行间距
\def\prob{\textrm{P}}       % Probability
\def\cov{\textrm{Cov}}      % Covariance
\def\var{\textrm{Var}}      % Variance

%%%% 定理类环境的定义 %%%%
\newtheorem{theorem}{定理}

%%%% 基本信息 %%%%
\newcommand{\RQ}{\today} % 日期
\newcommand{\km}{随机过程} % 科目
\newcommand{\bj}{人工智能B2480} % 班级
\newcommand{\xm}{吴天阳} % 姓名
\newcommand{\xh}{4124136039} % 学号

\begin{document}

%\pagestyle{empty}
\pagestyle{plain}
\vspace*{-15ex}
\centerline{\begin{tabular}{*5{c}}
    \parbox[t]{0.25\linewidth}{\begin{center}\textbf{日期}\\ \large \textcolor{blue}{\RQ}\end{center}} 
    & \parbox[t]{0.2\linewidth}{\begin{center}\textbf{科目}\\ \large \textcolor{blue}{\km}\end{center}}
    & \parbox[t]{0.2\linewidth}{\begin{center}\textbf{班级}\\ \large \textcolor{blue}{\bj}\end{center}}
    & \parbox[t]{0.1\linewidth}{\begin{center}\textbf{姓名}\\ \large \textcolor{blue}{\xm}\end{center}}
    & \parbox[t]{0.15\linewidth}{\begin{center}\textbf{学号}\\ \large \textcolor{blue}{\xh}\end{center}} \\ \hline
\end{tabular}}
\begin{center}
    \zihao{3}\textbf{第三次作业\quad 连续时间的Markov链}
\end{center}\vspace{-0.2cm}
\begin{problem}
    \textbf{P332 1.}
    一个有机体的总体由雄性和雌性成员组成。在一个小的群体中,某个特定的雄性可能与一个特定的雌性以概率$\lambda h+o(h)$在任意长度为$h$的时间区间里交配。
    每次交配立即等可能产生一个雄性或雌性后代。以$N_1(t)$和$N_2(t)$分别记在时刻$t$总体中的雄性和雌性的个数。推导连续时间的Markov链$\{N_1(t),N_2(t)\}$的参数,
    即6.2节中的参数$v_i,P_{ij}$。
\end{problem}
\begin{solution}
    状态$i = (n_1,n_2)$,其中$n_1,n_2$分别为雄性和雌性的个数,且$n_1,\geqslant 0, n_2\geqslant 0$。

    离开状态$i$的速率为每次的交配速率和总交配对数之积,即
    \begin{equation*}
        v_i = v_{(n_1,n_2)} = \lambda n_1n_2
    \end{equation*}

    由于是等概率产生一个雄性或雌性后代,因此当$n_1>0, n_2>0$时
    \begin{equation*}
        P_{(n_1,n_2)\to(n_1+1,n_2)} = P_{(n_1,n_2)\to(n_1,n_2+1)} = \frac{1}{2},
    \end{equation*}

    当$n_1=0$或$n_2=0$时
    \begin{equation*}
        P_{(n_1,n_2)\to(n_1,n_2)} = 1.
    \end{equation*}
\end{solution}

\begin{problem}
    \textbf{P332 6.}
    考虑一个具有出生率$\lambda_i = (i+1)\lambda\ (i\geqslant 0)$与死亡率$\mu_i = i\mu\ (i\geqslant 0)$的生灭过程。
    
    (a) 确定从状态$0$到状态$4$的期望时间。

    (b) 确定从状态$2$到状态$5$的期望时间。

    (c) 确定(a)和(b)中的方差。
\end{problem}
\begin{solution}
    设$m_i$为从状态$i$到状态$i+1$的期望时间,则
    \begin{align*}
        &\ m_i = \frac{1}{\lambda_i}+\frac{\mu_i}{\lambda_i}m_{i-1} = \frac{1}{(i+1)\lambda} + \frac{i\mu}{(i+1)\lambda}m_{i-1} = \frac{1}{(i+1)\lambda}\sum_{k=0}^i r^k \\
        \Rightarrow &\ m_i = \frac{1}{(i+1)\lambda}\frac{1-r^{i+1}}{1-r}\quad(r\neq 1) \\
        \Rightarrow &\ m_i = \frac{1}{\lambda_i}\quad (r=1)
    \end{align*}
    其中$r = \frac{\mu}{\lambda}$

    (a) \vspace{-4ex}
    \begin{align*}
        \E[T_{0\to 4}] = \sum_{i=0}^3 m_i = \frac{1}{\lambda (1-r)}\left[\frac{1-r}{1}+\frac{1-r^2}{2}+\frac{1-r^3}{3}+\frac{1-r^4}{4}\right]
    \end{align*}

    (b) \vspace{-4ex}
    \begin{align*}
        \E[T_{2\to 5}] = \sum_{i=2}^4 m_i = \frac{1}{\lambda (1-r)}\left[\frac{1-r^3}{3}+\frac{1-r^4}{4}+\frac{1-r^5}{5}\right]
    \end{align*}

    (c) 设$T_i$为从状态$i$到状态$i+1$的时间,则
    \begin{equation*}
        \var(T_i) = \frac{1}{\lambda_i(\lambda_i+\mu_i)} + \frac{\mu_i}{\lambda_i}\var(T_{i-1}) + \frac{\mu_i}{\mu_i+\lambda_i}(\E[T_{i-1}]+\E[T_i])^2
    \end{equation*}
    由于当$r\neq 1$时情况过于复杂,下面仅考虑$r=1$的情况,则
    \begin{equation*}
        \var(T_i) = \frac{1}{\lambda_i^2}+\var(T_{i-1}) = \frac{i+1}{\lambda_i^2}
    \end{equation*}
    于是
    \begin{align*}
        \var(T_{0\to 4}) &= \sum_{i=0}^3 \var(T_i) = \frac{1}{\lambda^2}\left[1+2+3+4\right] = \frac{10}{\lambda^2} \\
        \var(T_{2\to 5}) &= \sum_{i=2}^4 \var(T_i) = \frac{1}{\lambda^2}\left[3+4+5\right] = \frac{12}{\lambda^2}
    \end{align*}
\end{solution}

\begin{problem}
    \textbf{P333 13.}
    一个理发师经营的小理发店最多能容纳两个顾客。潜在顾客以每小时$3$个的速度的Poisson过程到达,而相继的服务时间是均值为$1/4$小时的独立的指数随机变量。
    求解下面各项:

    (a) 在店中顾客的平均数。

    (b) 进入店中的潜在顾客比例。

    (c) 如果该理发师工作的速率快至两倍,他将多做多少生意?
\end{problem}
\begin{solution}
\end{solution}

\begin{problem}
    \textbf{P334 19.}
\end{problem}
\begin{solution}
\end{solution}

\begin{problem}
    \textbf{P336 35.}
\end{problem}
\begin{solution}
\end{solution}

\begin{problem}
    \textbf{P338 45.}
\end{problem}
\begin{solution}
\end{solution}
\end{document}
