\documentclass[12pt, a4paper, oneside]{ctexart}
\usepackage{amsmath, amsthm, amssymb, bm, color, graphicx, geometry, mathrsfs,extarrows, braket, booktabs, array, xcolor, fontspec, appendix, float, subfigure, wrapfig, enumitem, titlesec, algorithm, mathtools}
\usepackage[colorlinks,linkcolor=red,anchorcolor=blue,citecolor=blue,urlcolor=blue,menucolor=black]{hyperref}

%%%% 设置中文字体 %%%%
% fc-list -f "%{family}\n" :lang=zh >d:zhfont.txt 命令查看已有字体
\setCJKmainfont[
    BoldFont=方正黑体_GBK,  % 黑体
    ItalicFont=方正楷体_GBK,  % 楷体
    BoldItalicFont=方正粗楷简体,  % 粗楷体
    Mapping = fullwidth-stop  % 将中文句号“.”全部转化为英文句号“.”,
]{方正书宋简体}  % !!! 注意在Windows中运行请改为“方正书宋简体.ttf” !!!
%%%% 设置英文字体 %%%%
\setmainfont{Minion Pro}
\setsansfont{Calibri}
\setmonofont{Consolas}

%%%% 设置代码块 %%%%
% 在vscode中使用minted需要先配置python解释器, Ctrl+Shift+P, 输入Python: Select Interpreter选择安装了Pygments的Python版本. 再在setting.json中xelatex和pdflatex的参数中加入 "--shell-escape", 即可
% TeXworks中配置方法参考: https://blog.csdn.net/RobertChenGuangzhi/article/details/108140093
\usepackage{minted}
\renewcommand{\theFancyVerbLine}{
    \sffamily\textcolor[rgb]{0.5,0.5,0.5}{\scriptsize\arabic{FancyVerbLine}}} % 修改代码前序号大小
% 加入不同语言的代码块
\newmintinline{cpp}{fontsize=\small, linenos, breaklines, frame=lines}
\newminted{cpp}{fontsize=\small, baselinestretch=1, linenos, breaklines, frame=lines}
\newmintedfile{cpp}{fontsize=\small, baselinestretch=1, linenos, breaklines, frame=lines}
\newmintinline{matlab}{fontsize=\small, linenos, breaklines, frame=lines}
\newminted{matlab}{fontsize=\small, baselinestretch=1, mathescape, linenos, breaklines, frame=lines}
\newmintedfile{matlab}{fontsize=\small, baselinestretch=1, linenos, breaklines, frame=lines}
\newmintinline{python}{fontsize=\small, linenos, breaklines, frame=lines, python3}  % 使用\pythoninline{代码}
\newminted{python}{fontsize=\small, baselinestretch=1, linenos, breaklines, frame=lines, python3}  % 使用\begin{pythoncode}代码\end{pythoncode}
\newmintedfile{python}{fontsize=\small, baselinestretch=1, linenos, breaklines, frame=lines, python3}  % 使用\pythonfile{代码地址}

%%%% 设置行间距与页边距 %%%%
\linespread{1.2}
\geometry{left=2.5cm, right=2.5cm, top=2.5cm, bottom=2.5cm}
% \geometry{left=1.84cm,right=1.84cm,top=2.18cm,bottom=2.18cm}  % 更小的页边距

%%%% 定理类环境的定义 %%%%
\newtheorem{example}{例}            % 整体编号
\newtheorem{theorem}{定理}[section] % 定理按section编号
\newtheorem{definition}{定义}
\newtheorem{axiom}{公理}
\newtheorem{property}{性质}
\newtheorem{proposition}{命题}
\newtheorem{lemma}{引理}
\newtheorem{corollary}{推论}
\newtheorem{condition}{条件}
\newtheorem{conclusion}{结论}
\newtheorem{assumption}{假设}
% \numberwithin{equation}{section}  % 公式按section编号 (公式右端的小括号)
% \newtheorem{algorithm}{算法}

%%%% 自定义环境 %%%%
\newsavebox{\nameinfo}
\newenvironment{myTitle}[1]{
    \begin{center}
    {\zihao{-2}\bf #1\\}
    \zihao{-4}\it
}{\end{center}}  % \begin{myTitle}{标题内容}作者信息\end{myTitle}
\newcounter{problem}  % 问题序号计数器
\newenvironment{problem}[1][]{\stepcounter{problem}\par\noindent\textbf{题目\arabic{problem}. #1}}{\smallskip\par}
\newenvironment{solution}[1][]{\par\noindent\textbf{#1解答. }}{\smallskip\par}  % 可带一个参数表示题号\begin{solution}{题号}
\newenvironment{note}{\par\noindent\textbf{注记. }}{\smallskip\par}
\newenvironment{remark}{\begin{enumerate}[label=\textbf{注\arabic*.}]}{\end{enumerate}}
%\BeforeBeginEnvironment{minted}{\vspace{-0.5cm}}  % 缩小minted环境距上文间距
%\AfterEndEnvironment{minted}{\vspace{-0.2cm}}  % 缩小minted环境距下文间距

%%%% 自定义段落开头序号,间距 (titlesec) %%%%
% 中文序号:\zhnum{section}, 阿拉伯序号:\arabic
\titleformat{\section}{\Large\bfseries}{\arabic{section}}{1em}{}[]
\titlespacing{\section}{0pt}{1.2ex plus .0ex minus .0ex}{.6ex plus .0ex}
\titlespacing{\subsection}{0pt}{1.2ex plus .0ex minus .0ex}{.6ex plus .0ex}
\titlespacing{\subsubsection}{0pt}{1.2ex plus .0ex minus .0ex}{.6ex plus .0ex}

%%%% 图片相对路径 %%%%
\graphicspath{{figures/}} % 当前目录下的figures文件夹, {../figures/}则是父目录的figures文件夹
\setlength{\abovecaptionskip}{-0.2cm}  % 缩紧图片标题与图片之间的距离
\setlength{\belowcaptionskip}{0pt} 

%%%% 缩小item,enumerate,description两行间间距 %%%%
\setenumerate[1]{itemsep=0pt,partopsep=0pt,parsep=\parskip,topsep=5pt}
\setitemize[1]{itemsep=0pt,partopsep=0pt,parsep=\parskip,topsep=5pt}
\setdescription{itemsep=0pt,partopsep=0pt,parsep=\parskip,topsep=5pt}

%%%% 自定义公式 %%%%
\everymath{\displaystyle} % 默认全部行间公式, 想要变回行内公式使用\textstyle
\DeclareMathOperator*\uplim{\overline{lim}}     % 定义上极限 \uplim_{}
\DeclareMathOperator*\lowlim{\underline{lim}}   % 定义下极限 \lowlim_{}
\DeclareMathOperator*{\argmax}{arg\,max}  % 定义取最大值的参数 \argmax_{}
\DeclareMathOperator*{\argmin}{arg\,min}  % 定义取最小值的参数 \argmin_{}
\let\leq=\leqslant % 简写小于等于\leq (将全部leq变为leqslant)
\let\geq=\geqslant % 简写大于等于\geq (将全部geq变为geqslant)
\DeclareRobustCommand{\rchi}{{\mathpalette\irchi\relax}}
\newcommand{\irchi}[2]{\raisebox{\depth}{$#1\chi$}} % 使用\rchi将\chi居中

%%%% 一些宏定义 %%%%
\def\bd{\boldsymbol}        % 加粗(向量) boldsymbol
\def\disp{\displaystyle}    % 使用行间公式 displaystyle(默认)
\def\tsty{\textstyle}       % 使用行内公式 textstyle
\def\sign{\text{sign}}      % sign function
\def\wtd{\widetilde}        % 宽波浪线 widetilde
\def\R{\mathbb{R}}          % Real number
\def\N{\mathbb{N}}          % Natural number
\def\Z{\mathbb{Z}}          % Integer number
\def\Q{\mathbb{Q}}          % Rational number
\def\C{\mathbb{C}}          % Complex number
\def\K{\mathbb{K}}          % Number Field
\def\P{\mathbb{P}}          % Polynomial
\def\d{\mathrm{d}}          % differential operator
\def\e{\mathrm{e}}          % Euler's number
\def\i{\mathrm{i}}          % imaginary number
\def\re{\mathrm{Re}}        % Real part
\def\im{\mathrm{Im}}        % Imaginary part
\def\res{\mathrm{Res}}      % Residue
\def\ker{\mathrm{Ker}}      % Kernel
\def\vspan{\mathrm{vspan}}  % Span  \span与latex内核代码冲突改为\vspan
\def\L{\mathcal{L}}         % Loss function
\def\O{\mathcal{O}}         % big O notation
\def\wdh{\widehat}          % 宽帽子 widehat
\def\ol{\overline}          % 上横线 overline
\def\ul{\underline}         % 下横线 underline
\def\add{\vspace{1ex}}      % 增加行间距
\def\del{\vspace{-1.5ex}}   % 减少行间距

\makeatletter
\newenvironment{breakablealgorithm}
  {% \begin{breakablealgorithm}
   \begin{center}
     \refstepcounter{algorithm}% New algorithm
     \hrule height.8pt depth0pt \kern2pt% \@fs@pre for \@fs@ruled
     \renewcommand{\caption}[2][\relax]{% Make a new \caption
       {\raggedright\textbf{\fname@algorithm~\thealgorithm} ##2\par}%
       \ifx\relax##1\relax % #1 is \relax
         \addcontentsline{loa}{algorithm}{\protect\numberline{\thealgorithm}##2}%
       \else % #1 is not \relax
         \addcontentsline{loa}{algorithm}{\protect\numberline{\thealgorithm}##1}%
       \fi
       \kern2pt\hrule\kern2pt
     }
  }{% \end{breakablealgorithm}
     \kern2pt\hrule\relax% \@fs@post for \@fs@ruled
   \end{center}
  }
\makeatother

%%%% 正文开始 %%%%
\begin{document}

%%%% 以下部分是正文 %%%%  
\clearpage
\begin{myTitle}{机器学习实验报告}
    强基数学002\ 吴天阳\ 2204210460
\end{myTitle}
本部分将对K-均值聚类和混合高斯进行实验。
\section{K-均值聚类}
设数据集为$\{\bd{x}_1,\cdots,\bd{x}_N\},\ \bd{x}_i\in\R^D$,由$N$个在$\R^D$中的观测值构成。K-均值聚类(K-means Clustering)的目标是将数据集划分为$K$簇(Cluster),
假设$\bd{\mu}_k\in\R^D,\ (k=1,\cdots,K)$代表簇的中心,我们的目标是最小化每个数据点到最近$\bd{\mu}_k$的距离平方和。

为方便描述每个数据点的分类,引入二进制指标集$r_{nk}\in\{0,1\}$,如果数据点$\bd{x}_n$分配到簇$k$,那么$r_{nk} = 1$且$r_{nj}=0,\ (j\neq k)$。
根据该编码方法,可以定义以下最小化目标\textbf{失真度量(distortion measure)}:
\begin{equation}\label{eq-kmeansJ}
    J = \sum_{n=1}^N\sum_{k=1}^Kr_{nk}||\bd{x}_n-\bd{\mu}_k||^2
\end{equation}
K-均值聚类也是用了EM算法,首先给出如何划分为EM算法:
\begin{itemize}
    \item E步:固定$\mu_k$,求使得$J$最小化的$r_{nk}$(求出期望)。
    \item M步:固定$r_{nk}$,求使得$J$最小化的$\mu_k$(最大化)。
\end{itemize}
\paragraph{E步}当固定$\mu_k$时,由于$J$关于$r_{nk}$是线性的,即不同的数据$\bd{x}_n$之间相互独立,所以可以对于每个样本单独进行优化,对于样本$\bd{x}_n$的$r_{nk}$满足求解以下最优化问题:
\begin{equation*}
    \begin{aligned}
        \min_{r_{nk}\in\{0,1\}}&\ \sum_{k=1}^Kr_{nk}||\bd{x}_n-\bd{\mu}_k||^2\\
        s.t.&\ \sum_{k=1}^Kr_{nk} = 1
    \end{aligned}\Rightarrow r_{nk} = \begin{cases}
        1, &\text{当}\ k=\argmin_{1\leq j\leq K}||\bd{x}_n-\bd{\mu}_j||^2,\\
        0, &\text{否则}.
    \end{cases}
\end{equation*}
不难发现,最优化结果正好就表明只需将每个$\bd{x}_n$分配到最近的簇中心$\bd{\mu}_k$上。
\paragraph*{M步}当固定$r_{nk}$时,由于$J$是关于$\mu_k$的二次函数,所以可以通过导数为零确定最小化点:
\begin{equation*}
    2\sum_{n=1}^Nr_{nk}(\bd{x}_n-\bd{\mu}_k) = 0\Rightarrow \bd{\mu}_k=\frac{\disp\sum_{1\leq n\leq N}r_{nk}\bd{x}_n}{\disp\sum_{1\leq n\leq N}r_{nk}}
\end{equation*}
表达式中分母是簇$k$分配到的数据的个数,所以$\mu_k$的更新就是所有簇$k$分配到的数据点的平均值,因此被称为K-均值,算法总共两步:将每个数据点分配到最近的簇,
重新计算簇均值以替代新的簇。由于该方法每一步都会使得$J$单调递减,所以算法收敛性显然,但是它只能收敛到$J$的局部最小值。
\section{混合高斯模型}
混合高斯模型(Gaussian Mixture Model, GMM)是一种基于概率的聚类模型,首先我们可以将所有的变量均视为随机变量,包括观测变量$x$和隐变量$\theta,\omega$,
它们都服从某个概率分布,于是可以将模型参数求解转化为求解$p(\theta|D)$,即根据数据集$D$求出模型参数$\theta$的后验分布,所以可以用最大似然(MLE)方法求解。

用上述方法理解重新聚类问题:混合概率模型的隐变量就是$y$表示数据的类别种类,服从分布$p(y=k) = \pi_k$(表示全部的$\bd{x}$来自类别$k$的概率大小),
从\textbf{数据生成}的角度理解,第$k$个类别的数据$\bd{x}$应来自与$y$相关的某个分布$p(\bd{x}|y=k)$中(不妨令该分布为多维正态分布),
于是二者的联合分布为
\begin{equation*}
p(\bd{x},y) = p(y)p(\bd{x}|y) \xlongequal{y=k} \pi_k\mathcal{N}(\bd{x}|\bd{\mu}_k,\Sigma_k)   
\end{equation*}
通过联合分布我们又可以求出\textbf{数据预测}的结果:
\begin{equation*}
p(y|\bd{x}) = \frac{p(\bd{x},y)}{p(\bd{x})} \xlongequal{y=k} \frac{\pi_k\mathcal{N}(\bd{x}|\bd{\mu}_k,\Sigma_k)}{\sum_{k=1}^K\pi_k\mathcal{N}(\bd{x}|\bd{\mu}_k,\Sigma_k)}
\end{equation*}

将全部的参数简记为$\theta = (\mu_1,\cdots,\mu_K,\sigma^2,\pi_1,\cdots,\pi_K)$,于是关于$\theta$的MLE为
\begin{equation}\label{eq-MLE}
    \max_{\theta}\prod_{i=1}^Np(x_i|\theta) = \prod_{i=1}^N\sum_{k=1}^Kp(x_i,y_i=k|\theta) = \prod_{i=1}^N\sum_{k=1}^Kp(y_i=k|\theta)p(x_i|y_i=k,\theta)
\end{equation}
\subsection{由GMM导出K-均值}
我们考虑一个GMM的特殊情况,假设所有的方差均相同,即$\Sigma = \Sigma_k,\ (k=1,\cdots,K)$,并令$\pi_k = \frac{1}{n}\sum_{i=1}^nr_{nk}$,则$\bd{\mu}_k$似然函数为
\begin{align*}
    L =&\ \prod_{i=1}^{N}\sum_{k=1}^K\pi_kp(x_i) = \prod_{i=1}^{N}\sum_{k=1}^K\pi_k\mathcal{N}(x_i|\bd{\mu}_k,\Sigma)\\
    =&\ \prod_{i=1}^{N}\sum_{k=1}^K\pi_k(2\pi)^{-\frac{k}{2}}|\Sigma|^{-\frac{1}{2}}\exp\left\{-\frac{1}{2}(\bd{x}-\bd{\mu}_k)^T\Sigma^{-1}(\bd{x}-\bd{\mu}_k)\right\}
\end{align*}
取对数后得到MLE为
\begin{equation*}
    \max_{\bd{\mu}_k}-\sum_{i=1}^{N}\sum_{k=1}^K\pi_{k}||x_i-\bd{\mu}_k||^2 = \min_{\bd{\mu}_k}\sum_{i=1}^{N}\sum_{k=1}^K\pi_{k}||x_i-\bd{\mu}_k||^2
\end{equation*}
结果与K-均值(\ref{eq-kmeansJ})式中的失真度量$J$的区别仅需将$\pi_k$换为$r_{nk}$,而这个转换就会将聚类方法从GMM的\textbf{软分类}变为K-均值的\textbf{硬分类}。

其次我们可以利用GMM的预测方法证明:当方差相同时(K-均值的分类边界),分类边界是线性的。假设数据$\bd{x}$分到$i,j$类别具有相同的可能性时,也就是$p(y=i|\bd{x})=p(y=j|\bd{x})$,于是:
\begin{align*}
    0=&\ \log\frac{p(y=i|\bd{x})}{p(y=j|\bd{x})} = \log\frac{p(\bd{x}|y=i)\frac{p(y=i)}{p(\bd{x})}}{p(\bd{x}|y=j)\frac{p(y=j)}{p(\bd{x})}} = \log\frac{p(\bd{x}|y=i)\pi_i}{p(\bd{x}|y=j)\pi_j}\\
    =&\ \log\frac{\pi_i}{\pi_j} + ||\bd{x}-\bd{\mu}_i||^2 - ||\bd{x}-\bd{\mu}_j||^2 = \log\frac{\pi_i}{\pi_j} + 2(\bd{\mu}_j^T-\bd{\mu}_i^T)\bd{x}+\bd{\mu}_i^T\bd{\mu}_i - \bd{\mu}_i^T\bd{\mu}_j\\
    \Rightarrow&\  \bd{w}^T\bd{x} = \bd{b}
\end{align*}
说明如果$\bd{x}$分为类别$i,j$的可能性相同时,则$\bd{x}$一定处于直线$\bd{w}^T\bd{x}=\bd{b}$上。同理,对于一般情况$\Sigma_k$,计算得到的分类边界为$\bd{x}^TW\bd{x}+\bd{w}^T\bd{x}+\bd{c}$,
说明GMM的分类边界就是二次函数。
\subsection{使用EM算法进行参数求解}
观察(\ref{eq-MLE})式,对其取对数仍然无法将内部的求和符号展开成线性表示,所以难以求出极值,考虑基于$p(x,y)$求$\theta$的MLE:
\begin{align*}
    \max_{\theta}\prod_{i=1}^Np(\bd{x}_i,y_i|\theta)\propto\sum_{i=1}^N\log p(\bd{x}_i,y_i|\theta) = \sum_{i=1}^N\sum_{k=1}^Kp(y_i=k|\bd{x}_i,\theta)\log p(\bd{x}_i,y_i|\theta)
\end{align*}
\textbf{M步}:假设我们在$t-1$步已经得到了参数估计值$\theta^{t-1}$,于是可以在第$t$步建立$Q$函数,然后最大化该函数得到$\theta^t$
\begin{equation}\label{GMM-M}
    \max_{\theta^t}Q(\theta^t|\theta^{t-1}) = \sum_{i=1}^n\sum_{k=1}^Kp(y_i=k|\bd{x}_i,\theta^{t-1})\log p(\bd{x}_i,y_i=k|\theta^t)
\end{equation}
\textbf{E步}:就是在$t-1$步时,基于$\theta^{t-1}$计算数据$\bd{x}_i$从属于每个类别的概率:
\begin{equation*}
    R_{i,k}^{t-1} = p(y_i=k|\bd{x}_i,\theta^{t-1}) = \frac{\pi_k^{t-1}\mathcal{N}(\bd{x}_i|\bd{\mu}_k^{t-1},\Sigma_k^{t-1})}{\sum_{k=1}^K\pi_k^{t-1}\mathcal{N}(\bd{x}_i|\bd{\mu}_k^{t-1},\Sigma_k^{t-1})}
\end{equation*}
而E步的计算结果,就是M步中$Q$函数中$\log p(\bd{x}_i,y_i=k|\theta^t)$前的加权系数。

我们先不探讨上述方法的收敛性,通过Lagrange乘子法和令导数为零可以求解求解(\ref{GMM-M})式:
\begin{align}
    \text{由Lagrange乘子法:}
    \nonumber\nabla_{\pi_k^t}Q+\lambda\nabla_{\pi_k^t}\left(\sum_{k=1}^K\pi_k-1\right) = 0\Rightarrow&\ \frac{\sum_{i=1}^NR_{i,1}^{t-1}}{\pi_1^t} = \cdots=\frac{\sum_{i=1}^NR_{i,k}^{t-1}}{\pi_k^t}\\
    \nonumber\xRightarrow{\sum_{k=1}^KR_{i,k}^{t-1} = 1}&\ {\color{red}\pi_k^t = \frac{\sum_{i=1}^NR_{i,k}^{t-1}}{N}}\\
    \label{GMM-update}\nabla_{\bd{\mu}_k^t}Q = 0\Rightarrow \sum_{i=1}^NR_{i,k}^{t-1}(\bd{x}_i-\bd{\mu}_k^t) = 0\Rightarrow&\ {\color{red}\bd{\mu}_k^t = \sum_{i=1}^Nw_{ik}\bd{x}_i}\\
    \nonumber\nabla_{\Sigma_k^t}Q = 0\Rightarrow \sum_{i=1}^NR_{i,k}^{t-1}(-\Sigma_k^t + (\bd{x}_i-\bd{\mu}_k^t)^T(\bd{x}_i-\bd{\mu}_k^t)) = 0\Rightarrow&\ {\color{red}\Sigma_k^t = \sum_{i=1}^Nw_{ik}(\bd{x}_i-\bd{\mu}_k^t)^T(\bd{x}_i-\bd{\mu}_k^t)}
\end{align}
其中$w_{ik} = \frac{R_{i,k}^{t-1}}{\sum_{i=1}^NR_{i,k}^{t-1}}$。通过上式中的标红的部分,就可以得到第$t$步下的新参数值。
\subsection{EM算法收敛性证明}
首先引入一个求解函数极值的技术:若要求解$f(x)$的极小值,可以先取其定义域上任意一点$x_0$,再找到一个在$(x_0,f(x_0))$处与$f$相切的函数$g(x)$,并且要求$g(x)$是$f(x)$的上界,
令$x_1\gets \argmin_{x}g(x)$(求极大值反之亦然),根据该方法进行迭代即可得到$f(x)$的极小值点。

下面推到一个关于$\theta$的MLE重要结论:
\begin{align*}
\log p(x|\theta) =&\ \int_Yq(y)\log p(x|\theta)\,\mathrm{d}y = \int_Yq(y)\log\frac{p(x,y|\theta)}{p(y|x,\theta)}\frac{q(y)}{q(y)}\,\mathrm{d}y\\
=&\ \underbrace{\int_Yq(y)\log p(x,y|\theta)\,\mathrm{d}y}_{\text{核函数}} - \underbrace{\int_Yq(y)\log q(y)\,\mathrm{d}y}_{\text{与}\theta\text{无关}} + \underbrace{\int_Y q(y)\log\frac{q(y)}{p(y|x,\theta)}\,\mathrm{d}y}_{\text{KL}(q||p)}
\end{align*}
注意到右边第三项正好是$p,q$的KL散度,于是有$\text{KL}(q||p) \geq 0$,于是前两项构成$\log p(x|\theta)$的下界,要求极大似然的极大值,第二项与$\theta$无关,所以只需对第一项求即可。
注意上式只讨论了一个数据,极大似然是对所有数据的对数似然求和得到:
\begin{equation*}
    \max_{\theta}\sum_{i=1}^N\int_Yq(y)\log p(\bd{x}_i,y|\theta)\,\d y
\end{equation*}
于是当我们取$q(y) = p(y|\bd{x},\theta)$时,KL散度正好为$0$,极大似然对应的$\theta^*$可以通过以下迭代式求解:
\begin{equation*}
    \theta^*\gets\argmax_{\theta^*}\sum_{i=1}^N\int_Yp(y|\bd{x}_i,\theta)\log p(\bd{x}_i,y|\theta^*)\,\d y
\end{equation*}
将积分号换为求和符号就可以得到EM算法中M步的(\ref{GMM-M})式。由上面引入的函数极值求解技术,可以证明每次EM都可以使得MLE下降,从而达到极小值点,说明EM算法具有收敛性。

\clearpage
\section{实验步骤与结果分析}
\subsection{K-均值聚类}
数据集下载:\href{https://www.kaggle.com/code/niteshhalai/old-faithful-data-visualisation-and-modelling/input}{Kaggle - Old faithful}。
数据集规模:$N = 272$,$D = 2$,$N$行$D$列,使用K-means对其进行分类。
\begin{algorithm}
    \caption{K-均值聚类}
    \begin{pythoncode}
x = pd.read_csv("faithful.xls", index_col=0).to_numpy()
def normalize(x): return (x - np.mean(x)) / np.std(x)
x = np.apply_along_axis(normalize, 0, x)  # 按照列进行归一化
fig, axs = plt.subplots(2,3,figsize=(9,6))
def getdis(x, mu):
    dis = []
    for i in range(mu.shape[0]):
        dis.append(np.sqrt(np.sum(np.power(x-mu[i], 2))))
    return dis
mu = np.array([[-1, 1], [1, -1]])  # 初始化
for cnt, ax in enumerate(axs.reshape(-1)):
    # calculate the distance to each cluster
    dis = np.array([getdis(x[i], mu) for i in range(x.shape[0])])
    r = np.argmin(dis, axis=1)  # E步
    plot(...)  # 绘制图像
    # M步
    mu = np.array([np.mean(x[r==i], axis=0) for i in range(mu.shape[0])])
plt.show()
    \end{pythoncode}
\end{algorithm}
\begin{figure}[H]
    \centering
    \vspace{-1.0cm}
    \includegraphics[scale=0.7]{./code/figures/faithful/Kmeans.png}
    \caption{K-均值聚类}
\end{figure}
使用K-均值做图像分割与图像压缩方法非常直接,假设图像是$n\times m$的三通道像素,首先将图像拉直产生$N=n\times m$个数据,每个数据的维数均为$D=3$,
于是用K-均值找到$K$个簇中心$\bd{\mu}_k$,最后再用每个像素点从属的簇中心代替即可得到压缩后的图像。这样我们只需存储原图像每个像素从属的簇编号,
并记录下$\bd{\mu}_k$,从而对图像大小进行压缩。实现上使用的是scikit-learn库,因为图像较大,使用一般的线性算法速度太慢,在scikit中,
K-均值使用了KD树进行加速,底层用C++实现速度上有较大的提升。
\begin{algorithm}
    \caption{K-均值聚类图像分割}
    \begin{pythoncode}
from sklearn.cluster import KMeans
X = img.reshape(-1, 3)
fig, axs = plt.subplots(1, 4, figsize=(12, 4))
for k, ax in zip([2, 3, 10], axs):
    kmeans = KMeans(n_clusters=k).fit(X)  # 数据拟合
    pred = np.array([kmeans.cluster_centers_[i] for i in kmeans.labels_]).reshape(img.shape)  # 数据预测,转换为对应的聚类中心
    img_show(pred, ax, f"$K={k}$")
img_show(img, axs[-1], "Original Image")
plt.show()
    \end{pythoncode}
\end{algorithm}
\begin{figure}[H]
    \vspace{-0.5cm}
    \hspace{-2cm}
    \includegraphics[scale=0.65]{./code/figures/image_segmentation/kmeans_image_segmentation.png}
    \caption{K-均值聚类图像分割}
\end{figure}
\subsection{混合高斯模型GMM}
我对自己生成的数据使用了K-均值和GMM进行聚类,并比较二者的区别。数据生成方法:通过固定三个高斯分布,每个高斯下随机分布生成1000个数据
\begin{align*}
&\mathcal{N}\left(\mu_1=(1,1)^T,\Sigma=0.3I+\varepsilon\right),\\
&\mathcal{N}\left(\mu_1=(-1.5,0)^T,\Sigma=0.2I+\varepsilon\right),\\
&\mathcal{N}\left(\mu_1=(1,-1.5)^T,\Sigma=0.1I+\varepsilon\right).
\end{align*}
其中$\varepsilon$为Gauss噪声,即来自高斯分布的$2\times2$随机数据作为方差的偏移量。下面图\ref{fig-Kmeans}中展示了K-均值的聚类效果,
不难看到,最终聚类中心位置基本正确,但分类边界为硬分类,无法很好的对边界进行处理。而图\ref{fig-GMM}中展示了GMM的聚类效果,
可以看出,由于GMM的软分类性质,所以可以很好得处理边界数据,但GMM算法对初值点的选取很重要,否则容易发生两个聚类中心重合的问题。
\begin{figure}[H]
    \hspace{-1.5cm}
    %\vspace{-0.5cm}
    \includegraphics[scale=0.8]{./code/figures/GMM/Kmeans.png}
    \caption{K-均值分类结果}
    \label{fig-Kmeans}
\end{figure}
\begin{breakablealgorithm}
    \caption{GMM算法}
\begin{pythoncode}
X = ...  # 初始化数据集
# 参数初始化
mu = np.array([[-1, 2], [-1, -1], [1, -1]])
sigma = np.array([0.1*np.eye(2) for _ in range(K)])
pi = np.full(3, 1/K)
def calc_normal(x, mu, sigma):  # 计算多维正态分布
    return np.power(2*np.pi, -K/2) * np.linalg.det(sigma) * np.exp(-0.5*np.dot(np.dot(x-mu, np.linalg.inv(sigma)), (x-mu).T))
for T in range(6):
    # E步
    R = np.array([[calc_normal(X[i], mu[j], sigma[j]) for j in range(3)] for i in range(len(X))])
    plot(...)  # 绘制图像
    # M步
    w = np.concatenate([R[:,j].reshape(-1, 1)/np.sum(R[:,j]) for j in range(K)], axis=1)
    pi = np.concatenate([R[:,j].reshape(-1, 1)/N for j in range(K)], axis=1)
    sigma = np.array([np.sum([w[i,k] * np.dot((X[i]-mu[k]).reshape(-1,1), (X[i]-mu[k]).reshape(1,-1)) for i in range(N)], axis=0) for k in range(K)])
    mu = np.array([[np.dot(X[:,i], w[:,k]) for i in range(2)] for k in range(K)])
plt.show()
\end{pythoncode}
\end{breakablealgorithm}
\begin{figure}[H]
    \hspace{-1.5cm}
    %\vspace{-0.5cm}
    \includegraphics[scale=0.8]{./code/figures/GMM/GMM.png}
    \caption{GMM聚类结果}
    \label{fig-GMM}
\end{figure}
\section{结论与讨论}
\end{document}
