\documentclass[12pt, a4paper, oneside]{ctexart}
\usepackage{amsmath, amsthm, amssymb, bm, color, graphicx, geometry, mathrsfs,extarrows, braket, booktabs, array, xcolor, fontspec, appendix, float, subfigure, wrapfig, enumitem}
\usepackage[colorlinks,linkcolor=red,anchorcolor=blue,citecolor=blue,urlcolor=blue,menucolor=black]{hyperref}

%%%% 设置中文字体 %%%%
\setCJKmainfont{方正新书宋_GBK.ttf}[BoldFont = 方正小标宋_GBK, ItalicFont = 方正楷体_GBK, BoldItalicFont = 方正粗楷简体]
%%%% 设置英文字体 %%%%
\setmainfont{Times New Roman}
\setsansfont{Calibri}
\setmonofont{Consolas}

%%%% 设置代码块 %%%%
% 在vscode中使用minted需要先配置python解释器, Ctrl+Shift+P, 输入Python: Select Interpreter选择安装了Pygments的Python版本. 再在setting.json中xelatex和pdflatex的参数中加入 "--shell-escape", 即可
% TeXworks中配置方法参考: https://blog.csdn.net/RobertChenGuangzhi/article/details/108140093
\usepackage{minted}
\renewcommand{\theFancyVerbLine}{
    \sffamily\textcolor[rgb]{0.5,0.5,0.5}{\scriptsize\arabic{FancyVerbLine}}} % 修改代码前序号大小
% 加入不同语言的代码块
\newmintinline{cpp}{fontsize=\small, linenos, breaklines, frame=lines}
\newminted{cpp}{fontsize=\small, baselinestretch=1, linenos, breaklines, frame=lines}
\newmintedfile{cpp}{fontsize=\small, baselinestretch=1, linenos, breaklines, frame=lines}
\newmintinline{matlab}{fontsize=\small, linenos, breaklines, frame=lines}
\newminted{matlab}{fontsize=\small, baselinestretch=1, mathescape, linenos, breaklines, frame=lines}
\newmintedfile{matlab}{fontsize=\small, baselinestretch=1, linenos, breaklines, frame=lines}
\newmintinline{python}{fontsize=\small, linenos, breaklines, frame=lines, python3}  % 使用\pythoninline{代码}
\newminted{python}{fontsize=\small, baselinestretch=1, linenos, breaklines, frame=lines, python3}  % 使用\begin{pythoncode}代码\end{pythoncode}
\newmintedfile{python}{fontsize=\small, baselinestretch=1, linenos, breaklines, frame=lines, python3}  % 使用\pythonfile{代码地址}

%%%% 设置行间距与页边距 %%%%
\linespread{1.2}
\geometry{left=2.5cm, right=2.5cm, top=2.5cm, bottom=2.5cm}

%%%% 定理类环境的定义 %%%%
\newtheorem{example}{例}            % 整体编号
\newtheorem{theorem}{定理}[section] % 定理按section编号
\newtheorem{definition}{定义}
\newtheorem{axiom}{公理}
\newtheorem{property}{性质}
\newtheorem{proposition}{命题}
\newtheorem{lemma}{引理}
\newtheorem{corollary}{推论}
\newtheorem{condition}{条件}
\newtheorem{conclusion}{结论}
\newtheorem{assumption}{假设}
\numberwithin{equation}{section}  % 公式按section编号 (公式右端的小括号)
\newtheorem{algorithm}{算法}

%%%% 自定义环境 %%%%
\newsavebox{\nameinfo}
\newenvironment{myTitle}[1]{
    \begin{center}
    {\zihao{-2}\bf #1\\}
    \zihao{-4}\it
}{\end{center}}  % \begin{myTitle}{标题内容}作者信息\end{myTitle}
\newcounter{problem}  % 问题序号计数器
\newenvironment{problem}[1][]{\stepcounter{problem}\par\noindent\textbf{题目\arabic{problem}. #1}}{\smallskip\par}
\newenvironment{solution}[1][]{\par\noindent\textbf{#1解答. }}{\smallskip\par}  % 可带一个参数表示题号\begin{solution}{题号}
\newenvironment{note}{\par\noindent\textbf{注记. }}{\smallskip\par}
\newenvironment{remark}{\begin{enumerate}[label=\textbf{注\arabic*.}]}{\end{enumerate}}
\BeforeBeginEnvironment{minted}{\vspace{-0.5cm}}  % 缩小minted环境距上文间距
\AfterEndEnvironment{minted}{\vspace{-0.2cm}}  % 缩小minted环境距下文间距

%%%% 图片相对路径 %%%%
\graphicspath{{figure/}} % 当前目录下的figure文件夹, {../figure/}则是父目录的figure文件夹
\setlength{\abovecaptionskip}{-0.2cm}  % 缩紧图片标题与图片之间的距离
\setlength{\belowcaptionskip}{0pt} 

%%%% 缩小item,enumerate,description两行间间距 %%%%
\setenumerate[1]{itemsep=0pt,partopsep=0pt,parsep=\parskip,topsep=5pt}
\setitemize[1]{itemsep=0pt,partopsep=0pt,parsep=\parskip,topsep=5pt}
\setdescription{itemsep=0pt,partopsep=0pt,parsep=\parskip,topsep=5pt}

%%%% 自定义公式 %%%%
\everymath{\displaystyle} % 默认全部行间公式, 想要变回行内公式使用\textstyle
\DeclareMathOperator*\uplim{\overline{lim}}     % 定义上极限 \uplim_{}
\DeclareMathOperator*\lowlim{\underline{lim}}   % 定义下极限 \lowlim_{}
\DeclareMathOperator*{\argmax}{arg\,max}  % 定义取最大值的参数 \argmax_{}
\DeclareMathOperator*{\argmin}{arg\,min}  % 定义取最小值的参数 \argmin_{}
\let\leq=\leqslant % 简写小于等于\leq (将全部leq变为leqslant)
\let\geq=\geqslant % 简写大于等于\geq (将全部geq变为geqslant)
\DeclareRobustCommand{\rchi}{{\mathpalette\irchi\relax}}
\newcommand{\irchi}[2]{\raisebox{\depth}{$#1\chi$}} % 使用\rchi将\chi居中

%%%% 一些宏定义 %%%%
\def\bd{\boldsymbol}        % 加粗(向量) boldsymbol
\def\disp{\displaystyle}    % 使用行间公式 displaystyle(默认)
\def\tsty{\textstyle}       % 使用行内公式 textstyle
\def\sign{\text{sign}}      % sign function
\def\wtd{\widetilde}        % 宽波浪线 widetilde
\def\R{\mathbb{R}}          % Real number
\def\N{\mathbb{N}}          % Natural number
\def\Z{\mathbb{Z}}          % Integer number
\def\Q{\mathbb{Q}}          % Rational number
\def\C{\mathbb{C}}          % Complex number
\def\K{\mathbb{K}}          % Number Field
\def\P{\mathbb{P}}          % Polynomial
\def\d{\mathrm{d}}          % differential operator
\def\e{\mathrm{e}}          % Euler's number
\def\i{\mathrm{i}}          % imaginary number
\def\re{\mathrm{Re}}        % Real part
\def\im{\mathrm{Im}}        % Imaginary part
\def\res{\mathrm{Res}}      % Residue
\def\ker{\mathrm{Ker}}      % Kernel
\def\vspan{\mathrm{vspan}}  % Span  \span与latex内核代码冲突改为\vspan
\def\L{\mathcal{L}}         % Loss function
\def\O{\mathcal{O}}
\def\wdh{\widehat}          % 宽帽子 widehat
\def\ol{\overline}          % 上横线 overline
\def\ul{\underline}         % 下横线 underline
\def\add{\vspace{1ex}}      % 增加行间距
\def\del{\vspace{-1.5ex}}   % 减少行间距

%%%% 正文开始 %%%%
\begin{document}
%%%% 以下部分是正文 %%%%  
\clearpage
\begin{myTitle}{数据结构与算法 - 综合训练1\\用树解决问题}
    吴天阳\ 2204210460\ 强基数学002
\end{myTitle}
\section{实验目的}
\textbf{实验1}:实现二叉检索树(Binary Search Tree, BST),需要包含以下$8$种结构
\begin{pythoncode}
class BSTBase():  # BST基类
    def insert(self, key, value): pass  # 插入(key,value)键值对
    def remove(self, key): pass  # 删除key节点,若找到并删除节点,则返回对应的value,若无该节点则返回None
    def search(self, key): pass  # 查询key对应的value,若无key节点,则返回None
    def update(self, key, value): pass  # 更新key对应的value,若无key节点,返回False,否则返回True
    def isEmpty(self): pass  # 判断二叉搜索树是否为空
    def clear(self): pass  # 重新初始化BST
    def showStructure(self, file): pass  # 输出当前二叉树的节点总数和高度到文件file中
    def printInorder(self, file): pass  # 输出二叉树的中序遍历到文件file中
\end{pythoncode}

读入命令文件\pythoninline{BST_testcases.txt}并处理执行,将输出运行结果到文件中,并与\\\pythoninline{BST_result.txt}进行比较,判断程序是否正确.

\textbf{实验2}:使用BST为文稿建立单词索引表,基于给出的英文文本\pythoninline{article.txt},记录每个单词在文本中出现的行号(编号从$1$开始).
\section{实验原理}
二叉搜索树是一种二叉树形式的数据结构,支持\textbf{插入、查找、删除}键值对的功能,对于一个有$n$个节点的二叉搜索树,每次操作的最优复杂度为 $\O(\log n)$,最坏为 $\O(n)$,随机构成一颗二叉树的期望高度为 $\O(\log n)$.

\textbf{注}:平衡树是在二叉搜索树基础上进行的改进,可以保证每次查找的复杂度为 $\O(\log n)$.

二叉搜索树定义如下:
\begin{enumerate}
    \item 空树是二叉搜索树.
    \item 若二叉搜索树的左子树非空,则左子树上所有点的key值均小于根节点的key值.
    \item 若二叉搜索树的右子树非空,则右子树上所有点的key值均大于根节点的key值.
    \item 二叉搜索树的左右子树均为二叉搜索树.
\end{enumerate}
实现以上操作基本均通过递归即可实现,节点之间的关联可用指针实现,具体实现请见下文.

\section{实验步骤与结果分析}
使用\pythoninline{Python 3.9.12}进行实现. 在Python中数值类型数据\pythoninline{int, float, str}无法直接传入实参,但是使用\pythoninline{class, list, dict}则默认传入实参,所以需要利用该性质实现指针操作.

代码文件总共有三个:\pythoninline{my_bst.py}包含BST核心类,\pythoninline{main1.py, main2.py}分别为实验1和实验2的代码,对应的输出文件分别为\pythoninline{my_result1.txt, my_result2.txt}.
\subsection{实现BST类}
\subsubsection{初始化}
首先创建节点类\pythoninline{Node},具体有以下属性:
\begin{itemize}
    \item \pythoninline{key(int)}:存储key值.
    \item \pythoninline{val0}:用于初始化val值.(用于实验2,可以为list,存储多个value值)
    \item \pythoninline{val}:存储val值.
    \item \pythoninline{child(list)}:长度为2的list,\pythoninline{child[0],child[1]}分别表示左右子节点.
\end{itemize}
\begin{pythoncode}
class Node():  # 节点子类
    val0 = 0
    def __init__(self):
        self.key = None  # 初始化键值
        self.child = [None, None]  # 初始化左右孩子节点
        if isinstance(self.val0, list):
            self.val = []  # 由于list按照实参赋值,必须重新创建空list
        else: self.val = self.val0
\end{pythoncode}

初始化BST类
\begin{pythoncode}
def __init__(self, val0=0):
    self.Node.val0 = val0  # val0为每个节点值的初值
    self.file = None  # 将要写入的文件
    self.root = None  # 创建根节点
    self.height = 0  # 树的高度
    self.size = 0  # 树的节点总数
\end{pythoncode}

判断二叉搜索树是否为空,只需判断\pythoninline{BST.root}是否为\pythoninline{None}即可.
\begin{pythoncode}
def isEmpty(self):
    return self.root == None
\end{pythoncode}

重新初始化整个BST,由于需要删除掉全部节点,所以需要递归删除,这里使用中序遍历中的递归顺便完成该任务:
\begin{pythoncode}
def dfs(self, p, delete_node=False):  # 输出中序遍历结果
    if p.child[0]:
        self.dfs(p.child[0], delete_node)
    if not delete_node:
        self.file.write('[{} ---- < {} >]\n'.format(p.key, str(p.val)[1:-1]))  # 在文件中直接写入
    if p.child[1]:
        self.dfs(p.child[1], delete_node)
    if delete_node:  # 用于清空整棵树
        del p

def clear(self):
    self.dfs(self.root, delete_node=True)
    self.__init__()  # 调用初始化函数, 清空BST

def printInorder(self, file):
    if file is None:
        return None
    self.file = file
    self.dfs(self.root)
    return   # 返回中序遍历
\end{pythoncode}

\subsubsection{加入节点}
直接通过递归找到对应键值位置,然后修改value值即可.
\begin{pythoncode}
def add(self, p, key, val):
    if p is None:  # 当前节点为空节点, 开始创建
        p = self.Node()
        p.key = key
    if key == p.key:  # 找到key值对应节点, 更新节点val
        if isinstance(p.val, list):  # 如果val当前是列表, 则加入值
            p.val.append(val)
        else:  # 否则直接修改当前值
            p.val = val
    elif key < p.key:  # key节点在左子树中
        p.child[0] = self.add(p.child[0], key, val)
    else:  # key节点在右子树中
        p.child[1] = self.add(p.child[1], key, val)
    return p

def insert(self, key, val):
    if key is None or val is None:
        return
    self.root = self.add(self.root, key, val)  # 每次从根节点开始查找插入位置
\end{pythoncode}

\subsubsection{查询节点}
直接通过递归搜索即可.
\begin{pythoncode}
def find(self, p, key):
    if p == None:  # 空节点
        return None
    elif key == p.key:  # 找到了key节点
        return p.val
    elif key < p.key:  # key节点在左子树中
        return self.find(p.child[0], key)
    return self.find(p.child[1], key)  # 否则只能在右子树中

def search(self, key):
    if key is None:
        return None
    return self.find(self.root, key)
\end{pythoncode}

\subsubsection{删除节点}
删除节点稍微有点麻烦,若删除的节点有两个子节点,为了保持二叉搜索树的性质,需要用左子树中的最大值,或者右子树的最小值替代删除掉的节点;若删除的节点仅有一个儿子,则直接用它代替即可;若删除的节点为叶子结点,直接删去即可.
\begin{pythoncode}
def delete_min(self, p):  # 查找最小值
    if p.child[0] is None:  # 找到最小值
        return p, p.child[1]  # 返回的第一个参数为找到的最小值点, 第二个参数为更新点编号
    min_p, update_p = self.delete_min(p.child[0])
    p.child[0] = update_p  # 用第二个参数更新该点
    return min_p, p  # 返回的第一个参数为找到的最小值点, 第二个参数为更新点编号

def delete(self, p, key):
    if p.key == key:
        if p.child[0] and p.child[1]:  # 如果有两个儿子节点, 就需要找到左子树中的最大值或右子树的最小值替代
            # 这里找右子树最小值替代
            min_p, update_p = self.delete_min(p.child[1])
            min_p.child[0] = p.child[0]
            min_p.child[1] = update_p
            del p  # 删除掉当前节点p
            return min_p
        else:
            ret = p.child[0] if p.child[0] else p.child[1]
            del p  # 删除掉当前节点p
            return ret
    elif key < p.key:  # key节点在左子树中
        p.child[0] = self.delete(p.child[0], key)
    else:  # key节点在右子树中
        p.child[1] = self.delete(p.child[1], key)
    return p

def remove(self, key):
    val = self.search(key)
    if key is None or val is None:  # 如果找不到该key值也返回False
        return None
    self.root = self.delete(self.root, key)
    return val
\end{pythoncode}
\subsubsection{查询BST结构}
直接通过递归即可完成.
\begin{pythoncode}
def struct(self, p, h):  # 查找bst的节点数和高度
    self.size += 1  # 总节点数+1
    self.height = max(self.height, h)  # 更新树的深度
    if p.child[0]:
        self.struct(p.child[0], h + 1)
    if p.child[1]:
        self.struct(p.child[1], h + 1)

def showStructure(self, file):  # 返回树的总节点数, 返回树的高度
    if file is None:
        return None
    self.size, self.height = 0, 0  # 先初始化为0
    self.struct(self.root, 1)
    file.write('-----------------------------\n')
    file.write(f'There are {self.size} nodes in this BST.\nThe height of this BST is {self.height}.\n')
    file.write('-----------------------------\n')
\end{pythoncode}
\subsection{实验1}
主要是对读入字符串的划分和条件判断,执行对应的BST函数即可.
\begin{pythoncode}
import my_bst

with open('BST_testcases.txt', 'r', encoding='utf-8') as file,\
        open('my_result1.txt', 'w', encoding='utf-8') as outfile:
    bst = my_bst.BST()
    while True:
        line = file.readline()
        if not line:
            break
        opt = line[0]
        if opt == '#':
            bst.showStructure(outfile)
            continue
        key = line.split(' ')[1]  # 提取key值
        if opt == '+':
            value = line.split('\"')[1]  # 提取value值
            bst.insert(key, value)
        elif opt == '-':
            value = bst.remove(key)
            if value is not None:
                outfile.write(f'remove success ---{key} {value}\n')
            else:
                outfile.write(f'remove unsuccess ---{key}\n')
        elif opt == '?':
            value = bst.search(key)
            if value is not None:
                outfile.write(f'search success ---{key} {value}\n')
            else:
                outfile.write(f'search unsuccess ---{key}\n')
        elif opt == '=':
            value = line.split('\"')[1]  # 提取value值
            flag = bst.update(key, value)
            if flag:
                outfile.write(f'update success ---{key} {value}\n')
            else:
                outfile.write(f'update unsuccess ---{key}\n')
\end{pythoncode}
\subsection{实验2}
主要先对文本进行清洗,清洗方法非常简单,按空格划分出单词,然后去除掉首尾的非字母符号即可.
\begin{pythoncode}
import my_bst

def wash(word):  # 将其他字符都去掉, 只剩下拉丁字母
    while len(word):
        if not word[-1].isalpha():
            word = word[:-1]
        elif not word[0].isalpha():
            word = word[1:]
        else:
            break
    return word

with open('article.txt', 'r', encoding='utf-8', errors='ignore') as file:
    bst = my_bst.BST(val0=[])
    cnt = 0  # 用于记录行数
    while True:
        # print(cnt)
        line = file.readline()
        if not line:
            break
        cnt += 1
        line = wash(line)  # 将每一行也洗一下, 把多余的空格去掉
        words = line.split(' ')  # 将一行中的单词分离出来
        for key in words:  # 逐个遍历
            key = wash(key)  # 将key中的其他标点去掉, 只剩下单词
            if not len(key):  # 洗没了
                continue
            bst.insert(key, cnt)

with open("my_result2.txt", 'w') as file:
    bst.printInorder(file)
\end{pythoncode}

\section{结论与讨论}
最后通过比对,\pythoninline{my_reslut1.txt}与\pythoninline{BST_result.txt}完全一致,说明BST算法应该基本正确,实验2的输出文件为 \pythoninline{my_result2.txt},去重后总计有 $16580$ 个单词.

本次实验,我学会了如何使用Python实现BST的指针操作,相比C语言虽然不直观,但是实现效果非常简洁. Python在处理文本上相较于C也更为方便.
\end{document}
